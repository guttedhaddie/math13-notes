\graphicspath{{5induction/asy/}}
\section{Mathematical Induction and Well-ordering}\label{chap:induction}

In Section \ref{sec:proof} we discussed three methods of proof: direct, contrapositive, and contradiction. The fourth standard method, \emph{induction,} has a very different flavor. Before discussing this formally, we consider some contexts in which induction arguments often arise.

\subsection{Iterative Processes \& Proof by Induction}\label{sec:induction}

Recursive processes are very common in mathematics and its applications: an initial value $x_1$ determines a sequence of values $(x_n)$ via a \emph{recurrence relation} $x_{n+1}=f(x_n)$. A typical approach to such problems is to \emph{hypothesize} a general formula $x_n=g(n)$---spot a pattern!---before \emph{proving} the validity of the formula. Induction is the proof method often employed in such situations. To get us started, we investigate a famous game.


\boldsubsubsection{The Tower of Hanoi}

Circular disks of decreasing radius are stacked on three pegs. A single move consists of removing one disks from the top of a stack then placing it on an empty peg or on top of a larger disk. If we start with 10 disks on the first peg, how many moves are required to transfer all disks to another peg?\smallbreak

To get a feel for the problem, try playing the game with small numbers of disks. Suppose $n$ disks require $r_n$ moves. Then:
\begin{itemize}\itemsep0pt
  \item $r_1=1$ since there is only one disk to move!
  \item $r_2=3$ is demonstrated in the picture.
\end{itemize} 

\begin{center}
	\includegraphics[width=0.9\textwidth]{induction-03-hanoi2}
\end{center}

Further experimentation will hopefully convince you that $r_3=7$, at which point you might be ready to hypothesize a general formula---if not, experiment more!

\begin{conj}{}{hanoi}
	The Tower of Hanoi with $n$ disks requires $r_n=2^n-1$ moves.
\end{conj}

To make progress we need to think \emph{abstractly}. If we have a stack of $n+1$ disks, then to move the largest disk \emph{all others must be stacked on a single peg.} Moving $n+1$ disks to another peg is therefore a three-step process:
\begin{enumerate}\itemsep0pt
  \item Move the smallest $n$ disks to another peg ($r_n$ moves);
  \item Move the largest disk (one move);
  \item Move the remaining disks on top of the largest ($r_n$ moves).
\end{enumerate}
\begin{center}
	\href{http://www.math.uci.edu/~ndonalds/math13/induction-01-hanoi.html}{
		\includegraphics{induction-04-hanoirn3}\quad
		\includegraphics{induction-04-hanoirn4}\quad
		\includegraphics{induction-04-hanoirn5}
	}
\end{center}


The upshot is that $r_n$ satisfies a \emph{recurrence relation}: $r_{n+1}=r_n+1+r_n=2r_n+1$.

\goodbreak

We are now in a position to prove our conjecture.

\begin{proof}
	Certainly the formula $r_n=2^n-1$ holds when $n=1$ disk (one disk requires $r_1=1$ move).\smallbreak
	Now suppose that $n$ disks require $r_n=2^n-1$ moves, where $n\in\N$ is some fixed number. Then $n+1$ disks require
	\[
		r_{n+1}=2r_n+1=2(2^n-1)+1 =2^{n+1}-1 \tag{$\ast$}
	\]
	moves. Since $n$ was arbitrary, we see that we've proved an \emph{infinite family of implications}: 
	\[
		(r_1=2^1-1)\implies (r_2=2^2-1)\implies (r_3=2^3-1) \implies \cdots \implies (r_n=2^n-1)\implies \cdots
	\]
	Since the first proposition ($r_1=2^1-1$) holds, we conclude that all do: $r_n=2^n-1$ for all $n\in\N$.
\end{proof}

To answer the original question, 10 disks require $r_{10}=2^{10}-1=1023$ moves; at one move per second this would take 17 minutes, 3 seconds.



\boldsubsubsection{Proof by Induction}

The above argument is an example of a \emph{proof by induction.} We invoke this method when we want to prove a sequence of propositions $P(1)$, $P(2)$, $P(3)$, \ldots, one for each natural number. The abstract structure of an induction proof consists of two separate arguments:
\begin{enumerate}
  \item \emph{Base case}:\lstsp Prove $P(1)$.
  \item \emph{Induction step}:\lstsp Prove $P(n)\Longrightarrow P(n+1)$ (for each $n\in\N$). During this phase, $P(n)$ is termed the \emph{induction hypothesis.}
\end{enumerate}
The result is an infinite chain of implications:
\[
	P(1)\implies P(2)\implies P(3)\implies P(4)\implies P(5)\implies \cdots
\]
Since $P(1)$ is true (base case), \emph{all} remaining propositions $P(2)$, $P(3)$, $P(4)$, \ldots{} are also true.\smallbreak

Re-read the Tower of Hanoi proof; can you separate the base case and the induction step? Since this is an introduction, our presentation was informal. A few modifications should be made to produce a formal argument.

\begin{itemize}
  \item \emph{Set-up} the proof by stating, ``We prove by induction.'' It might also be helpful to spell out the propositions $P(n)$ and to tell the reader what variable ($n$) controls the induction.
  \item Label the \emph{base case} and \emph{induction step} to aid the reader.
  \item After the induction step is complete, state your \emph{conclusion.} In the above we would replace everything after ($\ast$) with, ``By mathematical induction, $r^n=2^n-1$ for all $n\in\N$.''
\end{itemize}

Here is a straightforward and famous result, where we write the proof in our new language.

\begin{thm}{}{ind1}
	The sum of the first $n$ positive integers is $\sum\limits_{k=1}^nk=\frac 12n(n+1)$
\end{thm}

You should be familiar with summation notation $\sum\limits_{k=1}^nk=1+2+3+\cdots+n$ from calculus: if not, \emph{ask}.

\goodbreak



\begin{proof}
	We prove by induction. For each $n\in\N$, let $P(n)$ be the proposition $\smash{\sum\limits_{k=1}^nk=\frac 12n(n+1)}$.
	\begin{description}
		\item[\normalfont\emph{Base case} ($n=1$):] \smash{$\sum\limits_{k=1}^1k=1=\frac 121(1+1)$,} says that $P(1)$ is true.
		\item[\normalfont\emph{Induction Step}:] Fix $n\in\N$ and assume $P(n)$ is true \textbf{for this $\pmb n$.} We compute the sum of the first $n+1$ positive integers and use the induction hypothesis $P(n)$ to simplify:
	  \begin{align*}
			\sum_{k=1}^{n+1}k
			&=\textcolor{blue}{\sum_{k=1}^nk}+(n+1)=\textcolor{blue}{\frac 12n(n+1)}+(n+1) \tag{\textcolor{blue}{induction hypothesis}}\\
			&=\left(1+\frac 12n\right)(n+1)=\frac 12(n+2)(n+1)\\
			&=\frac 12(n+1)\bigl[(n+1)+1\bigr]
	  \end{align*}
	Therefore $P(n+1)$ is true.
	\end{description}
	By mathematical induction, we conclude that $P(n)$ is true for all $n\in\N$. Otherwise said,
	\[
		\forall n\in\N,\quad \sum\limits_{k=1}^nk=\frac 12n(n+1)\tag*{\qedhere}
	\]
\end{proof}

Note how we grouped $\frac 12(n+1)\bigl[(n+1)+1\bigr]$ so that it is obviously the right hand side of $P(n+1)$.\smallbreak

We present several more examples in a similar vein, though done a little faster. As is typical, we don't explicitly introduce the notation $P(n)$, though you should feel free to continue doing so if you find it helpful. Aim to lay out your formal arguments in a similar style. 


\begin{example}{}{ind3}
	We prove by induction that, for all $n\in\N$,
	\[
		2+5+8+\cdots+(3n-1)=\frac 12n(3n+1) \tag{$\dag$}
	\]
	\begin{description}\itemsep0pt
		\item[\normalfont\emph{Base case} ($n=1$):] The proposition ($\dag$) is trivially true: $2=\frac 12\cdot 1\cdot(3\cdot 1+1)$.
		\item[\normalfont\emph{Induction Step}:] Fix $n\in\N$ and assume ($\dag$) holds for this value of $n$. Then
		\begin{align*}
			\textcolor{blue}{2+5+\cdots+(3n-1)}+[3(n+1)-1]&=\textcolor{blue}{\frac 12n(3n+1)}+3n+2\\
			&=\frac 12(3n^2+7n+4) =\frac 12(n+1)(3n+4)\\
			&=\frac 12(n+1)\bigl[3(n+1)+1\bigr]
		\end{align*}
		which is the required proposition for $n+1$.
	\end{description}
	By mathematical induction, ($\dag$) holds for all $n\in\N$.
\end{example}

For brevity we labelled the desired proposition (what we'd might call $P(n)$) by $(\dag)$ so it could be referenced. The structure is similar to Theorem \ref{thm:ind1}: since the goal is to evaluate a sum, the induction step is little more than adding the same thing ($3n+2$) to both sides of the \textcolor{blue}{induction hypothesis}. In fact, the example could have been proved directly as a corollary of Theorem \ref{thm:ind1}---can you see how?
\medbreak\goodbreak

Our next two examples are a little harder, requiring more creativity to invoke the induction hypothesis. Both can alternatively be proved directly using modular arithmetic (Chapter \ref{chap:gcd}).

\begin{examples}{}{ind2}
	\exstart We prove by induction: $\forall n\in\N$, the integer $17^n-4^n$ is divisible by 13.
	\begin{enumerate}\setcounter{enumi}{1}
		\item[]\begin{description}
			\item[\normalfont\emph{Base case} ($n=1$):] Plainly $17^1-4^1$ is divisible by 13.
			\item[\normalfont\emph{Induction Step}:] Fix $n\in\N$ and assume that $17^n-4^n=13k$ for some $k\in\Z$. Then
			\begin{align*}
				17^{n+1}-4^{n+1}
				&=17\bigl(\textcolor{blue}{17^n}\bigr)-4^{n+1}
					=17\bigl(\textcolor{blue}{13k+4^n}\bigr)-4^{n+1}
					\tag{\textcolor{blue}{induction hypothesis}}\\
				&=17\cdot 13k+(17-4)4^n 
					=13\bigl(17k+4^n\bigr)
			\end{align*}
			is divisible by 13.
		\end{description}
		By mathematical induction, $17^n-4^n$ is divisible by 13 for all $n\in\N$.		
		
		\item\label{ex:ind22} We prove by induction: if $n\in\N$, then $n(n+1)(2n+1)$ is divisible by 6.
		\begin{description}
			\item[\normalfont\emph{Base case} ($n=1$):] The proposition reads $1\cdot (1+1)\cdot (2\cdot 1+1)=6$, which is divisible by 6.
			\item[\normalfont\emph{Induction Step}:] Fix $n\in\N$ and assume that $\textcolor{blue}{n(n+1)(2n+1)=6k}$ for some $k\in\Z$. Then
			\begin{align*}
					(n+1)(n+2)\bigl[2(n+1)+1\bigr]-\textcolor{blue}{6k}&=\textcolor{blue}{(n+1)}\bigl[(n+2)(2n+3)-\textcolor{blue}{n(2n+1)}\bigr]\\
					&=(n+1)\bigl(2n^2+7n+6-(2n^2+n)\bigr)\\
					&=6(n+1)^2
				\end{align*}
			from which
			\[
				(n+1)\bigl[(n+1)+1\bigr]\bigl[2(n+1)+1\bigr] =6\bigl((n+1)^2+k\bigr)
			\]
			is divisible by 6.
		\end{description}
		By mathematical induction, $n(n-1)(2n-1)$ is divisible by 6 for all $n\in\N$.	
	\end{enumerate}
\end{examples}


\boldinline{Scratch work is your friend!}

Unless things are very simple, start with some scratch work for the hard part: the \emph{induction step.} Explicitly state the propositions $P(n)$ and $P(n+1)$ and try to manipulate one into the other. Here are the relevant propositions for Example \ref{ex:ind2}.1:
\begin{quote}
	\begin{tabular}{ll}
		$P(n)$: &$\exists k\in\Z$ such that $17^n-4^n=13k$\\[5pt]
		$P(n+1)$: &$\exists l\in\Z$ such that $17^{n+1}-4^{n+1}=13l$
	\end{tabular}
\end{quote}
Since $17^n$ is common to both, it is natural to try multiplying both sides of the equation in $P(n)$ by 17; if you re-read the example, you'll see that this is essentially the induction step! For Example \ref*{ex:ind2}.2, you might try multiplying out the cubic expressions
\[
	n(n+1)(2n+1)\ \text{ and }\ (n+1)(n+2)(2n+3)
\]
and comparing coefficients. Since the leading term in both is $n^3$, the \emph{difference} is quadratic and therefore much easier to think about\ldots\smallbreak

Remember that scratch work isn't a proof; while it might make perfect sense to you, it isn't a proof unless a reader can follow it without assistance. Once you think you understand the induction step, lay out the entire proof cleanly: \emph{set-up, base case, induction step, conclusion.} As an example of what happens when you don't, here is a typical attempt at Example \ref{ex:ind3} by someone new to induction:

\begin{tcolorbox}[exstyle]\vspace{-10pt}
	\begin{align*}
		P(n+1)=\hspace{4pt}
		&2+5+\cdots+(3n-1)+[3(n+1)-1] =\smash{\frac 12}(n+1)\bigl[3(n+1)+1\bigr]\\
		&\frac 12n(3n+1)+(3n+2) =\frac 12(n+1)(3n+4)\\
		&3n^2+n+6n+4 =3n^2+7n+4
	\end{align*}
\end{tcolorbox}

Is this a good argument? While there are many issues,\footnote{%
	\textbullet\lstsp There is no \emph{set-up, base case} or \emph{conclusion}, and the word \emph{induction} is missing. The argument also needs some English.
	\begin{itemize}\itemsep0pt
	  \item $P(n)$ has not been \emph{defined}. If you don't define it, don't write it.
	  \item $P(n+1)$ is a \emph{proposition}: it cannot \emph{equal} a number! Replacing ``$P(n+1)=$'' with ``$P(n+1)\Longleftrightarrow$'' would correct this.
	  \item There are no conditional connectives to indicate the logical flow. Moreover, read top to bottom, the argument is essentially $P(n)\wedge P(n+1)\Longrightarrow T$, rather than the correct induction step $P(n)\Longrightarrow P(n+1)$.
	\end{itemize}
} the work isn't without merit: the required calculation is present (left side of 1\st{} line $=$ left side of 2\nd). While helpful as scratch work, a substantial re-write is needed to make this convincing to a reader.
\medbreak

We finish this section with a trickier example of this thinking at work.

\begin{example}{}{}
	An \emph{L-shaped tromino} is an arrangement of three squares in an ``L'' shape. We claim:\par
	\begin{minipage}[t]{0.75\linewidth}\vspace{-4pt}
		\begin{quote}
			If \textcolor{Magenta}{any} single square is removed from a $2^n\times 2^n$ square gird, then the remaining grid may be tiled by L-shaped trominos.
		\end{quote}
		\vspace{-6pt}
		The claim has the form $\forall n\in\N,P(n)$, but note that $P(n)$ is itself \textcolor{Magenta}{universal}. The picture shows one of the \emph{sixteen} possible examples when $n=2$. To get an idea of how to structure the induction step, think how you might use $2\times 2$ grids to analyse a $4\times 4$ grid: the picture shows how!
		\begin{itemize}
		  \item Whatever square we remove, one quarter of the $2^2\times 2^2$ grid is a $2^1\times 2^1$ grid with one square removed: this is tilable (the \textcolor{blue}{blue tromino}).
		\end{itemize}
	\end{minipage}
	\hfill
	\begin{minipage}[t]{0.24\linewidth}\vspace{0pt}
		\flushright\includegraphics{induction-11-tilingexample}
	\end{minipage}
	\par

	\vspace{-2pt}
		
	\begin{itemize}
	 	\item Place a single tromino (\textcolor{orange}{orange} in the picture) so that one of its squares lies in each remaining quadrant. What's left of each quadrant is a $2^1\times 2^1$ grid with one missing square: again tilable.
	\end{itemize}
	
	This scratch work is really an argument $P(1)\Longrightarrow P(2)$! It remains only to formalize this intuition into a general proof. We proceed by induction on $n$.
	\begin{description}
		\item[\normalfont\emph{Base case} ($n=1$):] If a single square is removed from a $2\times 2$ grid, the three remaining squares form single L-shaped tromino.
		\item[\normalfont\emph{Induction step}:] Fix $n\in\N$ and assume that after removing \emph{any} square from \emph{any} $2^n\times 2^n$ grid, the remainder is tilable. Now take any $2^{n+1}\times 2^{n+1}$ grid and remove a square.
	\begin{itemize}
	  \item By the induction hypothesis, the $2^n\times 2^n$ quadrant containing the removed square is tilable.
	  \item Place a single tromino in the center so that one of its squares lies in each remaining quadrant. What's left of each quadrant is a $2^n\times 2^n$ grid with one missing square, each of which is tilable by the induction hypothesis.
	\end{itemize}  
	\end{description}
	By induction, we conclude that every $2^n\times 2^n$ grid is tilable by trominos after any square is removed.
\end{example}


\goodbreak

\begin{exercises}{}{}
	A reading quiz and several questions with linked video solutions can be found \href{http://www.math.uci.edu/~ndonalds/math13/selftest/5-1-induction.html}{online}.

	\begin{enumerate}
	  \item Suppose you move one disk on the Tower of Hanoi per second.
	  \begin{enumerate}
	    \item One of the oldest versions of the problem has monks transferring a tower of 64 disks. Roughly how many years would this take?
	    \item In a realistic human lifetime, how large a tower could be moved?
	  \end{enumerate}
	  
	  
	  \item Imagine you cut a large large piece of paper in half and stack the two pieces on top of each other. You then repeat the process, cutting all sheets in half and making a single taller stack.
		\begin{center}
			\includegraphics{induction-02-paper}
		\end{center}
		If a single sheet of paper has thickness $0.1$\,mm, how many times would you have to repeat the cut-and-stack process until the stack of paper reached to the sun? ($\approx 150$ million kilometers). \emph{Prove} that you are correct.
	  
	
	  \item A room contains $n$ people. Everybody wants to shake everyone else's hand (but not their own).
	  \begin{enumerate}
	    \item Suppose $n$ people require $h_n$ handshakes. If person $n+1$ enters the room, how many \emph{additional} handshakes are required? Obtain a recurrence relation for $h_{n+1}$ in terms of $h_n$.
	    \item Hypothesize a general formula for $h_n$, and prove it by induction.
	  \end{enumerate}
	  
	  
	  \item\begin{enumerate}
	    \item In Example \ref*{ex:ind2}.\ref{ex:ind22}, what is the proposition $P(n+1)$?
	    \item In the induction step of Example \ref*{ex:ind2}.\ref{ex:ind22}, explain why it would be incorrect to write
	    \begin{align*}
				P(n+1)-P(n)
				&=(n+1)\bigl[(n+2)(2n+3)-n(2n+1)\bigr]\\
				&=(n+1)(2n^2+7n+6-2n^2-n)\\
				&=6(n+1)^2
			\end{align*}
			\item Extend the Example: prove by induction that $\sum\limits_{k=1}^nk^2=\frac 16n(n+1)(2n+1)$.
	  \end{enumerate}
  
  
	  \item Prove by induction that for each natural number $n$, we have $\smash{\sum\limits_{k=0}^n2^k=2^{n+1}-1}$.
		
		
		\item Consider the statement: If $n$ is a natural number, then $\smash{\sum\limits_{k=1}^nk^3=\frac 14n^2(n+1)^2}$
		\begin{enumerate}
		  \item What, explicitly, is $\smash{\sum\limits_{k=1}^4k^3}$?
		  \item What would be meant by the expression $\smash{\sum\limits_{k=1}^nn^3}$, and why is it different to $\smash{\sum\limits_{k=1}^nk^3}$?
		  \item Proof the statement by induction.
	  \end{enumerate}
	
	
  	\item\begin{enumerate}
    	\item Prove by induction that $\forall n\in\N$ we have $3\mid(2^n+2^{n+1})$.
    	\item Give a direct proof that $3\mid(2^n+2^{n+1})$ for all integers $n\ge 1$.
  	\end{enumerate}

  
		\item Prove \emph{by induction} that for every $n\in\N$ we have $n\equiv 5$ or $n\equiv 6$ or $n\equiv 7\spmod 3$.
	
		\item Prove by induction that, for all $n\in\N$,
		\[
			1\cdot 2+2\cdot 3+3\cdot 4+\cdots +n(n+1) =\frac 13n(n+1)(n+2)
		\]

		\item\begin{enumerate}
		  \item Show, by induction, that for all $n\in\N$, the number 4 divides the integer $11^n-7^n$.
			\item More generally, prove by induction that $(a-b)\mid (a^n-b^n)$ for any $a,b,n\in\N$.
		\end{enumerate}
	
		
		\item\begin{enumerate}
		  \item Find a formula for the sum of the first $n$ odd natural numbers. Prove your assertion.
		 	\item Use Theorem \ref{thm:ind1} to give an alternative direct proof of your formula.
		\end{enumerate} 
  

		\item Find the error in the following ``proof'' of the statement, ``All cats have the same color fur.''
		\begin{proof}
	    Let $P(n)$ be the proposition, ``Any set of $n$ cats have the same color fur.'' We prove by induction on $n$. 
	    \begin{description}\itemsep0pt
	    	\item[\normalfont\emph{Base case} ($n=1$):] Any cat has the same color fur as itself.
	    	\item[\normalfont\emph{Induction step}:] Fix $n\in\N$ and assume $P(n)$. Take any set $S= \{C_1,C_2,\ldots,C_{n+1}\}$ of $n+1$ cats. The set $S\setminus\{C_1\}$ has $n$ cats; by the induction hypothesis all have the same color fur. Again by the induction hypothesis, all cats in $S\setminus\{C_2\}$ have the same color fur. Combining these observations, we see that all cats in $S$ have the same color fur. Since $S$ was arbitrary, we see that $P(n+1)$ holds.
	    \end{description}
	    By induction, $P(n)$ is true for all $n\in\N$, which establishes the claim.
		\end{proof}

		\item Use induction, the product rule, and the fact that $\diff xx=1$ to prove the power law from calculus:
		\[
	    \forall n\in\N,\ \diff x x^n =nx^{n-1}
		\]


		\item A (real) \emph{polynomial} of degree $n$ is a function $p(x)=a_nx^n+a_{n-1}x^{n-1}+ \cdots +a_1x+a_0$, whose \emph{coefficients} $a_k$ are real numbers and where $a_n\neq 0$.
		\begin{enumerate}
    	\item Prove: for all $n\in\N$,
    	\[
        \diff[{}^n]{x^n} e^{x^2} = p_n(x) e^{x^2}
    	\]
    	where $p_n(x)$ is some polynomial of degree $n$.

			\item (Hard)\lstsp Let $p(x)$ be a polynomial of degree $n\ge 1$. Show $p$ has at most $n$ roots.\par
			(\emph{Hint: induct on the degree $n$})
		\end{enumerate}


		\item Consider the following scratch work. Determine what result is being proved, then convert the scratch work into a formal proof of that result.
	  \begin{align*}
	    (1+x)^{n+1}
	    &=(1+x)^n(1+x)\ge (1+nx)(1+x)\\
	    &=1+x+nx+nx^2=1+(n+1)x+nx^2\\
	    &\ge 1+(n+1)x
	  \end{align*}

	\end{enumerate}

\end{exercises}

\clearpage



\subsection{Well-ordering and the Principle of Mathematical Induction}\label{sec:wellorder}


In this section we think more carefully about the logic behind induction, and tie it to a fundamental property of the natural numbers.

\begin{defn}{}{wellorder}
	A non-empty set of real numbers $A$ is \emph{well-ordered} if every non-empty subset of $A$ contains a minimum element.
\end{defn}

To test if a set $A$ is well-ordered, we need to check \emph{all} of its non-empty subsets. The definition could be written as equivalently as follows (in the second line we expand what is meant by a \emph{minimum}):
\begin{itemize}
  \item If $B\subseteq A$ and $B\neq\emptyset$, then $\min B$ exists.
  \item $\bigl((B\subseteq A)\wedge (B\neq\emptyset)\bigr)\implies \bigl(\exists b\in B,\ \forall x\in B, \ b\le x\bigr)$
\end{itemize}
To show that $A$ is \emph{not} well-ordered, we need only exhibit a non-empty subset $B$ with \emph{no minimum.}

\begin{examples}{}{}
	\exstart $A=\{4,-7,\pi,19,\ln 2\}$ is a well-ordered set. There are 31(!) non-empty subsets of $A$, each of which has a minimum element.\vspace{-5pt}
	\begin{enumerate}\setcounter{enumi}{1}
	  \item[] Can you justify this fact \emph{without} listing the subsets? It might be easier to think about why any \emph{finite} set $A=\{a_1,\ldots,a_n\}\subseteq\R$ is well-ordered\ldots 
	  
	  \item The interval $A=[3,10)$ is not well-ordered. Indeed $B=(3,4)$ is a non-empty subset with no minimal element. While you should believe this, let's prove it anyway!\par
	  We need to prove that $\forall b\in B, \exists x\in B$ with $x<b$. Given any $b\in(3,4)$, observe that $x:=\frac{b+3}2$ satisfies
	  \[
	  	3<x<b<4 \quad\text{from which}\quad x\in B\text{ and }x<b
	  \]
	  You could also argue by contradiction (if $b\in B$ is minimal, then\ldots).
	  
	  \item The integers $\Z$ are not well-ordered. For instance, $\Z$ is a non-empty subset of itself and there is no minimal integer.
	\end{enumerate}
\end{examples}

You might suspect (wrongly!) that every well-ordered set is finite. That the natural numbers form a well-ordered \emph{infinite} set is, for us, an axiom,\footnote{%
	There are many ways to define the natural numbers. Typically well-ordering is either an axiom (essentially part of the definition) or a theorem. Compare with Exercise \ref{exs:peano} for an alternative approach.
}
a foundational claim forming part of our basic conception of the natural numbers.

\begin{axiom}{Well-ordering Principle}{}
	$\N$ is well-ordered.
\end{axiom}


Also known as the \emph{least natural number principle}, well-ordering is applied widely throughout mathematics. In fact we've already done so in this text! Consider the set of positive remainders generated by the Euclidean algorithm (Theorem \ref{thm:euclidalg}) when applied to natural numbers $a>b$:
\[
	\{\ldots,r_2,r_1,b,a\}\subseteq\N
\]
Well-ordering guarantees that this set has a minimal element $r_t$ (which turns out to be $\gcd(a,b)$); this is essentially the argument for Exercise \ref*{sec:gcd}.\ref{exs:euclidalgproof}(a).

\goodbreak  

%  When written in roster notation in increasing order, any set that `looks like' $\N$, is also well-ordered. For example
% \[
% 	A=\left\{0,\frac 12,\frac 23,\frac 34,\frac 45,\ldots\right\}=\left\{\frac n{n+1}:n\in\N\right\}
% \]
Armed with the well-ordering principle, we can justify the method of proof by induction. 

\begin{thm}{Principle of Mathematical Induction}{ind}
	For each $n\in\N$, let $P(n)$ be a proposition. Additionally make the two standard assumptions:
	\begin{quote}
		\begin{tabular}{@{}ll}
			Base case: &$P(1)$ is true\\[5pt]
			Induction step: &$\forall n\in\N,\ P(n)\Longrightarrow P(n+1)$
		\end{tabular}
	\end{quote}
	Then $P(n)$ is true for all $n\in\N$.
\end{thm}

Before attempting a proof, consider how the theorem could be written as a pure implication:
\[
	\textcolor{blue}{P(1)\wedge\bigl(\forall n\in\N,P(n)\Longrightarrow P(n+1)\bigr)}\implies \textcolor{Green}{\bigl(\forall n\in\N, P(n)\bigr)}
\]
This helps us select a proof strategy: a direct approach seems hard since the \textcolor{Green}{conclusion} is universal; a contrapositive approach requires an ugly negation of the \textcolor{blue}{hypothesis}; a proof by contradiction seems most sensible since negation of the \textcolor{Green}{conclusion} is straightforward.

\begin{proof}
	We argue by contradiction. Assume the base case, the induction step, and that $\exists n\in\N$ for which $P(n)$ is \emph{false.} The set of natural numbers
	\[
		S:=\bigr\{k\in\N:P(k)\text{ is false}\bigr\}
	\]
	is therefore non-empty ($n\in S$). Well-ordering guarantees that $s:=\min S$ exists. Observe:
	\begin{itemize}\itemsep0pt\parskip2pt
	  \item $s\in S\Longrightarrow$ \textcolor{red}{$P(s)$ is \emph{false}}.
	  \item The \emph{base case} tells us that $s\neq 1$. Thus $s\ge 2$ and $s-1\in\N$.
	  \item $s-1<\min S\Longrightarrow P(s-1)$ is \emph{true.}
	  \item The \emph{induction step} ($P(s-1)\Longrightarrow P(s)$) tells us that \textcolor{red}{$P(s)$ is \emph{true}}.
	\end{itemize}
	\textcolor{red}{Contradiction}: $P(s)$ cannot be both true and false!
\end{proof}


Now we have the proof, it is straightforward to extend the principle of induction. For any integer $m$ (positive, negative or zero), the set
\[
	\Z_{\ge m}=\{n\in\Z:n\ge m\}=\{m,m+1,m+2,m+3,\ldots\}
\]
is also well-ordered. By changing the base case to $P(m)$ and replacing $\N$ with $\Z_{\ge m}$, we immediately obtain the proof of a more general principle of induction.

\begin{cor}{Induction with base case $m$}{indbasem}
	Fix an integer $m$. For each integer $n\ge m$, let $P(n)$ be a proposition. Suppose:
	\begin{quote}
		\begin{tabular}{@{}ll}
			Base case: &$P(m)$ is true\\[5pt]
			Induction step: &$\forall n\in\Z_{\ge m},\ P(n)\Longrightarrow P(n+1)$
		\end{tabular}
	\end{quote}
	Then $P(n)$ is true for all $n\in\Z_{\ge m}$.
\end{cor}

The intuitive concept is exactly as before, just with a different base case!
\[
	P(m)\implies P(m+1)\implies P(m+2)\implies P(m+3)\implies\cdots
\]
\goodbreak



\begin{examples}{}{}
	\exstart For all integers $n\ge 2$, we prove that\footnotemark
	\[
		\sum\limits_{k=2}^n\frac 1{k(k-1)} =1-\frac 1n\tag{$\ast$}
	\]	
	\begin{enumerate}\setcounter{enumi}{1}
		\item[]\begin{description}
			\item[\normalfont\emph{Base case} ($n=2$):] When $n=2$, ($\ast$) reads $\smash{\sum\limits_{i=2}^2}\frac 1{i(i-1)}=\frac 12=1-\frac 12$.
			\item[\normalfont\emph{Induction step}:] Assume that ($\ast$) is true for some fixed $n\ge 2$. Then
			\begin{align*}
				\sum_{i=2}^{n+1}\frac 1{k(k-1)}
				&=\sum_{i=2}^{n}\frac 1{k(k-1)}+\frac 1{(n+1)n} 
					=1-\frac 1n+\frac 1{n(n+1)}
					\tag{induction hypothesis}\\
				&=1-\frac{(n+1)-1}{n(n+1)}
					=1-\frac 1{n+1}
			\end{align*}
			which is precisely ($\ast$) when $n$ is replaced by $n+1$.
		\end{description}
		By induction, ($\ast$) holds for all integers $n\ge 2$.
	
	  \item For all integers $n\ge 4$, we claim that $3^n>n^3$.\par
	  We really do need the given base case: when $n=3$, the claim $3^3>3^3$ is false! As is often the case, it helps to do some scratch work. The \textcolor{blue}{induction hypothesis} allows us to see that 
	  \[
	  	3^{n+1}=3\cdot \textcolor{blue}{3^n} >3 \textcolor{blue}{n^3}
	  \] 
	  The proof of the induction step thus hinges on being able to show that $3n^3\ge (n+1)^3$. There are many ways to convince yourself of this, for instance
	  \[
	  	3n^3\ge(n+1)^3\iff 3\ge\left(\frac{n+1}n\right)^3 =\left(1+\frac 1n\right)^3 \tag{$\dag$}
	  \]
	  The right side \emph{decreases} as $n$ increases; since $n\ge 4$, the right side is at most $\left(\frac 54\right)^3=\frac{125}{64}<2$, whence ($\dag$) holds for all $n\ge 4$.\smallbreak
	  
	  We now prove the original claim by induction.	  
	  \begin{description}
	  	\item[\normalfont\emph{Base case} ($n=4$):] Observe that $3^4=81>64=4^3$.
	  	\item[\normalfont\emph{Induction step}:] Fix $n\in\Z_{\ge 4}$ and suppose that $3^n>n^3$. By ($\dag$), we see that
			\[
				3^{n+1}=3\cdot 3^n>3n^3\ge (n+1)^3
			\]
	  \end{description}
	  By induction, we conclude that $3^n>n^3$ whenever $n\in\Z_{\ge 4}$.
	\end{enumerate}
\end{examples}


\footnotetext{%
	You might have encountered this example in calculus as a \emph{telescoping series}:
	\[
		\sum\limits_{k=2}^n\frac 1{k(k-1)}=\frac 1{2\cdot 1}+\frac 1{3\cdot 2}+\cdots +\frac 1{n(n-1)} =\left(1-\frac 12\right) +\left(\frac 12-\frac 13\right) +\cdots +\left(\frac 1{n-1}-\frac 1n\right) =1-\frac 1n
	\]
	Taking the limit as $n\to\infty$ results in $\smash{\sum\limits_{k=2}^\infty\frac 1{k(k-1)}=1}$. Induction does this without the ambiguous ellipses ($\cdots$).
}


\goodbreak


\begin{example}{}{}
	We prove an extended version of de Morgan's law for sets (Theorem \ref{thm:setcomp}(a)): for any collection of sets $A_1,\ldots,A_n$ where $n\ge 2$, we have
	\[
	 	\comp{(A_1\cap \cdots\cap A_n)}=\comp{A_1}\cup\cdots\cup\comp{A_n} \tag{$\ddag$}
	\]
	\begin{description}
		\item[\normalfont\emph{Base case} ($n=2$):] $\comp{A_1\cap A_2}=\comp{A_1}\cup\comp{A_2}$ is precisely the standard de Morgan identity.
		\item[\normalfont\emph{Induction step}:] Fix $n\in\N_{\ge 2}$ and suppose ($\ddag$) holds for \emph{all} collections of $n$ sets. Given a collection of $n+1$ sets, we see that
		\begin{align*}
			\comp{(A_1\cap \cdots\cap A_n\cap A_{n+1})}
			&=\comp{\bigl((A_1\cap \cdots\cap A_n)\cap A_{n+1}\bigr)}\\
			&=\comp{(A_1\cap \cdots\cap A_n)}\cup \comp{A_{n+1}} \tag{de Morgan again!}\\
			&=\comp{A_1}\cup\cdots\cup\comp{A_n}\cup\comp{A_{n+1}} \tag{induction hypothesis}
		\end{align*}
	\end{description}
	By induction, the claim ($\ddag$) holds for any collection of $n$ sets.
\end{example}

We could have approached the argument as a standard induction with base case $n=1$. Instead we deliberately chose $n=2$, both to avoid confusion (the $n=1$ case $\comp{A_1}=\comp{A_1}$ isn't helpful or interesting) and to highlight the importance of de Morgan's law for two sets to the entire argument.


\boldsubsubsection{Proof by Minimal Counter-example}

Sometimes authors present induction arguments as contradiction proofs in the style of Theorem \ref{thm:ind}, with $s=\min S$ being known as the \emph{minimal counter-example.} Here are two variations on this idea; the first is a straight translation of an induction where the base case and induction step are clear.

\begin{examples}{}{}
	\exstart We prove: for all $n\in\N_0$, $\sum\limits_{k=0}^n 2^k=2^{n+1}-1$.\vspace{-10pt}
	
	\begin{enumerate}\setcounter{enumi}{1}
	  \item[]Suppose to the contrary and let $s\ge 0$ be the minimal counter-example.
	  \begin{itemize}
	    \item Since $\smash[t]{\sum\limits_{k=0}^0} 2^k=2^0=1=2^{0+1}-1$, we see that $s\neq 0$.\hfill (\emph{base case})
	    \item Since the claim holds for $s-1$, we see that
	    \[
	    	\sum\limits_{k=0}^s2^k=2^s+\sum\limits_{k=0}^{s-1} 2^k =2^s+2^s-1=2^{s+1}-1 \tag{\emph{induction step}}
	    \]
	    This contradicts the fact that $s$ is a counter-example!
	  \end{itemize} 
	  
	  
	  \item We re-prove Theorem \ref{thm:sqrt2}: $\sqrt 2\notin\Q$. Suppose $\sqrt 2$ is rational and consider the set
	  \[
	  	S=\bigl\{x\in\N:\exists y\in \N\text{ with }x^2=2y^2\bigr\}
	  \]
	  If $S$ is non-empty, let $s=\min S$, and let $t\in\N$ be such that $s^2=2t^2$. Plainly $\textcolor{red}{t<s}$. Since $s^2$ is even, $s$ is also even, and we can write $s=2k$. But then
	  \[
	  	4k^2=2t^2\implies t^2=2k^2\implies \textcolor{red}{t\in S}
	  \]
	  But this contradicts the minimality of $s$.\par
	  This approach is often used in number theory in the guise of Fermat's \emph{method of infinite descent.}
	\end{enumerate}
\end{examples}



\begin{aside}{}{}
	\boldinline{Aside: Well-ordering more generally}

	Definition \ref{defn:wellorder} is a weak version of a much deeper concept. Informally, to \emph{well-order} a set means to list its elements in some order so that every non-empty subset has an initial element \emph{with respect to that order.}\par
	For instance, the set of negative integers $\Z^-=\{\ldots,-4,-3,-2,-1\}$ is \emph{not} well-ordered with respect to the standard ordering of the integers, but is well-ordered with respect to the \emph{reverse} ordering
	\[
		\Z^-=\{-1,-2,-3,-4,\ldots\}
	\]
	The principle of mathematical induction is easily modified to accommodate theorems of the form $\forall n\in\Z^-,\ P(n)$: the base case is $P(-1)$ and the induction step justifies the chain
	\[
		P(-1)\implies P(-2)\implies P(-3)\implies\cdots
	\]
	All the infinite well-ordered sets we've thus far seen have ``looked like'' the natural numbers, however more esoteric examples exist. For instance, the following well-ordered set looks like two copies of the natural numbers, one following the other:
	\[
		A=\left\{0,\frac 12,\frac 23,\frac 34,\frac 45,\ldots,1,\frac 32,\frac 53,\frac 74,\frac 95,\ldots\right\}=\left\{1-\frac 1n:n\in\N\right\}\cup\left\{2-\frac 1n:n\in\N\right\}
	\]
	Every non-empty subset of $A$ really does have a minimum!	It is possible to modify induction to apply to propositions indexed by well-ordered sets like this, though an extra step is required to deal with \emph{limit elements} (like $1\in A$) with no immediate predecessor. If your further studies include set theory, you'll likely spend much time considering well-orders and their associated \emph{ordinals.}
\end{aside}


\begin{exercises}
	A reading quiz and several questions with linked video solutions can be found \href{http://www.math.uci.edu/~ndonalds/math13/selftest/5-2-wellorder.html}{online}.

	\begin{enumerate}
  	\item Prove that the interval $(-2,5]$ has no minimum element.
	
	
		\item Prove that every finite set of real numbers is well-ordered.
  
  
  	\item\begin{enumerate}
    	\item Suppose that $n\ge 3$. Prove that $\left(\frac{n+1}n\right)^2<2$.
    	\item Hence or otherwise, prove that $n^2<2^n$ for all natural numbers $n\ge 5$.
  	\end{enumerate}
  	

	  \item Consider the following result. For every natural number $n\ge 2$,
		\[
			\left(1-\frac{1}{4}\right) \left(1-\frac{1}{9}\right) \left(1-\frac{1}{16}\right) \cdots \left(1-\frac{1}{n^2}\right) = \frac{n+1}{2n}
		\]
	  \begin{enumerate}
	    \item If the statement is written in the form $\forall n\in\N_{\ge 2},\ P(n)$, what is the proposition $P(n)$?
% 	    \item $\Pi$-notation is used for products in the same way as $\Sigma$-notation for sums: for example
% 	    \[\prod_{k=1}^5(k+1)^k=2^1\cdot 3^2\cdot 4^3\cdot 5^4\cdot 6^5\]
% 	    Rewrite the statement using $\Pi$-notation.
	    \item Prove the result by induction.
	  \end{enumerate}
	  
	  
	
		\item Prove the geometric series formula: if $r\neq 1$ and $n\in\N_0$, then
			$\smash{\sum\limits_{k=0}^n}r^k=\frac{1-r^{n+1}}{1-r}$
			
		
		\item For all integers $n\ge 3$, prove that $\smash{\sum\limits_{k=3}^n}\frac 1{k(k-2)} =\frac 34-\frac{2n-1}{2n(n-1)}$	
		
		
		\item Prove: for any $n\in\N$, $\smash{\sum\limits_{i=1}^n}\frac{1}{i^2}<2$\par
	(\emph{Hint: prove the stronger fact that $\smash{\sum\limits_{i=1}^n} \frac{1}{i^2} < 2 - \frac{1}{n}$ for all $n \ge 2$})
	
	
		\goodbreak
		
	
  	\item The set $A_3=\{1,2,3\}$ satisfies the property that the sum of its elements ($1+2+3=6$) is divisible by every element of $A_3$.
  	\begin{enumerate}
  	  \item Use induction to prove that for any $n\ge 3$, there is a set $A_n$ of $n$ natural numbers such that the sum of its elements is divisible by every element of $A_n$.
  	  \item Prove by contradiction that no set of \emph{two} natural numbers satisfies this property.
  	\end{enumerate}
 	
	
		\item Suppose that $x^2+4y^2=3z^2$ has a solution $(x,y,z)$ where all three are \emph{positive integers.}
		\begin{enumerate}
	  	\item By considering remainders modulo 3, prove that $3\mid z$. Thus create a new solution $(X,Y,Z)$ in positive integers, where $Z<z$.
	  	\item Use the method of minimal counter-example to prove that $x^2+4y^2=3z^2$ has no solutions where $x,y,z\in\N$.
		\end{enumerate}
  
	
		\item We use the fact that $\N_0$ is well-ordered to prove the division algorithm (Theorem \ref{thm:div}).
		\begin{quote}
			\emph{If $m\in\Z$ and $n\in\N$, then $\exists$ unique $q,r\in\Z$ such that $m=qn+r$ and $0\le r<n$.}
		\end{quote}
	
		Given $m,n$, define
		\[
			S=\N_0\cap\bigl(m+n\Z\bigr) =\bigl\{k\in\N_0:k=m-qn\text{ for some } q\in\Z\bigr\}
		\]	
		\begin{enumerate}
			\item (Existence)\lstsp Show that $S$ is a \emph{non-empty} subset of $\N_0$. By well-ordering, \emph{define} $r:=\min S$. Prove that $0\le r<n$.
			\item (Uniqueness)\lstsp Suppose two pairs of integers $(q_1,r_1)$ and $(q_2,r_2)$ satisfy $m=q_in+r_i$ and $0\le r_1,r_2<n$. Prove that $r_1=r_2$.
		\end{enumerate}
		
	
	  \item\label{exs:peano} (Hard)\lstsp We consider a version of Peano's axioms for the natural numbers.
		\begin{itemize}
			\item[i.] (\emph{Initial element})\lstsp $1\in\N$
			\item[ii.] (\emph{Successor function})\lstsp $f(n)=n+1$ is a function $f:\N\to\N$
			\item[iii.] (\emph{No predecessor of the initial element})\lstsp $1\not\in\range(f)$
			\item[iv.] (\emph{Unique predecessor/order})\lstsp $f$ is injective: $m+1=n+1\Longrightarrow m=n$
			\item[v.] (\emph{Induction})\lstsp Any subset $A\subseteq\N$ with the following properties \emph{equals} $\N$:
			\[
				1\in A\quad\text{and}\quad \forall a\in A,\ a+1\in A
			\]
		\end{itemize}
		\begin{enumerate}
			\item Replace $\N$ with $\Z$ in each axiom. Which are true and which false?
			\item Let $T=\bigl\{(m,n):m,n\in\N\bigr\}$ be the set of all ordered pairs of natural numbers.
			\begin{enumerate}
			  \item Let $f:T\to T$ be the function $f(m,n)=(m+1,n)$. Letting the pair $(1,1)$ play the role of 1, and $f$ the successor function, decide which of Peano's axioms are satisfied by $T$.
				\item Repeat the question for the same initial element and 
				\[
					f:T\to T:(m,n)\mapsto
					\begin{cases}
						(m-1,n+1)&\text{if }m\ge 2\\
						(m+n,1)&\text{if }m=1
					\end{cases}
				\]
			\end{enumerate}
			\item Prove that $\range(f)=\N\setminus\{1\}$: every element except 1 is the successor of something.\par
			(\emph{Hint: let $A=\{1\}\cup\range(f)$ in the induction axiom})
			\item Prove that $\N$, as defined by Peano, is well-ordered (with respect to $x<x+1$, etc.).
		\end{enumerate}

	\end{enumerate}

\end{exercises}


% The induction arguments in the above examples are so simple that they hardly seem worth mentioning. In other situations things can be much harder.\\
% Bob
% 
% \begin{example}
% Recall the monotone convergence theorem from sequences. If $(x_n)$ is an increasing (decreasing) sequence bounded above (below), then it is convergent. Here we use this theorem to prove that the following sequence converges to $\frac 12$:
% \[\begin{cases}
% x_{n+1}=\frac 13(x_n+1)+(x_n-\tfrac 12)^2,\\
% x_1=1.
% \end{cases}\]
% You can try hunting for a general formula for $x_n$ (if you find one, let us know\ldots). Instead we observe the first few terms of the sequence: $(x_n)=(1,\frac{11}{12},\frac{13}{16},\frac{539}{768},\ldots)$ and hypothesize:\\[10pt]
% Conjecture:\quad $(x_n)$ is a decreasing sequence and $x_n>\frac 12$ for all $n\in\N$.\\
% We prove by induction.\\
% Certainly $x_1=1>\frac 12$. Now if $x_n>\frac 12$, we have $x_n-\frac 12\neq 0$, whence
% \[x_{n+1}>\frac 13\left(x_n+1\right)>\frac 13\left(\frac 12+1\right)=\frac 13\cdot\frac 32=\frac 12.\]
% Thus all $x_n>\frac 12$ by induction.\\[10pt]
% Given this, we can now see that
% \[x_{n+1}-x_n=\frac 13(1-2x_n)-\left(x_n-\frac 12\right)^2<0,\]
% thus $(x_n)$ is decreasing. Since $(x_n)$ is also bounded below (by $\frac 12$), the monotone convergence theorem says that the sequence converges.\\
% Call the limit $x$. Clearly $x\ge 1$. But then
% \[x=\frac 13(x+1)+\left(x-\frac 12\right)^2\iff x=\frac 12\text{ or }\frac 76.\]
% Since $(x_n)$ is decreasing from 1, it is clear that $\lim_{n\to\infty}x_n=\frac 12$.\\
% Note that it is essential that we establish the existence of the limit before calculating it: the same sequence but with initial value $x_1=2$ is \emph{divergent} to $\infty$.
% \end{example}

% Our final example involves a little abstraction.
% 
% \begin{thm}{}{polygon}
% 	The interior angles of an $n$-gon ($n$-sided polygon) sum to $180(n-2)$ degrees.
% \end{thm}
% 
% We will take the initial case ($n=3$) that the angles of a triangle sum to \ang{180} as given (can you prove it?) and merely prove the induction step. The main logical difficulty is that we must consider \emph{all} $n$-gons simultaneously. If we were to write the induction step in the form
% \[
% 	\forall n\in\Z_{\ge 3},\ P(n)\implies P(n+1)
% \]
% then the proposition $P(n)$ would be
% \[
% 	P(n):\quad \forall n\text{-gons }\cP_n,\text{ the sum of the interior angles of $\cP_n$ is $180(n-2)$\textdegree}
% \]
% To prove our induction step for a \emph{fixed} integer $n$, we must show that \emph{all} $(n+1)$-gons have the correct sum of interior angles. We therefore assume that we are given some $(n+1)$-gon $\cP_{n+1}$ and proceed to compute its interior angles in terms of a related $n$-gon.
% 
% \begin{proof}
% 	Fix an integer $n\ge 3$, and suppose that \emph{all} $n$-gons have interior angles summing to $180(n-2)$\textdegree. Suppose we are given an $(n+1)$-gon $\cP_{n+1}$. Select any vertex $A$ and label the adjacent vertices $B$ and $C$. Delete $A$, and join $B$ and $C$ with a straight edge. The result is an $n$-gon $\cP_n$. There are two cases to consider.\footnotemark{}\par
% 
% 	\begin{minipage}[t]{0.64\linewidth}\vspace{0pt}
% 		Case 1: The deleted point $A$ is \emph{outside} $\mathcal{P}_n$. The sum of the interior angles of $\mathcal{P}_{n+1}$ exceeds those of $P_n$ by $\alpha+\beta+\gamma=180$\textdegree. Therefore $\mathcal{P}_{n+1}$ has interior angles summing to $180(n-2)\text{\textdegree}+180\text{\textdegree}=180[(n+1)-2]$\textdegree.\par
% 	
% 		Case 2: The deleted point $A$ is \emph{inside} $\mathcal{P}_n$. To obtain the sum of the interior angles of $\mathcal{P}_{n+1}$, we take the sum of the interior angles of $\mathcal{P}_n$ and do three things:
% 		\begin{itemize}\setlength{\itemsep}{0pt}
% 	  	\item \emph{Subtract} $\beta$
% 	  	\item \emph{Subtract} $\gamma$
% 	  	\item \emph{Add the reflex angle $360$\textdegree$-\alpha$ at $A$}
% 		\end{itemize}
% 		We are therefore adding an additional
% 		\[
% 			-\beta-\gamma+(360\text{\textdegree}-\alpha)=360\text{\textdegree}-(\alpha+\beta+\gamma)=180\text{\textdegree}
% 		\]
% 	\end{minipage}
% 	\hfill
% 	\begin{minipage}[t]{0.35\linewidth}\vspace{0pt}
% 		\includegraphics[width=\textwidth]{induction-05-polygon}\\
% 		Case 1: $A$ outside $\mathcal{P}_n$\\[25pt]
% 		\includegraphics[width=\textwidth]{induction-06-polygon}\\
% 		Case 2: $A$ inside $\mathcal{P}_n$
% 	\end{minipage}\\[8pt]
% 	$\mathcal{P}_{n+1}$ again has interior angles summing to $180[(n+1)-2]$\textdegree.
% \end{proof}
% 
% \footnotetext{We are obscuring two subtleties here. It is a fact, though not an obvious one, that it is always possible to choose a vertex $A$ so that the new polygon $\cP_n$ doesn't cross itself. Read about `ears' and `mouths' of polygons and triangulation if you're interested. There are also two other, less likely, cases which we didn't consider: when deleting a point from an ($n+1$)-gon it is possible to obtain  an $(n-1)$-gon, or even an $(n-2)$-gon. To think it out, try drawing a 12-gon in the shape of a Star of David. Deleting one of the outer corners creates a 9-gon! Dealing with these cases strictly requires strong induction, so we return to them later.}

%Wellorder stuff!
%     \item Prove: $n! \le \left(\frac{n + 1}{2}\right)^n$.\par
%      (\emph{Hint: find a formula for the sum of the first $n$ positive integers})
%   


% \subsubsection*{Optional: Density of the Rationals}
% 
% In our last example, we offer a more direct application of $\N$ being well-ordered. One of the  key properties of the rational numbers $\Q$ is their density in the real line. Intuitively, the idea is that no matter how close you "zoom in" on the real line, you can always locate a rational number. We formalize this with the following definition.
% 
% \begin{defn}
% We say a set $A \subseteq \R$ is \textbf{dense} (in $\R$) if for any real numbers $x$ and $y$ such that $x < y$, there is  $a \in A$ such that $x < a < y$.
% \end{defn}
% 
% So if you take two real numbers, you can always find an element from $A$ in between them, no matter how close the two real numbers are from each other. Our goal will be to prove that the rational numbers $\Q$ are dense in $\R$. For this, we will use the well-orderedness of $\N$ along with the following:
% 
% \begin{axiom}
% The real numbers $\R$ have the \textbf{Archimedean property}, that is, for any real numbers $x,y > 0$, there is $n \in \N$ such that $nx > y$.
% \end{axiom}
% 
% It is not really necessary to take this as an axiom as the Archimedean property of $\R$ can be proved from more basic principles. However, this requires some knowledge about how to construct the real numbers which lies beyond the scope of this course. Back to our goal, we need the following lemma which states that if two real numbers differ by more than $1$, then there must be an integer between them.
% 
% \begin{lemm}\label{lemm:rationaldensityprep}
% Suppose we have $x,y \in \R$ with $y - x > 1$. Then there exists $k \in \Z$ such that $x < k < y$.
% \end{lemm}
% 
% \begin{proof}
% The idea is to take $k$ to be the least integer greater than $x$. We will show such an integer exists using the fact that $\N$ is well-ordered. Let $A = \{n \in \Z : n > x\}$. Then $A \neq \emptyset$ by the Archimedean property (why?). Let $m \in \Z$ be a number such that $m < x$ (this is another application of the Archimedean property), and thus $m < n$ for all $n \in A$, by definition of $A$. Let 
% \[
% S = \{n - m + 1 : n \in A\}.
% \]
% So $S \subseteq \N$ and since $A \neq \emptyset$, we have $S \neq \emptyset$. Since $\N$ is well-ordered, $S$ has  a minimum element $s$. Then $k = s + m - 1$ is the minimum element of $A$ (why?).
% 
% By definition $x < k$. But by minimality of $k$, $k - 1 \notin A$, i.e., $x \geq k - 1$. Thus $x < k \leq x + 1$. Finally, since $y - x > 1$, we have $x + 1 < y$. All together, $x < k \leq x + 1 < y$. So $k$ is as required.
% \end{proof}
% 
% Now we can prove our main result.
% 
% \begin{thm}
% The rational numbers $\Q$ are dense in $\R$.
% \end{thm}
% 
% \begin{proof}
% Let $x,y \in \R$ with $x < y$ be arbitrary. We need to find $r \in \Q$ with $x < r < y$. Then $y - x > 0$. By the Archimedean property, there is $n \in \N$ such that $n(y - x) > 1$. Since $ny - nx > 1$, we can apply Lemma \ref{lemm:rationaldensityprep} to get $k \in \Z$ such that $nx < k < ny$. As $n \geq 1 > 0$, dividing yields $x < \frac{k}{n} < y$. Take $r = \frac{k}{n} \in \Q$.
% \end{proof}


\clearpage



\subsection{Strong Induction}\label{sec:strongind}

The principle of mathematical induction (Theorem \ref{thm:ind}) is often known as \emph{weak} induction. \emph{Strong} induction differs primarily in that the induction step can assume more than one previous proposition.

\begin{thm}{Principle of Strong Induction}{indstrong}
	Let $l\ge m$ be fixed integers and suppose $P(n)$ is a proposition, one for each $n\in\Z_{\ge m}$. Suppose:
	\begin{quote}
		\begin{tabular}{@{}ll}
			Base case(s): &$P(m),\ P(m+1),\ \ldots,\ P(l)$ are true\\[5pt]
			Induction step: &$\forall n\ge l,\ \bigl(P(m)\wedge P(m+1)\wedge\cdots\wedge P(n)\bigr)\Longrightarrow P(n+1)$
		\end{tabular}
	\end{quote}
	Then $P(n)$ is true for all $n\ge m$.
\end{thm}

Exercise \ref{exs:strongindproof} provides a proof by showing that strong and weak induction are equivalent. We instead concentrate on a few examples. The additional difficulty of strong induction comes from determining how many base cases are required and in phrasing the induction hypothesis: in practice one rarely needs to employ all the propositions $P(m),\ldots,P(n)$.


\begin{example}{Fibonacci numbers}{fibonacci}
	This famous sequence $(f_n)_{n=1}^\infty=(1,1,2,3,5,8,13,21,\ldots)$ is defined by the 2\nd-order recurrence relation
	\[
		\begin{cases}
			f_{n+1}=f_n+f_{n-1}&\text{if }n\ge 2\\
			f_1=f_2=1&
		\end{cases}
	\]
	While the Fibonacci sequence seems to be increasing, it also appears to be less than doubling at each step, suggesting the claim
	\[
		\tcbhighmath{\forall n\in\N,\ f_n<2^n}
	\]
	We prove this using strong induction. \emph{Two base cases} are suggested since the sequence is defined by two initial conditions ($f_1=f_2=1$): in the language of the Theorem, $m=1$ and $l=2$. Moreover, the fact that each term from $f_3$ onwards is the sum of its \emph{two predecessors} suggests that the induction step requires only the explicit use of two propositions.
	\begin{quote}
		\begin{proof}
			\begin{description}
				\item[\normalfont\emph{Base cases} ($n=1,2$):] $f_1=1<2^1$ and $f_2=1<2^2$.
				\item[\normalfont\emph{Induction step}:] Fix $n\ge 2$ and suppose\footnotemark{} that $f_{n-1}<2^{n-1}$ and $f_n<2^n$. Then
				\[
					f_{n+1}=f_n+f_{n-1}<2^n+2^{n-1}<2^n+2^n=2^{n+1}
				\]
			\end{description} 
			By induction, $f_n<2^n$ for all $n\in\N$.
		\end{proof}
	\end{quote}
	The Fibonacci numbers satisfy many identities which can often be established by induction (see, for instance, Exercises \ref{exs:fib} \& \ref{exs:fib2}).
\end{example}

\vspace{-5pt}

\footnotetext{%
	To follow Theorem \ref{thm:indstrong} precisely, we should assume that $f_k<2^k$ for \emph{all} $k\le n$. Do so if you like, though our phrasing is more typical. Since we only make explicit use of two cases in the induction step, it is clearer to state these concretely rather than introducing the new variable $k$.%
}

\goodbreak

It is instructive to consider why we really needed strong induction to prove our Fibonacci example. Here are two broken attempts to prove the claim by weak induction.

\begin{proof}[Broken Proof A]
	\begin{description}
		\item[\normalfont\emph{Base case} ($n=1$):] $f_1=1<2^1$.
		\item[\normalfont\emph{Induction step}:] Fix $n\ge 2$ and suppose that $f_n<2^n$. Then\footnotemark
		\[
			f_{n+1}=f_n+f_{n-1}<2^n+\text{????} \tag*{\phantom{\qedhere}}
		\]
	\end{description} 
\end{proof}

\footnotetext{%
	The induction step requires $n\ge 2$: since $f_{n-1}=f_0$ doesn't exist, $f_{n+1}=f_n+f_{n-1}$ is meaningless when $n=1$.%
}


What is the problem? The induction hypothesis assumes $f_n<2^n$, but nothing about $f_{n-1}$: we are stuck! Let's correct this flaw by making the induction hypothesis as in the correct proof.

\begin{proof}[Broken Proof B]
	\begin{description}
		\item[\normalfont\emph{Base case} ($n=1$):] $f_1=1<2^1$.
		\item[\normalfont\emph{Induction step}:] Fix $n\ge 2$ and suppose that $f_{n-1}<2^{n-1}$ and $f_n<2^n$. Then
		\[
			f_{n+1}=f_n+f_{n-1}<2^n+2^{n-1}<2^n+2^n=2^{n+1}
		\]
	\end{description} 
	By induction, $f_n<2^n$ for all $n\in\N$.\phantom{\qedhere}
\end{proof}

Where is the problem now? Consider the first instance, $n=2$, in which the induction step is invoked:
\[
	f_3=\textcolor{red}{f_2}+f_1<\textcolor{red}{2^2}+2^1
\] 
We haven't proved enough base cases to get us started: the single base case establishes $f_1<2^1$, but not $\textcolor{red}{f_2<2^2}$. The induction step correctly establishes the chain of implications
\[
	P(1)\wedge P(2)\Longrightarrow P(3),\quad P(3)\wedge P(4)\Longrightarrow P(5),\quad P(4)\wedge P(5)\Longrightarrow P(6),\ \ldots
\]
but the process only gets started if we prove \emph{both} base cases $P(1)$ \emph{and} $\textcolor{red}{P(2)}$.\bigbreak

The moral here is to try the induction step as scratch work. Your attempt should tell you \emph{if} you need strong induction and \emph{how many} base cases are required.


\begin{example}{}{}
	A sequence of integers $(a_n)_{n=0}^\infty$ is defined by
	\[
		\begin{cases}
			a_{n+1}=5a_n-6a_{n-1}&\text{if }n\ge 1\\
			a_0=0,\ a_1=1
		\end{cases}
	\]
	We prove by induction that $a_n=3^n-2^n$ for all $n\in\N_0$.
	
	\begin{quote}
		\begin{proof}
			\begin{description}
				\item[\normalfont\emph{Base cases} ($n=0,1$):] $a_0=0=3^0-2^0$ and $a_1=1=3^1-2^1$.
				\item[\normalfont\emph{Induction step}:] Fix $n\ge 1$ and suppose that $a_k=3^k-2^k$ for all $k\le n$. Then
				\begin{align*}
					a_{n+1}
					&=5a_n-6a_{n-1}
						=5(3^n-2^n)-6(3^{n-1}-2^{n-1})\\
					&=(15-6)3^{n-1}+(10-6)2^{n-1}
						=3^{n+1}-2^{n+1}
				\end{align*}
			\end{description} 
			By induction, $a_n=3^n-2^n$ for all $n\in\N_0$.
		\end{proof}
	\end{quote}
\end{example}

\goodbreak


For another sequential induction example in the same vein, see Exercise \ref{ex:ind3} where \emph{three} base cases are required and the induction step explicitly uses \emph{three} propositions.\bigbreak

To see strong induction in all its glory, with the induction step making use of \emph{all} previous propositions, we prove the existence part of the Fundamental Theorem of Arithmetic, which states that all integers $\ge 2$ can be (uniquely) expressed as a product of primes: e.g.,\ $3564=2^2\times 3^4\times 11$. 


\begin{thm}{}{fundarith}
	Every integer $n\ge 2$ is either prime or a product of primes.
\end{thm}

This provides the missing piece in our discussion of Euclid's Theorem (\ref{thm:euclidprime}) on the existence of infinitely many primes. First recall Definition \ref{defn:irreducible}: $p\in\N_{\ge 2}$ is \emph{prime} if and only if its only positive divisors are itself and 1. A non-prime $q\in\N_{\ge 2}$ is said to be \emph{composite}: $\exists a,b\in\N_{\ge 2}$ such that $q=ab$.


\begin{proof}
	We prove by induction.
	\begin{description}
		\item[\normalfont\emph{Base case} ($n=2$):] The only positive divisors of 2 are itself and 1, hence 2 is prime.
		\item[\normalfont\emph{Induction step}:] Fix $n\ge 2$ and suppose that \emph{every} natural number $k$ satisfying $2\le k\le n$ is either prime or a product of primes. There are two possibilities:
		\begin{itemize}
		  \item $n+1$ is prime. Certainly $n+1$ is divisible by a prime: itself!
		  \item $n+1$ is composite. Then $n+1=ab$ for some natural numbers $a,b\ge 2$. Plainly $a,b\le n$. By the induction hypothesis, \emph{both} $a,b$ are prime or products of primes. Therefore $n+1$ is also the product of primes.
		\end{itemize}
	\end{description}
	By induction we see that all natural numbers $n\ge 2$ are either prime, or a product of primes.
\end{proof}

Think carefully about why \emph{only one} base case is required!


\begin{exercises}{}{}
	A reading quiz and several questions with linked video solutions can be found \href{http://www.math.uci.edu/~ndonalds/math13/selftest/5-3-strongind.html}{online}.

	\begin{enumerate}  
  	\item Define sequence $(b_n)_{n=1}^\infty$ as follows:
  	\[
  		\begin{cases}
				b_{n+1}=2b_n+b_{n-1} &\text{if } n\ge 2\\
				b_1=3, \ b_2=6
			\end{cases}
		\]
		Prove by induction that $b_n$ is divisible by 3 for all $n\in\N$.
	
	
		\item\label{ex:ind3strong} Define a sequence $(c_n)_{n=0}^\infty$:
	 	\[
	 		\begin{cases}
				c_{n+1}=\frac{49}8c_n-\frac{225}8c_{n-2}&\text{if }n\ge 2\\
				c_0=0, \ c_1=2, \ c_2=16
			\end{cases}
		\]
		Prove by induction (use three base cases!) that $c_n=5^n-3^n$ for all $n\in\N_0$.
		
		
		\item\label{exs:fib} Let $f_n$ be the $n\th$ Fibonacci number (Example \ref{ex:fibonacci}). Prove the following by induction $\forall n\in\N$:
		\begin{enumerate}
		  \item $\sum\limits_{k=1}^nf_k^2=f_nf_{n+1}$ \qquad\qquad (b)\lstsp $f_n\ge\left(\frac 32\right)^{n-2}$
		\end{enumerate}
		(\emph{Hints: Weak induction is good enough for (a); why?})
	
	
		\goodbreak
		

	  \item\label{exs:fib2} Extending Exercise \ref{exs:fib}(b), prove \emph{Binet's formula} for the $n\th$ Fibonacci number:
	  \[
	  	f_n=\frac 1{\sqrt 5}\bigl(\phi^n-\hat\phi^n\bigr) \quad \text{where}\quad \phi=\frac 12(1+\sqrt{5})\text{ and } \hat\phi=\frac 12(1-\sqrt{5})
	  \]
		(\emph{$\phi$ is the famous \emph{golden ratio}: $\phi,\hat\phi$ are the solutions to the quadratic equation $x^2=x+1$})
  		
		
		\item Prove by induction that every $n\in\N$ can be written in the form
	  \[
	    n=2^{r_1}+2^{r_2}+\cdots+2^{r_\ell}
	  \]
	  for some $\ell\in\N$ and \emph{distinct} integers $r_1,r_2,\ldots r_\ell \ge 0$.\par
	  (\emph{Hints: try proving in the style of Theorem \ref{thm:fundarith}; consider the cases when $n+1$ is even/odd separately})
	
	
		\item\label{exs:strongindproof} Prove the principle of strong induction (Theorem \ref{thm:indstrong}) by applying \emph{weak induction} to a new family of propositions $Q(n)$ via:
		\[
			Q(n)\iff P(m)\wedge P(m+1)\wedge\cdots\wedge P(n)
		\]
	
	
		\item Consider the proof of Theorem \ref{thm:fundarith}.
		\begin{enumerate}
	  	\item If the Theorem is written in the form $\forall n\in\N_{\ge 2}, P(n)$, what is the proposition $P(n)$?
	  	\item Explicitly carry out the induction step for the three situations $n+1=9$, $n+1=106$ and $n+1=45$. How many different ways can you perform the calculation for $n+1=45$? 
	  	\item Explain why it is only necessary in the induction step to assume that all integers $k$ satisfying $2\le k\le\frac{n+1}2$ are prime or products of primes.
		\end{enumerate}

		\item\label{exs:primedef2} In this question we use the alternative definition of prime (Exercise \ref*{sec:gcd}.\ref{exs:primedef1}).\footnotemark
		\begin{quote}
			An integer $p\ge 2$ is \emph{prime} if and only if $\forall a,b\in\N,\ p\mid ab\implies p\mid a$ or $p\mid b$.
		\end{quote}
		Let $p$ be prime, let $n\in\N$, and let $\lst an$ be natural numbers such that $p\mid a_1a_2\cdots a_n$. Prove by induction that,
		\[
			p\mid a_i\ \text{ for some }\ i\in\{1,2,\ldots,n\}
		\]
	  (\emph{Hint: $n=1$ isn't really part of the induction, but you can treat it as a base case})
	  
	    
	  \item The \emph{Fundamental Theorem of Arithmetic} states that every integer $n\ge 2$ can be written as a product of prime factors in a \emph{unique} way (up to reordering of the prime factors). In other words, 
		\begin{itemize}
	    \item[i.] $n=p_1p_2\cdots p_k$ for some primes $p_1,p_2,\ldots,p_k$, \ and,
	    \item[ii.] If $n=q_1q_2\cdots q_\ell$ for primes $q_1,q_2,\ldots,q_\ell$, then $k=\ell$ and $p_i=q_i$ after possibly reordering the prime factors. 
		\end{itemize}
	  Part i.{} is Theorem \ref{thm:fundarith}. Using Exercise \ref{exs:primedef2}, or otherwise, supply a proof of part ii.
		
	\end{enumerate}
\end{exercises}

\footnotetext{%
	Strictly, this is definition of \emph{prime,} whereas Definition \ref{defn:irreducible} defines a subtly different concept: \emph{irreducibility.} Within the integers, Exercise \ref*{sec:gcd}.\ref{exs:primedef1} says that these concepts are synonymous.
}

