\section{Logic and the Language of Proofs}\label{chap:logic}

\subsection{Propositions}\label{sec:prop}

To read and construct proofs, we need to develop the language of \emph{logic.} This is to mathematics what grammar is to English.


\begin{defn}{}{}
	A \emph{proposition} or \emph{statement} is a sentence that is either true or false.
\end{defn}

\begin{examples}{}{}
	\exstart $17-24=7$. \hfill \makebox[280pt][l]{2. \ $39^2$ is an odd integer.\hfill\hfill 3. \ God exists.\hfill\vspace{-4pt}}
	\begin{enumerate}\setcounter{enumi}{3}\itemsep0pt
		\item The moon is made of cheese. \hfill \makebox[280pt][l]{5. \ Every cloud has a silver lining.\hfill}
	\end{enumerate}
\end{examples}

For a proposition to make sense, readers must agree on the meaning of each concept it references. In the real world, arguments about propositions are often disagreements over \emph{definitions.} For instance, the question of whether God exists is meaningless unless we agree on which conception (Shiva, Yahweh, Allah, Zeus, all/any of them?) is being discussed! This also illustrates that the truth status of a proposition \emph{need not be known} when stated, a particularly common situation in mathematics.



\boldsubsubsection{Logical Expressions: Truth Tables and Combining Propositions}

To develop basic terminology, we represent abstract propositions by letters $P,Q,R,\ldots$ (similarly to the use of  $x,y,z$ in algebra). Combinations of such (logical expressions) are easily described in tabular form.

\begin{defn}{}{andornot}
	Let $P$ and $Q$ be propositions. The following \emph{truth tables} define three new propositions:\par
	\begin{minipage}[t]{0.5\linewidth}\vspace{0pt}
		\begin{itemize}
		  \item The \emph{conjunction} $\textcolor{red}{P\wedge Q}$ is read ``$P$ and $Q$.''
		  \item The \emph{disjunction} $\textcolor{Green}{P\vee Q}$ is read ``$P$ or $Q$.''
		  \item The \emph{negation} $\textcolor{blue}{\neg P}$ is read ``not $P$.''
		\end{itemize} 
	\end{minipage}
	\hfill
	\begin{minipage}[t]{0.49\linewidth}\vspace{-6pt}
		$\begin{array}{cc||c|c}
			P & Q & \textcolor{red}{P\wedge Q} & \textcolor{Green}{P\vee Q}\\\hline
			T & T & \textcolor{red}{T} & \textcolor{Green}{T}\\
			T & F & \textcolor{red}{F} & \textcolor{Green}{T}\\
			F & T & \textcolor{red}{F} & \textcolor{Green}{T}\\
			F & F & \textcolor{red}{F} & \textcolor{Green}{F}
		\end{array}
		\qquad\qquad
		\begin{array}[b]{c||c}
			P & \textcolor{blue}{\neg P}\\\hline
			T & \textcolor{blue}{F}\\
			F & \textcolor{blue}{T}
		\end{array}$
	\end{minipage}\medbreak
	A \emph{tautology} is a logical expression that is always true (truth table has a column of $T$'s), regardless of its component propositions. A \emph{contradiction} is a logical expression that is always false.
\end{defn}

The letters T/F stand for \emph{true/false.} For instance, the second line of the first table says that if $P$ is true and $Q$ is false, then the proposition ``$P$ and $Q$" is \textcolor{red}{false}; similarly ``$P$ or `$Q$'' is \textcolor{Green}{true.}

\begin{examples}{}{logiccolor}
	\exstart By choosing explicit propositions, we may compare the logical and/or/not with their plain English meanings. Suppose $P$ and $Q$ are the propositions
	\begin{enumerate}\setcounter{enumi}{1}
	  \item[]\begin{quote}
		$P$: ``I like purple.''\qquad\qquad $Q$: ``I like chartreuse.''
		\end{quote}
		The new propositions from Definition \ref{defn:andornot} might then be written
		\begin{quote}
			$P\wedge Q$: ``I like purple and chartreuse.''\qquad \qquad
			$P\vee Q$: ``I like purple or chartreuse.''\medbreak
			$\neg P$: ``I do not like purple.''
		\end{quote}
		Alternative phrasings might aid readability, though be careful: ``Not, I like purple,'' is terrible English! Note also that the logical or is \emph{inclusive} (first line of the truth table): with a logical or, ``I like purple or chartreuse,'' means you might like \emph{both.}
	
		\goodbreak
		
		\item We continue the previous example by adding a third proposition $R$: ``It is 9am.''
		What logical expression might be represented by the following sentence?
		\begin{quote}
			``I like purple and I like chartreuse or it is 9am.''
		\end{quote}
		Is it $P\wedge(Q\vee R)$ or is it $(P\wedge Q)\vee R$? Without brackets the sentence is unclear. In fact, as the next truth table shows, these logical expressions \textcolor{red}{mean different things}.
		\[
			\begin{array}{ccc||cc||cc}
				P & Q & R & Q\vee R & P\wedge (Q\vee R) & P\wedge Q & (P \wedge Q)\vee R\\\hline
				T & T & T & T & T & T & T\\
				T & T & F & T & T & T & T\\
				T & F & T & T & T & F & T\\
				T & F & F & F & F & F & F\\
				F & T & T & T & \textcolor{red}{F} & F & \textcolor{red}{T}\\
				F & T & F & T & F & F & F\\
				F & F & T & T & \textcolor{red}{F} & F & \textcolor{red}{T}\\
				F & F & F & F & F & F & F
			\end{array}
		\]
		The moral here is that English is terrible for logic! Clear identification of propositions is essential if you want to avoid ambiguous sentences such as the above.
		
	
		\begin{minipage}[t]{0.64\linewidth}\vspace{0pt}
  		\item Let $S$ be an abstract proposition. Then $S\vee (\neg S)$ is a tautology. Otherwise said, no matter the meaning of $S$, either it \emph{or} its negation must be true.
  	\end{minipage}
  	\hfill
  	\begin{minipage}[t]{0.33\linewidth}\vspace{0pt}
			$\begin{array}{cc||c|c}
				S & \neg S & S\vee (\neg S) & S\wedge(\neg S)\\\hline
				T & F & T & F\\
				F & T & T & F
			\end{array}$
  	\end{minipage}\par
  	Similarly, $S\wedge(\neg S)$ is a contradiction: regardless of $S$, it \emph{and} its negation cannot both be true.
	\end{enumerate}
\end{examples}


\boldsubsubsection{Conditional and Biconditional Connectives}

Having one proposition lead to another is of critical importance to mathematics.

\begin{defn}[lower separated=false, sidebyside, sidebyside align=top seam, sidebyside gap=0pt, righthand width=0.37\linewidth]{}{implies}
	Given propositions $P,Q$, the \emph{conditional} ($\Rightarrow$) and \emph{biconditional} ($\Leftrightarrow$) \emph{connectives} define new propositions as described in the truth table.\smallbreak
	For the proposition $P\Longrightarrow Q$, we call $P$ the \emph{hypothesis} and $Q$ the \emph{conclusion.}
	\tcblower
	\flushright%
	$\begin{array}{cc||c|c}
		P & Q & P\Longrightarrow Q & P\Longleftrightarrow Q\\\hline
		T & T & T & T\\
		T & F & F & F\\
		F & T & T & F\\
		F & F & T & T
	\end{array}$
\end{defn}


Connective propositions can be read and written in many different ways: for instance,
\begin{quote}
	\def\arraystretch{1.05}
	\begin{tabular}{@{}cc|c}
		\multicolumn{2}{c|}{$P\Longrightarrow Q$} & $P\Longleftrightarrow Q$\\\hline
		$P$ implies $Q$ & $P$ therefore $Q$ & $P$ if and only if $Q$\\
		If $P$, then $Q$ & $Q$ follows from $P$ & $P$ iff $Q$\\
		$P$ only if $Q$ & $Q$ if $P$ & $P$ and $Q$ are (logically) equivalent\\
		$P$ is sufficient for $Q$ & $Q$ is necessary for $P$ & $P$ is necessary and sufficient for $Q$
	\end{tabular}
\end{quote}
\emph{Logical equivalence} is often used to describe logical expressions whose \emph{truth tables are identical.}


\goodbreak


\begin{examples}{}{conditionalbasic}
	\exstart Here are six English sentences expressing the same conditional $P\Longrightarrow Q$:\vspace{-1pt}
	\begin{enumerate}\setcounter{enumi}{1}
	\item[]\begin{itemize}\itemsep1pt
			\item If you are born in Rome, then you are Italian. 
			\item You are Italian if you are born in Rome. 
			\item You are born in Rome only if you are Italian. 
			\item Being born in Rome is sufficient for being Italian. 
			\item Being Italian is necessary for being born in Rome. 
		\end{itemize}
		Are you comfortable with what the propositions $P$ and $Q$ are in this situation?
	
		\item\label{ex:taut2} $\bigl(P\wedge(P\implies Q)\bigr)\Longrightarrow Q$ is a tautology.
		\[
			\begin{array}{cc||c|c||c}
				P & Q & P\implies Q & P\wedge(P\implies Q) & \bigl(P\wedge(P\implies Q)\bigr)\implies Q\\\hline
				T & T & T & T& T\\
				T & F & F & F& T\\
				F & T & T & F& T\\
				F & F & T & F& T
			\end{array}
		\]
		%\item $(P\wedge \neg Q\implies F)\iff (P\implies Q)$ is a tautology. This tautology is the basis for \emph{proof by contradiction,} as we'll see in the next section. The expression $P\wedge \neg Q\implies F$ can be thought of as saying that $P\wedge\neg Q$ implies a contradiction.
	\end{enumerate}
\end{examples}



Connectives are central to mathematics for many reasons. In particular:
\begin{enumerate}
  \item The vast majority of simple theorems may be written in the form $P\Longrightarrow Q$. For instance, revisit Theorem \ref{thm:sumeven} and the discussion that follows:
  \[
  	\text{If \textcolor{red}{$x$ and $y$ are even integers}, then \textcolor{blue}{$x+y$ is even}.} \tag{If \textcolor{red}{$P$}, then \textcolor{blue}{$Q$}}
  \]
  Identifying the hypothesis and conclusion is essential if you want to understand a theorem!
  \item The simplest (``direct'') proofs typically involve chaining a sequence of connectives:
  \[
  	\textcolor{red}{P}\implies P_2\implies \cdots \implies P_n\implies \textcolor{blue}{Q}
  \]
\end{enumerate}
We'll revisit these ideas in Section \ref{sec:proof}, and repeatedly throughout the course.
\bigbreak


While the biconditional should be easy to remember, it is harder to make sense of the conditional connective. Short of simply memorizing the truth table, here are two examples that might help.

\begin{examples}{}{condmeaning}
	\exstart Suppose your professor says, ``If the class earns a B average on the midterm, then I'll bring doughnuts.'' The only situation in which the teacher will have lied is if the class earns a B average but she fails to provide doughnuts.
	\begin{enumerate}\setcounter{enumi}{1}
	  \item ($F\Longrightarrow T$ really can be true!) Let $P$ be the proposition ``$7=3$'' and $Q$ be ``$0=0$.'' Since multiplication of both sides of an equation by zero is algebraically valid, we see that
	  \begin{align*}
			7=3\implies\ &0\cdot 7=0\cdot 3\tag*{(If $7=3$, then 0 times 7 equals 0 times 3)}\\
			\implies\ &0=0\tag*{(then 0 equals 0)}
		\end{align*}
	  This argument is perfectly correct: the \emph{implication} $P\Longrightarrow Q$ is \emph{true.} It (rightly!) makes us uncomfortable because the hypothesis is \emph{false.}\par
	  If we instead add 1 to each side of $7=3$, we'd obtain a example where $F\Longrightarrow F$ is true.
	\end{enumerate}
\end{examples}


\goodbreak


\boldsubsubsection{The Converse and Contrapositive}

\begin{defn}{}{contra}
	The \emph{converse} of $P\Longrightarrow Q$ is the reversed conditional $Q\Longrightarrow P$.\smallbreak
	The \emph{contrapositive} of $P\Longrightarrow Q$ is the conditional $\neg Q\Longrightarrow\neg P$.
\end{defn}

\begin{example}{}{}
	Let $P$ and $Q$ be the statements
	\begin{quote}
	  $P$: \ ``Claudia has a peach.''\qquad\qquad
	  $Q$: \ ``Claudia has a fruit.''
	\end{quote}
	Plainly, every peach is a fruit, so $P\Longrightarrow Q$ is \emph{true}: ``If Claudia has a peach, then she has a fruit.''
	\par
	The \emph{converse} of $P\Longrightarrow Q$ is the sentence
	\begin{quote}
	  $Q\Longrightarrow P$: \ ``If Claudia has a fruit, then she has a peach.''
	\end{quote}
	This is palpably \emph{false}: Claudia might have an apple! The \emph{contrapositive}, however, is \emph{true}:
	\begin{quote}
	  $\neg Q\Longrightarrow \neg P$: ``If Claudia does \emph{not} have a fruit, then she does \emph{not} have a peach.''
	\end{quote} 
\end{example}

Understanding the distinction between the converse and contrapositive is vital. In general, the truth status of the converse bears no relation to that of the original: the converse of a true proposition could be either true or false. By re-reading the example, however, you should feel that the \emph{contrapositive} (in English) is just another way to state the original proposition; our next result makes this formal.

\begin{thm}{}{contrapos}
	The contrapositive of an conditional is logically equivalent to the original.
\end{thm}

\begin{proof}
	Compute the truth table and observe that its \textcolor{red}{third} and \textcolor{blue}{sixth} columns are identical:
	\begin{gather*}
		\begin{array}{cc|c||cc|c}
			P & Q & \textcolor{red}{P\Longrightarrow Q} & \neg Q & \neg P & \textcolor{blue}{\neg Q\Longrightarrow\neg P}\\\hline
			T & T & \textcolor{red}{T} & F & F & \textcolor{blue}{T}\\
			T & F & \textcolor{red}{F} & T & F & \textcolor{blue}{F}\\
			F & T & \textcolor{red}{T} & F & T & \textcolor{blue}{T}\\
			F & F & \textcolor{red}{T} & T & T & \textcolor{blue}{T}
		\end{array}
	\end{gather*}
	Otherwise said, $(\textcolor{red}{P\Longrightarrow Q})\Longleftrightarrow (\textcolor{blue}{\neg Q\Longrightarrow\neg P})$ is a tautology.
\end{proof}


 
\boldsubsubsection{Negating Logical Expressions}

Mathematics often requires us to negate propositions. What would you suspect to be the negation of a conditional $P\Longrightarrow Q$? Can we simply say ``$P$ doesn't imply $Q$"? But what does this mean? 

\begin{minipage}[t]{0.64\linewidth}\vspace{-2pt}
	We again rely on a truth table: to get the last column, recall that negation simply swaps $T$ and $F$. Can we write this column in another way? The single $T$ in the final column provides a proof of an important result:
\end{minipage}
\hfill
\begin{minipage}[t]{0.35\linewidth}\vspace{-5pt}
	\flushright%
	$\begin{array}{cc|c|c}
		P & Q & P\Longrightarrow Q & \neg(P\Longrightarrow Q)\\\hline
		T & T & T & F\\
		T & F & F & T\\
		F & T & T & F\\
		F & F & T & F
	\end{array}$
\end{minipage}

\begin{thm}{}{negconditional}
	$\neg(P\Longrightarrow Q)$ is logically equivalent to $P\wedge\neg Q$ \ (``$P$ and not $Q$").
\end{thm}

\vspace{-5pt}


\goodbreak


\begin{example}{}{}
	Consider the implication
	\begin{quote}
	  It's morning so I'll have coffee.
	\end{quote}
	Hopefully its negation is clear:
	\begin{quote}
	  It's morning \emph{and} I \emph{won't} have coffee.
	\end{quote}
	As in Example \ref{ex:condmeaning}, it might help to think about what it means for the original statement to be \emph{false}.
\end{example}

\begin{tcolorbox}
	{\bf \textcolor{red}{Warning!}} \ The negation of $P\Longrightarrow Q$ is \emph{not a conditional.} In particular it is \emph{neither}:
	\begin{itemize}\itemsep2pt
	  \item The converse $Q\Longrightarrow P$
	  \item The contrapositive of the converse $\neg P\Longrightarrow\neg Q$
	\end{itemize}
	If you are unsure about this, compare the truth tables!
\end{tcolorbox}

\bigbreak


Our final results in basic logic also involve negations; they are named for Augustus de Morgan, a British logician of the 19\th{} century.

\begin{thm}{de Morgan's laws}{demorgan}
	Let $P$ and $Q$ be propositions.
	\begin{enumerate}\itemsep2pt
	  \item $\neg(P\wedge Q)$ is logically equivalent to $\neg P\vee\neg Q$
	  \item $\neg(P\vee Q)$ is logically equivalent to $\neg P\wedge\neg Q$
	\end{enumerate}
\end{thm}

\begin{proof}
	For the first law, compute the truth table: the \textcolor{red}{fourth} and \textcolor{blue}{seventh} columns are identical.
	\[
		\begin{array}[t]{cc||cc||cc||c}
			P & Q & P\wedge Q & \textcolor{red}{\neg(P\wedge Q)} & \neg P & \neg Q & \textcolor{blue}{\neg P\vee\neg Q}\\\hline
			T & T & T & \textcolor{red}{F} & F & F & \textcolor{blue}{F}\\
			T & F & F & \textcolor{red}{T} & F & T & \textcolor{blue}{T}\\
			F & T & F & \textcolor{red}{T} & T & F & \textcolor{blue}{T}\\
			F & F & F & \textcolor{red}{T} & T & T & \textcolor{blue}{T}
		\end{array}
	\]
	The second law is an exercise.
\end{proof}

\begin{example}{}{}
	Consider the sentence\par
	\begin{minipage}[t]{0.57\linewidth}\vspace{-1pt}
		\begin{quote}
			I rode the subway \emph{and} I had coffee.
		\end{quote}
		To negate this, we might write
		\begin{quote}
			I \emph{didn't} ride the subway \emph{or} I \emph{didn't} have coffee.
		\end{quote}
	\end{minipage}
	\hfill
	\begin{minipage}[t]{0.42\linewidth}\vspace{-20pt}
		\flushright	
		\begin{tabular}{c|c||c@{}}
			Subway&Coffee&Subway and Coffee\\\hline\hline
			T & T & T\\
			\textcolor{blue}{T} & \textcolor{blue}{F} & \textcolor{blue}{F}\\
			\textcolor{blue}{F} & \textcolor{blue}{T} & \textcolor{blue}{F}\\
			\textcolor{blue}{F} & \textcolor{blue}{F} &\textcolor{blue}{F}
		\end{tabular}
	\end{minipage}\bigbreak
	
	This reads awkwardly because the negation encompasses \textcolor{blue}{three distinct possibilities}. Note how the logical (inclusive) use of \emph{or} includes the last row of the truth table: the possibility that one neither rode the subway nor had coffee. As with Example \ref{ex:logiccolor}, this is another advert for the use of logic over English.
\end{example}


\goodbreak


\begin{aside}{}{}
	\boldinline{Aside: Algebraic Logic}\phantomsection\label{pg:asidelogicalgebra}
	
	We can use truth tables to establish other laws of basic logic, e.g.:
	\[
		\def\arraystretch{1.2}
		\begin{array}{@{}lll@{}}
			\text{Double negation} & \neg(\neg P)\Longleftrightarrow P &\\
			\text{Commutativity} & P\wedge Q\Longleftrightarrow Q\wedge P & P\vee Q\Longleftrightarrow Q\vee P\\
			\text{Associativity} & (P\wedge Q)\wedge R\Longleftrightarrow P\wedge(Q\wedge R) & (P\vee Q)\vee R\Longleftrightarrow P\vee(Q\vee R)\\
			\text{Distributivity}&(P\wedge Q)\vee R\Longleftrightarrow (P\vee R)\wedge (Q\vee R) & (P\vee Q)\wedge R\Longleftrightarrow (P\wedge R)\vee (Q\wedge R)
		\end{array}
	\]
	To make things more algebraic, we've replaced ``is logically equivalent to" with a biconditional.\footnotemark{}\smallbreak
	
	Armed with these laws, one can often manipulate logical expressions without the laborious creation You are welcome to try memorizing these laws, though there is typically no need: De Morgan's laws, together with your intuitive understanding of \emph{and, or} and \emph{not} mean you'll likely perform correct manipulations regardless.
\end{aside}


\footnotetext{Stating the laws in this fashion is to assert that each expression is a tautology (Definition \ref{defn:andornot}). For instance, to claim that ``$\neg(\neg P)$ is logically equivalent to $P$'' is to assert that $\neg(\neg P)\Longleftrightarrow P$ is a tautology.}



\begin{exercises}{}{}
	A reading quiz and several questions with linked video solutions can be found \href{http://www.math.uci.edu/~ndonalds/math13/selftest/2-1-props.html}{online}.


	\begin{enumerate}
	  \item Express each statement in the form, ``If $\dots$, then $\dots$'' There are many possible correct answers.
		\begin{enumerate}
		  \item You must eat your dinner if you want to grow.
		  
		  \item Being a multiple of 12 is a sufficient condition for a number to be even.
		  
		  \item It is necessary for you to pass your exams in order for you to obtain a degree. 
		  
		  \item A triangle is equilateral only if all its sides have the same length.
		\end{enumerate}
		
	
	  \item Suppose ``$x$ is an even integer'' and ``$y$ is an irrational number'' are true statements, and that ``$z\geq 3$'' is a false statement. Which of the following are true?\par
	  (\emph{Hint: Label each statement and think about each using connectives})
		\begin{enumerate}
		  \item If $x$ is an even integer, then $z\geq 3$.
		  
		  \item If $z\geq 3$, then $y$ is an irrational number.
		  
		  \item If $z\geq 3$ or $x$ is an even integer, then $y$ is an irrational number.
		  
		  \item If $y$ is an irrational number and $x$ is an even integer, then $z\geq 3$.
		\end{enumerate}


	  \item Orange County is considering two competing transport plans: widening the 405 freeway and constructing light rail down its median. A local politician is asked, ``Would you like to see the 405 widened or would you like to see light rail?'' The politician wants to sound positive, but to avoid being tied to one project. What is their response?\par
	  (\emph{Hint: Think about how the word `OR' is used in logic})
  
  
	  \item Consider the proposition: ``If the integer $m$ is greater than 3, then $2m$ is not prime.''
	  \begin{enumerate}
	    \item Rewrite the proposition using the word `necessary.'
	    
	 		\item Rewrite the proposition using the word  `sufficient.'
	 		
	 		\item Write the negation, converse and contrapositive of the proposition. 
	  \end{enumerate}
	  

  	\item Suppose the following sentence is true: ``If Amy likes art, then no-one likes history." What, if anything, can we conclude if we discover that someone likes history.
  
  
  	\goodbreak
	  
	  
	  \item Construct the truth tables for the propositions $P\vee(Q\wedge R)$ and $(P\vee Q)\wedge R$. Are they the same?
   
   
  	\goodbreak
  
  
		\item Use truth tables to establish the following laws of logic:
		\begin{enumerate}
		  \item Double negation: \lstsp $\neg(\neg P)\Longleftrightarrow P$
		  
		  \item Idempotent law: \lstsp $P\wedge P\Longleftrightarrow P$
		  
		  \item Absorption law: \lstsp $P\wedge(P\vee Q)\Longleftrightarrow P$
		  
		  \item Distributive law: \lstsp $(P\wedge Q)\vee R\Longleftrightarrow (P\vee R)\wedge(Q\vee R)$
		\end{enumerate}
  
  
	  \item\begin{enumerate}
	    \item Decide whether $(P\wedge \neg P) \Longrightarrow Q$ is a tautology, a contradiction, or neither.
	    
	    \item Explain why $\neg P \vee \neg Q$ is logically equivalent to $P \Longrightarrow (P \wedge\neg Q)$.
	    
	    \item\label{exs:contradictproof} Prove: $\bigl((P\wedge \neg Q)\Longrightarrow F\bigr)\Longleftrightarrow  (P\Longrightarrow Q)$ is a tautology. Here $F$ represents a \emph{contradiction.}
	  \end{enumerate}
	  
	  
	  \item\begin{enumerate}
	    \item\label{exs:iff} Prove that the expressions $(P\Longrightarrow Q)\wedge (Q\Longrightarrow P)$ and $P\Longleftrightarrow Q$ are logically equivalent.
	    
	    \item\label{exs:iftransitive} Prove that $\bigl((P\Longrightarrow Q)\wedge (Q\Longrightarrow R)\bigr)\Longrightarrow \bigl(P\Longrightarrow R\bigr)$ is a tautology.
	  \end{enumerate}
	  Why do these make intuitive sense?

  
  	\item Use logical algebra (page \pageref{pg:asidelogicalgebra}) to show that $\bigl((P\vee Q)\wedge \neg P\bigr)\wedge\neg Q$ is a contradiction.

	
  	\item\label{exs:ttconverse} Compare the truth tables for $P\Longrightarrow Q$ and its converse. Do there exist propositions $P,Q$ for which both $P\Longrightarrow Q$ and its converse are \emph{false}? Explain.
  
    
		\item A friend insists that the negation of ``Mark and Mary have the same height,'' is ``Mark or Mary do not have the same height.'' What is the correct negation? Where did your friend go wrong?

		
		\item Suppose that the following statements are \emph{true}:
	  \begin{enumerate}
	    \item Every octagon is magical. 
	    \item If a polygon is not a rectangle, then is it not a square. 
	    \item A polygon is a square if it is magical.
	  \end{enumerate}
	  Is it true that ``Octagons are rectangles"? Explain your answer.\par
	  (\emph{Hint: try rewriting each of the statements as an implication})
  

		\item The connective $\downarrow$ (the \emph{Quine dagger}, \emph{NOR}) is defined by a truth table.\par
		\begin{enumerate}
		\begin{minipage}[t]{0.75\linewidth}\vspace{0pt}
	     \item Prove that $P\downarrow Q$ is logically equivalent to $\neg (P\vee Q)$. 
	     
	     \item Find a logical expression built using only $P$ and the connective $\downarrow$ which is logically equivalent to $\neg P$.
		\end{minipage}
		\hfill
		\begin{minipage}[t]{0.24\linewidth}\vspace{-25pt}
		\flushright%
		$\begin{array}{cc|c}
			P & Q & P \downarrow Q\\\hline
			T & T & F\\
			T & F & F\\
			F & T & F\\
			F & F & T
		\end{array}$
		\end{minipage}
		
			\item Find an expression built using only $P$, $Q$ and $\downarrow$ which is logically equivalent to $P\wedge Q$.
	  \end{enumerate}
  
  
	  \item (Just for fun)\lstsp Augustus de Morgan satisfied his own problem:
		\begin{quote}
			I turn(ed) $x$ years of age in the year $x^2$.
		\end{quote}
		\begin{enumerate}
		  \item Given that de Morgan died in 1871, and that he wasn't the beneficiary of some miraculous anti-aging treatment, find the year in which he was born.
		  
		  \item Suppose you have an acquaintance who satisfies the same problem. When were they born and how old will they turn this year?
		\end{enumerate}
		Do your best to give a formal proof of your correctness.
  
	\end{enumerate}

\end{exercises}

\clearpage




\subsection{Propositional Functions \& Quantifiers}\label{sec:quant}

Mathematical propositions are typically more complicated that those seen in Section \ref{sec:prop}. In particular, they often involve \emph{variables}: for instance, ``$x$ is an integer greater than 5.''

\begin{defn}{}{}
	A \emph{propositional function} is a family of propositions depending on one or more variables. The collection of permitted variables is the \emph{domain.}
\end{defn}

If $P$ is a propositional function depending on a single variable $x$, then for each object $a$ in the domain $P(a)$ is a proposition. Typically $P(x)$ is true for some $x$ and false for others.

\begin{example}{}{easyquantprop}
	Suppose $P(x)$ is the propositional function ``$x^2>4$" with domain the real numbers. Plainly $P(1)$ is false (``$1^2>4$'' is nonsense), while $P(6)$ is true (''$6^2>4$'').
\end{example}


Propositional functions are often \emph{quantified.} English contains various quantifiers (\emph{all, some, many, few, several,} etc.), but in mathematics we are primarily concerned with just two.


\begin{defn}{}{quant}
	The \emph{universal quantifier} $\forall$ is read ``for all.'' The \emph{existential quantifier} $\exists$ is read ``there exists.'' Given a propositional function $P(x)$, we may define two new \emph{quantified propositions}:
	\begin{itemize}
	  \item ``$\forall x, P(x)$'' is true if and only if $P(x)$ is true for \emph{every} $x$ in its domain.
	  \item ``$\exists x, P(x)$'' is true if and only if $P(x)$ is true for \emph{at least one} $x$ in its domain.
	\end{itemize}
	When quantifying propositions it is common to describe the domain by including a descriptor after the quantifier (\emph{bounding the quantifier}---see the example below).
\end{defn}

As with connectives, there are multiple ways to express quantified propositions both mathematically and in English. Symbolic quantifiers involve a trade-off so consider your audience: compact statements can improve clarity but are harder to read for the uninitiated. %Whether to employ them is your choice, however being able to \emph{read/recognize} them is essential.

\begin{example*}{\ref{ex:easyquantprop} cont.}{}
	To gain some practice with bounded quantifiers, we introduce the notation $x\in\R$ which simply means that $x$ is a real number.\footnotemark{}
	\begin{itemize}
	  \item ``$\forall x\in\R,\ x^2>4$'' might be read, ``The square of every real number is greater than 4.''\par
	  The quantified proposition is \emph{false} since $1^2>4$ is false: we call $x=1$ a \emph{counter-example.}
	  \item ``$\exists x\in\R,\ x^2>4$'' might be read, ``There is a real number whose square is greater than 4.''\par
	  The quantified proposition is \emph{true} since $6^2>4$ (is true): we call $x=6$ an \emph{example.}
	\end{itemize}
\end{example*}

\footnotetext{This notation should be familiar. Don't worry if not, for it will be properly discussed in chapter \ref{chap:sets}.}

Due to their importance, it is worth defining these last concepts formally.

\begin{defn}{}{example-cex}
	An \emph{example} of ``$\exists x, P(x)$'' is an element $x_0$ in the domain of $P$ for which $P(x_0)$ is \emph{true.} To \emph{prove} an existential statement is to provide an \emph{example.}\smallbreak
	A \emph{counter-example} to ``$\forall x, P(x)$'' is an element $x_0$ in the domain of $P$ for which $P(x_0)$ is \emph{false.} To \emph{disprove} a universal statement is to provide an \emph{counter-example.}
\end{defn}


\goodbreak


% \boldinline{Must I always use all these symbols?}
% 
% Absolutely not! Though you do have to be able to \emph{read} and \emph{understand} them. Remember that the purpose of writing mathematics is to \emph{convince the reader}; your chosen presentation style will have a huge impact on whether you succeed! Here are three presentations based on the previous example:
% \begin{quote}\def\arraystretch{1.2}
% 	\begin{tabular}{l|l}
% 		Pure English & There is a real number whose square is greater than four\\\hline
% 		Pure Logic & $\exists x\in\R, x^2>4$\\\hline
% 		Hybrid & $\exists x\in\R$ such that $x^2>4$
% 	\end{tabular}
% \end{quote}
% There are benefits and drawbacks to all three approaches; each might be entirely appropriate depending on the audiences. In these notes we'll typically follow a hybrid approach, aiming to replace words with symbols when it improves clarity while preserving readability.



\boldsubsubsection{Universal Quantifiers and Connectives: Hidden Quantifiers}\phantomsection\label{pg:univproof}

Universally quantified propositions are interchangeable with implications. To see this, suppose a propositional function $Q(x)$ is given and let $P(x)$ be ``$x$ lies in the domain of $Q$." Then
\[
	\tcbhighmath{\text{$\textcolor{blue}{\forall x,Q(x)}$ is logically equivalent to $\textcolor{Magenta}{P(x)\Longrightarrow Q(x)}$}}
\]
By convention, \textcolor{Magenta}{connectives containing variables} are assumed to be \textcolor{blue}{universal}. When written as a connective, the universal quantifier is often \emph{hidden.}\footnote{By contrast, the existential quantifier is never hidden: it is always explicit either symbolically ($\exists$) or as a phrase in English (\emph{there is, there exists, some, at least one,} etc.).}

\begin{examples}{}{oddsquared}
	\exstart The universal statement, ``Every cat is neurotic,'' may instead be written,
	\begin{enumerate}\setcounter{enumi}{1}
	  \item[]\begin{quote}
			If $x$ is a cat, then $x$ is neurotic.
		\end{quote} 
		
		\item\label{ex:easyquantprop2} Revisiting Example \ref{ex:easyquantprop}, we could rewrite ``$\forall x\in\R, x^2>4$'' as a connective,
		\[
			x\in\R\implies x^2>4 \tag{If $x$ is a real number, then $x^2>4$}
		\]
		
	  \item\label{ex:oddsquared2} The following three sentences have identical meaning:
	  \begin{quote}
	  	The square of an odd integer is odd.\qquad $\forall n$ odd, $n^2$ is odd.\qquad $n$ odd $\Longrightarrow n^2$ odd.
	  \end{quote}
	  The universal quantifier is explicit in only one of the sentences! For even more variety, the third sentence could be viewed as a universal statement about all \emph{integers}; including the \textcolor{red}{hidden quantifier} in this case results in
		\[
			\textcolor{red}{\forall n\in\Z},\ n\text{ odd} \implies n^2 \text{ odd.}
		\]
		where the symbol $\Z$ represents the (set of) integers.
	\end{enumerate}
\end{examples}

We've already seen that \emph{disproving} a universal statement requires only that we supply a \emph{counter-example.} While such might require some effort to find, often the resulting argument is very simple. By contrast, \emph{proving} a universal statement is the same as proving a conditional connective, typically a more involved activity. We therefore largely postpone this to the next section. Regardless, a simple proof of the above \emph{oddness} claim should be easy to follow.

\begin{proof}[Proof of Example \ref*{ex:oddsquared}.\ref{ex:oddsquared2}]
	If an integer $n$ is odd, then it may be written in the form $n=2k+1$ for some integer $k$. But then
	\[
		n^2 =(2k+1)^2 =4k^2+4k+1 =2(2k^2+2k)+1
	\]
	is plainly also odd.
\end{proof}

Similarly, \emph{proving} an existential statement (by providing an \emph{example}) is typically more straightforward than \emph{disproving} such. To help understand this duality, we need to see how to \emph{negate} quantified propositions.


\goodbreak


\boldsubsubsection{Negating Quantified Propositions}

To negate a proposition is to consider what it means for the proposition to be \emph{false.} We already understand what this means for a universal proposition:
\begin{quote}
	``$\forall x,P(x)$ false'' means that \emph{there exists a counter-example.}
\end{quote}
Otherwise said, the negation of a universal statement is \emph{existentially quantified}:
\begin{quote}
	The negation of ``$P(x)$ is \emph{always} true'' is ``$P(x)$ is \emph{sometimes} false.''
\end{quote}
Replacing $P(x)$ with $\neg P(x)$ provides a related observation:
\begin{quote}
	The negation of ``$P(x)$ is \emph{always} false" is ``$P(x)$ is \emph{sometimes} true.''
\end{quote}
To summarize:

\begin{thm}{}{negquant}
	For any propositional function $P(x)$:
	\begin{enumerate}
	  \item $\neg\bigl(\forall x, P(x)\bigr)$ is logically equivalent to $\exists x, \neg P(x)$.
	  \item $\neg\bigl(\exists x, P(x)\bigr)$ is logically equivalent to $\forall x, \neg P(x)$.
	\end{enumerate}
\end{thm}


\begin{examples}{}{}
	\exstart ``Everyone owns a bicycle," has negation ``Someone does not own a bicycle.'' 
		It is ugly, but we could write this semi-symbolically:
		\[
			\neg\bigl(\forall\text{ people }x,\ x\text{ owns a bicycle}\bigr)\iff \exists\text{ a person $x$ such that $x$ does not own a bicycle}
		\]
		
	\begin{enumerate}\setcounter{enumi}{1}
		\item The quantified proposition\footnotemark{} ``$\exists x>0, \sin x=4$,''
		has the form $\exists x,\,P(x)$. Its negation is therefore $\forall x,\ \neg P(x)$: explicitly,
		\[
			\forall x>0,\ \sin x\neq 4
		\]
		Since sine satisfies $-1\le\sin x\le 1$, the original proposition is \emph{false} and its negation \emph{true.}
		\[
			\tcbhighmath{\text{\textcolor{red}{Warning!}\lstsp \emph{Don't} change the \textcolor{Magenta}{domain} when negating: $\forall x\,\textcolor{Magenta}{\le 0}, \sin x\neq 4$ is wrong!}}
		\]
		
		\item Take special care negating connectives: a negated hidden quantifier becomes \textcolor{red}{explicit}.
		\[
			\tcbhighmath{\neg\bigl(\textcolor{blue}{P(x)}\Longrightarrow \textcolor{Green}{Q(x)}\bigr)\ \text{ is logically equivalent to }\ \textcolor{red}{\exists x}, \textcolor{blue}{P(x)} \wedge \textcolor{Green}{\neg Q(x)}}
			\tag{Theorem \ref{thm:negconditional}}
		\]
		\begin{enumerate}
		  \item (Example \ref*{ex:oddsquared}.\ref{ex:oddsquared2})\lstsp The negation of ``\textcolor{blue}{$n$ odd} $\Longrightarrow \textcolor{Green}{n^2}$ \textcolor{Green}{odd},'' is the (false) claim
			\[
				\textcolor{red}{\exists n\in\Z}\text{ with \textcolor{blue}{$n$ odd} \emph{and} \textcolor{Green}{$n^2$ even}.}
			\]
			
			\item (Example \ref*{ex:oddsquared}.\ref{ex:easyquantprop2})\lstsp The negation of the false claim ``$\textcolor{red}{x\in\R}\Longrightarrow \textcolor{Green}{x^2>4}$'' is the true assertion
			\[
				\textcolor{red}{\exists x\in\R}\text{ for which }\textcolor{Green}{x^2\le 4}
				\tag{Proof: for \emph{example} $x=1$}
			\]
		\end{enumerate}
	\end{enumerate}
\end{examples}

\vspace{-5pt}

\footnotetext{%
	``$\exists x>0$'' indicates that the \emph{domain} of the proposition ``$\sin x=4$'' is the \emph{positive} real numbers.%
}


\goodbreak


\boldsubsubsection{Multiple Quantifiers}\phantomsection\label{pg:multquant}

A propositional function can have several variables, each of which may be quantified.

\begin{examples}{}{multiplequant}
	\exstart The quantified proposition
	\[
		\forall x>0,\exists y>0\text{ such that }xy=4 \tag{$\ast$}
	\]
	\begin{enumerate}\setcounter{enumi}{1}
	  \item[]might be read, ``Given any positive number, there is another such that their \emph{product} is four.''	Hopefully you believe that this is \emph{true}! Here is a simple argument which comes from viewing ($\ast$) as an implication, ``$x>0 \Longrightarrow\exists y>0$ such that $xy=4$.''
	\begin{quote}
		\begin{proof}
			Suppose we are given $x>0$. Let $y=\frac 4x$. Then $xy=4$, as required.
		\end{proof}
	\end{quote}Being clear about \emph{domains} is critical. Suppose we modify the original proposition:
		\[
			\forall x\in\R,\exists y\in\R\text{ such that }xy=4 \tag{$\dag$}
		\]
		Our proof now fails! The new statement ($\dag$) is \emph{false}: indeed $x=0$ provides a \emph{counter-example.}
	\begin{quote}
		\begin{proof}[Disproof]
			Let $x=0$. Since $xy=0$ for any real number $y$, we cannot have $xy=4$.
		\end{proof}
	\end{quote}
	  \item[] Alternatively, we could \emph{negate} ($\dag$): following Theorem \ref{thm:negquant}, we switch the symbols $\forall\leftrightarrow\exists$ and negate the final proposition,\footnotemark{}
		\[
			\neg\bigl(\forall x\in\R,\exists y\in\R, xy=4\bigr) \iff \exists x\in\R, \forall y\in\R,\ xy\neq 4 \tag{$\neg (\dag)$}
		\]
		Our \emph{disproof} of ($\dag$) is really a \emph{proof} of the negation: we provided the \emph{example} $x=0$, thus demonstrating the truth of a $\exists$-statement. Since the negation is true, the original $(\dag)$ is false.
		
		
	  \item\label{ex:multiplequant2} \textcolor{red}{Order of quantifiers matters!}\lstsp The meaning of a sentence---\emph{and its truth state}---can change if we alter the order of quantification.
	  \begin{enumerate}
	  	\item $\forall x\in\R,\exists y\in\R,x^2<y$
		\end{enumerate}
		\begin{quote}
			\begin{proof}
				Suppose a real number $x$ is given. Let $y=x^2+1$. Then $x^2<y$, as required.
			\end{proof}
		\end{quote}
	  \begin{enumerate}\setcounter{enumii}{1}\itemsep8pt
	    \item[] As above, we prove this by viewing it as an implication, ``If $x\in\R$, then $\exists y\in\R,x^2<y$.''
	  	\item $\exists y\in\R,\forall x\in\R$, $x^2<y$
		\end{enumerate}
		\begin{quote}
			\begin{proof}[Disproof]
				We demonstrate the truth of the negation, ``$\forall y\in\R,\exists x\in\R,x^2\ge y$.''\smallbreak
				Suppose a real number $y$ is given. Let $x=\sqrt{\nm y}$. Then $x^2=y\ge y$, as required.
			\end{proof}
		\end{quote} 
	\end{enumerate}
\end{examples}

\footnotetext{%
	For an abstract justification of this heuristic, consider a propositional function $P(x)$: ``$\exists y,Q(x,y)$,'' then
	\[
		\neg\bigl(\forall x, \exists y,Q(x,y)\bigr) \iff  \neg\bigl(\forall x,P(x)\bigr) \iff \exists x,\neg P(x)  \iff \exists x,\neg\bigl(\exists y,Q(x,y)\bigr) \iff \exists x, \forall y,\neg Q(x,y)
	\]
}

Don't worry if these arguments seem difficult at the moment; much more practice is coming.


\goodbreak


%\boldinline{Putting it all together}

We finish with two harder examples you might have encountered elsewhere. For this course, \emph{you do not have to know what these statements mean,} though you do have to be able to \emph{negate} them.  

\begin{examples}{}{}
	\exstart (Linear Algebra)\lstsp Vectors $\vx,\vy,\vz$ in $\R^3$ are said to be \emph{linearly independent} if
	\[
		\forall a,b,c\in\R,\ a\vx+b\vy+c\vz=\V0\implies a=b=c=0
	\]
	Since this is a conditional proposition, the expression $\forall a,b,c\in\R$ would likely be hidden. The negation of this statement, what it means for $\vx,\vy,\vz$ to be \emph{linearly dependent,} is
	\[
		\exists a,b,c\in\R,\text{ such that }a\vx+b\vy+c\vz=\V0\text{ and $a,b,c$ are not all zero}
	\]
	\begin{enumerate}\setcounter{enumi}{1}
	  \item (Analysis/Calculus)\lstsp A function $f:\R\to\R$ is said to be \emph{continuous at $a\in\R$} if
		\[
			\forall\epsilon >0,\ \exists\delta>0\text{ such that }|x-a|<\delta\implies |f(x)-f(a)|<\epsilon
		\]
		The negation, what it means for $f$ to be \emph{discontinuous at $x=a$,} is
		\[
			\exists\epsilon>0\text{ such that }\forall\delta>0,\ \textcolor{red}{\exists x\in\R}\text{ with }|x-a|<\delta\text{ and }|f(x)-f(a)|\ge\epsilon
		\]
		The original statement contained a hidden quantifier $\textcolor{red}{\forall x}$ which became explicit upon negation.
	\end{enumerate}
\end{examples}



\begin{exercises}{}{}
	A self-test quiz and several worked questions can be found \href{http://www.math.uci.edu/~ndonalds/math13/selftest/2-2-quants.html}{online}.

	\begin{enumerate}
		\item Rewrite each sentence using quantifiers. Then write the negation (use words and quantifiers).
			\begin{enumerate}
			  \item All mathematics exams are hard.
		  	\item No football players are from San Diego.
		  	\item There is a odd number that is a perfect square.
			\end{enumerate}
			
			
		\item Let $P$ be the proposition: ``Every positive integer is divisible by thirteen.''
	    \begin{enumerate}
	      \item Write $P$ using quantifiers.
	      \item What is the negation of $P$?
	      \item Is $P$ true or false? Prove your assertion.
	    \end{enumerate}
	  
	  
	  \item A friend claims that the sentence ``$x^2>0\implies x>0$'' has negation ``$x^2>0$ and $x\le 0$.'' Why is this incorrect? What is the correct negation?
	  
	
		\item Consider the quantified statement
	    \[
	    	\forall x,y,z\in\R,\ (x-3)^2+(y-2)^2+(z-7)^2>0 \tag{$\ast$}
	    \]
			\begin{enumerate}
		    \item Express ($\ast$) in words.
		    \item Is ($\ast$) true or false? Explain.
		    \item Express the negation of ($\ast$) in symbols, and then in words.
		    \item Is the negation of ($\ast$) true or false? Explain.
		  \end{enumerate}
	    
	    
	  \item Suppose $P, Q,R$ are propositional functions. Compute the negations of the following:
	  \begin{enumerate}
	    \item $\forall x,\exists y, P(x)\wedge Q(y)$ \qquad\qquad (b) \ $\forall x,\exists y, \forall z,R(x,y,z)$
	  \end{enumerate}
	  
	  
	  \goodbreak
	  
	  
	  \item Revisit Example \ref*{ex:multiplequant}.\ref{ex:multiplequant2}. Decide whether each of the following is true or false:
	  \begin{enumerate}
		  \item \makebox[170pt][l]{$\exists x\in\R,\forall y\in\R$, $x^2<y$ \hfill (b)} \ $\forall y\in\R,\exists x\in\R,x^2<y$
		\end{enumerate}
	
	  
		\item The following are statements about positive real numbers $x,y$. Which is true? Explain.
		\begin{enumerate}
		  \item \makebox[170pt][l]{$\forall x$, $\exists y$ such that $xy<y^2$\hfill (b)} \ $\exists x$ such that $\forall y$, $xy<y^2$
		\end{enumerate}
	
	
		\item Which of the following statements are true? Explain.
		\begin{enumerate}
		  \item $\exists$ a married person $x$ such that $\forall$ married people $y$, $x$ is married to $y$.
		  \item $\forall$ married people $x$, $\exists$ a married person $y$ such that $x$ is married to $y$.
		\end{enumerate}
		
		
		\item Prove or disprove:
		\begin{enumerate}
		  \item For every two points $A$ and $B$ in the plane, there exists a circle on which both $A$ and $B$ lie.
		  \item There exists a circle in the plane on which lie any two points $A$ and $B$.
		\end{enumerate}
	  
	  
		\item Consider the following proposition (\emph{you do not have to know what is meant by a field}).
		\begin{quote}
			All non-zero elements $x$ in a field $\F$ have an inverse: some $y\in\F$ for which $xy=1$.
		\end{quote}
		\begin{enumerate}
		  \item Restate the proposition using quantifiers.
		  \item Find the negation of the proposition, again using quantifiers.
		\end{enumerate}
			
			
		\item\label{ex:decreasing} ``A function $f:\R\to\R$ is \emph{decreasing}'' means: $x\le y\Longrightarrow f(x)\ge f(y)$.
		\begin{enumerate}
		  \item State what it means for $f$ not to be decreasing (\emph{where is the hidden quantifier?})
		  \item Give an example to show that \emph{not decreasing} and \emph{increasing} do not mean the same thing.
		\end{enumerate}
		
		
		\item\label{exs:contradictproof2} Prove that $P(x)\Longrightarrow Q(x)$ is logically equivalent to $\bigl(\exists x, P(x)\wedge\neg Q(x)\bigr)\Longrightarrow F$.\par
		(\emph{This extends Exercise \ref*{sec:prop}.\ref{exs:contradictproof}---be careful of the quantifiers!})
		
			
		\item Consider the proposition: $\forall m,n\in\R,\ m>n\Longrightarrow m^2>n^2$.
		\begin{enumerate}
	  	\item State the negation of the proposition.
	  	\item Prove that the original proposition is \emph{false.}
	  	\item Suppose you rewrite the proposition: $\forall m,n\in A, \ m>n\Longrightarrow m^2>n^2$.\par
	  	What is the largest collection (set) of real numbers $A$ for which the proposition is \emph{true}?
		\end{enumerate}
	
		
		\item (Hard)\lstsp Let $(x_n)=(x_1,x_2,x_3,\ldots)$ denote a sequence of real numbers.
		\begin{quote}
			\makebox[150pt][l]{``$(x_n)$ \emph{diverges to $\infty$}'' means:\hfill}$\forall M>0,\,\exists N\in\R$ such that $n>N\Longrightarrow x_n>M$\smallbreak
			\makebox[150pt][l]{``$(x_n)$ \emph{converges to $L$}" means:\hfill}$\forall\epsilon>0,\,\exists N\in\R$ such that $n>N\Longrightarrow \nm{x_n-L}<\epsilon$
		\end{quote}
		\begin{enumerate}
		  \item State what it means for a sequence $(x_n)$ not to diverge to $\infty$. \emph{Beware of the hidden quantifier!}
		  \item State what it means for a sequence $(x_n)$ not to converge to $L$.
		  \item State what it means for a sequence $(x_n)$ not to converge at all.
		  \item (Challenge: non-examinable)\lstsp Use the definitions to prove that the sequence defined by $x_n=n$ diverges to $\infty$, and that the sequence defined by $y_n=\frac 1n$ converges to zero.
		\end{enumerate}
	
	\end{enumerate}

\end{exercises}


\clearpage



\subsection{Methods of Proof}\label{sec:proof}

The previous sections covered some of the language of foundational logic. While one can study this more deeply, we focus on putting it to work in the service of mathematics. The real work begins now.\medbreak

A mathematical \emph{theorem} is\footnote{It might be awkward to fit a theorem into this format but it can always be done. Often all that is stated is the conclusion $Q$, in which case $P$ would be the assertion ``All mathematics we already know/assume to be true.''} a justified assertion of the truth of an implication $P\Longrightarrow Q$. A \emph{proof} is any logical argument justifying the theorem. The first step in analyzing or strategizing a proof is to identify the hypothesis $P$ and conclusion $Q$.\medbreak 

There are four standard methods of proof; in practice longer arguments combine several of these.
\begin{description}
	\item[Direct] Assume the hypothesis $P$ and deduce the conclusion $Q$.\footnote{To \emph{assume} a proposition is to suppose its \emph{truth.} To suppose $P$ is false, we ``assume/suppose $\neg P$.''} This structure should be intuitive, though it may help to revisit the truth table in Definition \ref{defn:implies} and the tautology of Example \ref*{ex:conditionalbasic}.\ref{ex:taut2}.
	\item[Contrapositive] Directly prove the contrapositive $\neg Q\Longrightarrow\neg P$ (logically equivalent to $P\Longrightarrow Q$ by Theorem \ref{thm:contrapos}).
	\item[Contradiction] Assume $P\wedge\neg Q$ and deduce a \emph{contradiction} (directly prove $(P\wedge\neg Q)\Rightarrow F$). Theorem \ref{thm:negconditional} and Exercise \ref*{sec:prop}.\ref{exs:contradictproof} show that $P\Longrightarrow Q$ is true.	If $P(x)\Longrightarrow Q(x)$ has a hidden universal quantifier, the negation means we start by assuming $\textcolor{red}{\exists x}, P(x)\wedge\neg Q(x)$ (Exercise \ref*{sec:quant}.\ref{exs:contradictproof2}).
	\item[Induction] This has a completely different flavor; we will consider it in Chapter \ref{chap:induction}.
\end{description}

Each method has advantages and disadvantages: direct proofs typically have the simplest logical flow; contrapositive/contradiction approaches are useful when the negations $\neg P$, $\neg Q$ are easier to work with than $P$, $Q$ themselves. All methods are equally valid, and, as we'll see shortly, one can often prove a simple theorem using all three approaches!\smallbreak

As you work through this section, pay special attention to the logical structure---to encourage this, the mathematical level is very low. Refer to the previous sections if the logical terminology feels unfamiliar. Now is also a good time to re-read \emph{Planning and Writing Proofs} (page \pageref{sec:proofplan}).

\boldsubsubsection{Direct Proofs}

We begin by generalizing Example \ref*{ex:oddsquared}.\ref{ex:oddsquared2}.

\begin{thm}{}{oddodd}
	The product of any pair of odd integers is odd.
\end{thm}

To make sense of this, we first need to identify the logical structure by writing the theorem in terms of propositions and connectives. One way is to view the Theorem in the form $P\Longrightarrow Q$:
\begin{itemize}
  \item $P(x,y)$ is ``$x$ and $y$ are both odd.'' This is our assumption, the hypothesis.
  \item $Q(x,y)$ is ``The product $xy$ is odd.'' This is what we wish to demonstrate, the conclusion.
  \item Both propositional functions are statements about \emph{integers.} The Theorem is \emph{universal} (``any pair"), and so contains a (hidden) quantifier $\forall x,y\in\Z$.
\end{itemize}
We also need a clear understanding of the meaning of all necessary terms. To keep things simple, we'll treat \emph{integer} and \emph{product} as understood and be explicit only as to the meaning of \emph{oddness.}\goodbreak


A direct proof can be viewed as a \textcolor{blue}{proof sandwich} whose bread slices are the \textcolor{blue}{hypothesis and conclusion} ($P$ and $Q$): write these down as a first step. Next \textcolor{Green}{define} any useful terms in the hypothesis. All that remains is to perform a simple calculation!


\begin{proof}
	\newlength\pflen\settowidth\pflen{(computation/algebra)}
	\def\cmt#1{\hfill\makebox[\pflen][l]{#1}}
	\textcolor{blue}{Let $x$ and $y$ be odd integers.} \cmt{(state \textcolor{blue}{hypothesis} $P$)}\par
	There are integers $k$, $l$ for which $x=2k+1$ and $y=2l+1$. Then, \cmt{(\textcolor{Green}{definition} of \emph{odd})}
	\begin{align*}
		xy&=(2k+1)(2l+1) =4kl+2k+2l+1 \tag*{\cmt{(computation/algebra)}}\\
		&=2(2kl+k+l)+1
	\end{align*}
	Since $2kl+k+l$ is an integer, we conclude that \textcolor{blue}{$xy$ is odd.} \qedhere \cmt{(state \textcolor{blue}{conclusion} $Q$)}
\end{proof}

Observe how we wrote $xy$ in the form 2(integer)$+1$ so as to make the conclusion absolutely clear.


\boldinline{Insufficient Generality}

Before leaving this example, it is worth highlighting the most common mistake seen in such arguments.

\begin{proof}[Fake Proof 1]
	$x=3$ and $y=5$ are both odd, hence $xy=15$ is odd.\phantom{\qedhere}
\end{proof}

This is an \emph{example} of the theorem. Since the theorem is \emph{universal,} a single example does not constitute a proof (recall, however, that an example proves an \emph{existential} statement: Definition \ref{defn:example-cex}). 

\begin{proof}[Fake Proof 2]
	Let $x=2k+1$ and $y=2k+1$ be odd. Then
	\[
		xy=(2k+1)(2k+1)=2(2k^2+2k)+1\quad \text{is odd.}\tag*{\phantom\qedhere}
	\]
\end{proof}

This only verifies the special case where both odd integers are \emph{equal}: it proves $x$ odd $\Longrightarrow x^2$ odd. There is nothing wrong with trying out examples or sketching incomplete thoughts---indeed both are encouraged!---but you need to be aware of when your argument isn't sufficiently general.\bigbreak

For another simple direct proof, consider the sum of two consecutive integers.

\begin{thm}{}{}
	The sum of any pair of consecutive integers is odd.
\end{thm}

The theorem is again a universal claim (``any") of the form $P\Longrightarrow Q$ about two integers $x,y$:
\begin{itemize}
	\item $P(x,y)$ is ``$x,y$ are consecutive integers.''
	\item $Q(x,y)$ is ``$x+y$ is odd.''
\end{itemize}

The trick is to observe that, being consecutive, we may write one integer in terms of the other. The \textcolor{blue}{proof sandwich} is still visible, though it would be hard to write down the last sentence without already having settled on the trick, which is essentially the \textcolor{Green}{definition} of ``consecutive integers.''
	
\begin{proof}
	\textcolor{blue}{Suppose we are given two consecutive integers.} Label the smaller of these $\textcolor{Green}{x}$ and the other $\textcolor{Green}{x+1}$. Their sum is then
	\[
		x+(x+1)=2x+1
	\]
	\textcolor{blue}{which is odd.}
\end{proof}


\goodbreak


\boldsubsubsection{Proof by Contrapositive}

Here is another straightforward result about odd and even integers.

\begin{thm}{}{oddsum}
	If the sum of two integers is odd, then they have opposite parity.
\end{thm}

The theorem is yet another universal statement ($\forall x,y\in\Z$) of the form $P\Longrightarrow Q$:
\begin{quote}
	\makebox[220pt][l]{$P(x,y)$: ``$x+y$ is odd.''\hfill }$Q(x,y)$: ``$x,y$ have opposite parity.''
\end{quote}
\emph{Parity} means \emph{evenness or oddness}: the conclusion is that one of the integers is even and the other odd.\smallbreak

Naïvely attempting a direct proof produces an immediate difficulty:

\begin{proof}[Direct Proof?]
	Suppose $x+y$ is odd. Then $x+y=2k+1$ for some integer $k$\ldots\phantom{\qedhere}
\end{proof}

We want to conclude something about $x$ and $y$ \emph{separately,} but the direct approach lumps them together in the same algebraic expression.\smallbreak

A contrapositive approach ($\neg Q\Longrightarrow\neg P$) suggests itself as a remedy, since the new hypothesis $\neg Q$, by treating $x$ and $y$ separately, gives us twice as much to start with.\footnote{\textcolor{red}{Warning!} The contrapositive is still a \emph{universal} statement:  $\forall x,y\in\Z, \neg Q(x,y)\Longrightarrow \neg P(x,y)$. We are not negating the theorem so \textcolor{red}{do not convert $\forall$ to $\exists$!}}
\begin{quote}
	\makebox[220pt][l]{$\neg Q(x,y)$: ``$x,y$ have the same parity.''\hfill }$\neg P(x,y)$: ``$x+y$ is even.''
\end{quote}

The remaining difficulty is that ``same parity'' encompasses \emph{two} possibilities: $x,y$ are either both even, or both odd. The proof therefore contains two cases.

\begin{proof}
	\settowidth\pflen{(state hypothesis $\neg Q$)}
	\def\cmt#1{\hfill\makebox[\pflen][l]{#1}}
	\textcolor{blue}{Suppose $x$ and $y$ have the same parity.} There are two cases.\cmt{(state \textcolor{blue}{hypothesis} $\neg Q$)}
	\begin{description}\itemsep0pt
	  \item[\normalfont Case 1:] Assume $x$ and $y$ are both even.\newline
	  Write $x=2k$ and $y=2l$, for some integers $k,l$. \cmt{(\textcolor{Green}{definition} of \emph{even})}\newline
	  Then $x+y=2(k+l)$ is even. \cmt{(computation)}
	  \item[\normalfont Case 2:] Assume $x$ and $y$ are both odd.\newline
	  Write $x=2k+1$ and $y=2l+1$ for some integers $k,l$. \cmt{(\textcolor{Green}{definition} of \emph{odd})}\newline
	  Then $x+y=2(k+l+1)$ is even. \cmt{(computation)}
	\end{description}
	In both cases \textcolor{blue}{$x+y$ is even.} \hspace*{150pt}\qedhere \cmt{(state \textcolor{blue}{conclusion} $\neg P$)}
\end{proof}

Again observe the \textcolor{blue}{proof sandwich} and how the argument depends on little more than the \textcolor{Green}{definitions} of even and odd.\smallbreak

When presenting a lengthier contrapositive argument, consider orienting the reader by starting with the phrase, ``We prove the contrapositive." For simple proofs (like the above) this is unnecessary, since the logical structure should be clear without such assistance. It is also unnecessary to define and spell out the propositions $P$ and $Q$ or include any of the bracketed commentary. However, feel free to continue this practice if you think it aids your explanation, of if you are nervous about your proof skills.\footnote{\ldots and want to guarantee some partial credit!}


\goodbreak


For another example of a contrapositive argument, we extend the first result of this section.

\begin{thm}{}{oddodd2}
	The product of two integers is odd \emph{if and only if} both integers are odd.
\end{thm}

This is a \emph{biconditional} $P\Longleftrightarrow Q$, comprising \emph{two theorems in one}: $P\Longrightarrow Q$ and $Q\Longrightarrow P$ (Exercise \ref*{sec:prop}.\ref{exs:iff}). A contrapositive argument for $P\Longrightarrow Q$ is again suggested because $Q$ (\emph{both} integers are odd) treats the two integers individually.

\begin{proof}
	\begin{description}
		\item[\normalfont ($\Rightarrow$)] We prove the contrapositive. Let $x,y$ be integers, at least one of which is even. Suppose, \emph{without loss of generality,} that $x=2k$ is even. Then $xy=2ky$ is also even.
		\item[\normalfont ($\Leftarrow$)] This is precisely Theorem \ref{thm:oddodd}, which we've already proved.\qedhere
	\end{description}
\end{proof}

Note de Morgan's law: $\neg(x$ odd \emph{and} $y$ odd) is equivalent to ``$x$ even \emph{or} $y$ even'' (``at least one'').\par

The common phrase \emph{without loss of generality}, often abbreviated WLOG, saves us from performing a second, almost identical, argument assuming $y=2l$ is even. WLOG is stated when one makes a choice which does not materially affect the argument.


\boldsubsubsection{Proof by Contradiction}

Here is a simple result considered in several ways.\footnote{%
	From now on we'll reserve `Theorem' for results that are worth remembering in their own right.
}


\begin{example}{}{x odd if 3x+5 is odd}
	Let $x$ be an integer. We prove that if $3x+5$ is even, then $5x+2$ is odd.\medbreak

	We could proceed directly according to the following sketch:
	\[
		3x+5\text{ even }\operatorname{\textcolor{blue}{\implies}} 3x\text{ odd }\operatorname{\textcolor{blue}{\implies}} x\text{ odd }\operatorname{\textcolor{blue}{\implies}} 5x\text{ odd }\implies 5x+2\text{ odd} \tag{$\ast$}
	\]
	This isn't wrong! You should believe each implication; indeed we've proved \textcolor{blue}{most of them}. It would be nice, however, if we didn't have to rely on so many other results. A similar contrapositive proof (reverse the arrows and negate the propositions) would have the same weakness.\medbreak
	
	The advantage of a contradiction approach is that we have twice as much to work with: the hypothesis ($3x+5$ even) \emph{and} the negation of the conclusion ($5x+2$ \emph{even}).
	
	\begin{quote}
		\begin{proof}
			Suppose $x$ is an integer for which both $3x+5$ and $5x+2$ are even. Then their sum is also even. However,
		  \[
		  	(3x+5)+(5x+2)=8x+7=2(4x+3)+1 \tag{$\dag$}
		  \]
		  is odd. Contradiction (an integer cannot be both even and odd!).
		\end{proof}
	\end{quote}
	
	Remember to write \emph{contradiction} at the end so the reader knows what you've done!\smallbreak
	
	A nice side-effect of this approach is that it suggests an alternative \emph{direct proof.}
	
	\begin{quote}
		\begin{proof}[Direct Proof]
			For any integer $x$, ($\dag$) says that $3x+5$ and $5x+2$, in summing to an odd number, have \emph{opposite parity.}
		\end{proof}
	\end{quote}
	
	The last argument in fact proves that $3x+5$ is even \emph{if and only if} $5x+2$ is odd; the converse of the original claim comes for free! Revisiting $(\ast)$, you should believe that all the arrows are reversible.
\end{example}


\goodbreak


Such variety is one of the things that makes proving theorems fun! While the choice of proof method is largely a matter of personal taste, remember your audience. Our final direct argument is very slick but risks confusing an elementary reader rather than empowering them.\footnote{The Hungarian mathematician Paul Erdős referred to simple, elegant proofs as `from the Book,' as if the  Almighty kept a tome of perfect proofs. As with all matters spiritual, one person's Book is likely very different to another's\ldots} 



\boldinline{Three Proofs of the Same Result}

We finish this section with three proofs of the same result. All are based on the same factorization of a polynomial
\[
	x^3+4x^2-2x-20=(x-2)(x^2+6x+10)=(x-2)\bigl[(x+3)^2+1\bigr]
\]
and the well-known fact that $ab=0\iff a=0$ or $b=0$ (see Exercise \ref{exs:zerofactor}). Since the mathematics is so simple, pay attention to and compare the \emph{logical structures}---which do you prefer?

\begin{example}{}{polyroot}
	Let $x$ be a real number. We prove that $x^3+4x^2-2x-20=0\Longrightarrow x=2$.
	\begin{quote}
		\begin{proof}[Direct Proof]
			Suppose $x^3+4x^2-2x-20=0$. By factorization, $(x-2)[(x+3)^2+1]=0$, so at least one of the factors must be zero. Since $(x+3)^2+1\ge 1>0$, we conclude that $x-2=0$, from which $x=2$.
		\end{proof}
	
		\begin{proof}[Contrapositive Proof]
		Suppose $x\neq 2$. Since $(x+3)^2+1\ge 1>0$, we see that
			\[
				x^3+4x^2-2x-20=(x-2)[(x+3)^2+1]\neq 0 \tag*{\qedhere}
			\]
		\end{proof}
	
		\begin{proof}[Contradiction Proof]
			Suppose $x^3+4x^2-2x-20=0$ and $x\neq 2$. Then
			\[
				0=x^3+4x^2-2x-20=(x-2)[(x+3)^2+1]
			\]
			Since $x\neq 2$, we have $x-2\neq 0$. It follows that $(x+3)^2+1=0$. However, $(x+3)^2+1\ge 1$ for all real numbers $x$, so we have a contradiction.
		\end{proof}
	\end{quote}
\end{example}



\begin{exercises}{}{}
	A self-test quiz and several worked questions can be found \href{http://www.math.uci.edu/~ndonalds/math13/selftest/2-3-proofs.html}{online}.

	\begin{enumerate}
  	\item Prove or disprove the following conjectures.
		\begin{enumerate}
	  	\item There is an even integer which can be expressed as the sum of three even integers.
	  	\item Every even integer can be expressed as the sum of three even integers. 
	  	\item There is an odd integer which can be expressed as the sum of two odd integers.
	  	\item Every odd integer can be expressed as the sum of three odd integers.
		\end{enumerate}
		%\emph{To get a feel about whether a claim is true or false, try out some examples. If you believe a claim is false, provide a specific counterexample. If you believe a claim is true, give a formal proof.}

	
		\item For any given integers $a,b,c$, if $a$ is even and $b$ is odd, prove that $7a-ab+12c+b^2+4$ is odd.
	

		\item Prove that if $n$ is an integer greater than 1, then $n!+2$ is even.\par
  (\emph{$n!=n(n-1)(n-2)\cdots 1$ is the \emph{factorial} of the integer $n$})
  
  
		\item\begin{enumerate}
		  \item Let $x\in\Z$. Prove that $5x+3$ is even if and only if $7x-2$ is odd.
		  \item Can you conclude anything about $7x-2$ if $5x+3$ is odd?
		\end{enumerate}
		
  
  	\item Consider the following proposition, where $x$ is assumed to be a real number.
		\[
			x^3-3x^2-2x+6=0\implies x=3
		\]
		\begin{enumerate}
	  	\item Is the proposition true or false? Justify your answer. Is its converse true?
	  	\item Repeat part (a) for the proposition $x^3-3x^2-2x+6=0\implies x\neq 3$.
	%   	\item Does anything change about the truth status of ($\ast$) if we assume that it is a statement about \emph{rational numbers $x$}? Explain.
		\end{enumerate}
		
	  
	  
		\item Below is the proof of a result. What result is being proved?
	  \begin{proof}
	  	Assume that $x$ is odd. Then $x=2k+1$ for some integer $k$. Then
	  	\[
	  		2x^2-3x-4=2(2k+1)^2-3(2k+1)-4=8k^2+2k-5=2(4k^2+k-3)+1
	  	\]
	  	Since $4k^2+k-3$ is an integer, $2x^2-3x-4$ is odd.
	  \end{proof}
 
	
		\item Here is another proof. What is the result this time?
	  \begin{proof}
	  	Assume, without loss of generality, that $x=2a$ and $y=2b$ are both even. Then
	  	\[
	  		xy+xz+yz=(2a)(2b)+(2a)z+(2b)z=2(2ab+az+bz)
	  	\]
	  	Since $2ab+az+bz$ is an integer, $xy+xz+yz$ is even.
	  \end{proof}
	
	
    \item Consider the following proof of the fact that (for $m$ an integer) if $m^2$ is even, then $m$ is even. Can you re-write the proof so that it doesn't use contradiction?
    \begin{proof}
    	Suppose that $m^2$ is even and $m$ is odd. Write $m=2k+1$ for some integer $k$. Then
			\[
				m^2=(2k+1)^2 =4k^2+4k+1 =2(2k^2+2k) + 1
			\]
      is odd. Contradiction.
    \end{proof}	 
    
    
    \item Here is a `proof' that every real number $x$ equals zero. Find the mistake.
		\begin{align*}
			x=y\implies &x^2=xy \implies x^2-y^2=xy-y^2\\
			\implies &(x-y)(x+y)=(x-y)y\\
			\implies &x+y=y\\
			\implies &x=0
		\end{align*}
		
		
		\item\label{exs:cuberoot2} Prove or disprove: An integer $n$ is even if and only if $n^3$ is even. 
    
   
		\item Let $n$ and $m$ be positive integers. Prove $n^2m$ is even if and only if $n$ and $m$ are not both odd.
		
		
    \item Let $x$ and $y$ be integers. Prove $x^2+y^2$ is even  \textbf{if and only if} $x$ and $y$ have the same parity.
    
    
    \item Let $n$ be an integer. Prove $n^2+n+58$ is even. 

  
		\item\label{exs:zerofactor} Suppose $a,b\in\R$. Prove that $ab=0\iff a=0$ or $b=0$.
  
    
    \item Numbers of the form $\frac{k(k+1)}2$, where $k$ is a positive integer, are called \emph{triangular numbers}. Prove that $n$ is the square of an odd number if and only if $\frac{n-1}{8}$ is triangular or zero.
         
	\end{enumerate}
\end{exercises}


\clearpage




\subsection{Further Proofs \& Strategies}\label{sec:proof2}


The arguments in this section are slightly trickier and more representative of typical mathematics. Some of these results are indeed quite famous and worth knowing in their own right. We will also introduce \emph{lemmas} and \emph{corollaries} which are used to break up the presentation of complex results.


\boldsubsubsection{Proving Universal Statements}

Most of the results we've seen thus far have been universal. As discussed on page \pageref{pg:univproof}, any theorem $P(x)\Longrightarrow Q(x)$ is implicitly universal, albeit with the quantifier $\forall x$ typically hidden. Revisit Examples \ref{ex:multiplequant} to consider how multiple quantifiers fit into our proof framework; here is another example.

\begin{example}{}{allexists}
	We prove: $\forall x\ge 0$, $\exists y<0$ such that $x^3<y^2$\medbreak
	View the claim as an implication ``$x\ge 0\Longrightarrow (\textcolor{blue}{\exists y<0}$ \textcolor{blue}{such that} $\textcolor{blue}{x^3<y^2)}$'' and prove directly.
	\begin{quote}
		\begin{proof}
			Suppose $x\ge 0$ is given. Define $y=-\sqrt{x^3}-1$. Then $y\le -1<0$ and 
			\[
				y^2=x^3+1+2\sqrt{x^3}\ge x^3+1>x^3 \tag*{\qedhere}
			\]
		\end{proof}
	\end{quote}
	The \textcolor{blue}{conclusion} is existentially quantified, so we deduce it by supplying an \emph{example} (``\emph{Define} $y$\ldots''). Since $y$ is existentially quantified \emph{after} $x$, note that it is allowed to depend on $x$!	An argument such as this likely needs some scratch work to find a suitable $y$: don't expect to create such in one shot!
\end{example}


For a more involved example of a universal result, here is a famous inequality relating the \textcolor{blue}{arithmetic} and \textcolor{Magenta}{geometric} means of two numbers.

\begin{thm}{AM--GM inequality}{amgm}
	If $x,y$ are non-negative real numbers, then
	\[
		\textcolor{blue}{\frac{x+y}{2}}\ge\textcolor{Magenta}{\sqrt{xy}}
	\]
	with equality if and only if $x=y$.
\end{thm}

This requires some unpacking! First try an example to calm the nerves (e.g., $\frac{3+5}2=4\ge \sqrt{15}$). It should also be clear that both sides are equal whenever $x=y$. Now consider the logical structure: $x,y\ge 0\Longrightarrow Q(x,y)$; the challenge lies in making sense of $Q$. There are really two separate results:
\begin{enumerate}
  \item If $x,y\ge 0$, then $\frac{x+y}{2}\ge\sqrt{xy}$
  \item If $x,y\ge 0$, then $\frac{x+y}{2}=\sqrt{xy}\iff x=y$
\end{enumerate}

Concentrate on the first since it is simpler. The hypothesis ($x,y\ge 0$) doesn't give us much to work with, so it seems sensible to play with the inequality and try to eliminate the ugly square-root:
\[
	\frac{x+y}2\ge \sqrt{xy} \implies (x+y)^2\ge 4xy \implies x^2-2xy+y^2\ge 0\implies (x-y)^2\ge 0
\]
Now we have something believable! The question is whether we can reverse the arrows. Only the first should give you any pause; it is here that we use the \emph{non-negativity} of $x,y$.

\begin{proof}
	Suppose $x,y\ge 0$. Multiply out a trivial inequality:
	\begin{align*}
		(x-y)^2\ge 0&\iff x^2-2xy+y^2\ge 0 \iff x^2+2xy+y^2\ge 4xy\\
		&\iff (x+y)^2\ge 4xy\\
		&\iff \frac{x+y}{2}\mathrel{\textcolor{red}{\ge}} \sqrt{xy}
	\end{align*}
	The square-root is well-defined because $x,y\ge 0$, and the inequality is preserved since the square-root function is \emph{increasing.} For the second result, observe that the \textcolor{red}{final inequality} is an \emph{equality} precisely when \emph{all} the inequalities are equalities; this is if and only if $x=y$.
\end{proof}

The scratch work really helped us figure out how and where to apply the hypothesis. Notice also how the second result came almost for free! Result 1 only needed the $\Rightarrow$ direction in the proof, but the second result used the fact that all arrows are biconditionals.
\medbreak

For variety, here is a contradiction proof incorporating the same calculations in a different order.

\begin{proof}[Contradiction Proof]
	Let $x,y\ge 0$ and suppose that $\frac{x+y}{2}<\sqrt{xy}$. Since $x+y\ge 0$, the second inequality holds if and only if $(x+y)^2<4xy$. Now multiply out and rearrange:
	\begin{align*}
		(x+y)^2<4xy&\iff x^2+2xy+y^2<4xy\\
		&\iff x^2-2xy+y^2<0\\
		&\iff (x-y)^2<0
	\end{align*}
	Contradiction (squares of real numbers are non-negative). We conclude that $\frac{x+y}{2}\ge \sqrt{xy}$.\par
	Now suppose that $\frac{x+y}{2}=\sqrt{xy}$. Following the biconditionals in the above calculation, we see that equality holds if and only if $(x-y)^2=0$, from which we recover $x=y$.
\end{proof}

The AM--GM inequality in fact holds for any finite collection of non-negative numbers $x_1,\ldots, x_n$:
\[
	\frac{x_1+x_2+\cdots+x_n}n\ge\sqrt[n]{x_1x_2\cdots x_n}
\]
with equality if and only if all $x_i$ are equal. Proving this is a lot harder (see Exercise \ref{exs:amgm-full}).



\boldsubsubsection{Disproving Existential Statements: Non-existence Proofs}

Of the four combinations ``prove/disprove a universal/existential statement,'' we've now tackled three; it remains to consider how to prove that something does not, or cannot, exist. Recall our basic rule of negation (Theorem \ref{thm:negquant}):
\[
	\neg\bigl(\exists x,Q(x)\bigr)\iff \forall x,\neg Q(x)
\]
To show that $\exists x,Q(x)$ is false is to \emph{prove a universal statement}.\footnote{%
	The notation $\nexists x,Q(x)$ is discouraged because it obscures this essential fact.%
} As before (page \pageref{pg:univproof}), let $P(x)$ be the proposition ``$x$ lies in the domain of $Q$.'' We conclude that
\[
	\tcbhighmath{\text{$\textcolor{blue}{\neg\bigl(\exists x,Q(x)\bigr)}$ is logically equivalent to $\textcolor{Magenta}{P(x)\Longrightarrow \neg Q(x)}$}}
\]
Viewed in this way as an implication, any of our proof strategies might be applicable to a non-existence proof. Contradiction and contrapositive arguments are particularly common however, since the right hand side $(\neg Q)$ is already a \emph{negative} statement.


\begin{example}{}{easypolynosolns}
	We prove that the equation $x^{17}+12x^3+13x+3=0$ has no positive solutions.\medbreak

	Before seeing a proof, consider several ways in which this claim could be presented.
	\begin{quote}
		\def\arraystretch{1.1}
		\begin{tabular}{@{}ll}
			Non-existence ($\neg(\exists x, Q(x))$)& There are no $x>0$ for which $x^{17}+12x^3+13x+3=0$.\\
			Universal ($\forall x,\neg Q(x)$)&For all $x>0$, we have $x^{17}+12x^3+13x+3\neq 0$.\\
			Direct ($P\Rightarrow \neg Q$)&If $x>0$, then $x^{17}+12x^3+13x+3\neq 0$.\\
			Contrapositive ($Q\Rightarrow\neg P$)&If $x^{17}+12x^3+13x+3=0$, then $x\le 0$.\\
			Contradiction ($P\wedge Q$ false)&$x>0$ and $x^{17}+12x^3+13x+3=0$ is \emph{impossible.}
		\end{tabular}
	\end{quote}
	
	We present two similar arguments based on the direct and contradiction structures.  
	
	\begin{quote}
		\begin{proof}[Direct proof]
			Suppose that $x>0$. Then $x^{17}+12x^3+13x+3>0$ since all terms are positive. We conclude that $x^{17}+12x^3+13x+3\neq 0$.
		\end{proof}
		\begin{proof}[Contradiction proof]
			Assume that $x>0$ satisfies $x^{17}+12x^3+13x+3=0$. Since all terms on the left hand side are positive, we have a contradiction.
		\end{proof}
	\end{quote}
	
	Both arguments were very easy, with all difficulty coming from understanding \emph{why} they work: precisely the above discussion! Learning/practicing how to recognize/translate a claim into an actionable format is the essential skill here, both for the presenter and the reader.
\end{example}


\boldsubsubsection{Subdividing Theorems: Lemmas \& Corollaries}

It is sometimes helpful to break a proof into pieces, akin to viewing a computer program as a collection of subroutines. Often the intention is to improve the readability of a difficult/complex argument, though you may also wish to (de-)emphasize the relative importance of certain parts of a discussion. One way to do this is to utilise \emph{lemmas} and \emph{corollaries.}

\begin{quote}
	\begin{description}
	  \item[\normalfont\emph{Lemma:}] A theorem whose importance you want to downplay or which will later be used to help prove a more significant result. 
	  \item[\normalfont\emph{Corollary:}] A theorem which follows quickly from a previous result, either as a special case or by modifying the proof in a straightforward way.
	\end{description}
\end{quote}

Presentation style varies: some authors and journals reserve \emph{theorem} for only the most important results, with everything else presented as a lemma or corollary; others never use these terms or just call everything a \emph{proposition}! Regardless, lemmas and corollaries are useful to have in your toolkit if readability is your goal.\medbreak
 
 
Here is a very simple result in preparation for a much more important upcoming theorem.

\begin{lemm}{}{root2prep}
	Suppose $n$ is an integer. Then $n^2$ is even $\Longleftrightarrow n$ is even.
\end{lemm}

You should be able to prove this yourself, since the lemma is just a special case of Theorem \ref{thm:oddodd2}. If you are completely unsure how to start, revisit that result and the rest of Section \ref{sec:proof}.
 


\boldsubsubsection{Irrational Numbers}

Since their definition is inherently negative, irrational numbers provide good examples of non-existence/contradiction arguments. They are also interesting in their own right.

\begin{defn}{}{}
	A real number $x$ is said to be \emph{rational} if it may be written in the form $x=\frac mn$ for some integers $m,n$. A real number is \emph{irrational} if no such integers exist.
\end{defn}
 
You likely know of a few irrational numbers ($\sqrt 2,\pi,e$), but how do we \emph{prove} that a given number is irrational? Our next result is very famous, with versions dating back at least to Aristotle (c.\,340\BCE). 

\begin{thm}{}{sqrt2}
	$\sqrt 2$ is irrational.
\end{thm}

We must \emph{disprove} the existence claim $\exists m,n\in\Z, \sqrt 2=\frac mn$.
As before, consider several restatements:\vspace{-5pt}
\begin{quote}
	\begin{tabular}{@{}ll}
		Non-existence& There are no integers $m,n$ for which $\sqrt 2=\frac mn$.\\
		Universal&For all integers $m,n$, we have $\sqrt 2\neq \frac mn$.\\
		Direct&If $m,n\in\Z$, then $\sqrt 2\neq\frac mn$.\\
		Contrapositive&If $\sqrt 2=\frac mn$, then $m,n$ are not both integers.\\
		Contradiction&$m,n\in\Z$ and $\sqrt 2=\frac mn$ is \emph{impossible.}
	\end{tabular}
\end{quote}

Are the drawbacks of a direct or contrapositive approach obvious? We prove by contradiction. To improve readability, we outsource a repeated step to the ($\Rightarrow$) direction of Lemma \ref{lemm:root2prep}.

\begin{proof}
	Suppose $m,n\in\Z$ and that \textcolor{blue}{$\sqrt 2=\frac mn$}. \emph{Without loss of generality}, assume $m,n$ have \textcolor{red}{no common factors}. Cross-multiply and square:
	\[
		m^2=2n^2\text{ is even} \implies m \text{ is \textcolor{red}{even}} \tag{Lemma \ref{lemm:root2prep}}
	\]
	whence $m=2k$ for some integer $k$. But then
	\[
		2n^2=m^2=4k^2\implies n^2=2k^2 \text{ is even} \implies n \text{ is \textcolor{red}{even}} \tag{Lemma \ref{lemm:root2prep}}
	\]
	We see that $m$ and $n$ have a \textcolor{red}{common factor of 2}. Contradiction.
\end{proof}


Just as we can simplify $\frac 46=\frac 23$, the \textcolor{red}{no common factors assumption} is \emph{without loss of generality}: it costs nothing \emph{once we suppose \textcolor{blue}{$\sqrt 2=\frac mn$ is rational}.} This \textcolor{blue}{last} is what we contradicted! A (wrong) belief that the \textcolor{red}{no common factors assumption} was contradicted means the calculation continues forever!
\[
	m^2=2n^2\implies n^2=2k^2\implies k^2=2l^2\implies\cdots
\]

The irrationality of various surds ($\sqrt 3,\sqrt[3]{2}$, etc.), can be proved similarly ($\pi$ and $e$ are \emph{much} harder). We may also apply the theorem to demonstrate the irrationality of many other numbers.

\begin{example}{}{}
	Suppose $\sqrt 2-5\sqrt 3=x$ were rational: $\exists m,n\in\Z$ such that $x=\smash{\frac mn}$. Then
	\[
		75=(5\sqrt 3)^2=(\sqrt 2-x)^2=2+x^2-2\sqrt 2x \implies \sqrt 2=\frac{x^2-73}{2x} =\frac{m^2-73n^2}{2mn}
	\]
	Otherwise said, $\sqrt 2$ is \emph{rational}: contradiction.
\end{example}


\goodbreak


\boldsubsubsection{Non-constructive Existence Proofs}

Every existence proof we've thus far seen has been \emph{constructive}: we've exhibited/constructed an \emph{explicit example} $x$ for which $Q(x)$ is true. Sometimes this is asks too much. Indeed it is often far easier to show the existence of something \emph{without} explicitly stating what it is. We present two famous examples of this situation.

\begin{thm}{}{}
	There exist irrational numbers $a,b$ for which $a^b$ is rational.
\end{thm}

\begin{proof}
	Consider the number $x=(\sqrt 2)^{\sqrt 2}$. There are two possibilities:
	\begin{enumerate}
	  \item $x$ is rational. Let $a=b=\sqrt 2$ and we're done.
	  \item $x$ is irrational. Let $a=x$ and $b=\sqrt 2$. Apply the usual exponential laws to see that
	  \[
	  	a^b=\bigl((\sqrt 2)^{\sqrt 2}\bigr)^{\sqrt 2}=(\sqrt 2)^{\sqrt 2\cdot\sqrt 2}= (\sqrt 2)^2=2
	  \]
	\end{enumerate}\vspace{-5pt}
	In either case, $a,b$ are irrational and $a^b$ is rational.
\end{proof}

The proof is very sneaky: it does not provide an explicit example and does not answer the begged question of whether $(\sqrt 2)^{\sqrt 2}$ is rational or irrational! In fact this number is irrational, though demonstrating such is massively harder.\footnote{If you're interested, look up the Gelfond--Schneider Theorem (1934), Hilbert's Seventh Problem, and what they say about \emph{algebraic} and \emph{transcendental numbers.} Such ideas are far beyond the level of this text.}
\bigbreak

We finish with a particularly famous example of a non-constructive existence proof. This argument dates back to Euclid's \emph{Elements} (300\BCE), the most influential textbook in mathematical history. As ever, we need a solid definition before trying to prove anything.

\begin{defn}{}{irreducible}
	An integer $\ge 2$ is \emph{prime} if the only positive integers it is divisible by are itself and 1.
\end{defn}

 The first few primes are $2,3,5,7,11,13,17,19,\ldots$ It follows, though it is not completely obvious, that every integer $\ge 2$ is either prime or a product of primes (\emph{composite}). In particular, every integer $\ge 2$ is divisible by at least one prime. We now state Euclid's result, and prove it by contradiction.

\begin{thm}{\emph{Elements}, Book IX, Prop.\,20}{euclidprime}
	There are infinitely many prime numbers.
\end{thm}

\begin{proof}
	Assume there are exactly $n$ primes $\lst p n$ and define the integer
	\[
		\Pi:=p_1\cdots p_n+1
	\]
	Certainly $\Pi$ is divisible by some prime $p_i$ (in our list by assumption!), as is the product $p_1\cdots p_n$. But then the difference
	\[
		1=\Pi-p_1\cdots p_n
	\]
	must be divisible by $p_i$, contradicting the fact that $p_i\ge 2$.
\end{proof}


\goodbreak


\begin{exercises}{}{}
	A self-test quiz and worked questions can be found \href{http://www.math.uci.edu/~ndonalds/math13/selftest/2-4-proofs2.html}{online}.

	\begin{enumerate}
		\item Prove or disprove:
		\begin{enumerate}
		  \item There exist integers $m$ and $n$ such that $2m-3n=15$. 
		  
			\item There exist integers $m$ and $n$ such that $6m-3n=11$. 
	  \end{enumerate}
	   
	   
	  \item Prove or disprove: There exists a line $L$ in the plane such that, for all points $A,B$ in the plane, we have that $A,B$ lie on $L$. 
	   
	   
		\item Prove: For every positive integer $n$, the integer $n^2+n+3$ is odd and greater than or equal to 5.
		
	
		\item Let $p$ be an odd integer. Prove that the equation $x^2-x-p=0$ has no \emph{integer} solutions.
		
		
		\item (Example \ref{ex:allexists}, cont.)\lstsp Prove or disprove: $\exists y<0$ such that $\forall x\ge 0,\ x^3<y^2$.
		
		
		\item Prove or disprove: $\sqrt 2-\sqrt 7$ is rational.
	   
	   
		\item Prove or disprove the following conjectures about real numbers $x,y$.
		\begin{enumerate}
		  \item If $3x+5y$ is irrational, then at least one of $x$ and $y$ is irrational.\par
		  (\emph{Be careful! This isn't a logical `and': what happens when you negate?})
		  
		  \item If $x$ and $y$ are rational, then $3x+4xy+2y$ is rational.
		  
		  \item If $x$ and $y$ are irrational, then $3x+4xy+2y$ is irrational.
		\end{enumerate}
	  
	  
		\item Prove by contradiction: if $x$ and $y$ are positive real numbers, then $\sqrt{x+y}\neq\sqrt{x}+\sqrt{y}$.\par
		How would you change things to make a \emph{contrapositive} argument?
	
	  
		\item Prove that between any two distinct rational numbers there exists another rational number.\label{ex:rationalsdenseinthemselves}
		
	
		\item Consider the proposition:
	  \begin{quote}
	      For any non-zero rational $r$ and any irrational number $t$, the number $rt$ is irrational.
	  \end{quote}
	  \begin{enumerate}
	      \item Translate this statement into logic using quantifiers and propositional functions.
	      \item Prove the statement.
	  \end{enumerate}
	
		
		\item\begin{enumerate}
		  \item An integer $n$ is \emph{not divisible by 3} if and only if $\exists k\in\Z$ for which $n=3k+1$ or $3k-1$.\par
		  Prove that if $n^2$ is divisible by 3, then $n$ is divisible by 3.
		  \item Prove that $\sqrt 3$ is irrational.
		  \item Prove that $\sqrt[3]{2}$ is irrational. (\emph{Hint: revisit Exercise \ref*{sec:proof}.\ref{exs:cuberoot2}}) 
		\end{enumerate}
	

	  \item (Recall Example \ref{ex:easypolynosolns})\lstsp You are given the following facts:
	  \begin{itemize}
	    \item All polynomials are continuous.
	    \item (Intermediate Value Theorem) If $f$ is continuous on the interval $[a,b]$ and $L$ lies between $f(a)$ and $f(b)$, then there is some $x$ in the interval $(a,b)$ for which $f(x)=L$.
	    \item If $f'(x)>0$ on an interval, then $f$ is an increasing function.
		\end{itemize}
		Use these facts to give formal proofs of two claims that should be familiar from calculus:
		\begin{enumerate}
		  \item $x^{17}+12x^3+13x+3=0$ has a solution $x$ in the interval $(-1,0)$.
		  \item $x^{17}+12x^3+13x+3=0$ has \emph{exactly one} real number solution $x$.
		\end{enumerate}
	
	
		\goodbreak
	
	
		\item\label{exs:archimedes} The real numbers satisfy the \emph{Archimedean property}:
		\begin{quote}
			For any $x,y>0$, there exists a positive integer $n$ such that $nx>y$.
		\end{quote}
		\begin{enumerate}
		  \item Use the Archimedean property to show that there are no positive real numbers which are less than $\frac 1n$ for all positive integers $n$. 
	  
	  	\item Consider the following `proof' of the fact that every real number is less than some positive integer:
	  	\begin{proof}
	    	Consider a real number $x$. For example, $x = 19.7$. Then $x<20$ and 20 is a positive integer. 
	 		\end{proof}
	  	What is wrong with this argument? Give a correct proof.
		
			\item Suppose $x<y$ are real numbers. Prove that there exists a positive integer $n$ for which $n(y-x)>1$.
			
			\item (Hard)\lstsp Prove: $\forall x,y\in\R$ with $x<y$, $\exists m,n\in\Z$ for which $nx<m<ny$. Hence conclude an extension of Exercise \ref{ex:rationalsdenseinthemselves}: between any two \emph{real} numbers there exists a rational number.
		\end{enumerate}
	
	
		\item Suppose $x,y,z\ge 0$ satisfy $x+y+z=1$. Use the AM--GM inequality (two-variable or full version) to answer the following.
		\begin{enumerate}
		  \item What is the largest possible value of $xyz$ and when does it occur?
		  \item (Hard)\lstsp Prove that $(1-x)(1-y)(1-z)\ge 8xyz$.
		\end{enumerate}
    
    
		\item\label{exs:amgm-full} (Hard)\lstsp We prove the full AM--GM inequality.
		\begin{enumerate}
		  \item When $n=3$, try mimicking our earlier approach by cubing the desired inequality. Why does this seem unwise?
		  \item Prove that $x\le e^{x-1}$ for all real numbers $x$, with equality if and only if $x=1$.\par
		  (\emph{Hint: Consider $f(x)=e^{x-1}-x$ and apply a derivative test from calculus})
		  \item Let $\mu=\frac{x_1+x_2+\cdots +x_n}{n}$ be the arithmetic mean. Apply part (b) to each expression $x=\frac{x_i}\mu$ to conclude that $x_1\cdots x_n\le \mu^n$ and hence complete the proof.
		\end{enumerate}

	\end{enumerate}

\end{exercises}