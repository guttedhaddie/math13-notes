\section{Logic and the Language of Proofs}\label{sec:logic}


\subsection{Propositions}\label{sec:prop}


In order to read and construct proofs, we need to start with the language in which they are written: \emph{logic.} This is to mathematics what grammar is to English.


\begin{defn}{}{}
	A \emph{proposition} or \emph{statement} is a sentence that is either true or false.
\end{defn}

\begin{examples}{}{}
\exstart $17-24=7$. \hfill \makebox[250pt][l]{2. \ $39^2$ is an odd integer.\hfill}\vspace{-3pt}
	\begin{enumerate}\setcounter{enumi}{2}\itemsep0pt
		\item The moon is made of cheese. \hfill \makebox[250pt][l]{4. \ Every cloud has a silver lining.\hfill} 
		\setcounter{enumi}{4} 
		\item God exists.
	\end{enumerate}
\end{examples}

For a proposition to make sense, we must agree on the meaning of each concept it contains. When people argue over propositions, in practice they are often disagreeing about \emph{definitions.} There are many concepts of God; we cannot begin to consider whether or not They exist until we agree \emph{which} concept is being discussed! This also illustrates that the truth status of a proposition \emph{need not be known} at the moment you state it; this is particularly common in mathematics.\footnote{More surprisingly, there are even propositions whose truth state is impossible to determine!}



\boldsubsubsection{Truth Tables and Combining Propositions}

To develop basic rules and terminology, it is helpful to consider \emph{abstract} propositions: $P,Q,R,\ldots$. Given a small number of propositions, all possible combinations of truth states may be easily represented in tabular format: in a \emph{truth table.} These are useful for defining new propositions.

\begin{defn}{}{}
	Let $P$ and $Q$ be propositions. The truth tables below define three new propositions modeled on the words \emph{and, or} and \emph{not.}\par
	\begin{minipage}[t]{0.5\linewidth}\vspace{0pt}
		\begin{itemize}
		  \item The \emph{conjunction} $\textcolor{red}{P\wedge Q}$ is read ``$P$ and $Q$.''
		  \item The \emph{disjunction} $\textcolor{Green}{P\vee Q}$ is read ``$P$ or $Q$.''
		  \item The \emph{negation} $\textcolor{blue}{\neg P}$ is read ``not $P$.'' %or \emph{denial} (NOT, $\neg$, $\sim$, $\cl{\phantom{P}}$)
		\end{itemize} 
	\end{minipage}
	\hfill
	\begin{minipage}[t]{0.49\linewidth}\vspace{-2pt}
		$\begin{array}{cc||c|c}
			P & Q & \textcolor{red}{P\wedge Q} & \textcolor{Green}{P\vee Q}\\\hline
			T & T & \textcolor{red}{T} & \textcolor{Green}{T}\\
			T & F & \textcolor{red}{F} & \textcolor{Green}{T}\\
			F & T & \textcolor{red}{F} & \textcolor{Green}{T}\\
			F & F & \textcolor{red}{F} & \textcolor{Green}{F}
		\end{array}
		\qquad\qquad
		\begin{array}[b]{c||c}
			P & \textcolor{blue}{\neg P}\\\hline
			T & \textcolor{blue}{F}\\
			F & \textcolor{blue}{T}
		\end{array}$
	\end{minipage}
\end{defn}

The letters T/F stand for \emph{true/false.} E.g., the second line of the first table says that if $P$ is true and $Q$ is false, then the proposition ``$P$ and $Q$" is \textcolor{red}{false}, while ``$P$ or `$Q$'' is \textcolor{Green}{true.} %Hopefully this gels with your intuitive understanding.

\begin{example}{}{logiccolor}
	Suppose $P$ and $Q$ are the following propositions:
	\begin{quote}
		$P$: ``I like purple.''\qquad\qquad $Q$: ``I like chartreuse.''
	\end{quote}
	We form the new propositions described in the definition:
	\begin{quote}
		$P\wedge Q$: ``I like purple and chartreuse.''\qquad \qquad
		$P\vee Q$: ``I like purple or chartreuse.''\smallbreak
		$\neg P$: ``I do not like purple.''
	\end{quote}
	It is typical to modify phrasing to aid readability: ``Not, I like purple'' just sounds weird! Note also that or is \emph{inclusive} in logic: if ``I like purple or chartreuse'' is true, then you might like \emph{both}!\medbreak
	
	Let's continue by adding a third proposition:
	\begin{quote}
		$R$: ``It's 9am.''
	\end{quote}
	What proposition is represented by the following English sentence?
	\begin{quote}
		``I like purple and I like chartreuse or it's 9am.''
	\end{quote}
	Is it $P\wedge(Q\vee R)$ or is it $(P\wedge Q)\vee R$? Without brackets, the sentence is unclear; the moral is that English is terrible at logic! Indeed, as the truth table shows, the two logical expressions \textcolor{red}{really do mean different things!}\vspace{-1pt}
	 \[
	 \begin{array}{ccc||cc||cc}
			P & Q & R & Q\vee R & P\wedge (Q\vee R) & P\wedge Q & (P \wedge Q)\vee R\\\hline
			T & T & T & T & T & T & T\\
			T & T & F & T & T & T & T\\
			T & F & T & T & T & F & T\\
			T & F & F & F & F & F & F\\
			F & T & T & T & \textcolor{red}{F} & F & \textcolor{red}{T}\\
			F & T & F & T & F & F & F\\
			F & F & T & T & \textcolor{red}{F} & F & \textcolor{red}{T}\\
			F & F & F & F & F & F & F
		\end{array}
	 \]
\end{example}


\boldsubsubsection{Conditional and Biconditional Connectives}

Of critical importance is the ability to have one proposition lead to another.

\begin{defn}[lower separated=false, sidebyside, sidebyside align=top seam, sidebyside gap=0pt, righthand width=0.37\linewidth]{}{implies}
	Given propositions $P,Q$, the \emph{conditional} ($\Longrightarrow$) and \emph{biconditional} ($\Longleftrightarrow$) \emph{connectives} define new propositions as described in the truth table.\smallbreak
	For the proposition $P\Longrightarrow Q$, we call $P$ the \emph{hypothesis} and $Q$ the \emph{conclusion.}
	\tcblower
	\flushright$\begin{array}{cc||c|c}
	P & Q & P\implies Q & P\iff Q\\\hline
	T & T & T & T\\
	T & F & F & F\\
	F & T & T & F\\
	F & F & T & T
	\end{array}$
\end{defn}


%Remember that the expressions $P\Longrightarrow Q$ and $P\Longleftrightarrow Q$ are themselves \emph{propositions}: sentences which are either true or false. 
Connective propositions can be read and written in many different ways:
\begin{quote}
	\def\arraystretch{1.05}
	\begin{tabular}{@{}cc|c}
		\multicolumn{2}{c|}{$P\implies Q$} & $P\iff Q$\\\hline
		$P$ implies $Q$ & $P$ therefore $Q$ & $P$ if and only if $Q$\\
		If $P$, then $Q$ & $Q$ follows from $P$ & $P$ iff $Q$\\
		$P$ only if $Q$ & $Q$ if $P$ & $P$ and $Q$ are (logically) equivalent\\
		$P$ is sufficient for $Q$ & $Q$ is necessary for $P$ & $P$ is necessary and sufficient for $Q$
	\end{tabular}
\end{quote}


\begin{example}{}{}
	The following sentences express, in English, the same conditional $P\implies Q$.\vspace{-1pt}
	\begin{itemize}\itemsep1pt
		\item If you are born in Rome, then you are Italian. 
		\item You are Italian if you are born in Rome. 
		\item You are born in Rome only if you are Italian. 
		\item Being born in Rome is sufficient for being Italian. 
		\item Being Italian is necessary for being born in Rome.\vspace{-1pt} 
	\end{itemize}
	Are you comfortable with what the propositions $P$ and $Q$ are here?
\end{example}

%\goodbreak

Connectives are central to mathematics for many reasons. In particular:
\begin{enumerate}
  \item The vast majority of theorems we'll encounter may be written as a connective $P\Longrightarrow Q$. For instance, revisit Theorem \ref{thm:sumeven} and the discussion that follows:
  \begin{quote}
  	If $x$ and $y$ are even integers, then $x+y$ is even.
  \end{quote}
  Identifying the hypothesis and conclusion is essential if you want to understand a theorem!
  \item Simple proofs typically involve chaining a sequence of connectives:
  \[P\implies P_2\implies \cdots \implies P_n\implies Q\]
\end{enumerate}
We'll revisit these ideas in Section \ref{sec:proof}, and repeatedly throughout the course.\bigbreak


While the biconditional should be easy to remember, %($P\Longleftrightarrow Q$ is true precisely when $P$ and $Q$ have identical truth states)
it is harder to make sense of the conditional connective. Short of simply memorizing the truth table, here are two examples that might help.

\begin{examples}{}{condmeaning}
	\exstart Suppose your professor says, ``If the class earns a B average on the midterm, then I'll bring doughnuts.'' The only situation in which the teacher will have lied is if the class earns a B average but she fails to provide doughnuts.
	\begin{enumerate}\setcounter{enumi}{1}
	  \item ($F\Longrightarrow T$ really can be true!) Let $P$ be the proposition ``$7=3$'' and $Q$ be ``$0=0$.'' Since multiplication of both sides of an equation by zero is algebraically valid, we see that
	%   \[7=3\implies 0\cdot 7=0\cdot 3\implies 0=0\]
	  \begin{align*}
			7=3\implies\ &0\cdot 7=0\cdot 3\tag*{(If $7=3$, then 0 times 7 equals 0 times 3)}\\
			\implies\ &0=0\tag*{(then 0 equals 0)}
		\end{align*}
	  This argument is perfectly correct: the \emph{implication} $P\Longrightarrow Q$ is \emph{true.} It (rightly!) makes us uncomfortable because the hypothesis is \emph{false.}\par
	  If we instead add 1 to each side of $7=3$, we'd obtain a example where $F\Longrightarrow F$ is true.
	\end{enumerate}
\end{examples}


\boldsubsubsection{Tautologies and Contradictions}

\begin{defn}{}{tautology}
	A \emph{tautology} is a logical expression that is always true, regardless of what the component statements might be.\smallbreak
	A \emph{contradiction} is a logical expression that is always false.
\end{defn}

%The easiest way to detect these is to construct a truth table.

\begin{examples}{}{}
\exstart $P\wedge(\neg P)$ is a contradiction.
	
\begin{enumerate}\setcounter{enumi}{1}
  \begin{minipage}[t]{0.65\linewidth}\vspace{-8pt}
  	\item[] Regardless of the proposition $P$, it cannot be true at the same time as its negation!
  \end{minipage}
  \hfill
  \begin{minipage}[t]{0.29\linewidth}\vspace{-27pt}
	$\begin{array}{cc|c}
	P & \neg P & P\wedge(\neg P)\\\hline
	T & F & F\\
	F & T & F
	\end{array}$
  \end{minipage}\par
  
	\item $\bigl(P\wedge(P\implies Q)\bigr)\Longrightarrow Q$ is a tautology.% This is essentially how we understand a direct proof: if $P$ is true and we have a correct argument $P\implies Q$, then $Q$ must also be true.
	\[\begin{array}{cc||c|c||c}
	P & Q & P\implies Q & P\wedge(P\implies Q) & \bigl(P\wedge(P\implies Q)\bigr)\implies Q\\\hline
	T & T & T & T& T\\
	T & F & F & F& T\\
	F & T & T & F& T\\
	F & F & T & F& T
	\end{array}\]
	%\item $(P\wedge \neg Q\implies F)\iff (P\implies Q)$ is a tautology. This tautology is the basis for \emph{proof by contradiction,} as we'll see in the next section. The expression $P\wedge \neg Q\implies F$ can be thought of as saying that $P\wedge\neg Q$ implies a contradiction.
	\end{enumerate}
\end{examples}

\goodbreak


\boldsubsubsection{The Converse and Contrapositive}

%The following constructions are used regularly; it is vitally important to understand the distinction.

\begin{defn}{}{contra}
	The \emph{converse} of $P\Longrightarrow Q$ is the reversed implication $Q\Longrightarrow P$.\smallbreak
	The \emph{contrapositive} of $P\Longrightarrow Q$ is the implication $\neg Q\Longrightarrow\neg P$.
\end{defn}

It is vital to understand the distinction between these. In general, the truth status of the converse bears no relation to that of the original, though the contrapositive is much better behaved.

\begin{thm}{}{contrapos}
	The contrapositive of an implication is logically equivalent to the original.
\end{thm}

\begin{proof}
	Compute the truth table and observe that the third and sixth columns are identical:\footnotemark
	\begin{gather*}
		\begin{array}{cc|c||cc|c}
			P & Q & P\Longrightarrow Q & \neg Q & \neg P & \neg Q\Longrightarrow\neg P\\\hline
			T & T & T & F & F & T\\
			T & F & F & T & F & F\\
			F & T & T & F & T & T\\
			F & F & T & T & T & T
		\end{array}\\[-15pt]
		\phantom{bob}\tag*{\qedhere}
	\end{gather*}
	%Otherwise said, $(P\Longrightarrow Q)\Longleftrightarrow (\neg Q\Longrightarrow\neg P)$ is a tautology.
\end{proof}

\footnotetext{Otherwise said, $(P\Longrightarrow Q)\Longleftrightarrow (\neg Q\Longrightarrow\neg P)$ is a tautology.}

\vspace{-5pt}


\begin{example}{}{}
	Let $P$ and $Q$ be the following statements:
	\begin{quote}
	  $P$: \ ``Claudia is holding a peach.''\qquad\qquad
	  $Q$: \ ``Claudia is holding a piece of fruit.''
	\end{quote}
	Since a peach is indeed a piece of fruit, the proposition $P\Longrightarrow Q$ is \emph{true}:
	\begin{quote}
		$P\Longrightarrow Q$: \ ``If Claudia is holding a peach, then she is holding a piece of fruit.''
	\end{quote}
	The \emph{converse} of $P\Longrightarrow Q$ is the sentence
	\begin{quote}
	  $Q\Longrightarrow P$: \ ``If Claudia is holding a piece of fruit, then she is holding a peach.''
	\end{quote}
	This is palpably false: Claudia could be holding an apple! However, in accordance with Theorem \ref{thm:contrapos}, the \emph{contrapositive} is \emph{true}:
	\begin{quote}
	  $\neg Q\Longrightarrow \neg P$: ``If Claudia is \emph{not} holding any fruit, then she is \emph{not} holding a peach.''
	\end{quote} 
\end{example}

 
\boldsubsubsection{Negating Logical Expressions}

Mathematics often requires us to negate propositions. What would you suspect to be the negation of a conditional $P\Longrightarrow Q$? Is it enough to say ``$P$ doesn't imply $Q$"? But what does this mean? 

\begin{minipage}[t]{0.64\linewidth}\vspace{-2pt}
We again rely on a truth table: to get the last column, recall that negation simply swaps $T$ and $F$. Can we write this column in another way? Since there is only a single $T$ in the final column, we see that we've proved the following.
\end{minipage}
\hfill
\begin{minipage}[t]{0.35\linewidth}\vspace{-5pt}
\flushright $\begin{array}{cc|c|c}
		P & Q & P\Longrightarrow Q & \neg(P\Longrightarrow Q)\\\hline
		T & T & T & F\\
		T & F & F & T\\
		F & T & T & F\\
		F & F & T & F
	\end{array}$
\end{minipage}



\begin{thm}{}{negconditional}
	$\neg(P\Longrightarrow Q)$ is logically equivalent to $P\wedge\neg Q$ \ (``$P$ and not $Q$").
\end{thm}

\vspace{-5pt}

\goodbreak

\begin{example}{}{}
	Consider the implication
	\begin{quote}
	  It's the morning therefore I'll have coffee.
	\end{quote}
	Hopefully its negation is clear:
	\begin{quote}
	  It's the morning \emph{and} I \emph{won't} have coffee.
	\end{quote}
	As in Example \ref{ex:condmeaning}, it might help to think about what it means for the original statement to be \emph{false}.
\end{example}

\begin{tcolorbox}
{\bf \textcolor{red}{Warning!}} The negation of $P\Longrightarrow Q$ is \emph{not a conditional.} In particular it is \emph{neither} of the following:
\begin{quote}
  The converse $Q\Longrightarrow P$.\smallbreak
  The contrapositive of the converse $\neg P\Longrightarrow\neg Q$. 
\end{quote}
If you are unsure about this, write down the truth tables and compare.
\end{tcolorbox}

\bigbreak


Our final results in basic logic also involve negations; they are named for Augustus de Morgan, a famous 19\th{} century logician.

\begin{thm}{de Morgan's laws}{demorgan}
	Let $P$ and $Q$ be propositions.
	\begin{enumerate}\itemsep0pt
	  \item $\neg(P\wedge Q)$ is logically equivalent to $\neg P\vee\neg Q$
	  \item $\neg(P\vee Q)$ is logically equivalent to $\neg P\wedge\neg Q$
	\end{enumerate}
\end{thm}

\begin{proof}
	For the first law, observe that the fourth and seventh columns of the truth table are identical.
	\[
		\begin{array}[t]{cc||cc||cc||c}
			P & Q & P\wedge Q & \neg(P\wedge Q) & \neg P & \neg Q & \neg P\vee\neg Q\\\hline
			T & T & T & F & F & F & F\\
			T & F & F & T & F & T & T\\
			F & T & F & T & T & F & T\\
			F & F & F & T & T & T & T
		\end{array}
	\]
	The second law is an exercise.
\end{proof}

\begin{example}{}{}
	Consider the sentence:\par
	\begin{minipage}[t]{0.6\linewidth}\vspace{-1pt}
	\begin{quote}
		I rode the subway \emph{and} I had coffee.
	\end{quote}
	To negate this using de Morgan's first law, we might write:
	\begin{quote}
		I \emph{didn't} ride the subway \emph{or} I \emph{didn't} have coffee.
	\end{quote}
	\end{minipage}
	\hfill
	\begin{minipage}[t]{0.39\linewidth}\vspace{-20pt}
		\flushright	
		\begin{tabular}{c|c||c}
			Subway&Coffee&Su and Co\\\hline\hline
			T & T & T\\
			\textcolor{blue}{T} & \textcolor{blue}{F} & \textcolor{blue}{F}\\
			\textcolor{blue}{F} & \textcolor{blue}{T} & \textcolor{blue}{F}\\
			\textcolor{blue}{F} & \textcolor{blue}{F} &\textcolor{blue}{F}
		\end{tabular}
	\end{minipage}\bigbreak
	
	This feels awkward in English because the negation encompasses \textcolor{blue}{three distinct possibilities}. Note how the logical (inclusive) use of \emph{or} includes the last row of the truth table: the possibility that you neither rode the subway nor had coffee.
\end{example}

As with Example \ref{ex:logiccolor}, this is another advert for the use of logic: English simply isn't very helpful for precisely stating complex logical statements.


\begin{aside}{}{}
\boldinline{Aside: Algebraic Logic}\phantomsection\label{pg:asidelogicalgebra}

%Similarly to de Morgan's laws, 
We can use truth tables to establish other laws of basic logic, for instance:
\[
	\def\arraystretch{1.2}
	\begin{array}{@{}lll@{}}
	\text{Double negation} & \neg(\neg P)\iff P &\\
	\text{Commutativity} & P\wedge Q\iff Q\wedge P & P\vee Q\iff Q\vee P\\
	\text{Associativity} & (P\wedge Q)\wedge R\iff P\wedge(Q\wedge R) & (P\vee Q)\vee R\iff P\vee(Q\vee R)\\
	\text{Distributivity}&(P\wedge Q)\vee R\iff (P\vee R)\wedge (Q\vee R) & (P\vee Q)\wedge R\iff (P\wedge R)\vee (Q\wedge R)
	\end{array}
\]
To make things more algebraic, we've replaced ``is logically equivalent to" with a biconditional.\footnotemark{}\smallbreak

Armed with such laws, one can often suitably manipulate logical expressions without laboriously creating truth tables. This is not the focus of this course, though you might find it fun!\smallbreak

For this course, it is probably not worth memorizing these laws. Your intuitive understanding of \emph{and, or} and \emph{not} mean you'll likely apply the laws correctly whenever necessary.
\end{aside}

\footnotetext{Stating the laws in this fashion is to assert that each expression is a tautology (Definition \ref{defn:tautology}). For instance, to claim that ``$\neg(\neg P)$ is logically equivalent to $P$'' is to assert that $\neg(\neg P)\iff P$ is a tautology.}

\goodbreak

\begin{exercises}{}{}
	A reading quiz and several questions with linked video solutions can be found \href{http://www.math.uci.edu/~ndonalds/math13/selftest/2-1-props.html}{online}.


	\begin{enumerate}
	  \item Express each statement in the form, ``If $\dots$, then $\dots$'' There are many possible correct answers.
			\begin{enumerate}
		  	\item You must eat your dinner if you want to grow.
		  	\item Being a multiple of 12 is a sufficient condition for a number to be even.
		  	\item It is necessary for you to pass your exams in order for you to obtain a degree. 
		  	\item A triangle is equilateral only if all its sides have the same length.
			\end{enumerate}
		
	
	  \item Suppose ``$x$ is an even integer'' and ``$y$ is an irrational number'' are true statements, and that ``$z\geq 3$'' is a false statement. Which of the following are true?\par
	  (\emph{Hint: Label each statement and think about each using connectives})
			\begin{enumerate}
		  	\item If $x$ is an even integer, then $z\geq 3$.
		  	\item If $z\geq 3$, then $y$ is an irrational number.
		  	\item If $z\geq 3$ or $x$ is an even integer, then $y$ is an irrational number.
		  	\item If $y$ is an irrational number and $x$ is an even integer, then $z\geq 3$.
			\end{enumerate}


	  \item Orange County is considering two competing transport plans: widening the 405 freeway and constructing light rail down its median. A local politician is asked, ``Would you like to see the 405 widened or would you like to see light rail?'' The politician wants to sound positive, but to avoid being tied to one project. What is their response?\par
	  (\emph{Hint: Think about how the word `OR' is used in logic})
  
  
	  \item Consider the proposition: ``If the integer $m$ is greater than 3, then $2m$ is not prime.''
	  \begin{enumerate}
	    \item Rewrite the proposition using the word `necessary.'
	 		\item Rewrite the proposition using the word  `sufficient.'
	 		\item Write the negation, converse and contrapositive of the proposition. 
	  \end{enumerate}
	  

  \item Suppose the following sentence is true: ``If Amy likes art, then no-one likes history." What, if anything, can we conclude if we discover that someone likes history.
  
  
  \goodbreak
	  
	  
	  \item Construct the truth tables for the propositions $P\vee(Q\wedge R)$ and $(P\vee Q)\wedge R$. Are they the same?
   
   
  \goodbreak
  
	\item Use truth tables to establish the following laws of logic:
	\begin{enumerate}
	  \item Double negation: \lstsp $\neg(\neg P)\iff P$.
	  \item Idempotent law: \lstsp $P\wedge P\iff P$.
	  \item Absorption law: \lstsp $P\wedge(P\vee Q)\iff P$.
	  \item Distributive law: \lstsp $(P\wedge Q)\vee R\iff (P\vee R)\wedge(Q\vee R)$.
	\end{enumerate}
  
  
  \item\begin{enumerate}
    \item Decide whether $(P\wedge \neg P) \Longrightarrow Q$ is a tautology, a contradiction, or neither.
    \item Explain why $\neg P \vee \neg Q$ is logically equivalent to $P \Longrightarrow (P \wedge\neg Q)$.
    \item Prove: $\bigl((P\wedge \neg Q)\Longrightarrow F\bigr)\Longleftrightarrow  (P\Longrightarrow Q)$ is a tautology. Here $F$ represents a \emph{contradiction.}
  \end{enumerate}
  
  
  \item\begin{enumerate}
    \item Prove that the expressions $(P\Longrightarrow Q)\wedge (Q\Longrightarrow P)$ and $P\Longleftrightarrow Q$ are logically equivalent.
    \item Prove that $\bigl((P\Longrightarrow Q)\wedge (Q\Longrightarrow R)\bigr)\Longrightarrow \bigl(P\Longrightarrow R\bigr)$ is a tautology.
  \end{enumerate}
  Why do these make intuitive sense?

  
  \item Use logical algebra (e.g., page \pageref{pg:asidelogicalgebra}) to show that $\bigl((P\vee Q)\wedge \neg P\bigr)\wedge\neg Q$ is a contradiction.

	
  \item Do there exist propositions $P,Q$ for which both $P\Longrightarrow Q$ and its converse are \emph{false}? Explain.
  
    
	\item Your friend insists that the negation of the sentence ``Mark and Mary have the same height'' is ``Mark or Mary do not have the same height.'' What is the correct negation? Where did your friend go wrong?

		
	\item Suppose that the following statements are \emph{true}:
  \begin{enumerate}
    \item Every octagon is magical. 
    \item If a polygon is not a rectangle, then is it not a square. 
    \item A polygon is a square, if it is magical.
  \end{enumerate}
  Is it true that ``Octagons are rectangles"? Explain your answer.\par
  (\emph{Hint: try rewriting each of the statements as an implication})
  

	\item The connective $\downarrow$ (the \emph{Quine dagger}, \emph{NOR}) is defined by the truth table:\par
	\begin{enumerate}
	\begin{minipage}[t]{0.75\linewidth}\vspace{0pt}
      \item Prove that $P\downarrow Q$ is logically equivalent to $\neg (P\vee Q)$. 
      \item Find a logical expression built using only $P$ and the connective $\downarrow$ which is logically equivalent to $\neg P$.
	\end{minipage}
	\hfill
	\begin{minipage}[t]{0.24\linewidth}\vspace{-25pt}
	\flushright$  \begin{array}{cc|c}
		P & Q & P \downarrow Q\\\hline
		T & T & F\\
		T & F & F\\
		F & T & F\\
		F & F & T
	\end{array}$
	\end{minipage}
	
		\item Find an expression built using only $P$, $Q$ and $\downarrow$ which is logically equivalent to $P\wedge Q$.
  \end{enumerate}
  
\end{enumerate}

\end{exercises}

\clearpage



\subsection{Propositional Functions \& Quantifiers}\label{sec:quant}

The majority of mathematical propositions are more complicated that those seen in Section \ref{sec:prop}. In particular, they typically involve \emph{variables,} for instance
\begin{quote}
	``$x$ is an integer greater than 5.''
\end{quote}

\begin{defn}{}{}
	A \emph{propositional function} is a family of propositions which depend on one or more variables. The collection of permitted variables is the \emph{domain.}
\end{defn}

If $P$ is a propositional function depending on a single variable $x$, then for each object $a$ in the domain, $P(a)$ is a proposition. Typically $P(x)$ is true for some $x$ and false for others.

\begin{example}{}{easyquantprop}
	Consider the propositional function $P(x)$: ``$x^2>4$" with domain the real numbers. Plainly $P(1)$ is false (``$1^2>4$'') and $P(6)$ is true (''$6^2>4$'').
% 	\begin{itemize}
% 	  \item $P(-1)$ is false: it is the sentence ``$(-1)^2>4$.''
% 	  \item $P(6)$ is true: it is the sentence ``$6^2>4$.''
% 	\end{itemize}
\end{example}

%\boldsubsubsection{Quantified Propositions}

In mathematics, propositional functions are often \emph{quantified.} English contains various quantifiers (\emph{all, some, many, few, several,} etc.), but in mathematics we are primarily concerned with just two.


\begin{defn}{}{quant}
	The \emph{universal quantifier} $\forall$ is read `for all'. The \emph{existential quantifier} $\exists$ is read `there exists.' Given a propositional function $P(x)$, we define two new \emph{quantified propositions}:
	\begin{itemize}
	  \item ``$\forall x, P(x)$'' is true if and only if $P(x)$ is true for \emph{every} $x$ in its domain.
	  \item ``$\exists x, P(x)$'' is true if and only if $P(x)$ is true for \emph{at least one} $x$ in its domain.
	\end{itemize}
	It is common to describe the domain when quantifying propositions by including a descriptor after the quantifier (\emph{bounding the quantifier}---see below).
\end{defn}

As with connectives, there are many ways to express quantified propositions both mathematically and in English. The use of symbolic quantifiers involves a trade-off: compact statements can improve clarity, but they are harder to read for the uninitiated, so consider your audience! While it is your choice whether to employ symbolic quantifiers in your own \emph{writing,} it is essential that you know how to \emph{read/recognize} them and that you can \emph{translate} between various incarnations.

\begin{example*}{\ref{ex:easyquantprop} cont.}{}
	To gain some practice with bounded quantifiers, we introduce the notation $x\in\R$: this means that $x$ is a real number.%\footnotemark{} 
	\begin{itemize}
	  \item ``$\forall x\in\R,\ x^2>4$'' might be read, ``The square of every real number is greater than 4.''\par
	  The quantified expression is \emph{false} since $1^2>4$ is false: we call $x=1$ a \emph{counter-example.}
	  \item ``$\exists x\in\R,\ x^2>4$'' might be read, ``There is a real number whose square is greater than 4.''\par
	  The quantified expression is \emph{true} since $6^2>4$ (is true): we call $x=6$ an \emph{example.}
	\end{itemize}
\end{example*}

Due to their importance, it is worth defining these last concepts formally.

\begin{defn}{}{}
	An \emph{example} of $\exists x, P(x)$ is an element $x_0$ in the domain of $P$ for which $P(x_0)$ is \emph{true.}\smallbreak
	A \emph{counter-example} to $\forall x, P(x)$ is an element $x_0$ in the domain of $P$ for which $P(x_0)$ is \emph{false.}
\end{defn}


\goodbreak


% \boldinline{Must I always use all these symbols?}
% 
% Absolutely not! Though you do have to be able to \emph{read} and \emph{understand} them. Remember that the purpose of writing mathematics is to \emph{convince the reader}; your chosen presentation style will have a huge impact on whether you succeed! Here are three presentations based on the previous example:
% \begin{quote}\def\arraystretch{1.2}
% 	\begin{tabular}{l|l}
% 		Pure English & There is a real number whose square is greater than four\\\hline
% 		Pure Logic & $\exists x\in\R, x^2>4$\\\hline
% 		Hybrid & $\exists x\in\R$ such that $x^2>4$
% 	\end{tabular}
% \end{quote}
% There are benefits and drawbacks to all three approaches; each might be entirely appropriate depending on the audiences. In these notes we'll typically follow a hybrid approach, aiming to replace words with symbols when it improves clarity while preserving readability.



\boldinline{Universal Quantifiers and Connectives: Hidden Quantifiers}

Universally quantified statements are interchangeable with implications. Given a propositional function $Q(x)$, let $P(x)$ be the proposition ``$x$ lies in the domain of $Q$." Then
\[
	\tcbhighmath{\text{$\textcolor{blue}{\forall x,Q(x)}$ is logically equivalent to $\textcolor{Magenta}{P(x)\Longrightarrow Q(x)}$}}
\]
\textcolor{Magenta}{Connectives containing variables} are therefore assumed to be \textcolor{blue}{universal}. When written as a connective, the universal quantifier is typically \emph{hidden.}\footnote{By contrast, the existential quantifier is never hidden: it is always explicitly written as a symbol ($\exists$), or as a phrase in English (\emph{there is, there exists, some, at least one,} etc.).}

\begin{examples}{}{oddsquared}
	\exstart The universal statement ``Every cat is neurotic,'' may also be written
	\begin{enumerate}\setcounter{enumi}{1}
	  \item[]\begin{quote}
			If $x$ is a cat, then $x$ is neurotic.
		\end{quote} 
		
		\item Revisiting Example \ref{ex:easyquantprop}, we could rewrite ``$\forall x\in\R, x^2>4$'' as a connective
		\[
			x\in\R\Longrightarrow x^2>4 \tag{If $x$ is a real number, then $x^2>4$}
		\]
		
	  \item\label{ex:oddsquared2} The following three sentences have identical meaning:
	  \begin{quote}
	  	The square of an odd integer is odd.\qquad $\forall n$ odd, $n^2$ is odd.\qquad $n$ odd $\Longrightarrow n^2$ odd.
	  \end{quote}
	  In only one of the sentences is the universal quantifier explicit. For even more variety, the third sentence can also be viewed as a universal statement about all \emph{integers}; including the \textcolor{red}{hidden quantifier} in this case results in
		\[
			\textcolor{red}{\forall n\in\Z},\ n\text{ odd} \implies n^2 \text{ odd.}
		\]
		where the symbol $\Z$ represents the (set of) integers.
	\end{enumerate}
\end{examples}

We've already seen that \emph{disproving} a universal statement requires only that we supply a \emph{counter-example.} While such might require some effort to find, often the resulting argument is very simple. By contrast, \emph{proving} a universal statement is the same as proving a connective, an activity that is typically much more involved. We therefore largely postpone this to the next section. Regardless, a simple proof of our \emph{oddness} claim should be easy to follow.

\begin{proof}[Proof of Example \ref*{ex:oddsquared}.\ref{ex:oddsquared2}]
	If an integer $n$ is odd, then it may be written in the form $n=2k+1$ for some integer $k$. But then
	\[
		n^2 =(2k+1)^2 =4k^2+4k+1 =2(2k^2+2k)+1
		\]
	is plainly also odd.
\end{proof}

Similarly, \emph{proving} an existential statement (by providing an \emph{example}) is typically more straightforward than \emph{disproving} such. To understand this duality, we need to understand how to \emph{negate} quantified propositions.

\goodbreak


\boldsubsubsection{Negating Quantified Propositions}

To negate a proposition, we consider what it means for it to be \emph{false.} We already understand what this means for a universal proposition:
\begin{quote}
	``$\forall x,P(x)$'' is false if and only if \emph{there exists a counter-example.}
\end{quote}
The negation of a universal statement is \emph{existentially quantified}:
\begin{quote}
	The negation of ``$P(x)$ is \emph{always} true'' is ``$P(x)$ is \emph{sometimes} false.''
\end{quote}
Repeating this with $\neg P(x)$ results in a related observation:
\begin{quote}
	The negation of ``$P(x)$ is \emph{always} false" is ``$P(x)$ is \emph{sometimes} true.''
\end{quote}
%In summary:

\begin{thm}{}{negquant}
	For any propositional function $P(x)$:
	\begin{enumerate}
	  \item $\neg\bigl(\forall x, P(x)\bigr)$ is logically equivalent to $\exists x, \neg P(x)$.
	  \item $\neg\bigl(\exists x, P(x)\bigr)$ is logically equivalent to $\forall x, \neg P(x)$.
	\end{enumerate}
\end{thm}


\begin{examples}{}{}
	\exstart  ``Everyone owns a bicycle," has negation, ``Someone does not own a bicycle.'' 
		It is somewhat ugly, but we could write this symbolically:
		\[
			\neg\bigl(\forall\text{ people }x,\ x\text{ owns a bicycle}\bigr)\iff \exists\text{ a person $x$ such that $x$ does not own a bicycle}
		\]
		
	\begin{enumerate}\setcounter{enumi}{1}
		\item The quantified proposition\footnotemark{} ``$\exists x>0$ such that $\sin x=4$,''
		has the form $\exists x,\,P(x)$. Its negation has the form $\forall x,\ \neg P(x)$. Explicitly:
		\[\forall x>0,\ \sin x\neq 4\]
		Since the sine function satisfies the inequalities $-1\le\sin x\le 1$, the original proposition is \emph{false} and its negation \emph{true.}
		\[
			\tcbhighmath{\text{\textcolor{red}{Warning!}\lstsp \emph{Never} negate a quantifier's \textcolor{Magenta}{bounds}: $\forall x\,\textcolor{Magenta}{\le 0}$\ldots{} is completely wrong!}}
		\]
		
		\item Be especially careful when negating connectives: after negation, a \textcolor{red}{hidden quantifier $\forall x$} becomes \emph{explicit.}
		\[
			\tcbhighmath{\neg\bigl(P(x)\Longrightarrow Q(x)\bigr)\ \text{ is logically equivalent to }\ \textcolor{red}{\exists x}, P(x) \wedge\neg Q(x)}
		\]
		\begin{enumerate}
		  \item (Example \ref*{ex:oddsquared}.\ref{ex:oddsquared2})\lstsp The negation of ``$n$ odd $\Longrightarrow n^2$ odd''is the (false) claim
			\[
				\textcolor{red}{\exists n\in\Z}\text{ with $n$ odd \emph{and} $n^2$ even.}
			\]
			
			\item (Example \ref{ex:easyquantprop})\lstsp The negation of the false claim ``$x\in\R\Longrightarrow x^2>4$'' is the true assertion
			\[
				\textcolor{red}{\exists n\in\R}\text{ for which }x^2\le 4
			\]
			\vspace{-28pt}
		\end{enumerate}
	\end{enumerate}
\end{examples}

\footnotetext{``$\exists x>0$'' indicates that the domain of the proposition ``$\sin x=4$'' is the \emph{positive} real numbers.}

\goodbreak


\boldsubsubsection{Multiple Quantifiers}

A propositional function can have several variables, each of which may be quantified.

\begin{examples}{}{multiplequant}
	\exstart The quantified proposition
	\[
		\forall x>0,\exists y>0\text{ such that }xy=4
	\]
	\begin{enumerate}\setcounter{enumi}{1}
	  \item[]might be read, ``Given any positive number, there is another such that their \emph{product} is four.''	Hopefully you believe that this is \emph{true}! Here is a simple argument which comes from viewing it as an implication, ``If $x>0$, then $\exists y>0$ such that $xy=4$.''
	\begin{quote}
		\begin{proof}
			Suppose we are given $x>0$. Let $y=\frac 4x$, then $xy=4$, as required.
		\end{proof}
	\end{quote}Being clear about \emph{domains} is critical. Suppose we modify the original proposition:
		\[
			\forall x\in\R,\exists y\in\R\text{ such that }xy=4 \tag{$\dag$}
		\]
		Our proof now fails! The new statement ($\dag$) is \emph{false}: indeed $x=0$ provides a \emph{counter-example.}
	\begin{quote}
		\begin{proof}[Disproof]
			Let $x=0$. Since $xy=0$ for any real number $y$, we cannot have $xy=4$.
		\end{proof}
	\end{quote}
	  \item[] Alternatively, we could \emph{negate} ($\dag$): following Theorem \ref{thm:negquant}, we switch the symbols $\forall\leftrightarrow\exists$ and negate the final proposition,\footnotemark{}
		\[
			\neg\bigl(\forall x\in\R,\exists y\in\R, xy=4\bigr) \iff \exists x\in\R, \forall y\in\R,\ xy\neq 4 \tag{$\neg (\dag)$}
		\]
		Our \emph{disproof} of ($\dag$) is really a \emph{proof} of the negation: we provided the \emph{example} $x=0$, thus demonstrating the truth of a $\exists$-statement. Since the negation is true, the original $(\dag)$ is false.
		
		
	  \item\label{ex:multiplequant2} \textcolor{red}{Order of quantifiers matters!}\lstsp The meaning of a sentence will likely change if we alter the order of quantification. This might also change the truth state of a proposition.
	  \begin{enumerate}
	  	\item $\forall x\in\R,\exists y\in\R,x^2<y$
		\end{enumerate}
		\begin{quote}
			\begin{proof}
				Suppose a real number $x$ is given. Let $y=x^2+1$, then $x^2<y$, as required.
			\end{proof}
		\end{quote}
	  \begin{enumerate}\setcounter{enumii}{1}
	    \item[]We proved this by viewing it as an implication, ``If $x\in\R$, then $\exists y\in\R,x^2<y$.''
	  	\item $\exists y\in\R,\forall x\in\R$, $x^2<y$
		\end{enumerate}
		\begin{quote}
			\begin{proof}[Disproof]
				We demonstrate the truth of the negation, ``$\forall y,\exists x,x^2\ge y$.''\smallbreak
				Suppose a real number $y$ is given. Let $x=\sqrt{\nm y}$, then $x^2=y\ge y$, as required.
			\end{proof}
		\end{quote} 
	\end{enumerate}
\end{examples}

\footnotetext{Here is an abstract justification for this heuristic. Consider a propositional function $P(x)$: ``$\forall y,Q(x,y)$,'' then
\begin{align*}
	\neg\bigl(\forall x, \exists y,Q(x,y)\bigr) &\iff  \neg\bigl(\forall x,P(x)\bigr) \iff \exists x,\neg P(x)  \iff \exists x,\neg\bigl(\exists y,Q(x,y)\bigr)\\
	&\iff \exists x, \forall y,\neg Q(x,y)
\end{align*}}


\goodbreak


\boldinline{Putting it all together}

We finish with two examples you might have seen elsewhere. For this course, \emph{you do not have to know what these statements mean,} though you do have to be able to \emph{negate} them.  

\begin{examples}{}{}
	\exstart Vectors $\vx,\vy,\vz$ in the vector space $\R^3$ are \emph{linearly independent} if
	\[
		\forall a,b,c\in\R,\ a\vx+b\vy+c\vz=\V0\implies a=b=c=0
	\]
	In a linear algebra course, the expression $\forall a,b,c\in\R$ would often be hidden. The negation of this statement, what it means for $\vx,\vy,\vz$ to be \emph{linearly dependent,} is
	\[
		\exists a,b,c\in\R,\text{ not all zero, such that }a\vx+b\vy+c\vz=\V0
	\]
	\begin{enumerate}\setcounter{enumi}{1}
	  \item A function $f:\R\to\R$ is said to be \emph{continuous at $a\in\R$} if
		\[
			\forall\epsilon >0,\ \exists\delta>0\text{ such that }|x-a|<\delta\implies |f(x)-f(a)|<\epsilon
		\]
		The negation, what it means for $f$ to be \emph{discontinuous at $x=a$,} is
		\[
			\exists\epsilon>0\text{ such that }\forall\delta>0,\ \textcolor{red}{\exists x\in\R}\text{ with }|x-a|<\delta\text{ and }|f(x)-f(a)|\ge\epsilon
		\]
		The original statement contained a hidden quantifier $\textcolor{red}{\forall x}$ which became explicit upon negation.
	\end{enumerate}
\end{examples}


\goodbreak


\begin{exercises}{}{}
	A self-test quiz and several worked questions can be found \href{http://www.math.uci.edu/~ndonalds/math13/selftest/2-2-quants.html}{online}.

	\begin{enumerate}
		\item Rewrite each sentence using quantifiers. Then write the negation (use words and quantifiers).
			\begin{enumerate}
			  \item All mathematics exams are hard.
		  	\item No football players are from San Diego.
		  	\item There is a odd number that is a perfect square.
			\end{enumerate}
			
			
		\item Let $P$ be the proposition: ``Every positive integer is divisible by thirteen.''
	    \begin{enumerate}
	      \item Write $P$ using quantifiers.
	      \item What is the negation of $P$?
	      \item Is $P$ true or false? Prove your assertion.
	    \end{enumerate}
	  
	  
	  \item A friend claims that the sentence ``$x^2>0\implies x>0$'' has negation ``$x^2>0$ and $x\le 0$.'' Why is this incorrect? What is the correct negation?
	  
	
		\item Consider the quantified statement
	    \[\forall x,y,z\in\R,\ (x-3)^2+(y-2)^2+(z-7)^2>0 \tag{$\ast$}\]
			\begin{enumerate}
		    \item Express ($\ast$) in words.
		    \item Is ($\ast$) true or false? Explain.
		    \item Express the negation of ($\ast$) in symbols, and then in words.
		    \item Is the negation of ($\ast$) true or false? Explain.
		  \end{enumerate}
	    
	    
	  \item Suppose $P, Q,R$ are propositional functions. Compute the negations of the following:
	  \begin{enumerate}
	    \item $\forall x,\exists y, P(x)\wedge Q(y)$ \qquad\qquad (b) \ $\forall x,\exists y, \forall z,R(x,y,z)$
	  \end{enumerate}
	  
	  
	  \goodbreak
	  
	  
	  \item Revisit Example \ref*{ex:multiplequant}.\ref{ex:multiplequant2}. Decide whether each of the following is true or false:
	  \begin{enumerate}
		  \item \makebox[170pt][l]{$\exists x\in\R,\forall y\in\R$, $x^2<y$ \hfill (b)} \ $\forall y\in\R,\exists x\in\R,x^2<y$
		\end{enumerate}
	
	  
		\item The following are statements about positive real numbers $x,y$. Which is true? Explain.
		\begin{enumerate}
		  \item \makebox[170pt][l]{$\forall x$, $\exists y$ such that $xy<y^2$\hfill (b)} \ $\exists x$ such that $\forall y$, $xy<y^2$
		\end{enumerate}
	
	
		\item Which of the following statements are true? Explain.
		\begin{enumerate}
		  \item $\exists$ a married person $x$ such that $\forall$ married people $y$, $x$ is married to $y$.
		  \item $\forall$ married people $x$, $\exists$ a married person $y$ such that $x$ is married to $y$.
		\end{enumerate}
		
		
		\item Prove or disprove the following statements.
		\begin{enumerate}
		  \item For every two points $A$ and $B$ in the plane, there exists a circle on which both $A$ and $B$ lie.
		  \item There exists a circle in the plane on which lie any two points $A$ and $B$.
		\end{enumerate}
	  
	  
		\item Consider the following proposition (\emph{you do not have to know what is meant by a field}).
		\begin{quote}
			All non-zero elements $x$ in a field $\F$ have an inverse: some $y\in\F$ for which $xy=1$.
		\end{quote}
		\begin{enumerate}
		  \item Restate the proposition using quantifiers.
		  \item Find the negation of the proposition, again using quantifiers.
		\end{enumerate}
			
			
		\item\label{ex:decreasing} A function $f:\R\to\R$ is said to be \emph{decreasing} if:
		\[
			x\le y\implies f(x)\ge f(y)
		\]
		\begin{enumerate}
		  \item State what it means for $f$ not to be decreasing (\emph{where is the hidden quantifier?})
		  \item Give an example to show that \emph{not decreasing} and \emph{increasing} do not mean the same thing.
		\end{enumerate}
		
			
		\item Consider the proposition:
		\[
			\forall m,n\in\R,\quad m>n\implies m^2>n^2
		\]
		\begin{enumerate}
	  	\item State the negation of the proposition.
	  	\item Prove that the original proposition is \emph{false.}
	  	\item Suppose you rewrite the proposition:
	  	\[\forall m,n\in A, m>n\implies m^2>n^2\]
	  	What is the largest collection (set) of real numbers $A$ for which the proposition is \emph{true}?
		\end{enumerate}
	
		
		\item (Hard)\lstsp Let $(x_n)=(x_1,x_2,x_3,\ldots)$ denote a sequence of real numbers.
		\begin{quote}
			\makebox[150pt][l]{``$(x_n)$ \emph{diverges to $\infty$}'' means:\hfill}$\forall M>0,\,\exists N\in\R$ such that $n>N\Longrightarrow x_n>M$\smallbreak
			\makebox[150pt][l]{``$(x_n)$ \emph{converges to $L$}" means:\hfill}$\forall\epsilon>0,\,\exists N\in\R$ such that $n>N\Longrightarrow \nm{x_n-L}<\epsilon$
		\end{quote}
		\begin{enumerate}
		  \item State what it means for a sequence $(x_n)$ not to diverge to $\infty$. \emph{Beware of the hidden quantifier!}
		  \item State what it means for a sequence $(x_n)$ not to converge to $L$.
		  \item State what it means for a sequence $(x_n)$ not to converge at all.
		  \item (Challenge: non-examinable)\lstsp Use the definitions to prove that the sequence defined by $x_n=n$ diverges to $\infty$, and that the sequence defined by $y_n=\frac 1n$ converges to zero.
		\end{enumerate}
	
	\end{enumerate}

\end{exercises}

\clearpage



\subsection{Methods of Proof}\label{sec:proof}

\iffalse

EXTRAS!!!

\boldsubsubsection{The Order of Quantifiers Matters!}

We conclude this section with an important observation: the order of quantifiers matters critically! Consider, for example, the following propositions: 
\begin{enumerate}
\item For every person $x$, there exists a person $y$ such that $y$ is a friend of $x$. 
\item There exists a person $y$ such that, for every person $x$, $y$ is a friend of $x$. 
\end{enumerate}
Assuming that $x$ and $y$ always represent people, we can rewrite the sentences as follows:
\begin{enumerate}
\item $\forall x,\,\exists y$ such that $y$ is a friend of $x$.
\item $\exists y$ such that, $\forall x$, we have that $y$ is a friend of $x$.
\end{enumerate}


 All we've done is to switch the order of the two quantifiers! How does this affect the meaning? Written entirely in English, the statements become:
\begin{enumerate}
\item Everyone has at least one friend.
\item There is someone who is friends with everybody.
\end{enumerate}
Quite different! The critical observation is that if $\exists y$ comes after $x$, then $y$ is \emph{allowed to depend on $x$.} Each person might have a friend, but that friend is likely to be different depending on the person. If $\forall x$ comes after $y$, then $x$ cannot depend on $y$.\\

 Play around with the pairs of examples below. What are the meanings? Which ones are true? 
\begin{itemize}\setlength{\itemsep}{0cm}
\item $\forall\,\text{days}\,x,\,\exists\text{ a person }y$ such that $y$ was born on day $x$.
\item $\exists\text{ a person }y$ such that, $\forall\text{ days }x$, $y$ was born on day $x$.\\
\item $\forall\,\text{ circles }x,\,\exists\text{ a point }y$ such that $y$ is the center of $x$.
\item $\exists\text{ a point }y$ such that, $\forall\text{ circles }x$, $y$ is the center of $x$.\\
\item $\forall\,x\in\Z,\,\exists y\in\Z$ such that $y\le x$.
\item $\exists y\in\Z$ such that, $\forall x\in\Z$, $y\le x$.
\end{itemize}
What happens in the last two examples if we replace the integers $\Z$ with the natural numbers $\N$?



% 
% The famous \emph{sum of four squares} theorem can be stated in several ways:\footnote{$\N$ represents the natural numbers (positive integers), and $\Z$ the integers.}
% \begin{quote}\def\arraystretch{1.2}
% 	\begin{tabular}{l|l}
% 		English & Every positive integer may be written as the sum of four squared integers\\\hline
% 		Pure Logic & $\forall n\in\N,\ \exists a,b,c,d\in\Z:n=a^2+b^2+c^2+d^2$\\\hline
% 		Hybrid & $\forall n\in\N,\, \exists a,b,c,d\in\Z$ such that $n=a^2+b^2+c^2+d^2$
% 	\end{tabular}
% \end{quote}


\begin{example}{}{}
Let $x$ be an integer. What is the negation of the following sentence?
\begin{itemize}\setlength{\itemsep}{0pt}
  \item[] If $x$ is even, then $x^2$ is even.
\end{itemize}
Written in terms of propositions, we wish to negate $P\implies Q$, where $P$ and $Q$ are:
\begin{itemize}\setlength{\itemsep}{0pt}
  \item[]{$P.$} $x$ is even.
  \item[]{$Q.$} $x^2$ is even.
\end{itemize}
The negation of $P\implies Q$ is $P\wedge\neg Q$, namely:
\begin{itemize}\setlength{\itemsep}{0pt}
  \item[] $x$ is even and $x^2$ is odd.
\end{itemize}
This is very different to $\neq Q\implies\neg P$ (if $x$ is odd then $x^2$ is odd).\\[5pt]
Keep yourself straight by thinking about the meaning of these sentences. It should be obvious that `$x$ even $\implies x^2$ even' is true. It negation should therefore be \emph{false.} The fact that it is false should make reading the negation feel a little uncomfortable.
\end{example}




\begin{example}{}{}
Let $P,Q$ and $R$ be the following propositions:
\begin{itemize}
\item[]$P$. Irvine is a city in California.
\item[]$Q$. Irvine is a town in Ayrshire, Scotland.
\item[]$R$. Irvine has seven letters.
\end{itemize}
Clearly $P$ is true while $R$ is false. If you happen to know someone from Scotland, you might know that $Q$ is true.\footnote{The second syllable is pronounced like the i in bin or win. Indeed the first Californian antecedent of the Irvine family which gave its name to UCI was an Ulster-Scotsman named James Irvine (1827--1886). Probably the family name was originally pronounced in the Scottish manner.} We can now compute the following (increasingly grotesque) combinations\ldots
\[\def\arraystretch{1.3}
\begin{array}{c|c|c|c|c|c|c}
P\wedge Q&P\vee Q&P\wedge R&\neg R&(\neg R)\wedge P&\neg(R\vee P)&(\neg P)\vee [((\neg R)\vee P)\wedge Q] \\\hline
T & T & F & T & T & F & T
\end{array}\]
\end{example}

 How did we establish these facts? Some  are quick,  and can be done in your head. Consider, for instance, the statement  $(\neg R)\wedge P$. Because $R$ is false, $\neg R$ is true. Thus $(\neg R)\wedge P$ is the conjunction of two true statements, hence it is true. Similarly, we can argue that $R\vee P$ is true (because $R$ is false and $P$ is true), so the negation $\neg(R\vee P)$ is false.

  Establishing the truth value of the final proposition $(\neg P)\vee [((\neg R)\vee P) \wedge Q]$ requires more work. You may want to set up a truth table with several auxiliary columns to help you compute: 
\[\def\arraystretch{1.3}
\begin{array}{c|c|c|c|c|c|c|c}
P & Q & R & \neg P & \neg R & (\neg R) \vee P & ((\neg R)\vee P) \wedge Q &  (\neg P) \vee [ ((\neg R)\vee P) \wedge Q] \\\hline
 T & T & F & F            & T         & T                        & T                                               & T
 \end{array}\]
The importance of parentheses in a logical expressions cannot be stressed enough. For example, try building the truth table for the propositions $P\vee(Q\wedge R)$ and $(P\vee Q)\wedge R$. Are they the same?\pagebreak[4]







\boldsubsubsection{Theorems and Direct Proofs}

Truth tables and connectives are very abstract. To apply them to mathematics we need the following basic notions of theorem and proof.

\begin{defn}{}{}
A \emph{theorem} is a justified assertion that some statement of the form $P\implies Q$ is true.\\
A \emph{proof} is an argument that justifies the truth of a theorem.
\end{defn}

 Think back to the truth table for $P\implies Q$ in Definition \ref{defn:implies}. Suppose that the hypothesis $P$ is true and that $P\implies Q$ is true: that is, $P\implies Q$ is a \emph{theorem.} We must be in the \emph{first row} of the truth table, and so the conclusion $Q$ is also true. This is how we think about proving basic theorems. In a \emph{direct proof} we start by assuming the hypothesis ($P$) is true and make a logical argument ($P\implies Q$) which asserts that the conclusion ($Q$) is true. As such, it often convenient to rewrite the statement of a theorem as an implication of the form $P\implies Q$. Here is a very simple theorem which we prove directly. 

\begin{thm}{}{}
The product of two odd integers is odd.
\end{thm}

 The first thing to do is to write the theorem in terms of propositions and connectives: that is, in the form $P\implies Q$.
\begin{itemize}
  \item $P$ is `$x$ and $y$ are odd integers.' This is our assumption, the hypothesis.
  \item $Q$ is `The product of $x$ and $y$ is odd.' This is what we want to show, the conclusion.
  \item Showing that $P\implies Q$ is true, that (the truth of) $P$ implies (the truth of) $Q$ requires an argument. This is the \emph{proof.}
\end{itemize}

\begin{proof}
Let $x$ and $y$ be \emph{any} two odd integers. We want to show that product $x\cdot y$ is an odd integer. \\
By definition, an integer is odd if it can be written in the form $2k+1$ for \emph{some} integer $k$. Thus there must be integers $n$, $m$ such that $x=2n+1$ and $y=2m+1$. We compute:
\[x\cdot y=(2n+1)(2m+1)=4mn+2n+2m+1=2(2mn+n+m)+1.\]
Because $2mn+n+m$ is an integer, this shows that $x\cdot y$ is an odd integer.
\end{proof}

 It is common to place a symbol (in this case \smash{\raisebox{7pt}{$\qedsymbol$}}) at the end of a proof to tell the reader that your argument is complete. Traditionally the letters Q.E.D. (from the Latin \emph{quod erat demonstrandum,} literally `which is what had to be demonstrated') were used, but this has gone out of style.You may also feel that you want to write more, or less than the above. This is a difficult thing to judge. What do you feel is a convincing argument? Test your argument on your classmates. The appropriate level of detail will depend on your readership: a middle school student will need more detail than a graduate student! At the moment, the best guide is to write for someone with the same mathematical sophistication as yourself. If, in three weeks' time, you can return to what you've written and understand it, then it's probably good!





\boldsubsubsection{Proof by Contrapositive}

The fact that $P\implies Q$ and $\neq Q\implies\neg P$ are logically equivalent allows us, when convenient, to prove $P\implies Q$ by instead proving its contrapositive. As an example, consider another basic theorem.

\begin{thm}{}{}
Let $x$ and $y$ be integers. If $x+y$ is odd, then exactly one of $x$ or $y$ is odd.
\end{thm}

 The theorem is an implication of the form $P\implies Q$ where
\begin{itemize}\setlength{\itemsep}{0pt}
  \item[]$P$. The sum $x+y$ of integers $x$ and $y$ is odd.
  \item[]$Q$. Exactly one of $x$ or $y$ is odd.
\end{itemize}

 A direct proof would require that we assume $P$ to be true and logically deduce the truth of $Q$. For instance we might start our argument with:
\[\text{Suppose that $x+y=2n+1$ for some integer $n$}\]
The problem is that this doesn't really tell us anything about $x$ and $y$, which we need to think about in order to demonstrate the truth of $Q$. Instead we consider the negations of our propositions:
\begin{itemize}\setlength{\itemsep}{0pt}
  \item[]$\neg Q$.\quad $x$ and $y$ are both even or both odd \ (they have the same parity).
  \item[]$\neg P$.\quad The sum $x+y$ of integers $x$ and $y$ is even.
\end{itemize}
Since $P\implies Q$ is logically equivalent to the seemingly simpler contrapositive $\neg Q\implies\neg P$, we choose to prove the latter. This is, by Theorem \ref{thm:contrapos}, equivalent to proving the original implication.

\begin{proof}
Assume that $x$ and $y$ have the same parity. There are two cases: $x$ and $y$ are both even, or both odd.
\begin{description}\setlength{\itemsep}{0pt}
  \item[Case 1:] Let $x=2m$ and $y=2n$ be even. Then $x+y=2(m+n)$ is even.
  \item[Case 2:] Let $x=2m+1$ and $y=2n+1$ be odd. Then $x+y=2(m+n+1)$ is even.
\end{description}
In both cases $x+y$ is even, and the result is proved.
\end{proof}

 


\clearpage




%\subsection{Methods of Proof}\label{sec:proof}

There are four standard methods for proving a theorem $P\implies Q$. In practice, long proofs will use several such arguments joined together. We have already discussed the first two types of proof in Section \ref{sec:prop}.

\begin{description}
	\item[Direct] Assume $P$ is true and deduce that $Q$ is true.
	\item[Contrapositive] Assume $\neg Q$ and deduce $\neg P$. This is enough since the contrapositive $\neq Q\implies\neg P$ is logically equivalent to $P\implies Q$
	\item[Contradiction] Assume that $P$ and $\neg Q$ are true and deduce a \emph{contradiction}. Since $P\wedge\neg Q$ implies a contradiction, it follows that $P\wedge\neg Q$ must be \emph{false.} By Theorem \ref{thm:negconditional}, we see that $P\implies Q$ is true.
	\item[Induction] This has a completely different flavor: we will consider it in Chapter \ref{sec:ind}.
\end{description}

 Each of the methods has advantages and disadvantages. For instance, the direct method has the advantage of a straightforward logical flow. The contrapositive method is useful when the \emph{negations} $\neg P$, $\neg Q$ are simpler than $P$, $Q$ themselves. This is often the case when one or both statements involve the \emph{non-existence} of something. Working with their negations might give you the existence of ingredients with which you can calculate. Proof by contradiction has a similar advantage: assuming both $P$ and $\neg Q$ gives you two pieces of information with which you can calculate. Logically speaking there is no difference between the three methods, beyond how you visualize your argument.\\

 To illustrate the difference between direct proof, proof by contrapositive, and proof by contradiction, we prove the same simple theorem in three different ways. 


\begin{thm}{}{3xodd}
Suppose that $x$ is an integer. If $3x+5$ is even, then $3x$ is odd.
\end{thm}

\begin{proof}[Direct Proof]
We show that if $3x+5$ is even then $3x$ is odd.\\[5pt]
Assume that $3x+5$ is even, then $3x+5=2n$ for some integer $n$. Hence
\[3x=2n-5=2(n-3)+1.\]
This is clearly odd, because it is of the form `an even integer plus one.'
\end{proof}

\begin{proof}[Contrapositive Proof]
We show that if $3x$ is even then $3x+5$ is odd.\\[5pt]
Assume that $3x$ is even, and write $3x=2n$ for some integer $n$. Then
\[3x+5=2n+5=2(n+2)+1.\]
This is odd, because $n+2$ is an integer.
\end{proof}

\begin{proof}[Contradiction Proof]
We assume that $3x+5$ and $3x$ are both even, and we deduce a falsehood.\\[5pt]
Write $3x+5=2m$ and $3x=2n$ for some integers $m$ and $n$. Then
\[5=(3x+5)-3x=2m-2n=2(m-n).\]
Since $m-n$ is an integer, this says that 5 is even: a contradiction.
\end{proof}


\boldsubsubsection{Some simple proofs}

We now give several examples of simple proofs. The only notation needed to speed things along is that of some basic sets of numbers: $\N$ for the positive integers, $\Z$ for the integers, $\R$ for the real numbers, and $\in$ for `is a member of the set'. Thus $2\in\Z$ is read as `2 is a member of the set of integers', or more concisely, `2 is an integer'. We will properly cover this notation in Chapter \ref{chap:sets}.

\begin{thm}{}{oddprod}
Let $m,n\in\Z$. Both $m$ and $n$ are odd if and only if the product $mn$ is odd.
\end{thm}

 There are really two theorems here:
\begin{itemize}
\item[]{($\Rightarrow$)} If $m$ and $n$ are both odd integers, then the product $mn$ is odd.
\item[]{($\Leftarrow$)} If the product $mn$ of two integers is odd, then both $m$ and $n$ are odd.
 \end{itemize}
 
 Often when there are two directions you'll have to prove them separately. Here we give a direct proof for ($\Rightarrow$) and a contapositive proof for ($\Leftarrow$).

\begin{proof}
\begin{enumerate}
  \item[($\Rightarrow$)] Let $m$ and $n$ be odd. Then $m=2k+1$ and $n=2l+1$ for some $k,l\in\Z$. Then
  \[mn=(2k+1)(2l+1)=4kl+2k+2l+1=2(2kl+k+l)+1.\]
  This is odd, because $2kl+k+l\in\Z$.
  \item[($\Leftarrow$)] Suppose that the integers $m$ and $n$ are \emph{not} both odd. That is, assume that \emph{at least one} of $m$ and $n$ is even. We show that the product $mn$ is even. Without loss of generality,\footnote{See `Potential Mistakes' below for what this means.} we may assume that $n$ is even, from which $n=2k$ for some integer $k$. Then,
  \[mn=m(2k)=2(mk)\quad\text{is even}.\tag*{\qedhere}\]
\end{enumerate}
\end{proof}

 In the second part of the proof, we did not need to consider whether $m$ was even or odd: if $n$ is even, the product $mn$ is even regardless. The second part would have been very difficult to prove directly. For instance, you might have tried to start a direct proof with:
\[\text{Assume that $mn$ is odd, then $mn=2k+1$ for some integer $k$. Then\ldots}\]
We are stuck!

\begin{thm}{}{x odd if 3x+5 is odd}
If $3x+5$ is even, then $x$ is odd.
\end{thm}

We can prove this directly, by the contrapositive method, or by contradiction. We'll do all of them, so you can appreciate the difference. 

\begin{proof}[Direct Proof] Simply quote the two previous theorems. Because $3x+5$ is even, $3x$ must be odd by Theorem \ref{thm:3xodd}. Now, since   $3x$ is odd,  both $3$ and $x$ are odd by Theorem \ref{thm:oddprod}.
\end{proof}
 
\begin{proof}[Contrapositive Proof] Suppose that $x$ is even. Then $x=2m$ for some integer $m$ and we get
  \[3x+5=6m+5=2(3m+2)+1.\]
Because $3m+2\in\Z$, we have $3x+5$ odd. 
\end{proof}

\begin{proof}[Contradiction Proof] Suppose that both $3x+5$ and $x$ are even. We can write $3x+5=2m$  and $x=2k$ for some integers $m$ and $k$. Then
  \[5= (3x+5)-3x = 2m - 6k=2(m-3k)\]
  is even. Contradiction.
\end{proof}

 Selecting a method of proof is often a matter of taste. You should be able to see the advantages and disadvantages of the various approaches. The direct proof is more logically straightforward, but it depends on two previous results. The contrapositive and the contradiction arguments are quicker and more self-contained, but they require a greater level of comfort with logic. Consider who you are writing for before you decide to present a slick difficult proof over a slow simple one.\footnote{The Hungarian mathematician Paul Erdős used to refer to simple, elegant proofs as being `from the Book,' as if the Almighty had a book of perfect proofs of which mere mortals might occasionally be permitted a glimpse. Of course, as with all matters spiritual, one person's Book may be very different to another's\ldots} For even more variety, here is a direct proof of Theorem \ref{thm: x odd if 3x+5 is odd} that does not use any previous result.

\begin{proof}[Alternative Direct Proof]
Suppose $3x+5$ is even, so $3x+5=2n$ for some integer $n$. Then
\begin{align*}
x&= (3x+5)-2x-5=2n-2x-5\\
&=2(x-n-3)+1
\end{align*}
is odd.
\end{proof} 
The fact that such variety is possible just makes proving theorems even more fun!

\boldsubsubsection{Common Mistake 1.\quad Generality and `Without Loss of Generality'}

There are many common mistakes in the writing of proofs that you should be careful to avoid. Here are two incorrect `proofs' of the $\implies$ direction of Theorem \ref{thm:oddprod}.

\begin{proof}[Fake Proof 1]
$m=3$ and $n=5$ are both odd, and so $mn=15$ is odd.
\end{proof}

 This is an \emph{example} of the theorem, not a proof. Examples are critical to helping you understand and believe what a theorem says, but they are no substitute for a proof! Recall the discussion in the Introduction on the usage of the word \emph{proof} in English.

\begin{proof}[Fake Proof 2]
Let $m=2k+1$ and $n=2k+1$ be odd. Then,
\[mn=(2k+1)(2k+1)=2(2k^2+2k)+1\]
is odd.
\end{proof}

 The problem with this second `proof' is that it is insufficiently general. $m$ and $n$ are supposed to be \emph{any} odd integers, but by setting both of them equal to $2k+1$, we've chosen $m$ and $n$ to be the same! Notice how the correct proof uses $m=2k+1$ and $n=2l+1$, where we place no restriction on the integers $k$ and $l$.\\


 By \emph{generality} we mean that we must make sure to consider all possibilities encompassed by the hypothesis. The phrase \emph{Without Loss of Generality,} often shorted to WLOG, is used when a choice is made which might at first appear to restrict things but, in fact, does not.

Think back to how this was used in the the proof of Theorem \ref{thm:oddprod}. Since the integers $m$ and $n$ appear symmetrically in the Theorem, if at least one of them is even, then we lose nothing by assuming that the second integer $n$ is even.\\

The phrase WLOG is used to pre-empt a challenge to a proof in the sense of \emph{Fake Proof 2,} as if to say to the reader:
\begin{center}
`You might be tempted to object that my argument is not general enough. However, I've thought about it, and there is no problem.'\\
\end{center}

\paragraph{Common Mistake 2.\quad Incorrect use of equals}

Remember that propositions should be joined by connectives: i.e., by $\implies$ or $\iff$. It is very common to see students write something like
\[\text{$m$ is odd $=m=2k+1$ for some integer $k$}\]
This is extremely confusing! If this is part of a longer argument, things will become very difficult to follow. Since `$m$ is odd' and `$m=2k+1$ for some integer $k$' are both \emph{propositions,} they should be linked by a \emph{connective.} We should instead write
\[\text{$m$ is odd $\iff m=2k+1$ for some integer $k$}\]

\paragraph{Common Mistake 3.\quad Becoming distracted by algebra}

Here is a palpably ludicrous `theorem' which illustrates another potential mistake.

\begin{thm*}{Fake Theorem}{}
The only number is zero.
\end{thm*}

\begin{proof}[Fake Proof]
Let $x$ be any number and let $y=x$, then
\begin{align*}
x=y\implies &x^2=xy\tag*{(Multiply both sides by $x$)}\\
\implies &x^2-y^2=xy-y^2\tag*{(Subtract $y^2$ from both sides)}\\
\implies &(x-y)(x+y)=(x-y)y\tag*{(Factorize)}\\
\implies &x+y=y\tag*{(Divide both sides by $x-y$)}\\
\implies &x=0 \tag*{\qedhere}
\end{align*}
\end{proof}

 Everything is fine up to the third line, but then we divide by $x-y$, which is zero! Don't let yourself become so enamoured of logical manipulations that you forget to check the basics.


\boldsubsubsection{More simple proofs}

We continue with more straightforward proofs. None of these results are particularly important, they are just exercises in deciding how to present an argument.

\begin{thm}{}{polyroot}
Suppose that $x\in\R$. Then $x^3+2x^2-3x-10=0\implies x=2$.
\end{thm}

 We can prove this theorem using any of the three methods. All rely on your ability to factorize the polynomial:
\[x^3+2x^2-3x-10=(x-2)(x^2+4x+5)=(x-2)\Bigl[(x+2)^2+1\Bigr],\]
and partly on your knowledge that $ab=0\iff a=0$ or $b=0$ (proof in the exercises).

\begin{proof}[Direct Proof]
If $x^3+2x^2-3x-10=0$, then $(x-2)[(x+2)^2+1]=0$. Hence at least one of the factors $x-2$ or $(x+2)^2+1$ is zero.\\
In the first case we conclude that $x=2$.\\
The second case is impossible, since $(x+2)^2\ge 0\implies (x+2)^2+1>0$.\\
Therefore $x=2$ is the only solution.
\end{proof}

\begin{proof}[Contrapositive Proof]
Suppose that $x\neq 2$. Then $x^3+2x^2-3x-10=(x-2)[(x+2)^2+1]\neq 0$ since neither of the factors is zero.
\end{proof}

\begin{proof}[Contradiction Proof]
Suppose that $x^3+2x^2-3x-10=0$ and $x\neq 2$. Then
\[0=x^3+2x^2-3x-10=(x-2)[(x+2)^2+1].\]
Since $x\neq 2$, we have $x-2\neq 0$.\\
It follows that $(x+2)^2+1$ must be zero. However, $(x+2)^2+1\ge 1$ for all real numbers $x$, so we have a contradiction.
\end{proof}

 On balance, the contrapositive proof is probably the most elegant, but you can decide for yourself.


\paragraph{Common Mistake 4.\quad Being excessively logical}

The statement of Theorem \ref{thm:polyroot} is an implication $P\implies Q$ where $P$ and $Q$ are:
\[P.\quad x^3+2x^2-3x-10=0, \qquad\qquad Q.\quad x=2.\]
You can make life very hard for yourself by being overly logical. For instance, you may wish take a third proposition $R$.\quad $x\in\R$, and state the theorem as $R\implies (P\implies Q)$. This is the way of pain! It's easier to assume, as a universal constraint, that you're always dealing with real numbers; you can then ignore said constraint within the argument.

 Indeed, one can always append a third proposition to the front of any theorem, namely, ``all math I already know.'' Try to resist the temptation to be so logical that your arguments become unreadable. The goal is to convince the reader that the theorem is true, not to confuse them!

\boldsubsubsection{Definition-Pushing}

The next example concerns divisibility. Before you can prove a theorem, it is critical that you know the \emph{meaning} of all of the words in its statement. We therefore state the definition of divisibility.

\begin{defn}{}{}
Let $n$ and $p$ be integers. We say that $n$ is \emph{divisible by} $p$ if $n=pk$ for some integer $k$.
\end{defn}

 Now we can present a theorem.

\begin{thm}{}{}
If $n\in\Z$ is divisible by $p\in\N$, then $n^2$ is divisible by $p^2$.
\end{thm}

\begin{proof}
We prove directly. Let $n$ be divisible by $p$. Then $n=pk$ for some $k\in\Z$. Then $n^2=p^2k^2$, and so $n^2$ is divisible by $p^2$.
\end{proof}

 This is an example of a \emph{definition-pushing} proof. If you simply state the the definition of everything important in the theorem, the proof will often be staring you in the face.

\boldsubsubsection{Proof by Cases}

The next proof is also in the definition-pushing vein. However, it requires that we consider several cases. The relevant definition here is that of \emph{remainder.}

\begin{defn}{}{}
An integer $n$ is said to have remainder $r=0,1,$ or 2 upon division by 3 if we can write $n=3k+r$ for some integer $k$.
\end{defn}

 With a little thought, it should be clear that every integer is of the form $3k$, $3k+1$ or $3k+2$. This is analogous to how all integers are either even ($2k$) or odd $(2k+1)$. We will consider remainders more carefully in Chapter \ref{sec:gcd}.

\begin{thm}{}{sqmod3}
If $n$ is an integer, then $n^2$ has remainder 0 or 1 upon dividing by 3.
\end{thm}

\begin{proof}
We again prove directly. There are three cases: $n$ has remainder 0, 1 or 2 upon dividing by 3.
\begin{enumerate}
  \item[(a)] If $n$ has remainder 0, then $n=3m$ for some $m\in\Z$ and so $n^2=9m^2=3(3m^2)$ has remainder 0.
  \item[(b)] If $n$ has remainder 1, then $n=3m+1$ for some $m\in\Z$ and so
  \[n^2=9m^2+6m+1=3(3m^2+2m)+1\quad\text{has remainder 1.}\]
  \item[(c)] If $n$ has remainder 2, then $n=3m+2$ for some $m\in\Z$ and so
  \[n^2=9m^2+12m+4=3(3m^2+4m+1)+1\quad\text{has remainder 1.}\]
\end{enumerate}
Thus $n^2$ has remainder 0 or 1.
\end{proof}

\boldsubsubsection{Non-existence Proofs}

When a Theorem claims that something does not exist, it is generally a good idea to try a contrapositive or a contradiction proof. This is since `does not exist' is already a \emph{negative} statement. A contradiction or contrapositive proof of a theorem $P\implies Q$ already involves the negated statement $\neg Q$. If $Q$ states that something does not exist, then $\neg Q$ states that it does, which often gives you something to play with! To see this in action, consider the following result.

\begin{thm}{}{}
The equation $x^{17}+12x^3+13x+3=0$ has no positive (real number) solutions.
\end{thm}

 First we interpret the theorem as an implication: throughout we assume that $x$ is a real number.
\[\text{If $x^{17}+12x^3+13x+3=0$, then $x\leq 0$.}\]
The theorem is now in the form $P\implies Q$, with:
\[P.\quad x^{17}+12x^3+13x+3=0,\qquad\qquad\qquad Q.\quad x\le 0.\]
The negation of $Q$ is simply `$x>0$.' To prove the theorem by contradiction, we assume $P$ and not $Q$, and deduce a contradiction.

\begin{proof}
Assume that a real number $x$ satisfies $x^{17}+12x^3+13x+3=0$, and that $x>0$. Because all terms on the left hand side are positive, we have $x^{17}+12x^3+13x+3>0$. A contradiction.
\end{proof}


 Note how quickly the proof is written: it assumes that any reader is familiar with the underlying logic of a contradiction proof without it needing to be spelled out. The discussion we undertook before writing the proof would be considered scratch work: you shouldn't include it a final write-up.\\

 If you want to extend the above result, and you can recall the Intermediate and Mean Value Theorems from Calculus, you should be able to prove that there is exactly one (necessarily negative!) solution to the above polynomial equation.

\boldsubsubsection{The AM-GM inequality}

Next we give several proofs of a famous inequality relating the arithmetic and geometric means of two or more numbers.

\begin{thm}{}{amgm}
If $x,y$ are positive real numbers, then
\[\frac{x+y}{2}\ge\sqrt{xy}\]
with equality if and only if $x=y$.
\end{thm}

 This is a little tricky to read: we really have two separate theorems:
\begin{enumerate}
  \item If $x,y>0$, then $\frac{x+y}{2}\ge\sqrt{xy}$
  \item If $x,y>0$, then $\frac{x+y}{2}=\sqrt{xy}\iff x=y$
\end{enumerate}

 First we give a direct proof: note how the implication signs are stacked to make the argument clearer.

\begin{proof}[Direct Proof]
Clearly $(x-y)^2\ge 0$ with equality $\iff x=y$. Now multiply out:
\begin{align*}
x^2-2xy+y^2\ge 0&\iff (x^2+2xy+y^2)- 4xy  \ge 0 \\
&\iff x^2+2xy+y^2\ge 4xy\\
&\iff (x+y)^2\ge 4xy\\
&\iff x+y\ge 2\sqrt{xy}\tag*{($\ast$)}\\
&\iff \frac{x+y}{2}\ge \sqrt{xy}.
\end{align*}
The square-root in $(\ast)$ is well-defined because $x+y$ is positive.\footnote{Moreover, the inequality is preserved since the function $f(t)=t^2$ is \emph{increasing} when $t$ is positive.} Moreover, it is clear that the final inequality is an equality if and only if all of them are, which is if and only if $x=y$.
\end{proof}

 Observe how the argument for `with equality if and only if $x=y$' depends on all of the implications in the proof being \emph{biconditionals.}\pagebreak

 The following contradiction proof incorporates exactly the same calculation, but is laid out in a different order. This is not always possible, and you have to take great care when trying it. You will likely agree that the direct proof is easier to follow.

\begin{proof}[Contradiction Proof]
Suppose that $\frac{x+y}{2}<\sqrt{xy}$. Since $x+y\ge 0$, this is if and only if $(x+y)^2<4xy$. Now multiply out and rearrange:
\begin{align*}
(x+y)^2<4xy&\iff x^2+2xy+y^2<4xy\\
&\iff x^2-2xy+y^2<0\\
&\iff (x-y)^2<0.
\end{align*}
Since squares of real numbers are non-negative, this is a contradiction. Thus $\frac{x+y}{2}\ge \sqrt{xy}$.\\
Now suppose that $\frac{x+y}{2}=\sqrt{xy}$. Following the biconditionals through the above argument, we see that this is if and only if $(x-y)^2=0$, from which we recover $x=y$. Hence result.
\end{proof}


\boldsubsubsection{Optional: The general AM-GM inequality}

Both the statement and the proof of the general inequality are significantly more difficult. You might be surprised that an argument involving `raising to the $n$th power' doesn't work. Try it and see why\ldots\ It is often the case that a very simple proof exists for a special case of a result, and that attempting to generalize the proof fails. A completely new approach is then required.\\
Since the general result is harder, we present it at a higher level, leaving out some of the more obvious details. This helps us view the proof as a whole, and makes the logical flow clearer. The only prerequisite is a little calculus, namely the First Derivative Test at the end of the first paragraph.

\begin{thm}{}{}
If $x_1,\ldots,x_n>0$ then
\[\frac{x_1+x_2+\cdots+x_n}n\ge\sqrt[n]{x_1x_2\cdots x_n}\]
with equality if and only if $x_1=\cdots =x_n$.
\end{thm}

\begin{proof}
Consider the function $f(x)=e^{x-1}-x$. Its derivative is $f'(x)=e^{x-1}-1$, which is zero if and only if $x=1$. The sign of the derivative changes from negative to positive at $x=1$, whence this is a local minimum. $f$ has no other critical points and its domain is the whole real line, whence $x=1$ is the location of the \emph{global} minimum of $f$. Since $f(1)=0$, we obtain the inequality
\[e^{x-1}\ge x\tag*{($\dag$)}\]
with equality if and only if $x=1$.

 Now consider the arithmetic mean $\mu=\frac{x_1+x_2+\cdots +x_n}{n}$. Applying ($\dag$) to $x=\frac{x_i}\mu$, we obtain
\[\frac{x_i}\mu\le\exp\left(\frac{x_i}\mu-1\right),\qquad\text{for each $i=1,2,\ldots,n$.}\tag*{($\ast$)}\]
Since all the $x_i$ are positive, we may multiply the expressions $(\ast)$ while preserving the inequality:
\[\frac{x_1}\mu\cdots\frac{x_n}\mu\le\exp\left(\frac{x_1}{\mu}-1+\cdots+\frac{x_n}{\mu}-1\right)=\exp(n-n)=1.\]
It follows that $\mu^n\ge x_1\cdots x_n$, from which the result, $\mu\ge\sqrt[n]{x_1\cdots x_n}$, is clear.\\
Equality is if and only if all the inequalities $(\ast)$ are equalities, which is if and only if $x_i=\mu$ for all $i=1,\ldots,n$. That is, all the $x_i$ are equal.
\end{proof}

 Given that the theorem and proof are both more difficult than the $n=2$ case, there are a few things you should do to help convince yourself of their legitimacy.
\begin{enumerate}\setlength{\itemsep}{0pt}
  \item Write down some examples. E.g. if $x_1=20,x_2=27,x_3=50$, the inequality reads
  \[\frac{97}3\ge\sqrt[3]{20\cdot 27\cdot 50}=30.\]
  \item Observe that Theorem \ref{thm:amgm} is a special case.
  \item Work through the proof, inserting comments and extra calculations until you are convinced that the argument is correct. For example, the calculation $\frac{x_1+\cdots+x_n}{\mu}=\frac{\mu n}\mu=n$ was omitted from the final displayed calculation: anyone with the prerequisite knowledge to read the rest of the proof should easily be able to supply this.
\end{enumerate}

 It is perfectly reasonable to ask how you would know to try such a proof. The answer is that you wouldn't. You should appreciate that a proof like this is a distillation of thousands of attempts and improvements, perhaps over many years. No-one came up with this argument as a first attempt!


\boldsubsubsection{Combining and Subdividing Theorems}

Sometimes it is useful to break a proof into pieces, akin to viewing a computer program as a collection of subroutines that you combine for some greater purpose. Usually the intention is to make the proof of a difficult result more readable, but you may also wish to emphasize the importance of certain aspects of your work. Mathematics does this by using \emph{lemmas} and \emph{corollaries.}

\begin{itemize}\setlength{\itemsep}{0pt}
  \item[]\emph{Lemma:} a theorem whose importance you want to downplay. Often the result is individually unimportant, but becomes more useful when referenced in the proof of a larger theorem.
  \item[]\emph{Corollary:} a theorem which follows quickly from another result. Corollaries can be used to draw attention to a particular aspect or a special case of a theorem.
\end{itemize}

 In many mathematical papers the word \emph{theorem} is reserved only for the most important results, everything else being presented as a lemma or corollary. The choice of what to call a result is entirely one of presentation. If you want your paper to be more readable, or to highlight what you think is important, then lemmas and corollaries are your friends!\pagebreak[2]

 Here is a famous example of a lemma at work.

\begin{lemm}{}{root2prep}
Suppose that $n\in\Z$. Then $n^2$ is even $\iff n$ is even.
\end{lemm}

 Prove this yourself: the ($\Rightarrow$) direction is easiest using the contrapositive method, while the ($\Leftarrow$) direction works well directly.

\begin{thm}{}{}
$\sqrt 2$ is irrational.
\end{thm}

 This is tricky for a few reasons. The theorem does not appear to be of the form $P\implies Q$, but in fact it is. Consider:
\begin{itemize}\setlength{\itemsep}{0pt}
  \item[]{$Q$.} $\sqrt 2$ is irrational.
  \item[]{$P$.} Everything you already know in mathematics!
\end{itemize}
Of course $P$ is entirely unhelpful; how can we start a direct proof when we don't know what to choose from the whole universe of mathematics? A contrapositive proof might also be difficult: $\neg Q$ straightforwardly states that $\sqrt 2$ is rational, but $\neg P$ is the cryptic statement, `something we know to be true is actually false.' But what is the \emph{something}?\\
Rather than worry about all this, we instead present a proof by contradiction.

\begin{proof}[Proof.\hspace{-27pt}]
\begin{itemize}\setlength{\itemsep}{-2pt}
  \item[] Suppose that $\sqrt 2=\frac mn$ for some $m,n\in\N$. Without loss of generality, we may assume that $m,n$ have no common factors.
  \item[] Then $m^2=2n^2$ which says that $m^2$ is even.
  \item[] By Lemma \ref{lemm:root2prep} we have that $m$ is even.
  \item[] Thus $m=2k$ for some $k\in\Z$.
	\item[] But now, $n^2=2k^2$, from which (Lemma \ref{lemm:root2prep} again) we see that $n$ is even.
	\item[] Now $m$ and $n$ have a common factor of 2. This is a contradiction.\qedhere
\end{itemize}
\end{proof}

 First observe how Lemma \ref{lemm:root2prep} was used to make the proof easier to read. Without the lemma, the essential shape of the proof would have been less clear.\\
Now try to make sense of the proof. In the first line we invoke the definition of \emph{rational,} being the \emph{ratio} of two integers. The main challenge comes immediately afterwards. Once we assume that $\sqrt 2=\frac mn$, we can immediately insist that $m,n$ have no common factors. Indeed this is no significant restriction \emph{once we assume that $m,n$ exist,} that is \emph{once we assume that $\sqrt 2$ is rational.} It is important to realize that the `no common factors' assumption is \emph{not} the assumption being contradicted. Because of this subtlety, we include the phrase `without loss of generality' so that the reader is forced to think carefully, and does not jump to the wrong conclusion.\\
It might seem difficult to completely understand, but if we hadn't made the observation, our calculation could have continued forever, telling us nothing!
\[m^2=2n^2\implies n^2=2k^2\implies k^2=2l^2\implies\cdots\]
If you find this approach difficult, you may prefer an alternative proof given in the exercises.

\paragraph{Prime Numbers}

Here is another famous result, dating back at least to the Ancient Greeks (Euclid's \emph{Elements}, Proposition IX.20). As ever, we need a solid definition before we try to prove anything.

\begin{defn}{}{irreducible}
An integer $\ge 2$ is \emph{prime} if the only positive integers it is divisible by are itself and 1.
\end{defn}

 The first few primes are $2,3,5,7,11,13,17,19,\ldots$ It follows\footnote{This is not completely obvious: we will prove it much later in Theorem \ref{thm:fundarith}.} from the definition that all positive integers $\ge 2$ are either primes or \emph{composites} (products of primes). In particular, every integer $\ge 2$ is divisible by at least one prime. We may now state Euclid's result.

\begin{thm}{}{}
There are infinitely many prime numbers.
\end{thm}

\begin{proof}
We prove by contradiction. Assume there are exactly $n$ prime numbers $\lst p n$ and consider the integer
\[\Pi:=p_1\cdots p_n+1.\]
Certainly $\Pi$ is divisible by some prime: since we are assuming that the list $\lst pn$ contains \emph{all} the primes, $\Pi$ must be divisible by some prime $p_i$ in the list. However, the product $p_1\cdots p_n$ is clearly divisible by $p_i$, whence so is the difference\footnote{Is this obvious? Can you prove it?!}
\[\Pi-p_1\cdots p_n=1.\]
We conclude that 1 is divisible by the prime $p_i$, contradicting\footnote{Euclid's original argument was not strictly by contradiction. Instead he asserted that, given any list of primes $p_1,\ldots,p_n$, the number $\Pi$ must be divisible by a new prime not in his list.} the fact that $p_i\ge 2$.
\end{proof}

\paragraph{Self-test Questions}

\begin{enumerate}
  \item Consider a theorem $P\implies Q$. We call $P$ the \underline{\phantom{hypothesis}\qquad\qquad} and $Q$ the \underline{\phantom{conclusion}\qquad\qquad}
  \item In a \emph{proof by contrapositive,} we assume that \underline{\phantom{$Q$ is false}\qquad\qquad} and deduce that \underline{\phantom{$P$ is false}\qquad\qquad}
  \item A \emph{proof by contradiction} begins by assuming that \underline{\phantom{$P$ is true and $Q$ is false}\qquad\qquad}
  %\item Propositions are joined by \underline{\phantom{connectives}\qquad\qquad}
  \item Explain in a sentence or two what is meant by \emph{without loss of generality,} and how the phrase is used.
\end{enumerate}

\begin{exercises}{}{}

\begin{enumerate}\renewcommand{\labelenumi}{\thesubsection.\theenumi}
	\item Show that for any given integers $a,b,c$, if $a$ is even and $b$ is odd, then $7a-ab+12c+b^2+4$ is odd.\goodbreak

  \item Prove or disprove the following conjectures.
	\begin{enumerate}
	  \item There is an even integer which can be expressed as the sum of three even integers.
	  \item Every even integer can be expressed as the sum of three even integers. 
	  \item There is an odd integer which can be expressed as the sum of two odd integers.
	  \item Every odd integer can be expressed as the sum of three odd integers.
		\item[]\emph{To get a feel about whether a claim is true or false, try out some examples. If you believe a claim is false, provide a specific counterexample. If you believe a claim is true, give a formal proof.}
\end{enumerate}

  \item Prove or disprove the following conjectures:
	\begin{enumerate}
	  \item The sum of any 3 consecutive integers is divisible by 3.
	  \item The sum of any 4 consecutive integers is divisible by 4.
	  \item The product of any 3 consecutive integers is divisible by 6.
	\end{enumerate}

  \item Augustus de Morgan satisfied his own problem:
	\begin{center}
	I turn(ed) $x$ years of age in the year $x^2$.
	\end{center}
	\begin{enumerate}
	  \item Given that de Morgan died in 1871, and that he wasn't the beneficiary of some miraculous anti-aging treatment, find the year in which he was born.
	  \item Suppose you have an acquaintance who satisfies the same problem. How old will they turn this year?
	\end{enumerate}%\vspace{-9pt}
	Give a formal argument which justifies that you are correct.
	
	\item Prove that if $n$ is a natural number greater than 1, then $n!+2$ is even.\\[5pt]
  \emph{Here $n!$ denotes the \emph{factorial} of the integer $n$. Look up the definition if you forgot about it.}
  
  \item Consider the following proposition, where $x$ is assumed to be a real number.
	\[x^3-3x^2-2x+6=0\implies x=3\tag*{($\ast$)}\]
	\begin{enumerate}
  	\item Is the proposition ($\ast$) true or false? Justify your answer. Is its converse true?
  	\item Repeat part (a) for the proposition
		\[x^3-3x^2-2x+6=0\implies x\neq 3\]
  	\item Does anything change about the truth status of ($\ast$) if we assume that it is a statement about \emph{rational numbers $x$}? Explain.
	\end{enumerate}
  
	\item\begin{enumerate}
	  \item Let $x\in\Z$. Prove that $5x+3$ is even if and only if $7x-2$ is odd.
	  \item Can you conclude anything about $7x-2$ if $5x+3$ is odd?
	  \end{enumerate}
	  
	  
	\item Below is the proof of a result. What result is being proved?
  \begin{proof}
  Assume that $x$ is odd. Then $x=2k+1$ for some integer $k$. Then
  \[2x^2-3x-4=2(2k+1)^2-3(2k+1)-4=8k^2+2k-5=2(4k^2+k-3)+1.\]
  Since $4k^2+k-3$ is an integer, $2x^2-3x-4$ is odd.
  \end{proof}
 
	
	\item Here is another proof. What is the result this time?
  \begin{proof}
  Assume, without loss of generality, that $x$ and $y$ are even. Then $x=2a$ and $y=2b$ for some integers $a,b$. Therefore,
  \[xy+xz+yz=(2a)(2b)+(2a)z+(2b)z=2(2ab+az+bz).\]
  Since $2ab+az+bz$ is an integer, $xy+xz+yz$ is even.
  \end{proof}
	  
	\item In this question, you should use the following definition of the rational numbers.
		\begin{defn*}{}{}
  	A real number $x$ is \emph{rational} if it may be written in the form $x=\frac pq$ where $p$ is an integer and $q$ is a positive integer. $x$ is \emph{irrational} if it is not rational.
  	\end{defn*}
  	Prove or disprove the following conjectures.
  	\begin{conj*}{1}{}
			If $x$ and $y$ are real numbers such that $3x+5y$ is irrational, then at least one of $x$ and $y$ is irrational.
		\end{conj*}
  	\begin{conj*}{2}{}
			If $x$ and $y$ are rational numbers, then $3x+4xy+2y$ is rational.
		\end{conj*}
  	\begin{conj*}{3}{}
			If $x$ and $y$ are irrational numbers, then $3x+4xy+2y$ is irrational.
		\end{conj*}
	
	\item Let $x$ and $y$ be integers. Prove: For $x^2+y^2$ to be even, it is necessary that $x$ and $y$ have the same parity (i.e. both even or both odd).
 
	\item Prove that if $x$ and $y$ are positive real numbers, then $\sqrt{x+y}\neq\sqrt{x}+\sqrt{y}$. \emph{Argue by contradiction.}
	
	\item Prove that $ab=0\iff a=0$ or $b=0$.

  \item You meet three old men, Alain, Boris, and César, each of whom is a Truthteller or a Liar. Truthtellers speak only the truth; Liars speak only lies. You ask Alain whether he is a Truthteller or a Liar. Alain answers with his back turned, so you cannot hear what he says.\\[5pt]
``What did he say?'' you ask Boris.\\[5pt]
Boris replies: ``Alain says he is a Truthteller.''\\[5pt]
César says: ``Boris is lying.''\\[5pt]
Is César a Truthteller or a Liar? Explain your answer.\goodbreak
	
  \item \emph{(Snake-like integers)} Let's say that an integer $y$ is \textit{Snake-like} if and only if there is some integer $k$ such that $y=(6k)^2+9$.
	\begin{enumerate}
	  \item Give three examples and three non-examples of Snake-like integers. 
	  \item Given $y\in\Z$, compute the negation of the statement, `$y$ is Snake-like.'
	  \item Show that every Snake-like integer is a multiple of $9$.
	  \item Show that the statements, `$n$ is Snake-like,' and, `$n$ is a multiple of nine,' are not equivalent.
	\end{enumerate}
	
	
	\item Assume that Ben's father lives in Peru. Consider the following implication $\beta$:
	\begin{center}
	If Ben's father is an artist and does not have any friends in Asia, then Ben plays tennis or ping-pong, or he appeared in at least one picture of the May 1992 Time magazine.
	\end{center}
	\begin{enumerate}
	  \item Find the contrapositive of $\beta$.
	  \item Find the converse of $\beta$.
	  \item Find the negation of $\beta$.
	  \item Imagine you are a detective and want to find the truth value of $\beta$. Describe your action-strategy in full detail.
	\end{enumerate}
  
  \item Here is an alternative argument that $\sqrt 2$ is irrational. Suppose that $\sqrt 2=\frac mn$ where $m,n\in\N$. This time we don't assume that $m,n$ have no common factors.
  \begin{enumerate}
    \item $m,n$ satisfy the equation $m^2=2n^2$. Prove that there exist positive integers $m_1,n_1$ which satisfy the following three conditions:
    \[m_1^2=2n_1^2,\qquad\qquad m_1<m,\qquad\qquad n_1<n.\] 
    \item Show that there exist two sequences of decreasing positive integers $m>m_1>m_2>\cdots$ and $n>n_1>n_2>\cdots$ which satisfy $m_i^2=2n_i^2$ for all $i\in\N$.
    \item Is it possible to have an infinite sequence of decreasing \emph{positive} integers? Why not? Show that we obtain a contradiction and thus conclude that $\sqrt 2\not\in\Q$.
	\end{enumerate}
	This is an example of the \emph{method of infinite descent,} which is very important in number theory. 
	
  \item You are given the following facts.
  \begin{enumerate}
    \item All polynomials are continuous.
    \item (Intermediate Value Theorem) If $f$ is continuous on $[a,b]$ and $L$ lies between $f(a)$ and $f(b)$, then $f(x)=L$ for some $x\in(a,b)$.
    \item If $f'(x)>0$ on an interval, then $f$ is an increasing function.
	\end{enumerate}
	Use these facts to give a formal proof that $x^{17}+12x^3+13x+3=0$ has \emph{exactly one solution} $x$, and that $x$ lies in the interval $(-1,0)$.
\end{enumerate}

\end{exercises}


\newpage

%\subsection{Quantifiers}\label{sec:quant}


\fi