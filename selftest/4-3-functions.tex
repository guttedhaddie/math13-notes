\pagestyle{empty}

\boldsubsection{Reading Quiz Section \ref*{sec:func1}}

\begin{enumerate}
	\item The range of a function $f:A\to B$ is (select all that apply):
	\begin{enumerate}
		\item \makebox[200pt][l]{A subset of the domain.\hfill (b)} \ A subset of the codomain.
	  \setcounter{enumii}{2}
		\item \makebox[200pt][l]{Always equal to the codomain.\hfill (d)}\ Also called the image of the function.
	  \setcounter{enumii}{4}
		\item Equal to $f(A)$.
	\end{enumerate}

	\item Suppose $f:A\to B$ and $g:B\to C$ are functions. If $g\circ f$ is bijective, which of the following \emph{must} be true?
	\begin{enumerate}
	  \item \makebox[180pt][l]{$f$ is injective.\hfill (b)} \ $g$ is injective.
	  \setcounter{enumii}{2}
    \item \makebox[180pt][l]{$f$ is surjective.\hfill (d)} \ $g$ is surjective.
  \end{enumerate}
    
  \item True or False: We can always make a function surjective by making its domain smaller.
    
  \item True or False: If $A\subseteq B$, there is an injective function $f:A\to B$.
\end{enumerate}


\boldsubsection{Practice Problems Section \ref*{sec:func1}}

\begin{enumerate}
	\item\begin{enumerate}
    \item Explain why the `rule' $g:\{\text{all lines in the plane}\}\to\R$ which sends a line $\ell$ to the slope of $\ell$ does \emph{not} define a function.
    
    \href{https://youtu.be/HHX_f3KIVUw}{Video Solution}
    
    
    \item Let $L$ be the set of all \emph{non-vertical} lines in the plane. The rule $f:L\to\R$ sending $\ell$ to its slope is a well-defined function.
    \begin{enumerate} 
      \item Find $f(Z)$ where $Z$ is the set of lines intersecting $y=2x+5$ at exactly one point. 
    
    	\href{https://youtu.be/evTJt6VCg1Y}{Video Solution}
    
    	\item Let $U=\{-2\}$. Describe the inverse image $f^{-1}(U)$.
    
    	\href{https://youtu.be/FTLwdH1NZ-k}{Video Solution}
    
    	\item Explain why $f$ is not bijective. Find a subset $B\subseteq L$ so that $f:B\to\R$ is a bijection.

    	\href{https://youtu.be/TklbTQpXrMo}{Video Solution 1}\qquad
    	\href{https://youtu.be/eubplxO4Y_w}{Video Solution 2}
		\end{enumerate}
	\end{enumerate}


	\item Suppose $f:A\to B$ and $g:B\to C$ are functions. For each of the following, either find an example or explain why no such example exists.
	\begin{enumerate}
    \item $f$ surjective, $g$ not surjective and $g \circ f$ surjective. 
    \item $f$ not surjective, $g$ surjective and $g \circ f$ surjective. 
    \item $f$ surjective, $g$ surjective and $g \circ f$ not surjective. 
    \item $f$ injective, $g$ not injective and $g \circ f$ injective.  
    \item $f$ not injective, $g$ injective and $g \circ f$ injective.   
    \item $f$ injective, $g$ injective and $g \circ f$ not injective.  
	\end{enumerate}

	\href{https://youtu.be/A6m4O8GvGy8}{Video Solution (parts (a)-(c))}


	\item Suppose $f:A\to B$ is a function. Prove or disprove the following statements:
	\begin{enumerate}
    \item Let $X$ and $Y$ be subsets of $A$. If $X\cap Y=\emptyset$ then $f(X)\cap f(Y)=\emptyset$.
    \item Let $W$ and $Z$ be subsets of $B$. If $W\cap Z=\emptyset$ then $f^{-1}(W)\cap f^{-1}(Z)=\emptyset$. 
	\end{enumerate}

	\href{https://youtu.be/pVmf28Cg_Y8}{Video Solution}
\end{enumerate}