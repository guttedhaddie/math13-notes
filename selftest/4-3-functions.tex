\pagestyle{empty}

\boldsubsection{Reading Quiz Section \ref*{sec:func1}}

\begin{enumerate}
    \item The range of a function $f : A \to B$ is (select all that apply)
    \begin{enumerate}
        \item a subset of the domain.
        \item a subset of the codomain.
        \item always equal to the codomain.
        \item also called the image of the function.
        \item equal to $f(A)$.
    \end{enumerate}

    \item Suppose $f : A \to B$ and $g : B \to C$ are functions. If $g \circ f$ is bijective, which of the following \emph{must} be true?
    \begin{enumerate}
        \item $f$ is injective.
        \item $g$ is injective.
        \item $f$ is surjective.
        \item $g$ is surjective.
    \end{enumerate}
    
    \item True or False: We can always make a function surjective by making its domain smaller.
    
    \item True or False: If $A \subseteq B$, there is an injective function $f : A \to B$.
\end{enumerate}


\boldsubsection{Practice Problems Section \ref*{sec:func1}}

\begin{enumerate}
\item \begin{enumerate}
    \item Explain why the map $g : \{\text{all lines in the planes}\} \to \R$ which sends a line $\ell$ to the slope of $\ell$ is not a function.
    
    \href{https://youtu.be/HHX_f3KIVUw}{Video Solution}
    
    \item Let $L$ be the set of all non-vertical lines in the plane. The map $f : L \to \R$ defined by $\ell \mapsto \text{ slope of } \ell$ is a well defined function. Find $f(Z)$ where $Z$ is the subset of $L$ consisting of the lines that intersect the line $y = 2x + 5$ at exactly one point. 
    
    \href{https://youtu.be/evTJt6VCg1Y}{Video Solution}
    
    \item Now let $U = \{ -2 \}$. Describe the inverse image $f^{-1}(U)$ of $U$ under the function $f$ defined in part (b).
    
    \href{https://youtu.be/FTLwdH1NZ-k}{Video Solution}
    
    \item Is the function  $f$ bijective? 
    
    \href{https://youtu.be/TklbTQpXrMo}{Video Solution}
    
    \item Find a subset $B$ of $L$ so that the function $f : B \to \R$  is a bijection.
    
    \href{https://youtu.be/eubplxO4Y_w}{Video Solution}
\end{enumerate}

\item Suppose $f : A \to B$ and $g : B \to C$ are functions. For each of the following, either find an example or explain why no such example exists.
\begin{enumerate}
    \item $f$ surjective and $g$ not surjective so that the composition $g \circ f$ is surjective. 
    \item $f$ not surjective and $g$ surjective so that the composition $g \circ f$ is surjective. 
    \item $f$ surjective and $g$ surjective so that the composition $g \circ f$ is not surjective. 
    \item $f$ injective and $g$ not injective so that the composition $g \circ f$ is injective.  
    \item $f$ not injective and $g$ injective so that the composition $g \circ f$ is injective.   
    \item $f$ injective and $g$ injective so that the composition $g \circ f$ is not injective.  
\end{enumerate}

\href{https://youtu.be/A6m4O8GvGy8}{Video Solution (Parts (a)-(c))}

\item Suppose $f : A \to B$ is a function. Prove or disprove each of the following statements:
\begin{enumerate}
    \item Let $X$ and $Y$ be subsets of $A$. If $X \cap Y = \emptyset$ then $f(X) \cap f(Y) = \emptyset$.
    \item Let $W$ and $Z$ be subsets of $B$. If $W \cap Z = \emptyset$ then $f^{-1}(W)  \cap f^{-1}(Z) = \emptyset$. 
\end{enumerate}

\href{https://youtu.be/pVmf28Cg_Y8}{Video Solution}
\end{enumerate}