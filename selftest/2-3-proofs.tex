\pagestyle{empty}

\boldsubsection{Reading Quiz Section \ref*{sec:proof}}

\begin{enumerate}
	\item In a \emph{proof by contrapositive} of $P\Longrightarrow Q$, we assume that (1) \underline{\phantom{$Q$ is false}\qquad\qquad} and deduce that (2) \underline{\phantom{$P$ is false}\qquad\qquad}.
  \begin{enumerate}
      \item \makebox[200pt][l]{(1) $\neg Q$ is true, \quad (2) $P$ is true \hfill (b) }
      (1) $Q$ is false, \quad (2) $P$ is true
      \setcounter{enumii}{2}
      \item \makebox[200pt][l]{(1) $\neg P$ is true, \quad (2) $\neg Q$ is true \hfill (d) }
      (1) $\neg Q$ is true, \quad (2) $\neg P$ is true
  \end{enumerate}
  
  
  \item A \emph{proof by contradiction} of $P\Longrightarrow Q$ begins by assuming that \underline{\phantom{$P$ is true and $Q$ is false}\qquad\qquad}.
  \begin{enumerate}
      \item \makebox[150pt][l]{$\neg P \vee Q$ is true \hfill (b) } $P \wedge \neg Q$ is true
      \setcounter{enumii}{2}
      \item \makebox[150pt][l]{$P\Longrightarrow Q$ is true \hfill (d) } $Q\Longrightarrow P$ is false
  \end{enumerate}
  
  
  \item You wish to prove that $x^2>4\Longrightarrow x>-2$. How might a contradiction argument begin?
  \begin{enumerate}
    \item Suppose all real numbers $x$ satisfy $x^2>4$ and $x>-2$.
    \item Suppose all real numbers $x$ satisfy $x^2>4$ and $x\le -2$.
    \item Suppose all real numbers $x$ satisfy $x^2\le 4$ and $x>-2$.
    \item Suppose $x$ is a real number satisfying $x^2>4$ and $x>-2$.
    \item Suppose $x$ is a real number satisfying $x^2>4$ and $x\le -2$.
    \item Suppose $x$ is a real number satisfying $x^2\le 4$ and $x>-2$.
  \end{enumerate}

	
  \item In which of the following situations would it be correct to invoke \emph{without loss of generality}? Select all that apply.
  \begin{enumerate}
		\item Suppose we are attempting to prove that for two integers $m$ and $n$, if either one is even, then so is the product. Without loss of generality we can assume that $n$ is even.
		\item We are trying to prove that for two integers $m$ and $n$, if both are odd, then so is the product. Without loss of generality we can assume that both $m$ and $n$ are equal to $2k+1$ for some integer $k$.
		\item Attempting to prove that if $m$ is even and $n$ is odd, then $mn$ is even. Without loss of generality we assume that $m=2$.
		\item Attempting to prove that if three boxes are painted either green or gold, there must be two boxes which are painted the same color. Without loss of generality, we can assume that the first box is painted green.
  \end{enumerate}
\end{enumerate}


\boldsubsection{Practice Problems Section \ref*{sec:proof}}

\begin{enumerate}
  \item Let $x$ and $y$ be integers. Prove: For $x^2+y^2$ to be even, it is necessary that $x$ and $y$ have the same parity (both even or both odd).
  
  \href{https://youtu.be/X3LG7pfEY_c}{Video Solution}
  
  
  \item Prove or disprove the following conjectures:
	\begin{enumerate}
	  \item The sum of any 3 consecutive integers is divisible by 3.
	  \item The sum of any 4 consecutive integers is divisible by 4.
	  \item The product of any 3 consecutive integers is divisible by 6.
	\end{enumerate}
	
  \href{https://youtu.be/R5zpsZoR16w}{Video Solution}
\end{enumerate}