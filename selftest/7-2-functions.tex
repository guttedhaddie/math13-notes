\pagestyle{empty}

\boldsubsection{Reading Quiz Section \ref*{sec:func2}}

\begin{enumerate}
  \item What does it mean for a \emph{relation} $\cR\subseteq A\times B$ to be a \emph{function}? Select all that apply.
  \begin{enumerate}
    \item $\dom(\cR)=A$
    \item $\range(\cR)=B$
    \item For any $a\in A$, if $(a,b_1),(a,b_2)\in\cR$, then $b_1=b_2$
    \item For any $b\in\range(\cR)$, if $(a_1,b),(a_2,b)\in\cR$, then $a_1=a_2$
  \end{enumerate}
    
  \item Let $f:A\to B$ be a function. If $f^{-1}:B\to A$ is a function, this means in particular that $\dom(f^{-1})=B$. This is equivalent to what property of $f$?
  \begin{enumerate}
		\item Injectivity
		\item Surjectivity
    \item $\dom(f)=A$
    \item That $f$ is a symmetric relation.
  \end{enumerate}
    
  \item True or False: a relation $\cR$ has a domain and range if and only if it is a function.
\end{enumerate}


\boldsubsection{Practice Problems Section \ref*{sec:func2}}

\begin{enumerate}
	\item Let $f:A\to A$ be a function. Viewed as a relation, if $f$ is symmetric, what can be said about $f$? 

	\item\begin{enumerate}
    \item Express the function $f:\R\to\R:x\mapsto x^2$ as a relation.
    \item What is the inverse relation $f^{-1}$?
    \item Use Definition \ref*{defn:func} to prove that the relation $f^{-1}$ is \emph{not} a function.
    \item Prove directly from Definition \ref*{defn:11} that $f$ is not injective and not surjective. Compare your arguments with your answer to part (c).
  \end{enumerate}
\end{enumerate}

