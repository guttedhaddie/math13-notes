\pagestyle{empty}

\boldsubsection{Reading Quiz Section \ref*{sec:equiv}}

\begin{enumerate}
	\item True or False: a relation $\sim$ on a set $X$ is \emph{reflexive} if $\exists x\in X$ such that $x\sim x$.


	\item Suppose that $x,y,z\in X$ and $\sim$ is an equivalence relation on $X$. Express each of the following assertions in terms of the properties satisfied by an equivalence relation.
  \begin{itemize}
    \item[(1)] $x\in[y]$ and $y\in[z]\implies x\in[z]$
    \item[(2)] $x\in[x]$
    \item[(3)] $x\in[y]\iff y\in[x]$
  \end{itemize}
  \begin{enumerate}
  	 \item (1) is reflexivity, (2) is symmetry, and (3) is transitivity
  	 \item (1) is transitivity, (2) is symmetry, and (3) is reflexivity
  	 \item (1) is transitivity, (2) is reflexivity, and (3) is antisymmetry
  	 \item (1) is transitivity, (2) is reflexivity, and (3) is symmetry
 	\end{enumerate}
  	
  	
	\item Suppose $\cR$ is an equivalence relation on a set $X$. Then $\cR^{-1}$ is \rule{2cm}{0.15mm} also an equivalence relation.
	\begin{enumerate}
    \item never
    \item sometimes
    \item always
	\end{enumerate}


	\item Which of the following statements are true? Select all that apply.
	\begin{enumerate}
    \item If $X$ is partitioned into the equivalence classes of some equivalence relation $\sim$, then each element of $X$ lies in some equivalence class $[x]$.
    \item Suppose that $X$ is partitioned into subsets and that $x,y,z\in X$. If $x,y$ lie in the same subset, and $y,z$ lie in the same subset of the partition, then it is possible for $x$ and $z$ to lie in different subsets.
    \item $\bigl\{\emptyset, \{a\}, \{b,c\}\bigr\}$ is a partition of $\{a,b,c\}$.
    \item Every subset in a partition of a set must have the same size.
	\end{enumerate}


	\item Which of the following describe the relationship between partitions and equivalence relations? Select all that apply.
	\begin{enumerate}
    \item Equivalence relations have nothing to do with partitions in general.
    \item For any set $X$ and equivalence relation $\sim$ on $X$, the quotient set $\quotient X\sim$ is a partition of $X$.
    \item There exists an infinite set $X$ and a partition $\{A_n\}$ of $X$ such that for any equivalence relation $\sim$ on $X$, there is $A \in \{A_n\}$ for which $A \neq [x]$ for any $x \in X$.
    \item Given any partition $\{A_n\}$ of $X$, there is an equivalence relation whose equivalence classes are exactly the subsets of $X$ in $\{A_n\}$.
	\end{enumerate}
	
	
	\item The set of real numbers $\R=\Q\cup(\R\setminus\Q)$ is partitioned into the subsets of rational and irrational numbers. Describe an equivalence relation on $\R$ whose equivalence classes form this partition:
	\begin{quote}
		$x\sim y\iff x-y\ \underline{\phantom{\in\Q\quad}}$
	\end{quote}
\end{enumerate}


\boldsubsection{Practice Problems Section \ref*{sec:equiv}}

\begin{enumerate}
	\item Define $\cR$ on $\N_{\ge 2}$ by $a\relR b$ if and only if $\gcd(a,b) > 1$. Determine whether $\cR$ is reflexive, symmetric, or transitive. 

	\item Let $\sim$ be the relation on $\R$ defined by $x \sim y$ if and only if $x - y \in \Z$.
	\begin{enumerate}
    \item Prove that $\sim$ is an equivalence relation.
    \item List three distinct elements of the equivalence class $[\frac 52]$. In general, what is an equivalence class $[x]$ as a set?
    \item Describe the quotient $\quotient{\R}\sim$.
	\end{enumerate}

	\item Let $X$ be a non-empty set. Then $\{X\}$ and $\bigl\{\{x\} : x \in X\bigr\}$ are both partitions of $X$. For both partitions, determine the equivalence relation whose equivalence classes form the subsets of the partition.

	\item Determine whether each collections $\{A_n\}$ partitions $\R^2$. Justify your answers and sketch several of the sets $A_n$.
	\begin{enumerate}
		\item $A_n=\bigl\{(x,y)\in\R^2:y=2x+n\bigr\}$, for $n\in\Z$.
	  \item $A_n=\bigl\{(x,y)\in\R^2:y=x^2+n\bigr\}$, for $n\in\R$.
	  \item $A_n=\bigl\{(x,y)\in\R^2:y=\cos(x-n)\bigr\}$, for $n\in\R$.
	\end{enumerate}
\end{enumerate}

