\pagestyle{empty}

\boldsubsection{Reading Quiz Section \ref*{sec:proof2}}

\begin{enumerate}
  \item When proving a non-existence statement, proof by contradiction is often useful because:
  \begin{enumerate}
    \item Contradiction is more powerful than direct proof.
    \item Direct and contrapositive proofs are too complicated.
    \item It allows us to assume such an object exists, hence providing an object that may be manipulated.
    \item It allows us to assume such an object does not exist, which is what the problem is asking for.
  \end{enumerate}
  
  
	\item State the following non-existence result in five different ways (as in Example \ref*{ex:easypolynosolns}).
	\begin{quote}
		There are no real numbers whose square is $-1$.
	\end{quote}
  
  
  \item In the proof that $\sqrt{2}$ is irrational, we assumed that $\sqrt{2}=\frac{m}{n}$ for integers $m$ and $n$ \emph{with no common factors}. Why is this justified?
  \begin{enumerate}
      \item Because no pair of integers ever has a common factor.
      \item Because any rational number $\frac mn$ can be seen, by canceling any common factors of $m$ and $n$, to be equal to a rational $\frac{m'}{n'}$ where $m'$ and $n'$ have no common factors.
      \item It is not justified, we have lost generality by making this assumption.
      \item Because $\sqrt{2}$ is irrational.
  \end{enumerate}
\end{enumerate}


Questions 3 and 4 are to help revise this chapter. Can you answer these without writing anything down? Can you persuade a friend that you are correct?

\begin{enumerate}\setcounter{enumi}{2}	
	\item We say that an integer $y$ is \emph{snake-like} if and only if there is some integer $k$ such that $y=(6k)^2+9$.
	\begin{enumerate}
	  \item Give three examples and three non-examples of snake-like integers.
	  \item Given $y\in\Z$, state the negation of the statement, ``$y$ is snake-like.''
	  \item Show that every snake-like integer is a multiple of 9.
	  \item Show that the statements, ``$n$ is snake-like,'' and, ``$n$ is a multiple of 9,'' are not equivalent.
	\end{enumerate}
  
  
	\item You meet three old men, Alain, Boris, and César, each of whom is a Truthteller or a Liar. Truthtellers speak only the truth; Liars speak only lies.\par
	You ask Alain whether he is a Truthteller or a Liar. Alain answers with his back turned, so you cannot hear what he says.
	\begin{quote}
		``What did he say?'' you ask Boris.\par
		Boris replies, ``Alain says he is a Truthteller.''\par
		César says, ``Boris is lying.''	
	\end{quote}
	Is César a Truthteller or a Liar? Explain your answer.
 
\end{enumerate}


\clearpage


\boldsubsection{Practice Problems Section \ref*{sec:proof2}}

\begin{enumerate}
  \item Prove: For an integer $n$ to be odd it is sufficient that its ones/units digit be odd.\par
  \href{https://youtu.be/9wHi0ojmvDg}{Video Solution}
  
  \item Consider the following claim and its ``proof.''
    \begin{thm*}{}{}
      If $x$ is a positive real number, then $x > 1$ if and only if $\frac 1x<1$.
    \end{thm*}
    \begin{proof}
      Suppose $\frac 1x<1$. Since $x$ is positive, multiplying both sides of this inequality by $x$ does not reverse the inequality and we obtain $1<x$.
    \end{proof}
    
    If the proof adequately demonstrates why the statement is true, explain why.
    Otherwise, identify any errors and explain how to correct them.\par
	\href{https://youtu.be/R_W3wwJRlmQ}{Video Solution}
\end{enumerate}