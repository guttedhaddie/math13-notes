\pagestyle{empty}

\boldsubsection{Reading Quiz Section \ref*{sec:proof2}}

\begin{enumerate}
  \item When proving a non-existence statement, i.e., proving that something \emph{does not} exist, proof by contradiction is often useful because \rule{5cm}{0.15mm}.
  \begin{enumerate}
    \item Contradiction is more powerful than a direct proof.
    \item Direct and contrapositive proofs are too complicated.
    \item It allows one to assume such an object exists, hence giving an object that can be manipulated.
    \item It allows one to assume such an object does not exist, which is exactly what the problem is asking for.
  \end{enumerate}
  
  
%   \item True or False: When proving a universal statement like $\forall x, P(x)$, it suffices to give an explicit example of an object $x$ for which $P(x)$ holds.
% 
% 
%   \item True or False: When proving an existential statement like $\exists x, P(x)$, it suffices to give an explicit example of an object $x$ for which $P(x)$ holds.


  \item In the proof that $\sqrt{2}$ is irrational, we started by assuming that $\sqrt{2}=\frac{m}{n}$ for integers $m$ and $n$ \emph{with no common factors}. Why is this justified?
  \begin{enumerate}
      \item Because no pair of integers ever has a common factor.
      \item Because any rational number $\frac mn$ can be seen, by canceling any common factors of $m$ and $n$, to be equal to a rational $\frac{m'}{n'}$ where $m'$ and $n'$ have no common factors.
      \item It is not justified, we have lost generality by making this assumption.
      \item Because $\sqrt{2}$ is irrational.
  \end{enumerate}
  
\end{enumerate}


\boldsubsection{Practice Problems Section \ref*{sec:proof2}}

\begin{enumerate}
  \item Let $n$ be an integer. Prove: for $n$ to be odd it is sufficient that its ones/units digit be odd.
  
  \href{https://youtu.be/9wHi0ojmvDg}{Video Solution}
  
  \item Critique the following proof.
    If the proof adequately demonstrates why the statement is true, explain why.
    Otherwise, identify any errors and explain how to correct them.
    \begin{thm*}{}{}
        If $x$ is a positive real number, then $x > 1$ if and only if $1/x < 1$.
    \end{thm*}
    \begin{proof}
        Suppose $1/x < 1$.
        Since $x$ is positive, multiplying both sides of this inequality by $x$ does not reverse the inequality and we obtain $1 < x$.
    \end{proof}
	
	\href{https://youtu.be/R_W3wwJRlmQ}{Video Solution}
\end{enumerate}