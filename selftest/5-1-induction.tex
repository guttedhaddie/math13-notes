\pagestyle{empty}

\boldsubsection{Reading Quiz Section \ref*{sec:induction}}

\begin{enumerate}
  \item In an induction proof of the fact that $P(n)$ is true for all $n \in \N$, the base case consists of proving that
	\begin{enumerate}
		\item $P(1)$ is false.
		\item $P(1)$ is true.
		\item for all $n$, $P(n) \implies P(n+1)$.
		\item $P(1) \implies P(2)$.
	\end{enumerate}
    
	\item In an induction proof of the fact that $P(n)$ is true for all $n \in \N$, the induction hypothesis is the assumption that
	\begin{enumerate}
		\item $P(1)$ is true.
		\item for all $n$, $P(n) \implies P(n+1)$.
		\item $P(n)$ is true for some fixed $n \in \N$.
		\item $P(n)$ is true for all $n \in \N$.
	\end{enumerate}
    
	\item True or False: in formal proofs, it is acceptable to write
  \[
		P(n)=\sum_{i=1}^n k=\frac{1}{2}n(n+1)
  \]
  as shorthand for ``$P(n)$ is the proposition $\sum_{i=1}^nk=\frac 12n(n+1)$.''
\end{enumerate}


\boldsubsection{Practice Problems Section \ref*{sec:induction}}

\begin{enumerate}
	\item\begin{enumerate}
    \item Prove by induction that $\forall n\in\N$ we have $3\mid(2^n+2^{n+1})$.
    \item Give a direct proof that $3\mid(2^n+2^{n+1})$ for all integers $n\ge 1$ \emph{and} for $n=0$.
    \item Look carefully at your proof for part (a). If you had started with the base case $n=0$ instead of $n=1$, would your proof still be valid?
    
    \href{https://youtu.be/Z4qxw2YTLzI}{Video Solution}
  \end{enumerate}
\end{enumerate}

