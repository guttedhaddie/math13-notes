\graphicspath{{4sets/asy/}}
\section{Sets and Functions}\label{chap:sets}


Sets are the fundamental building blocks of mathematics, supplying the language used to describe mathematical objects and to group objects according to shared characteristics. While our primary focus is learning to understand and employ set notation, the mathematical discipline of \emph{set theory} is far more ambitious: set theorists define all basic mathematical objects---\emph{numbers, addition, functions,} etc.---purely in terms of sets!\footnote{This is an impractical approach for most working mathematicians, most of the time. Within \emph{axiomatic set theory} it can take many, many pages of development to justify writing $1+1=2$: the issue is that rigorous definitions---using sets---are first required of the notions \emph{one, two, equals} and \emph{add}\ldots} We will only scratch the surface of set theory; indeed long before one can accept the benefit of such an approach, it is necessary to develop a significant level of familiarity with sets and their basic operations.


\subsection{Set Notation and Subsets}\label{sec:subset}

Without any attempt to define the meaning of \emph{object,} we offer a naïve definition.

\begin{defn}[lower separated=false, sidebyside, sidebyside align=top seam, sidebyside gap=0pt, righthand width=0.3\linewidth]{}{}
	A \emph{set} is a collection of objects, namely its \emph{elements} or \emph{members.}\smallbreak
	The proposition ``$x$ is an \emph{element}/\emph{member} of the set $A$,'' is written $x\in A$, usually read simply ``$x$ is in $A$.''\smallbreak
	If $y$ is a member of some other set, but not of $A$, we instead write $y\notin A$ (``$y$ is not in $A$''). In essence, this is the negation $\neg(y\in A)$.
	\tcblower
	\flushright\includegraphics{sets-01-venn}
\end{defn}

As in the definition, it is typical to use upper-case letters ($A,B,C,\ldots$) for abstract sets and lower-case letters for elements/members.\smallbreak

\emph{Venn diagrams} are useful for visualizing abstract sets. A set is represented by a region in the plane, with elements depicted by dots. The diagram in the definition represents a set $A$ comprising at least four elements $a_1,a_2,a_3$ and $x$. The element $y$ does not lie in $A$. 

\begin{example}{}{}
	Let $A$ be the set of (names of) US states. Then Michigan $\in A$ and Saskatchewan $\notin A$.
\end{example}

\begin{defn}{}{subset}
	Let $A$ and $B$ be sets.
	\begin{enumerate}
	  \item Sets are \emph{equal,} written $A=B$, when they have precisely the same elements.\par
	  \begin{minipage}[t]{0.74\linewidth}\vspace{-4pt}
	  	\item $A$ is a \emph{subset} of $B$, written $A\subseteq B$, when every element of $A$ is also an element of $B$.
	  	\item $A$ is a \emph{proper subset} of $B$ when both $A\subseteq B$ and $A\neq B$. To stress this, we could write $A\subsetneq B$. The Venn diagram represents a proper subset. 
	  \end{minipage}
	  \hfill
	  \begin{minipage}[t]{0.25\linewidth}\vspace{-15pt}
			\flushright\includegraphics{sets-02-vennsubset}
	  \end{minipage}
	\end{enumerate}
\end{defn}

\phantomsection\label{pg:setequalitysubset}
The following observations are merely translations of the definition---do they make sense to you?
\begin{gather*}
  \text{Equality:}\lstsp A=B \iff A\subseteq B\text{ and }B\subseteq A. \tag{$\ast$}\\
  \text{Subset:}\lstsp A\subseteq B \iff \bigl(x\in A\Longrightarrow x\in B\bigr) \iff \bigl(\forall x\in A, x\in B\bigr).\\
  \text{Not a subset:}\lstsp A\nsubseteq B \iff \exists x\in A\text{ for which } x\notin B.
\end{gather*}


\boldsubsubsection{Roster \& Set-Builder Notation}

\emph{Roster notation} is ideal for describing small sets: simply \emph{list} the elements \textbf{in any order} between curly brackets $\{\,,\,\}$.

\begin{example}{}{easysetnotation}
	$A=\{3,\frac 12\}$ is the set containing the numbers 3 and $\frac 12$. For instance $3\in A$, but $7\notin A$. Since \emph{order doesn't matter}, we could also write $A=\{\frac 12,3\}$. Now let Let $B=\{3\}$. Plainly,	
	\begin{itemize}
		\item $A\nsubseteq B$ since $\frac 12\in A$ and $\frac 12\notin B$\quad ($\exists x\in A$ for which $x\notin B$).
		\item $B\subseteq A$ since 3 (the only element of $B$) lies in $A$ ($\forall x\in A, x\in B$). Indeed $B\subsetneq A$ is a proper subset since $A\neq B$.
	\end{itemize}
\end{example}

Roster notation is rarely used in practice, due to its limited utility for larger sets. As an alternative\ldots\bigbreak

\emph{Set-builder notation} describes the elements of a set in terms of some common property. Suppose $\cU$ is some (already understood) set and $P(x)$ a propositional function with domain $\cU$, then
\[
	A:=\bigl\{x\in\cU:P(x)\bigr\} \tag{``$A$ is the set of $x$ in $\cU$ such that $P(x)$''}
\]
defines a set $A$ as \emph{the subset} of $\cU$ whose elements $x$ satisfy the property $P(x)$. A vertical separator $\mid$ can be used instead of a colon: in some contexts the choice is essential for clarity.

\begin{examples}{}{}
	\exstart We continue Example \ref{ex:easysetnotation}. Note that $2x^2-7x+3=(2x-1)(x-3)$ and recall that $\R$ represents the set of real numbers and $\Z$ the set of integers. In set-builder notation, our sets may be written
	\[
		A=\bigl\{x\in\R:2x^2-7x+3=0\bigr\},\qquad B=\bigl\{x\in\Z:2x^2-7x+3=0\bigr\}
	\]
	In this case the qualifying proposition $P(x)$ is ``$2x^2-7x+3=0$.''\vspace{-5pt}
	\begin{enumerate}\setcounter{enumi}{1}
	  \item[]We can also express the fact that $B\subseteq A$ in this notation (this time with a vertical separator),
	\[
		B=\bigl\{x\in A\bigm| x\in\Z\} \tag{``$B$ is the set of elements $x$ in $A$ such that $x$ is an integer"}
	\]
	
	\item Let $X=\{2,4,6\}$ and $Y=\{1,2,5,6\}$. There are many options for how to write these in set-builder notation. For instance:
	\[
		X=\bigl\{n\in\Z:\tfrac 12n\in\{1,2,3\}\bigr\},\qquad Y=\bigl\{n\in\Z\bigm|1\le n\le 6\text{ and }n\neq 3,4\bigr\}
	\]
	We now practice the opposite skill by converting five sets from set-builder to roster notation.
	\begin{align*}
		&S_1=\bigl\{x\in X:x\text{ is divisible by 4}\bigr\}=\{4\}
		&&
		S_2=\bigl\{y\in Y:y\text{ is odd}\bigr\}=\{1,5\}
		\\
		&S_3=\bigl\{x\in X\bigm| x\in Y\bigr\}=\{2,6\}
		&&
		S_4=\bigl\{x\in X:x\notin Y\bigr\}=\{4\}
		\\
		&S_5=\bigl\{y\in Y\bigm| y\text{ is odd and $y-1\in X$}\}=\{5\}
	\end{align*}
	Can you find alternative descriptions in set-builder notation for the sets $S_1,\ldots,S_5$ above? Take your time getting used to this notation, since translating between various descriptions of a set is essential to reading mathematics.
	
	\goodbreak
	
	\item We use the set $C=\{0,1,2,3,\ldots,24\}$ to describe $D=\{n\in\Z:n^2-3\in C\}$ in roster notation. Start by expanding the criterion for membership in $D$:
  \[
  	n^2-3\in C\iff n^2\in\bigl\{3,4,5,\ldots,25,26,27\bigr\}
  \]
  Since $n$ must be an integer, it follows that $D=\{\pm 2,\pm 3,\pm 4,\pm 5\}$.
  
  \item To express $E=\{0,2,6,12,\ldots\}$ in set-builder notation, we might spot a pattern and decide that
  \[
  	E=\bigl\{n\in\Z: n=m(m+1) \text{ for some integer }m\ge 0\bigr\}
  \]
  Unfortunately, we cannot guarantee our correctness! Perhaps the correct formula is
  \[
  	n=m(m+1)+m(m-2)(m-6)(m-12)\quad \text{(!!)}
  \]
	In the first case the next term in the sequence is $4\cdot 5=20$, whereas in the second it is $20+128=148$. For larger sets, the clarity afforded by set-builder notation is essential!

	\end{enumerate}
\end{examples}



\boldsubsubsection{Common Sets of Numbers}

We've used some of this notation already and much of the rest should be familiar.\vspace{-5pt}

\begin{quote}\def\arraystretch{1.15}
	\begin{tabular}{@{}ll}
		\emph{Natural numbers}&$\N=\bigl\{1,2,3,4,\ldots\bigr\}$ is the set of \emph{positive} integers.\\
		\emph{Integers}&$\Z=\bigl\{\ldots,-3,-2,-1,0,1,2,3,\ldots\bigr\}$.\\
		\emph{Rational numbers}&$\Q=\bigl\{\frac mn:m\in\Z\text{ and }n\in\N\bigr\} =\bigl\{\frac ab\bigm| a,b\in\Z\text{ and }b\neq 0\bigr\}$.\\
		\emph{Real numbers}&$\R$. Even a rudimentary definition is too involved for this text.\footnotemark\\
		\emph{Complex numbers}&$\C=\bigl\{x+iy:x,y\in\R\bigr\}$ where $i^2=-1$. We won't make use of these.
	\end{tabular}
\end{quote}

\vspace{-10pt}
	
\footnotetext{%
	We assume the reader is comfortable with the real line where number corresponds to \emph{length.} A rigorous development of $\R$ is a matter for an upper-division analysis course.
}

\begin{examples}{}{}
	\exstart For instance: \ $7\in\N$, \ $\pi\in\R$, \ $-\frac 79\notin\Z$, \ $\sqrt 2\notin\Q$ \ and \ $3+\sqrt 5i\in\C$.
	\begin{enumerate}\setcounter{enumi}{1}
	  \item The basic symbols can be modified is natural ways. For example:
\vspace{-3pt}
		\begin{itemize}
		  \item $\N_0=\bigl\{0,1,2,3,4,\ldots\bigr\}=\Z^+_0=\bigl\{x\in\Z:x\ge 0\bigr\}$. By some called the \emph{whole numbers} ($\mathbb W$).
			\item $\Z_{\ge 5}=\bigl\{5,6,7,8,\ldots\bigr\} =\bigl\{x\in\Z:x\ge 5\bigr\}$ denotes the integers greater than or equal to 5.
		  \item $\R^+=\bigl\{x\in\R:x>0\bigr\}$ is the set of positive real numbers.
			\item $4\Z=\bigl\{\ldots,-8,-4,0,4,8,12,\ldots\bigr\} =\bigl\{x\in\Z: 4\mid x\bigr\}$ is the set\footnotemark{} of integer multiples of 4.\par
			This can also be used for non-integer multiples, e.g. $\pi\Z=\bigl\{\ldots,-\pi,0,\pi,2\pi,\ldots\bigr\}$. 
			\item $2\Z+1=\bigl\{x\in\Z:x\equiv 1\pmod 2\bigr\}$ is the set of odd integers.
		\end{itemize}
		
		\item \emph{Intervals} are commonly encountered subsets of the real numbers. For instance:
% \vspace{-3pt}
		\begin{itemize}
		  \item $[1,\pi]=\bigl\{x\in\R\bigm|1\le x\le \pi\bigr\}$ is a \emph{closed} interval 
		  \item $[-4,7.21)=\{x\in\R\bigm|-4\le x<7.21\}$ is a \emph{half-open} interval.
		  \item $(-\infty, \sqrt 2) =\{x\in\R\bigm|x<\sqrt 2\}$ is an \emph{infinite (open)} interval.
		\end{itemize}
	\end{enumerate}
\end{examples}

\vspace{-5pt}

\footnotetext{Be careful with the second version---the colon is the \emph{such that} separator while $\mid$ denotes the \emph{property} ``4 divides $x$.''}

\goodbreak

In view of the natural subset relationships $\N\subsetneq\Z\subsetneq\Q\subsetneq\R\subsetneq\C$, we consider a simple result.

\begin{lemm}{Transitivity of Subset}{subsettrans}
	Suppose $A\subseteq B$ and $B\subseteq C$. Then $A\subseteq C$.
\end{lemm}

\begin{proof}
	Think back to the criteria following Definition \ref{defn:subset}. Suppose $A\subseteq B$ and $B\subseteq C$. Then
	\[
		x\in A \overset{(A\subseteq B)}{\implies} x\in B\overset{(B\subseteq C)}{\implies} x\in C
	\]
	We conclude that $A\subseteq C$.
\end{proof}

Compare this to Exercise \ref*{sec:prop}.\ref{exs:iftransitive}: if we rewrite each subset relation as an implication, the proof structure becomes $(x\in A\Rightarrow x\in B)\wedge (x\in B\Rightarrow x\in C)\Longrightarrow (x\in A\Rightarrow x\in C)$. This is typical of basic set results: a translation often reduces the problem to one of the standard rules of logic.



\boldsubsubsection{Cardinality and the Empty Set}

It is helpful to introduce some terminology to describe the \emph{size} of a set.

\begin{defn}{}{card1}
	A \emph{finite set} contains a finite number of elements: this number is its \emph{cardinality} $\nm A$.\par
	A set with infinitely many elements is said to be an \emph{infinite set.}\smallbreak
	The symbol $\emptyset$ denotes the \emph{empty set}: a set containing no elements (cardinality zero, $\nm\emptyset=0$).
\end{defn}

\begin{examples}{}{}
	\exstart If $A=\bigl\{1,3,\pi,\sqrt 2,103\bigr\}$, then $\nm A=5$.
	\begin{enumerate}\setcounter{enumi}{1}
		\item Let $B=\bigl\{4,\{1,2\},\{3\}\bigr\}$. The elements of $B$ are $4$, $\{1,2\}$ and $\{3\}$, therefore $\nm B=3$. It doesn't matter that the \emph{element} $\{1,2\}\in B$ is also a set!\par
		\begin{minipage}[t]{0.59\linewidth}\vspace{-5pt}
			\item Recall some basic trigonometry:
			\[
				\left\{x\in[0,4\pi]:\cos x=\frac 12\right\}=\left\{\frac{\pi}3,\frac{5\pi}3,\frac{7\pi}3,\frac{11\pi}3\right\}
			\]
			has cardinality 4.
		\end{minipage}
		\hfill
		\begin{minipage}[t]{0.4\linewidth}\vspace{-10pt}
			\flushright\includegraphics{sets-03-cos}
		\end{minipage}
		
		\item There are many, many representations of the empty set in set-builder notation: for example
		\[
			\emptyset =\bigl\{x\in\R:x^2=-1\bigr\} = \bigl\{x\in\N:x^2+3x+2=0\bigr\}=\bigl\{n\in\N:n<0\bigr\}
		\]
		In general, if $X$ is any set and $P(x)$ is false for all $x\in X$, then\footnotemark{} $\emptyset=\bigl\{x\in X:P(x)\bigr\}$.
	\end{enumerate}
\end{examples}

\footnotetext{%
	The existence of the empty set is sometimes considered an \emph{axiom}: an assumption made without proof. Provided one accepts that set-builder notation always defines a set (itself an axiom!) and that at least one set $X$ exists, the empty set may be \emph{defined} as in the example; a suitable property $P(x)$ might be something like ``$x\notin\{x\}$.''%
}


Cardinality is very simple for subsets of finite sets: if $B$ is finite, so is any subset, and we have
\[
	A\subseteq B\implies \nm A\le \nm B
\]
(Is it obvious why the converse is false?!) For infinite sets, cardinality is more subtle; we'll return to this matter and uncover some of its bizarre and fun consequences in Chapter \ref{chap:cantor}.
\bigbreak
\goodbreak

We finish with a couple of simple results regarding the empty set.

\begin{lemm}{}{emptysetunique}
	Let $A$ be a set.
	\begin{enumerate}
	  \item If $\nm A=0$, then $A=\emptyset$. The empty set is the \textbf{unique set} with cardinality zero.
	  \item $\emptyset\subseteq A$ and $A\subseteq A$
	\end{enumerate}
\end{lemm}

\begin{proof}
	Consider the claim $\emptyset\subseteq A$. By the observations following Definition \ref{defn:subset}, this means
	\[
		x\in\emptyset\implies x\in A
	\]
	This is true (for any set $A$!) since there are \emph{no elements} $x$ satisfying the hypothesis.\footnotemark{}
	\begin{enumerate}
	 	\item Suppose $A$ has cardinality zero. Repeating and combining with the above observation, we see that $\emptyset\subseteq A$ and $A\subseteq\emptyset$. We conclude that $A=\emptyset$.
	 	\item We already know that $\emptyset\subseteq A$. For the second part, simply observe that $x\in A\Longrightarrow x\in A$.\qedhere
	\end{enumerate}
\end{proof}

\footnotetext{If $P(x)$ is always false, then $(\forall x)\ P(x)\Longrightarrow Q(x)$ is true (Definition \ref{defn:implies}). This is called a \emph{vacuous} (empty) theorem.}

% 
% 
% \begin{examples}{}{}
% 	\begin{enumerate}\setcounter{enumi}{1}
%   \item Are the following sets equal?
%   \[E=\{n^2+2\in\Z:\text{$n$ is an odd integer}\},\qquad F=\{n\in\Z:n^2+2\text{ is an odd integer}\}.\]
%   It may help to first construct a table listing some of the values of $n^2+2$:
%   \[\begin{array}{c|c|c}
%   n&n^2&n^2+2\\\hline
%   \pm 1&1&3\\
%   \pm 3&9&11\\
%   \pm 5&25&27\\
%   \pm 7&49&51\\
%   \pm 9&81&83\\[-5pt]
%   \vdots&\vdots&\vdots
%   \end{array}\]
%   The set $E$ consists of those integers of the form $n^2+2$ where $n$ is an odd integer. By the table,
%   \[E=\{3,11,27,51,83,\ldots\}.\]
%   On the other hand, $F$ includes all those integers $n$ such that $n^2+2$ is odd. It is easy to see that
%   \[n^2+2\text{ is odd}\iff n^2\text{ is odd}\iff n\text{ is odd.}\]
%   Thus $F$ is simply the set of all odd integers:
%   \[F=\{\pm 1,\pm 3,\pm 5,\pm 7,\ldots\}=2\Z+1.\]
%   Plainly the two sets are not equal.
% 
%   \item $\{x\in\R:x^2-1=0\}\subseteq \{y\in\R:y^2\in\N\}$.\\
%   To make sense of this relationship, convert to roster notation: we obtain
%   \[\{-1,1\}\subseteq\{\pm\sqrt 1,\pm\sqrt 2,\pm\sqrt 3,\pm\sqrt 4,\ldots\}.\]
%   \item If $m$ and $n$ are positive integers, then $m\Z\subseteq n\Z\iff n\mid m$. Make sure you're comfortable with this! For example, $4\Z\subseteq 2\Z$ since every multiple of 4 is also a multiple of 2.
% \end{enumerate}
% \end{examples}


% \paragraph{Self-test Questions}
% 
% \begin{enumerate}
%   \item True or false: An open interval contains its endpoints.
%   \item True or false: $\{x\in\R:x^2<0\}$ is a representation of the empty set.
%   \item True or false: $\{x\in\Z:x\in[0,4)\}=\{0,1,2,3,4\}$.
% \end{enumerate}


\begin{exercises}{}{}
	A reading quiz and several questions with linked video solutions can be found \href{http://www.math.uci.edu/~ndonalds/math13/selftest/4-1-subset.html}{online}.

	\begin{enumerate}
	  \item Describe the following sets in roster notation: that is, list their elements.
		\begin{enumerate}
		  \item \makebox[215pt][l]{$\bigl\{x\in\N:x^2\le 3x\bigr\}$\hfill (b)} \ $\bigl\{n\in\{0,1,2,3,\ldots,19\}:n+3\equiv 5\spmod 4\bigr\}$
		  \setcounter{enumii}{2}
		  \item \makebox[215pt][l]{$\bigl\{n\in\{-2,-1,0,1,\ldots,23\}:4\mid n^2\bigr\}$\hfill (d)} \ $\bigl\{x\in \frac 12\Z: 0\le x\le 4\text{ and }4x^2\in 2\Z+1\bigr\}$
		  \setcounter{enumii}{4}
		  \item $\bigl\{y\in\R:y=x^2\text{ for some $x\in\R$ with } x^2-3x+2=0\bigr\}$
		\end{enumerate}
			
			
		\item Describe the following sets in set-builder notation (\emph{look for a pattern}).
		\begin{enumerate}
		  \item \makebox[180pt][l]{$\bigl\{\ldots,-3,0,3,6,9,\ldots\bigr\}$\hfill (b)} \ $\bigl\{-3,1,5,9,13,\ldots\bigr\}$
		  \setcounter{enumii}{2}
		  \item $\bigl\{1,\frac 13,\frac 17,\frac 1{15},\frac 1{31},\ldots\bigr\}$
		\end{enumerate}
		  
	
	  \item Each of the following sets of real numbers is a single interval. Determine the interval.
		\begin{enumerate}
		  \item \makebox[180pt][l]{$\bigl\{x\in\R:x>3\text{ and }x\le 17\bigr\}$\hfill (b)} \ $\bigl\{x\in\R:x\nleq 3\text{ or }x\le 17\bigr\}$
		  \setcounter{enumii}{2}
		  \item \makebox[180pt][l]{$\bigl\{x^2\in\R:x\neq 0\bigr\}$ \hfill (d)} \ $\bigl\{x\in\R^-:x^2\ge 16\text{ and }x^3\le 27\bigr\}$
		\end{enumerate}
			
			
		\item Is the set $\{x\in\Z:-1\le x<43\}$ finite or infinite? If finite, what is its cardinality?
				
					
		\item What is the cardinality of the set $\Bigl\{\emptyset,\bigl\{\emptyset\bigr\},\bigl\{\emptyset,\{\emptyset\}\bigr\}\Bigr\}$? \ What are its elements?
		
		\item Let $A=\emptyset$, \ $B=\{A\}$, \ $C=\bigl\{\{A\}\bigr\}$ \ and \ $D=\bigl\{A,\{0\},\{0,1\}\bigr\}$.\par
	  Answer the following true or false:
	  \begin{enumerate}
	    \item \makebox[80pt][l]{$0\in A$\hfill (b)} \ \makebox[80pt][l]{$A\in B$\hfill (c)} \ \makebox[80pt][l]{$A\in C$\hfill (d)} \ \makebox[80pt][l]{$B\in C$ \hfill (e)} \ $A\in D$
	    \setcounter{enumii}{5}
	    \item \makebox[80pt][l]{$B\in D$\hfill (g)} \ \makebox[80pt][l]{$0\in D$\hfill (h)} \ \makebox[80pt][l]{$\{0\}\in D$\hfill (i)} \ $\{1\}\in D$
	  \end{enumerate}
	  
	  
	  \item List all the \emph{proper} subsets of $\{1,2,3\}$.
	  
	    
		\goodbreak
	   
		   
		\item Let $A,B,C,D$ be the following sets:
	  \begin{align*}
	  	&A=\{-4,1,2,4,10\}
	  	&&B=\bigl\{m\in\Z:\nm m\le 12\bigr\}\quad \text{(\emph{absolute value of $m$})}\\
	  	&C=\bigl\{n\in\Z:n^2\equiv 1\spmod 3\bigr\}
	  	&&D=\bigl\{t\in\Z:t^2+3\in [4,20)\bigr\}  
	  \end{align*}
	  Of the 12 subset relations $A\subseteq B$,\ $A\subseteq C,\ldots, D\subseteq C$, which are true and which false?
	  
	  \item Let $A=\bigl\{1,2,\{1,2\},\{3\}\bigr\}$ and $B=\{1,2\}$. Answer the following true or false:
	  \begin{enumerate}
	    \item \makebox[100pt][l]{$B\in A$\hfill (b)} \ \makebox[100pt][l]{$B\subseteq A$\hfill (c)} \ \makebox[100pt][l]{$3\in A$\hfill (d)} \ $\{3\}\subseteq A$
	    \setcounter{enumii}{4}
	    \item \makebox[100pt][l]{$\{3\}\in A$\hfill (f)} \ \makebox[100pt][l]{$\emptyset\subseteq A$\hfill (g)} \ $\emptyset\in A$
	  \end{enumerate}
	  
	  \item Let $A=\{0,2,4,6,8,10\}$. Write the set $B=\{X\subseteq A:|X|=2\}$ in roster notation.
	  
	    
	  \item\begin{enumerate}
	    \item Suppose $A\subseteq B\subseteq C\subseteq A$. Show that $A=B=C$.
	    \item Is it possible for sets $A,B,C$ to satisfy $A\subsetneq B\subseteq C\subseteq A$? Why/why not?
	  \end{enumerate}

	
		\item Let $A=\{\text{1,2,3,4}\}$, and let $B =\bigl\{\{x,y\}:x,y\in A\bigr\}$.
		\begin{enumerate}
	  	\item Describe $B$ in roster notation (\emph{what happens when $x=y$?}).
			\item Find the cardinalities of the following sets:
			\[
				C=\Bigl\{\bigl\{x,\{y\}\bigr\}:x,y\in A\Bigr\}
				\quad\text{and}\quad
				D=\biggl\{\Bigl\{\bigl\{x,\{y\}\bigr\}:x,y\in A\Bigr\}\biggr\}
			\]
		\end{enumerate}
  
  
  	\item Let $A=\{x\in\R:x^3+x^2-x-1=0\}$ and $B=\{x\in\R:x^4-5x^2+4=0\}$. Are either of the relations $A\subseteq B$ or $B\subseteq A$ true? Explain.
  
  
  	\item For which real numbers $x>0$ do we have $[0,x]\subsetneq[0,x^2]$? Prove your assertion.
  
  
  	\item Let $m,n\in\N$. Prove: $m\Z\subseteq n\Z\iff n\mid m$.

  
  	\item\label{ex:mirrored} Given $A\subseteq\Z$ and $x\in\Z$, we say that $x$ is $A$-mirrored if and only if $-x\in A$. Also define
  	\[
			M_A:=\bigl\{x\in\Z: x\text{ is $A$-mirrored}\bigr\}
		\]
		\begin{enumerate}
	  	\item What does it mean for $x$ \emph{not} to be $A$-mirrored?
	  	\item Find $M_B$ given $B=\{0,1,-6,-7,7,100\}$.
	  	\item Assume that $A \subseteq\Z$ is closed under addition: for all $x,y\in A$, we have $x+y\in A$. Show that $M_A$ is closed under addition.
	  	\item In your own words, under which conditions is $A=M_A$?
		\end{enumerate}

  \item Define the set $[1]=\bigl\{x\in\Z: x\equiv 1\spmod 5\bigr\}$.
		\begin{enumerate}
		  \item Describe the set $[1]$ in roster notation.
		  \item Compute the set $M_{[1]}$, as defined in Exercise \ref{ex:mirrored}. Is $M_{[1]}$ equal to $[1]$?
			\item Now consider the set $[10]=\{x\in\Z:x\equiv 10\spmod 5\}$. Are the sets $[10]$ and $M_{[10]}$ equal? Prove or disprove.
  	\end{enumerate}


		\item Consider the set $A=\{a,b,c,d\}$. 
    \begin{enumerate}
      \item Of each cardinality 0, 1, 2, 3 and 4, how many subsets has $A$? Is there a pattern?
        
      \item Completely expand the polynomial $(1 + x)^4$. What do you notice about the coefficients? 
    \end{enumerate}

	\end{enumerate}
\end{exercises}

\clearpage



\subsection{Unions, Intersections and Complements}\label{sec:union}

In this section we construct new sets from old, modeled on the logical \emph{and, or,} and \emph{not} (Definition \ref{defn:andornot}).

\iffalse

\begin{defn}{}{unionint}
	Let $A,B$ be sets. 
	\begin{enumerate}\itemsep0pt\setcounter{enumi}{0}
	  \item The \emph{union} of $A,B$ is the set of elements in $A$ or in $B$:
		\[
			\makebox[210pt][l]{$A\cup B:=\bigl\{x:x\in A\text{ or }x\in B\bigr\}$\hfill} (x\in A\cup B\iff x\in A\text{ or }x\in B)
		\]
		\item The \emph{intersection} of $A,B$ is the set of elements lying in both $A$ and $B$:
		\[
			\makebox[210pt][l]{$A\cap B:=\bigl\{x:x\in A\text{ and }x\in B\bigr\}$\hfill} (x\in A\cap B\iff x\in A\text{ and }x\in B)
		\]
		We say that $A,B$ are \emph{disjoint} if $A\cap B=\emptyset$.
	  \item The \emph{complement} of $A$ is the set of elements not in $A$ (with respect to some \emph{universal set\footnotemark} $\cU$):
		\[
			\makebox[210pt][l]{$\comp A:=\bigl\{x\in\cU:x\notin A\bigr\}$ \hfill} (x\in\comp A\iff x\notin A \ (\text{and }x\in\cU))
		\]
		\item The \emph{complement of $A$ relative to $B$} is the set of elements in $B$ which are not in $A$:
		\[
			\makebox[210pt][l]{$B\setminus A=\bigl\{x\in B:x\notin A\bigr\} =B\cap\comp A$\hfill} (x\in B\setminus A\iff x\in B\text{ and } x\notin A)
		\]
		This can be read ``$B$ minus $A$'' (some authors write $B-A$). Similarly $\comp A=\cU\setminus A$, etc.
	\end{enumerate}
	
	\begin{center}
		\includegraphics[scale=0.95]{sets-05-venncomp}
		\qquad\qquad
		\includegraphics[scale=0.95]{sets-04-vennunion}
	\end{center}
\end{defn}

\fi

\begin{defn}{}{unionint}
	Let $A,B$ be sets. 
	\begin{enumerate}
	  \item The \emph{union} of $A,B$ is the set of elements lying in $A$ or in $B$ (or in both):
		\[
			\makebox[210pt][l]{$A\cup B=\bigl\{x:x\in A\text{ or }x\in B\bigr\}$\hfill} (x\in A\cup B\iff x\in A\text{ or }x\in B)
		\]
		\item The \emph{intersection} of $A,B$ is the set of elements lying in both $A$ and $B$:
		\[
			\makebox[210pt][l]{$A\cap B=\bigl\{x:x\in A\text{ and }x\in B\bigr\}$\hfill} (x\in A\cap B\iff x\in A\text{ and }x\in B)
		\]
		We say that $A,B$ are \emph{disjoint} if $A\cap B=\emptyset$.\par
		\begin{minipage}[t]{0.72\linewidth}\vspace{-5pt}
	  	\item The \emph{complement} of $A$ is the set of elements not in $A$ (with respect to some \emph{universal set\footnotemark} $\cU$):
	  	\begin{align*}
				&\comp A=\bigl\{x\in\cU:x\notin A\bigr\} \tag*{($x\in\comp A\iff x\in\cU \ \text{and }x\notin A$)\quad}
			\end{align*}
			In the Venn diagram, the outer box represents the universal set $\cU$.
			\item The \emph{complement of $A$ relative to $B$} is the set of elements in $B$ which do not lie in $A$:
			\begin{align*}
				B\setminus A&=B\cap \comp{A} \tag*{($x\in B\setminus A\iff  x\in B\text{ and } x\notin A$)\quad}\\
				&=\bigl\{x\in B:x\notin A\bigr\}
			\end{align*}
			This can be read ``$B$ minus $A$'' (some authors indeed write $B-A$). 
			Note also that $\comp A=\cU\setminus A$; the distinction is that the relative complement $B\setminus A$ does not require that $A$ be a subset of $B$.
	  \end{minipage}
	  \hfill
		\begin{minipage}[t]{0.25\linewidth}\vspace{-12pt}
			\flushright
			\includegraphics[scale=0.95]{sets-05-venncomp}
			\bigbreak
			\includegraphics[scale=0.95]{sets-04-vennunion}
		\end{minipage}
	\end{enumerate}
\end{defn}

\footnotetext{%
	The elements $x$ must live somewhere! Without a universal set, should, say, $\comp{\{7\}}$ be the set of \emph{integers} except 7, \emph{real numbers} except 7, etc.? Often $\cU$ is naturally assumed: e.g., $\R$ in calculus. The universal set is not needed for parts 1, 2 \& 4: that the union is a set is typically an axiom, while intersections and relative complements are subsets of pre-existing sets.%
}

Observe the notational similarities with logic: $\cup$ looks a bit like $\vee$ (OR); $\cap$ like $\wedge$ (AND). The second Venn diagram suggests the identities
\[
	A=(A\setminus B)\cup (A\cap B)
	\quad\text{and}\quad
	B=(B\setminus A)\cup(A\cap B)
\]
While these indeed hold, note that \textcolor{red}{a Venn diagram isn't a proof}: set identities must rigorously be proved in the style of the upcoming theorems.

\begin{examples}{}{basicunionint}
	Reading set notation is one of the most basic requirements of abstract mathematics. Make sure you understand why the following examples are correct before moving on!
	
	\begin{enumerate}
	  \item Let $\cU=\{1,2,3,4,5\}$, $A=\{1,2,3\}$, and $B=\{2,3,4\}$. Then
	\begin{align*}
		&\comp A=\{4,5\} &&\comp B=\{1,5\} &&B\setminus A=\{4\} &&A\setminus B=\{1\}\\
		&A\cup B=\{1,2,3,4\} &&A\cap B=\{2,3\} &&A\cap\comp B=\{1\} &&\comp A\cup\comp B=\{1,4,5\}
	\end{align*}
	
	
	\goodbreak
	  
		\item\label{ex:basicunionint2} Using interval notation, let $\cU=[-4,5]$, \ $A=[-3,2]$, \ and \ $B=[-4,1)$. Then\par
		\begin{minipage}[t]{0.3\linewidth}\vspace{-20pt}
			\begin{gather*}
				\comp A=[-4,-3)\cup (2,5]\\
				\comp B=[1,5]\\
				A\setminus B=[1,2]\\
				B\setminus A=[-4,-3)\\
				A\cup B=[-4,2]\\
				A\cap B=[-3,1) 
			\end{gather*}
		\end{minipage}
		\hfill
		\begin{minipage}[t]{0.69\linewidth}\vspace{-6pt}
			\flushright\includegraphics[scale=0.78]{sets-13-intervalex}
		\end{minipage}
		\smallbreak
		While you should believe these from the picture, they also make for good logic practice. E.g.,
		\begin{align*}
			x\in \comp A&\iff x\notin A\iff \neg\bigl(x\in A\bigr) \iff\neg\bigl(-3\le x \text{ and }x\le 2\bigr)\\
			&\iff x<-3\text{ or }x>2 \tag{de Morgan's law, Theorem \ref{thm:demorgan}}\\
			&\iff x\in[-4,-3)\cup (2,5] \tag{remember that $-4\le x\le 5$ ($x\in\cU$)}
		\end{align*}
		The argument illustrates the basic strategy for set computations \& proofs: convert claims to propositions (Definition \ref{defn:unionint} parentheses) and apply basic logic (Theorem \ref{thm:demorgan}, page \pageref{pg:asidelogicalgebra}, etc.). Alternatively, you can prove each direction separately: this would be to show that each of the sets $\comp A$, $[-4,-3)\cup(2,5]$ is a subset of the other (page \pageref{pg:setequalitysubset}, ($\ast$)).
		
 	\item\label{ex:basicunionint3} Let $A=(-\infty,3)$ and $B=[-2,\infty)$ in interval notation. Then $A\cup B=\R$ and $A\cap B=[-2,3)$. We show the first: for variety, this time we observe that each side is a subset of the other.
	\begin{description}
		\item[\normalfont($\subseteq$):] $A\cup B\subseteq\R$ is trivial, since everything in $A,B$ is a real number ($\R$ is the universal set).
		\item[\normalfont($\supseteq$):] Let $x\in\R$. If $x<3$, then $x\in A$. Otherwise, $x\ge 3\Longrightarrow x\ge -2\Longrightarrow x\in B$. Either way, $x\in A\cup B$.
	\end{description}

% 		\begin{gather*}
% 			x\in A\cap B\implies x<3\text{ and }x\ge -2\implies -2\le x<3\implies x\in [-2,3)\\
% 			x\in [-2,3)\implies 
% 		\end{gather*}
	\end{enumerate}
\end{examples}


For the remainder of this section, we summarize the basic rules of set algebra.

\begin{thm}{Union/intersection rules}{setbasic}
	Let $A,B,C$ be sets. Then:
	\begin{enumerate}\itemsep1pt
		\item \makebox[220pt]{$\emptyset\cup A=A$ \ and \ $\emptyset\cap A=\emptyset$ \hfill 2.} \ $A\cap B\subseteq A\subseteq A\cup B$
		\setcounter{enumi}{2}
		\item \makebox[220pt]{$A\cup B=B\cup A$ \ and \ $A\cap B=B\cap A$ \hfill 4.} \ $A\cup A=A\cap A=A$
		\setcounter{enumi}{4}
		\item $A\cup (B\cup C)=(A\cup B)\cup C$ \ and \ $A\cap (B\cap C)=(A\cap B)\cap C$
		%\item $A\cup A=A\cap A=A$
		\item $A\subseteq B\Longrightarrow A\cup C\subseteq B\cup C$ \ and \ $A\cap C\subseteq B\cap C$
	\end{enumerate}
\end{thm}

If you don't believe a result, \emph{visualize} it with a Venn diagram. We prove part 2 and half of 6.

\begin{proof}
\begin{enumerate}
  \item[2.] We offer direct proofs of the two results: $A\cap B\subseteq A$ and $A\subseteq A\cup B$.
	\begin{itemize}
		\item Suppose $x\in A\cap B$.\hfill(goal: want to prove $x\in A\cap B\Rightarrow x\in A$)\par
		Then $x\in A$ and $x\in B$.\hfill(definition of intersection)\par
		Plainly $x\in A$. We conclude that $A\cap B\subseteq A$\hfill(definition of subset)
		\item Suppose $y\in A$.\hfill(goal: prove $y\in A\Rightarrow y\in A\cup B$)\par
		Then ``$y\in A$ or $y\in B$'' (is true), whence $y\in A\cup B$.\hfill(definition of union/or)\par
	  We conclude that $A\subseteq A\cup B$.
	\end{itemize}
	
	\goodbreak
	
	\item[6.] (first half)\lstsp Suppose $A\subseteq B$. We wish to prove that $x\in A\cup B\Longrightarrow x\in A\cup C$. However,
	\begin{align*}
		x\in A\cup C&\implies x\in A\text{ or }x\in C\tag{definition of union}\\
		&\implies x\in B\text{ or }x\in C\tag{since $A\subseteq B$}\\
		&\implies x\in B\cup C \tag*{\qedhere}
	\end{align*}
\end{enumerate}
\end{proof}



Our next batch of rules describe how complements interact with other set operations: parts 1 and 2 are \emph{de Morgan's laws for sets}; unsurprisingly, their proofs depend on the corresponding laws of logic.


\begin{thm}[lower separated=false, sidebyside, sidebyside align=top seam, sidebyside gap=0pt, righthand width=0.4\linewidth]{Complement rules}{setcomp}
	Let $A,B$ be sets. Then:
	\begin{enumerate}\itemsep1pt
		\item $\comp{(\textcolor{Magenta}{A\cap B})}=\textcolor{blue}{\comp A}\cup \textcolor{red}{\comp B}$ (pictured)
		\item $\comp{(A\cup B)}=\comp A\cap \comp B$
		\item $\smash[t]{\comp{(\comp A)}}=A$
		\item $A\subseteq B\iff \comp B\subseteq \comp A$
	\end{enumerate}
	\tcblower
	\flushright
	\includegraphics[scale=0.95]{sets-08-venndemorgan2}
\end{thm}

\begin{proof}
	We prove only part 1. As before, the natural approach is to restate the result using propositions.
	\begin{align*}
		x\in\comp{(A\cap B)}&\iff \neg\bigl(x\in A\cap B\bigr) \iff \neg\bigl(x\in A\ \text{ and }\  x\in B\bigr)\\
		&\iff \neg\bigl(x\in A\bigr)\ \text{ or }\ \neg\bigl(x\in B\bigr) \tag*{(de Morgan's first law of logic)}\\
		&\iff x\in\comp A\ \text{ or }\ x\in\comp B\\
		&\iff x\in\comp A\cup\comp B\tag*{\qedhere}
	\end{align*}
\end{proof}

Our final results describe the interaction of unions and intersections.\par

\begin{thm}[lower separated=false, sidebyside, sidebyside align=top seam, sidebyside gap=0pt, righthand width=0.3\linewidth]{Distributive laws}{setdist}
	For any sets $A,B,C$:
	\begin{enumerate}\setlength{\itemsep}{2pt}
		\item $A\cap(B\cup C)=(A\cap B)\cup(A\cap C)$
		\item $A\cup(B\cap C)=(A\cup B)\cap(A\cup C)$
	\end{enumerate}
	The Venn diagram illustrates the second result: think about adding the colored regions.
	\tcblower
	\flushright\includegraphics[scale=0.95]{sets-07-venndist}
\end{thm}


\begin{proof}
	We prove only the first result.
	\begin{align*}
		x\in A\cap(B\cup C) &\iff x\in A\text{ and }x\in B\cup C\\
		&\iff x\in A \text{ and }\bigl(x\in B\text{ or }x\in C\bigr)\\
		&\iff \bigl(x\in A \text{ and }x\in B\bigr)\text{ or }\bigl(x\in A\text{ and }x\in C\bigr) \tag{distributive law, page \pageref{pg:asidelogicalgebra}}\\
		&\iff x\in A\cap B\text{ or }x\in A\cap C\\
		&\iff x\in (A\cap B)\cup(A\cap C)\tag*{\qedhere}
	\end{align*}
\end{proof}

Remember, if you prefer, you can prove these equalities in two stages: $S=T\Longleftrightarrow S\subseteq T$ and $S\supseteq T$.\vspace{-10pt}

\goodbreak

% \paragraph{Self-test Questions}
% 
% \begin{enumerate}
%   \item The set operations of complement, union and intersection are based, respectively, on the logical constructions \underline{\phantom{not\quad}}, \underline{\phantom{or\quad}}, and \underline{\phantom{and\quad}}.
%   \item The result $\comp{(A\cup B)}=\comp A\cap\comp B$ is one of \underline{\phantom{De Morgan's laws\qquad}}.
%   \item True or false: if $A$ and $B$ are finite sets, then $A\cap B$ has strictly smaller cardinality than $A$.
%   \item True or false: if $A$ is a finite set, then $\comp A$ is a finite set.
%   \item True of false: if $A$ and $B$ are finite sets, then $\nm{A\cup B}\le\max(\nm A,\nm B)$.
% \end{enumerate}

\begin{exercises}{}{}
	A reading quiz and several questions with linked video solutions can be found \href{http://www.math.uci.edu/~ndonalds/math13/selftest/4-2-union.html}{online}.
	
	\begin{enumerate}
	  \item Describe each set straightforwardly as you can: e.g., 
	  \[
	  	\bigl\{x\in\R:x^2<9\text{ and } x^3<8\bigr\}= (-3,3)\cap(-\infty,2) =(-3,2)
	  \]
	  \begin{enumerate}
		 	\item \makebox[235pt][l]{$\bigl\{x\in\R:x^2\neq x\bigr\}$\hfill (b)} \ \ $\bigl\{x\in\R:x^3-2x^2-3x\le 0\text{ or }x^2=4\bigl\}$
		 	\setcounter{enumii}{2}
			\item \makebox[235pt][l]{$\bigl\{y\in\R:\exists x\in\R \text{ with }y=x^2 \text{ and }x\neq 1\bigl\}$ \hfill (d)} \ $\bigl\{z\in\Z:z^2\text{ is even and $z^3$ is odd}\bigl\}$
		 	\setcounter{enumii}{4}
			\item $\bigl\{y\in 3\Z+2:y^2\equiv 1\spmod 3\bigl\}$
		\end{enumerate}
	  
	  \item Let $A=\{1,3,5,7,9\}$, $B=\{1,4,7,10\}$ and $\cU=\{1,2,\ldots,10\}$. What are the following sets?
	    \begin{enumerate}
		  	\item \makebox[100pt][l]{$A\cap B$\hfill (b)} \ \makebox[100pt][l]{$A\cup B$\hfill (c)} \ \makebox[100pt][l]{$B\setminus A$\hfill (d)} \ $\comp A$ 
		  	\setcounter{enumii}{4}
		  	\item \makebox[100pt][l]{$\comp{(A\setminus B)}$ \hfill (f)} \ \makebox[100pt][l]{$\comp A\cap \comp B$\hfill (g)} \ $(A\cup B)\setminus (A\cap B)$
			\end{enumerate}
			
		
		\item In Example \ref*{ex:basicunionint}.\ref{ex:basicunionint2}, use logic to formally justify the assertions $\comp B=[1,5]$, $A\cap B=[-3,1)$, and $A\cup B=[-4,2]$. If you prefer, use the `subset of each side' approach of Example \ref*{ex:basicunionint}.\ref{ex:basicunionint3}.
	
  	
% 	\item Consider Theorems \ref{thm:setbasic} and \ref{thm:setdist}. In all seven results, replace the symbols in the first row of the following table with those in the second. Which of the results seem familar? Which are false?
% \[\begin{array}{c|c|c|c|c}
% \emptyset&A,B,C\text{ sets}&\cup&\cap&\subseteq\\\hline
% 0&A,B,C\in\N_0&+&\cdot&\le
% \end{array}\]

	%\item Prove that $B\setminus A=B\iff A\cap B=\emptyset$.
	
		\item Give formal proofs of the following parts of Theorems \ref{thm:setbasic}, \ref{thm:setcomp} and \ref{thm:setdist}. %With practice you should be able to prove \emph{all} of parts of these theorems \emph{without} looking at the arguments in the notes!
		\begin{enumerate}
		  \item \makebox[180pt][l]{$\emptyset\cap A=\emptyset$\hfill (b)} \ $A\cap (B\cap C)=(A\cap B)\cap C$
		  \setcounter{enumii}{2}
			\item \makebox[180pt][l]{$\comp{(\comp A)}=A$\hfill (d)} \ $A\cup(B\cap C)=(A\cup B)\cap(A\cup C)$
			\setcounter{enumii}{4}
			\item $A\subseteq B\iff \comp B\subseteq \comp A$
		\end{enumerate}
	
	
		\item By showing that each side is a subset of the other, give a formal proof of the set identity
		\[
			A=(A\setminus B)\cup (A\cap B)
		\]
		Now repeat your argument using only results from set algebra (Theorems \ref{thm:setcomp} and \ref{thm:setdist}).
	
	
	\item Prove the identity $A\cup B=A\iff B\subseteq A$ for any sets $A,B$.
	
	
	\item Prove the identities for any sets $A,B,C$:
	\begin{enumerate}
	  \item \makebox[200pt][l]{$\comp{(A\cap B\cap C)}=\comp A\cup\comp B\cup\comp C$ \hfill(b)} \ 
	  $(A\cup B)\setminus (A\cap B)=(A\setminus B)\cup (B\setminus A)$
	\end{enumerate}
	
		
	\item Prove or disprove the following conjectures (\emph{Hint: revisit Section \ref{sec:proof2}}).
	 \begin{enumerate}
	    \item \makebox[200pt][l]{$\exists x\in\R\setminus\Q$ such that $x^2\in\Q$\hfill (b)} \ $\forall x\in\R\setminus\Q$ we have $x^2\in\Q$
		\end{enumerate}
	
	
	  \item Let $A\subseteq\R$, and let $x\in\R$. We say that $x$ is \emph{far away} from the set $A$ if and only if:
	  \[
	  	\exists d>0 \ \text{ such that } A\cap[x-d,x]=\emptyset
	  \] 
		If this does not happen, we say that $x$ is \emph{close to} $A$.
  	\begin{enumerate}
			\item Draw a picture of a set $A$ and elements $x,y$ such that $x$ is \emph{far away} from and $y$ is \emph{close to} $A$. 
			\item State the meaning of ``$x$ is close to $A$'' \ (negate ``$x$ is far away from $A$'').
			%\item Let $A=\{1,2,3\}$. Show that $x=4$ is \emph{far away} from $A$ using the definition.
			%\item Let $A=\{1,2,3\}$. Show that $x=1$ is \emph{close} to $A$.
			\item Let $A=\{1,2,3\}$.
			\begin{enumerate}
			  \item Show that $x=4$ is \emph{far away} from $A$ using the definition.
				\item Let $A=\{1,2,3\}$. Show that $x=1$ is \emph{close} to $A$.
			\end{enumerate}
			\item For general $A\subseteq\R$, show that if $x\in A$, then $x$ is \emph{close} to $A$.
			\item Let $A=(a,b)$ be a bounded interval. Is the end-point $a$ \emph{far away} from $A$?  What about $b$?
  	\end{enumerate}
  	
	\end{enumerate}

\end{exercises}

\clearpage



\subsection{Introduction to Functions}\label{sec:func1}

Sets become a lot more useful and interesting once you start transforming their elements! This is accomplished using \emph{functions.} In this section we introduce some basic concepts and notation, much of which should be familiar. A formal definition will be given in Chapter \ref{chap:relations}, but for the present a naïve notion will suffice.

\begin{defn}{}{function1}
	Let $A,B$ be sets. A \emph{function} $f:A\to B$ is a rule assigning to each input $a\in A$ a single output $f(a)\in B$.	Various sets are associated to $f$:\par
	\begin{minipage}[t]{0.61\linewidth}\vspace{-5pt}
		\emph{Domain}: $\dom(f)=A$ is the set of inputs to the function.\smallbreak
		\emph{Codomain}: $\operatorname{codom}(f)=B$ is the set of potential outputs.\smallbreak
		\emph{Image} of a subset $U\subseteq A$: the set of outputs given inputs in $U$
		\[
			f(U):=\bigl\{f(u)\in B:u\in U\bigr\}
		\]
		\emph{Range}: $\range(f)=f(A)=\{f(a)\in B:a\in A\}$ is the set of realized outputs, the image of the domain $A$.
	\end{minipage}
	\hfill
	\begin{minipage}[t]{0.38\linewidth}\vspace{-5pt}
		\flushright
   	\includegraphics[scale=0.95]{sets-16-funcdef2}
	\end{minipage}\medbreak
	\emph{Inverse image} (or \emph{pre-image}) of a subset $V\subseteq B$: the set of inputs which are mapped into $V$
	\[
		f^{-1}(V):=\bigl\{a\in A:f(a)\in V\bigr\}
	\]
\end{defn}

The rule defining a function can also be described using arrow notation $f:a\mapsto b$.

\begin{examples}{}{functions}
	\exstart It is common to \textcolor{blue}{graph} functions whose codomain is a subset of the real numbers: the \textcolor{Purple}{domain} and \textcolor{Green}{range} are found by projecting the graph onto the two axes.\par
	\begin{enumerate}\setcounter{enumi}{1}
		\begin{minipage}[t]{0.6\linewidth}\vspace{-8pt}
			\item[]For instance if $f:[-3,2)\to\R$ is the square function
			\[
				f:x\mapsto x^2 \tag{equivalently $f(x)=x^2$}
			\]
			then $\textcolor{Purple}{\dom(f)=[-3,2)}$ and $\textcolor{Green}{\range(f)=[0,9]}$. We could also calculate other images/pre-images, for example,
  		\begin{gather*}
  			f\bigl([-1,2)\bigr)=\bigl\{x^2:-1\le x<2\bigr\}=[0,4)\\[3pt]
  			\begin{aligned}
  				f^{-1}\bigl((-10,2]\bigr)&=\bigl\{x\in[-3,2):-10<x^2\le 2\bigr\}\\
  				&=[-\sqrt 2,\sqrt 2]
  			\end{aligned}
  		\end{gather*}
		\end{minipage}
		\hfill
		\begin{minipage}[t]{0.39\linewidth}\vspace{-8pt}
			\flushright
  		\includegraphics[scale=0.95]{sets-10-rangedom}
		\end{minipage}
		
 	  \bigbreak

	\begin{minipage}[t]{0.64\linewidth}\vspace{0pt}
		\item Define $f:\Z\to\{0,1,2\}$ by $f:n\mapsto n^2\pmod 3$. The table shows a few examples (remember $\dom(f)=\Z$ is infinite!).
	\end{minipage}
	\hfill
	\begin{minipage}[t]{0.35\linewidth}\vspace{0pt}
		\flushright
		$\begin{array}{c|cccccccc}
  		n&0&1&2&3&4&5&6&7\\\hline
  		f(n)&0&1&1&0&1&1&0&1
  	\end{array}$
	\end{minipage}\par
		Exercise \ref*{sec:cong}.\ref{exs:nsquaredrem} confirms what the table suggests, that $\range(f)=\{0,1\}$. Here also is a single inverse image (revisit the previous sections if you're unsure of the notation):
		\begin{align*}
			f^{-1}\bigl(\{1\}\bigr) &=\bigl\{x\in\Z:x^2\equiv 1\spmod 3\bigr\} =\bigl\{x\in\Z:x\equiv 1\text{ or }2\spmod 3\bigr\}\\
			&=(3\Z+1)\cup(3\Z+2)
		\end{align*}
		
		\goodbreak

  \begin{minipage}[t]{0.64\linewidth}\vspace{0pt}
    \item\label{ex:functmod} Let $A=\{0,1,2,\ldots,7\}$ be the set of remainders modulo 8 and define two functions $f,g:A\to A$:
    \[
    	f(n)=3n\spmod{8}\qquad g(n)=6n\spmod 8
    \]
  \end{minipage}
  \hfill
  \begin{minipage}[t]{0.35\linewidth}\vspace{0pt}
  	\flushright
  	$\begin{array}{c|cccccccccc}
  		n&0&1&2&3&4&5&6&7\\\hline
  		f(n)&0&3&6&1&4&7&2&5\\\hline
  		g(n)&0&6&4&2&0&6&4&2
  	\end{array}$	
  \end{minipage}
  \smallbreak
  This time the table completely describes the functions. Observe that
  \begin{align*}
  	&\range(f)=A &&f\bigl(\{1,5\}\bigr) =\{3,7\} &&f^{-1}\bigl(\{1,2,3,4\}\bigr) =\{1,3,4,6\}\\
  	&\range(g)=\{0,2,4,6\} &&g\bigl(\{1,5\}\bigr) =\{6\} &&g^{-1}\bigl(\{1,2,3,4\}\bigr) =\{2,3,6,7\}
  \end{align*}
  If you're unsure, compute these as we did in the last two examples.
  	
  \item Let $A=\{0,1,2,3,4\}$ and let $B=\{$two-element subsets of $A\}$. Define
  \[
  	f:A\to B:a\mapsto \{a,a+1\negthickspace \spmod 5\}
  \]
  where we take the remainder modulo 5. You should be able to convince yourself that
  \begin{gather*}
  	\range(f)=\big\{ \{0,1\}, \{1,2\}, \{2,3\}, \{3,4\}, \{4,0\} \big\}\\[3pt]
  	f\bigl(\{1,4\}\bigr)=\big\{f(1),f(4)\big\}=\big\{\{1,2\},\{4,0\}\big\} \quad\text{and}\quad f^{-1}\bigl\{\{2,4\},\{4,0\}\bigr\}=\{4\}
  \end{gather*}
	\end{enumerate}
  
\end{examples}


\boldsubsubsection{Injections, Surjections and Invertibility}

We turn our attention to perhaps the most important properties a function can possess.

\begin{defn}{}{11}
	Let $f:A\to B$ be a function. We say that $f$ is:
	\begin{enumerate}
	  \item \emph{Injective} (\emph{1--1}, an \emph{injection}) if distinct inputs produce distinct outputs. Otherwise said (we state the contrapositive),
		\[
			f(a_1)=f(a_2)\implies a_1=a_2 \tag{``$\forall a_1,a_2\in A$'' is typically hidden}
		\]
		\item \emph{Surjective} (\emph{onto}, a \emph{surjection}) if every potential output is realized: $B=\range(f)$. Equivalently,
		\[
			\forall b\in B,\ \exists a\in A\text{ such that }f(a)=b
		\]
		This merely expresses $B\subseteq\range(f)$; the reverse inclusion $\range(f)\subseteq B$ holds for any function.
		\item \emph{Bijective} (\emph{invertible}, a \emph{bijection}) if it is both injective and surjective. Equivalently,
		\[
			\forall b\in B,\ \exists \text{ a \emph{unique} }a\in A\text{ such that }f(a)=b
		\]
		When $f$ is bijective, its \emph{inverse function} is $f^{-1}:B\to A:b\mapsto a$.
	\end{enumerate}
\end{defn}


These are \emph{universal} statements, so \emph{counter-examples} are enough to demonstrate the negations:
\[
	\tcbhighmath{
		\begin{array}{@{}ll}
			\text{\emph{$f$ not injective}:}&\text{$\exists a_1\neq a_2\in A$ such that $f(a_1)=f(a_2)$}\\[5pt]
			\text{\emph{$f$ not surjective}:}&\text{$\exists b\in B$ such that $\forall a\in A$, $f(a)\neq b$}
		\end{array}
	}
\]

\goodbreak

\begin{examples*}{\ref{ex:functions}, cont.}{}
	We briefly revisit our previous examples.
	\begin{enumerate}
	  \item Let $f:[-3,2)\to\R:x\mapsto x^2$.
	  \begin{itemize}
	  	\item $f$ is non-injective: $f(1)=f(-1)$ provides a counter-example. \hfill ($a_1=1=-a_2$)
	  	\item $f$ is non-surjective: there is no $x\in[-3,2)$ for which $x^2=-5$. \hfill ($b=-5$)
	  \end{itemize}
	  
	  
	  \begin{minipage}[t]{0.73\linewidth}\vspace{-3pt}
	  	We can obtain a related injective function by shrinking the domain, for instance $g:\textcolor{Purple}{[0,2)}\to\R:x\mapsto x^2$.	Indeed
	  	\[
	  		g(x_1)=g(x_2) \implies x_1^2=x_2^2 \implies x_1=x_2
	  	\]
	  	since $x_1,x_2\in\textcolor{Purple}{[0,2)}$ are non-negative. By also shrinking the codomain, we get a surjective (now \emph{bijective}) function: $h:\textcolor{Purple}{[0,2)}\to \textcolor{Green}{[0,4)}:x\mapsto x^2$.
	  	\medbreak
	  	\emph{Proof of surjectivity}: Given $\textcolor{Green}{y\in[0,4)}$, let $\textcolor{Purple}{x}=\sqrt{\textcolor{Green}{y}}$, then $y=h(x)$.
	  \end{minipage}
	  \hfill
	  \begin{minipage}[t]{0.26\linewidth}\vspace{-3pt}
	  	\flushright\includegraphics{sets-18-rangedom2}
	  \end{minipage}

% \includegraphics[width=0.4\textwidth]{sets-19-rangedom3}


% 	  \begin{minipage}[t]{0.7\linewidth}\vspace{0pt}
% 	  	\item $f$ is neither injective ($f(a_3)=f(a_4)$) nor surjective $b_4\notin\range(f)$.
% 	  \end{minipage}
% 	  \hfill
% 	  \begin{minipage}[t]{0.29\linewidth}\vspace{0pt}
% 	  	\flushright
% 	  	$\begin{array}{c|cccc}
%   		a&a_1&a_2&a_3&a_4\\\hline
%   		f(a)&b_1&b_2&b_3&b_3
%   		\end{array}$
% 	  \end{minipage}


  	\item $f:\Z\to\{0,1,2\}:n\mapsto n^2\spmod 3$ is neither injective nor surjective.
  	\begin{itemize}
  		\item $f$ is non-injective: for instance $f(1)=f(2)$.
  		\item $f$ is non-surjective: $\range(f)=\{0,1\}\neq \{0,1,2\}=\operatorname{codom}(f)$.
		\end{itemize}
	
		\begin{minipage}[t]{0.64\linewidth}\vspace{-10pt}
    	\item Given $f,g:A\to A$ where $A=\{0,1,\ldots,7\}$ as in the table:
    	\begin{itemize}\itemsep0pt
      	\item $f$ is bijective: all elements of $\operatorname{codom}(f)$ appear exactly once in the $f$-row.
     	 	\item $g$ is non-injective: e.g., $g(0)=g(4)$.
      	\item $g$ is non-surjective: e.g., $1\notin\range(g)=\{0,2,4,6\}$.
    	\end{itemize}
  	\end{minipage}
  	\hfill
  	\begin{minipage}[t]{0.35\linewidth}\vspace{-10pt}
  		\flushright
  		$\begin{array}{c|cccccccccc}
  			n&0&1&2&3&4&5&6&7\\\hline
  			f(n)&0&3&6&1&4&7&2&5\\\hline
  			g(n)&0&6&4&2&0&6&4&2
  		\end{array}$	
  	\end{minipage}

  
	  \item Let $A=\{0,1,2,3,4\}$ and $f:A\to\bigl\{$two-element subsets of $A\bigr\}:a\mapsto \{a,a+1\spmod 5\}$
	  \begin{itemize}%\itemsep0pt
	    \item $f$ is injective: suppose $a_1,a_2\in A$, then
	    \[
	  		f(a_1)=f(a_2)\implies \{a_1,a_1+1\negthickspace\spmod 5\}=\{a_2,a_2+1\negthickspace\spmod 5\}\implies a_1=a_2
	  	\]
	  	\item $f$ is not surjective: e.g., $\{1,3\}\notin\range(f)$.
	  \end{itemize}
    
	\end{enumerate}
\end{examples*}


%To show that a function is bijective, it is enough to exhibit an inverse function. 
You should have seen the approach of the next example in other classes.

\begin{example}[lower separated=false, sidebyside, sidebyside align=top seam, sidebyside gap=0pt, righthand width=0.28\linewidth]{}{}
	We show that $f:(-\infty,2)\to (1,\infty): x\mapsto 1+\frac 1{(x-2)^2}$ is bijective by \emph{computing its inverse function.} Just solve for $x$ in terms of $y$: 
	\begin{align*}
		y=1+\frac 1{(x-2)^2}&\implies (x-2)^2=\frac 1{y-1}\\
		&\implies f^{-1}(y)=x=2-\frac 1{\sqrt{y-1}}
	\end{align*}
	The sign of the square-root was chosen so that $x\in\dom(f)=(-\infty,2)$.
	\tcblower
  \flushright\includegraphics{sets-09-bij2}
\end{example}

\goodbreak

% 
% \begin{aside}{}{}
% {\bf Inverse Functions}
% 
% The word \emph{invertible} is a synonym for bijective because bijective functions really have inverses! Indeed, suppose that $f:A\to B$ is bijective. Since $f$ is surjective, we know that $B=\range(f)$ and so every element of $B$ has the form $f(a)$ for some $a\in A$. Moreover, since $f$ is injective, the $a$ in question is unique. The upshot is that, when $f$ is bijective, we can construct a new \emph{function}
% \[f^{-1}:B\to A:f(a)\mapsto a.\]
% This may appear difficult at the moment but we will return to it in Chapter \ref{chap:relations}.
% 
% Instead, recall that in Calculus you saw that any injective function has an inverse. How does this fit with our definition? Consider, for example, $f:[0,2]\to\R:x\mapsto x^4$. This is injective but not surjective. To fix this, simply define a new function with the same formula but with codomain equal to the range of $f$. We obtain the bijective function
% \[g:[0,2]\to[0,16]:x\mapsto x^4,\]
% with inverse
% \[g^{-1}:[0,16]\to[0,2]:x\mapsto \sqrt[4]{x}.\]
% In Calculus we didn't nitpick like this and would simply go straight to $f^{-1}(x)=\sqrt[4]{x}$.\\[5pt]
% In general, if $f:A\to B$ is any injective function, then $g:A\to f(A):x\mapsto f(x)$ is automatically bijective, since we are forcing the codomain of $g$ to match its range.
% \end{aside}



\boldsubsubsection{Composition of Functions}

We consider how injectivity and surjectivity interact with composition of functions.

\begin{defn}[lower separated=false, sidebyside, sidebyside align=top seam, sidebyside gap=0pt, righthand width=0.4\linewidth]{}{}
	Given functions $f:A\to B$ and $g:B\to C$, their \emph{composition} is the function
	\[
		\textcolor{red}{g\circ f}:A\to C:a\mapsto g\bigl(f(a)\bigr)
	\]
	Note the order: to compute $(g\circ f)(a)$, first apply $f$, then $g$.
	\tcblower
	\flushright
	\includegraphics[scale=1]{sets-15-setcomp}
\end{defn}

In practice, some restriction of domains might be required in order to define a composition.

\begin{example}{}{}
	If $f(x)=x^2$ and $g(x)=\frac 1{x-1}$, then
	\[
		(g\circ f)(x)=\frac 1{x^2-1}\quad\text{and}\quad(f\circ g)(x)=\frac 1{(x-1)^2}
	\]
	Even though $\pm 1$ are legitimate inputs for $f$, $\dom(g\circ f) =\R\setminus\{\pm 1\}$ is implied so as to prevent division by zero. 
\end{example}


\begin{thm}{}{compinjsurj}
	Let $f:A\to B$ and $g:B\to C$ be functions. Then:
	\begin{enumerate}
	  \item If $f$ and $g$ are injective, then $g\circ f$ is injective.
	  \item If $f$ and $g$ are surjective, then $g\circ f$ is surjective.
	\end{enumerate}
	It follows that the composition of bijective functions is also bijective.
\end{thm}

\begin{proof}
	Suppose $f$ and $g$ are injective and let $a_1,a_2\in A$. Then
  \begin{align*}
	  (g\circ f)(a_1)=(g\circ f)(a_2)&\implies g\big(f(a_1)\big)=g\big(f(a_2)\big)\\
	  &\implies f(a_1)=f(a_2)\tag*{(since $g$ is injective)}\\
	  &\implies a_1=a_2\tag*{(since $f$ is injective)}\\[-25pt]
  \end{align*}
%   \item Now suppose that $f$ and $g$ are surjective. Let $c\in C$. We are required to show that $\exists a\in A$ such that $(g\circ f)(a)=c$.\\
%   Since $g$ is surjective, $\exists b\in B$ such that $g(b)=c$.\\
%   Similarly, since $f$ is surjective, $\exists a\in A$ such that $f(a)=b$.\\
%   Together we have $(g\circ f)(a)=g(f(a))=c$, as required.
%\end{enumerate}
	That is, $g\circ f$ is injective. We leave part 2 to the exercises.
\end{proof}

Somewhat surprisingly, the converse of this theorem is \emph{false.} If a composition is injective or surjective, \emph{only one} of the original functions is required also to be.

\begin{thm}[lower separated=false, sidebyside, sidebyside align=top seam, sidebyside gap=0pt, righthand width=0.4\linewidth]{}{}
	Suppose $f:A\to B$ and $g:B\to C$.
	\begin{enumerate}
	  \item If $g\circ f$ is injective, then $f$ is injective.
	  \item If $g\circ f$ is surjective, then $g$ is surjective.
	\end{enumerate}
	The picture illustrates what can happen: $f$ is \emph{only injective}, $g$ is \emph{only surjective,} but $g\circ f$ is \emph{bijective.}
	\tcblower
	\flushright
	\includegraphics[scale=1]{sets-17-injcomp}
\end{thm}


\goodbreak

\begin{example}{}{}
	Here is a formulaic version of the picture in the theorem. Make sure you're comfortable with the definitions and draw pictures or graphs to help make sense of what's going on.
	\begin{gather*}
		f:[0,2]\to[-4,4]:x\mapsto x^2\tag{injective only}\\
		g:[-4,4]\to [0,16]:x\mapsto x^2\tag{surjective only}\\
		g\circ f:[0,2]\to[0,16]:x\mapsto x^4\tag{bijective!}
	\end{gather*}
\end{example}


\begin{proof}
% \begin{enumerate}\setcounter{enumi}{1}
%   \item Suppose that $f(a_1)=f(a_2)$. Then $(g\circ f)(a_1)=(g\circ f)(a_2)$. Since $g\circ f$ is injective we conclude that $a_1=a_2$, whence $f$ is injective.\qedhere
	This time we leave part 1 for the Exercises. Let $c\in C$ and assume $g\circ f$ is surjective. But then
	\[
		\exists a\in A\text{ such that } c=(g\circ f)(a)=g\bigl(f(a)\bigr)
	\]
	Otherwise said, $\exists b(=f(a))\in B$ for which $c=g(b)$: that is, $g$ is surjective.%\qedhere
% \end{enumerate}
\end{proof}


\boldsubsubsection{Functions and Cardinality}

Injectivity and surjectivity are intimately tied to the notion of cardinality. In Chapter \ref{chap:cantor}, we will use such functions to \emph{define} cardinality for infinite sets. For the present we stick to finite sets. 

\begin{thm}{}{finitecard}
	Let $A$ and $B$ be finite sets. The following are equivalent:
	\begin{enumerate}\itemsep0pt
	  \item $\nm A\le\nm B$ \qquad\qquad 2. \ $\exists f:A\to B$ injective \qquad\qquad 3. \ $\exists g:B\to A$ surjective
	\end{enumerate}
	Moreover, $\nm A=\nm B\iff\exists f:A\to B$ bijective.
\end{thm}

\begin{minipage}[t]{0.84\linewidth}\vspace{-3pt}
	The theorem asserts that \emph{any one} of the three numbered statements is true if and only if \emph{all} are. It might appear that six arguments are required but, by proving in a circle, we only need three: for instance \circint 1 $\Rightarrow$ \circint 3 holds because \circint 1 $\Rightarrow$ \circint 2 and \circint 2 $\Rightarrow$ \circint 3.\par
	The proof is very abstract, but if you focus on the picture it should make sense.
\end{minipage}
\hfill
\begin{minipage}[t]{0.15\linewidth}\vspace{-3pt}
	\flushright\includegraphics{sets-14-circlearg}%\vspace{-10pt}
\end{minipage}
\par


\begin{proof}
	Suppose $\nm A=m$, $\nm B=n$ and label the elements in roster notation:\par
	\begin{minipage}[t]{0.63\linewidth}\vspace{-14pt}
	\[
		A=\{a_1,a_2,\ldots,a_m\}\qquad B=\{b_1,b_2,\ldots,b_n\}
	\]
	\begin{description}
		\item[\normalfont \circint 1 $\Rightarrow$ \circint 2]
		If $m\le n$, \emph{define} $f:A\to B$ by $f(a_k)=b_k$ as in the picture. This is injective since the $b_1,\ldots,b_m$ are distinct.
	\end{description}
	\end{minipage}
	\hfill
	\begin{minipage}[t]{0.3\linewidth}\vspace{-20pt}
		\flushright
		$\xymatrix @C0pt @R10pt @M1.4pt @H15pt{
			\{a_1, \ar@{|->}[d]_f &a_2,&\ldots,&a_m\} \ar@<-1ex>@{|->}[d]_f\\
			\{b_1, \ar@<-1ex>@{|->}[u]_g &b_2,&\ldots,&b_m, \ar@<0ex>@{|->}[u]_g&\overbrace{b_{m+1},\ldots,b_n}\} \ar@(u,ur)@{|->}@/_3pc/[ullll]_g
		}$
	\end{minipage}
	\vspace{-2pt}
	\begin{description}%\itemsep0pt
		\item[\normalfont \circint 2 $\Rightarrow$ \circint 3] Suppose $f:A\to B$ is injective. Without loss of generality, label the elements of $B$ such that $b_k=f(a_k)$ for $1\le k\le m$. \emph{Define} the surjective function $g:B\to A$ as in the picture:\footnotemark{}%\vspace{-2pt}
	  \[
	  	g(b_k):=
	  	\begin{cases}
	  		a_k&\text{ if }k\le m\\
	  		a_1&\text{ if }k>m
	  	\end{cases}
	  \]
	  \item[\normalfont \circint 3 $\Rightarrow$ \circint 1] Suppose $g:B\to A$ is surjective. Without loss of generality, label the elements of $B$ such that $a_k=g(b_k)$ for $1\le k\le m$. Since the $b_k$ must be distinct, we see that $n\ge m$.
	\end{description}
	When $m=n$, the constructed functions are plainly \emph{bijections} with $f^{-1}=g$.
\end{proof}

\vspace{-5pt}

\footnotetext{%
	The elements $b_{m+1},\ldots,b_n$ could be mapped \emph{anywhere;} we choose $a_1$ for simplicity.
}

\goodbreak



\begin{exercises}
	A reading quiz and several questions with linked video solutions can be found \href{http://www.math.uci.edu/~ndonalds/math13/selftest/4-3-functions.html}{online}.

	\begin{enumerate}
	  \item For each of the following functions $f:A\to B$ determine whether $f$ is injective, surjective or bijective. Prove your assertions.
	  \begin{enumerate}
	    \item $f:[0,3]\to\R$ where $f(x)=2x$.
	    \item $f:[3,12)\to[0,3)$ where $f(x)=\sqrt{x-3}$.
	    \item $f:(-4,1]\to(-5,-3]$ where $f(x)=-\sqrt{x^2+9}$.
	  \end{enumerate}
	  
	  
	  \item Suppose that $f:[-3,\infty)\to[-8,\infty)$ and $g:\R\to\R$ are defined by
		\[
			f(x)=x^2+6x+1,\qquad\qquad g(x)=2x+3
		\]
		Compute $g\circ f$ and show that it is injective.
	  
	  
	  \item\begin{enumerate}
			\item Find a set $A$ so that the function $f:A\to\R:x\mapsto\sin x$ is injective.
			\item Find a set $B$ so that the function $f:\R\to B:x\mapsto\sin x$ is surjective.
		\end{enumerate}
	

	  \item A function $f:\R\to\R$ is \emph{even} if and only if $\forall x\in\R,\ f(-x)=f(x)$.
		\begin{enumerate}
			\item Prove that $f:x\mapsto x^2$ is even.
			\item By negating the definition, state what it means for a function \emph{not to be even.}
			\item Give an example of a function that is \emph{not} even: prove it. 
			\item Prove or disprove: for every $f,\,g:\R\to\R$ even, the composition $h=f\circ g$ is even.
		\end{enumerate}
		
		
		\begin{minipage}[t]{0.6\linewidth}\vspace{0pt}
			\item The picture in Definition \ref{defn:function1} illustrates a function
			\[
				f:\{a_1,a_2,a_3,a_4\}\to\{b_1,b_2,b_3,b_4\}
			\]
			State the following:\vspace{-2pt}
		\end{minipage}
		\hfill
		\begin{minipage}[t]{0.39\linewidth}\vspace{0pt}
			\flushright
			$\begin{array}{c|cccc}
	   		a&a_1&a_2&a_3&a_4\\\hline
	   		f(a)&b_1&b_2&b_3&b_3
	  	\end{array}$
		\end{minipage}	
		\begin{enumerate}
		  \item $\range(f)$\qquad\qquad (b) \ $f\bigl(\{a_1,a_4\}\bigr)$\qquad\qquad (c) \ $f^{-1}\bigl(\{b_3\}\bigr)$\qquad\qquad (d) \ $f^{-1}\bigl(\{b_4\}\bigr)$
		\end{enumerate}
	
	
		\item\begin{enumerate}
	    \item Let $A=\{a,b,c\}$, $B=\{1,2,3,4\}$ and $f:A\to B$ be the function defined by $f(a)=f(c)=1$ and $f(b)=3$. State the following:
	    	\begin{enumerate}
	    	  \item $f^{-1}\bigl(\{1\}\bigr)$\qquad\qquad (b) \ $f^{-1}\bigl(\{3\}\bigr)$\qquad\qquad (c) \ $f^{-1}\bigl(\{1,3\}\bigr)$\qquad\qquad (d) \ $f^{-1}\bigl(\{2,4\}\bigr)$
	    	\end{enumerate} 
	    \item Let $g:[-1,\infty)\to\R:x\mapsto x^2+2x+1$. Compute $g^{-1}\bigl((0,2]\bigr)$. 
	    \item Let $h:\R\to\R:x\mapsto \sin x$. Find $h^{-1}\bigl(\{-1,1\}\bigr)$. 
		\end{enumerate}
	
		
	  \item Define $f:(-\infty,0]\to\R$ and $g:[0,\infty)\to\R$ by
	  \[
	  	f(x)=x^2,\qquad g(x)=
	  	\begin{cases}
	  		\frac{x}{1-x}&x<1\\
	  		1-x&x\ge 1
	  	\end{cases}
	  \]
	  Is $g\circ f:(-\infty,0]\to\R$ surjective? Justify your answer.
		  
	  
	  \item\label{ex:kfunc} Recall Example \ref*{ex:functions}.\ref{ex:functmod}.
	  Consider the nine functions $f_k:A\to A:x\mapsto kx\pmod{10}$, where $k=1,2,\ldots,9$. Find the range of each $f_k$. Can you find a relationship between $k$ and the cardinality of $\range(f_k)$?
			%\item More generally, let $A=\{0,1,2\ldots,n-1\}$ be the set of remainders modulo $n$. If $f_k:A\to A:x\mapsto kx\spmod n$, conjecture a relationship between $\nm{\range(f_k)}$, $k$ and $n$. You don't need to prove your assertions.
	  
	  
	 	\goodbreak
	 
	  
	  \item Let $f:\R\to\R^+$ be the function defined by $f(x)=e^x$. Explain why the following ``proof'' that $f$ is surjective is incorrect. Then, give a correct proof.  
		\begin{proof}[Proof?]
			Let $e^x\in\R^+$ be arbitrary. Then $f(x)=e^x$. We conclude that $f$ is surjective.
		\end{proof}
		
	
		\item\begin{enumerate}
		  \item Show there is a bijection between $\Z$ and $2\Z$.
			\item Let $S$ be the set of all circles in the plane which are centered at the origin. Find a bijection between $S$ and $\R^+$.
			\item Let $A,B$ be \emph{finite} sets. If $A\subsetneq B$, is it possible for there to be a bijection between $A$ and $B$?
		\end{enumerate}
	
	
	  \item Prove that the composition of two surjective functions is surjective.
	  
	  
	  \item Suppose that $g\circ f$ is injective. Prove that $f$ is injective.
	  
	
	  \item Let $f:A\to B$. Prove the following:
	  \begin{enumerate}
	    \item $f$ is injective if and only if $\forall b\in B,\ f^{-1}\bigl(\{b\}\bigr)$ has \emph{at most} one element.
	    \item $f$ is surjective if and only if $\forall b\in B,\ f^{-1}\bigl(\{b\}\bigr)$ has \emph{at least} one element.
	  \end{enumerate}
	  (\emph{Taken together: $f$ is bijective $\iff$ $f^{-1}$ is a function ($f$ is invertible).})
	  
	  
		\item Prove that functional composition is associative. That is, if $f:A\to B$, $g:B\to C$, and $h:C\to D$ are functions, then for all $a\in A$ we have
			\[
				\bigl(f\circ(g\circ h)\bigr)(a) = \bigl((f\circ g)\circ h\bigr)(a)
			\]
	  
	  \item Following Theorem \ref{thm:compinjsurj}, the composition of bijective functions $f,g$ is itself bijective. Give a \emph{brief} explanation as to why $(g\circ f)^{-1}=f^{-1}\circ g^{-1}$.
	
		
		\item Let $f:A\to B$ and $X\subseteq A$. Fill in the blanks to complete a proof of the following facts:
		\begin{enumerate}
	  	\item $X\subseteq f^{-1}\bigl(f(X)\bigr)$.\qquad\qquad\qquad (b) \ If $f$ is injective, then $X=f^{-1}\bigl(f(X)\bigr)$.
		\end{enumerate}
		\begin{proof}
			\begin{enumerate}
		  	\item $x\in X\Longrightarrow f(x)\in \underline{\phantom{f(X)}}$. Let $y=f(x)$, then $x\in \underline{\phantom{f^{-1}\bigl(\{y\}\bigr)}} \subseteq f^{-1}\bigl(f(X)\bigr)$.
		  	\item $a\in f^{-1}\bigl(f(X)\bigr) \Longrightarrow \underline{\phantom{f(a)\in f(X)}}$, whence $\exists x\in X$ with $f(a)=f(x)$. By injectivity, \underline{\phantom{$x=a$}}, whence $a\in X$. We conclude that \underline{\phantom{$f^{-1}\bigl(f(X)\bigr)\subseteq X$}}. Combine with part (a) for the result.\qedhere
			\end{enumerate}
		\end{proof}
		
	
		\item Let $f:A\to B$ be a function and let $Y\subseteq B$. Prove the following facts:
		\begin{enumerate}
	    \item $f\bigl(f^{-1}(Y)\bigr) \subseteq Y$.\qquad\qquad\qquad (b) \ If $f$ is surjective, then $f\bigl(f^{-1}(Y)\bigr)=Y$.
		\end{enumerate}
	  
	
	
		\item (Hard)\lstsp Let $f:A\to B$ be a function and $X,Y\subseteq A$.
		\begin{enumerate}
		  \item Prove that $f(X\cap Y)\subseteq f(X)\cap f(Y)$.
		  \item If $f$ is injective, prove that $f(X\cap Y) = f(X) \cap f(Y)$.
		  \item If $f(X\cap Y)=f(X)\cap f(Y)$ for \emph{all} $X,Y\subseteq A$, prove that $f$ is injective.
		\end{enumerate}
		
	\end{enumerate}

\end{exercises}

