\section{Divisibility and the Euclidean Algorithm}\label{chap:gcd}

\def\divides{\mid}
\def\ndivides{\nmid}

In this section we introduce \emph{congruence,} which generalizes the notion of an integer's parity (evenness/oddness). The study of congruence is of fundamental importance to the sub-discipline of number theory, and provides some of the most straightforward examples of groups and rings. We will cover the basics in this section, returning in Chapter \ref{chap:relations} for more formal observations.


\subsection{Divisors, Remainders and Congruence}\label{sec:cong}


\begin{defn}{}{div}
	Let $m,n$ be integers. The proposition $n\mid m$ is read ``\emph{$n$ divides $m$,}'' and means
	\[
		\exists k\in\Z\text{ such that }m=kn
	\]
	We could also say that ``$n$ is a \emph{divisor} of $m$,'' that ``$m$ is \emph{divisible} by $n$,'' or that ``$m$ is a \emph{multiple} of $n$.''
\end{defn}

The negated symbol $\ndivides$ is read \emph{does not divide.}

\begin{examples}{}{}
	\exstart We write $4\divides 20$ since $20=4\times 5$. The same equation also says that $5\mid 20$.
	\begin{enumerate}\setcounter{enumi}{1}
		\item The proposition $9\ndivides 7$ is read ``\emph{9 does not divide 7.}'' It is shorthand for $\neg(9\mid 7)$.
	\end{enumerate}
\end{examples}

When integers do not divide, there is a \emph{remainder} left over. Your study of remainders likely goes back to elementary school when you first learned division and wrote division in the form 
\[
	33\div 5=6\mathbin{\mathsf r}3 \tag{read ``6 remainder 3''}
\]

An important foundational result says that unique remainders always exist.


\begin{thm}{Division Algorithm\footnotemark}{div}
	Suppose $m,n\in\Z$ with $n$ positive. Then there exist \emph{unique} integers $q,r$ (the \emph{quotient} and \emph{remainder}) for which
	\[
		m=qn+r\quad\text{and}\quad 0\le r<n
	\]
\end{thm}


In elementary school language: $m\div n=q\mathbin{\mathsf r}r$.%might be read ``$n$ goes $q$ times into $m$ with $r$ left over.''

\begin{examples}{}{}
	\exstart Given $m=23$ and $n=7$, we have $23=3\cdot 7+2$; that is $q=3$ and $r=2$.
	\begin{enumerate}\setcounter{enumi}{1}
		\item If $m=-11$ and $n=3$, then $-11=(-4)\cdot 3+1$; that is $q=-4$ and $r=1$.
		\item The formula $m=6q+4$, where $q\in\Z$, describes all the integers that have remainder 4 after division by 6.
	\end{enumerate}
\end{examples}

An \emph{algorithm} is typically a computational process: if $m>0$ one could view this as the repeated subtraction of $n$ from $m$ until the result $r=m-qn$ satisfies $0\le r<n$. To prove this more rigorously requires foundational ideas related to induction to which we will return in Chapter \ref{chap:induction}. For our current purposes, it is enough for the division algorithm to guarantee that every integer has a nicely-defined remainder when divided by some positive integer $n$. We use this to construct an alternative form of arithmetic.


\begin{defn}{}{}
	Let $a,b$ and $n$ be integers with $n$ positive. The proposition
	\[a\equiv b\spmod n\qquad \text{``\emph{$a$ is congruent to $b$ modulo $n$}''}\]
	means that $a$ and $b$ have the \emph{same remainder} on division by $n$. The integer $n$ is called the \emph{modulus.}
\end{defn}

\begin{examples}{}{}
	Consider remainders modulo \textcolor{blue}{3} (division by \textcolor{blue}{3}).
	\begin{enumerate}\itemsep1pt
	  \item We write $7\equiv 10\pmod{\textcolor{blue}{3}}$, since $7=2\cdot \textcolor{blue}{3}+\textcolor{red}{1}$ and $13=4\cdot \textcolor{blue}{3}+\textcolor{red}{1}$ have the same remainder ($\textcolor{red}{r=1}$).
	  \item We write $6\not\equiv 10\pmod{\textcolor{blue}{3}}$, since $6=2\cdot\textcolor{blue}{3}+\textcolor{red}{0}$ and $17=5\cdot\textcolor{blue}{3}+\textcolor{red}{2}$ have \textcolor{red}{different remainders}.
	\end{enumerate}
\end{examples}

Calculating using the division algorithm is tedious. Our next result is crucial in that it permits the direct comparison of remainders. This can be treated as an equivalent definition of congruence. 

\begin{thm}{}{congdiv}
	$a\equiv b\pmod n\mathrel{\textcolor{red}{\iff}} n\divides (b-a) \mathrel{\textcolor{blue}{\iff}} b=a+kn\text{ for some integer }k$
\end{thm}

\begin{proof}
	The \textcolor{blue}{second} biconditional is nothing more than an application of Definition \ref{defn:div}:
	\begin{align*}
		n\mid(b-a)&\iff \exists k\in\Z\text{ such that }b-a=kn\\
 		&\iff b=a+kn\text{ for some integer }k
	\end{align*}

	Before presenting direct arguments for each direction of the \textcolor{red}{first} biconditional, it is helpful to introduce notation from the division algorithm:
	\begin{align*}
		&a=q_1n+r_1\qquad b=q_2n+r_2\qquad 0\le r_1,r_2<n\\[1pt]
		\implies &b-a=(q_2-q_1)n+(r_2-r_1) \tag{$\ast$}
	\end{align*}
	
	($\textcolor{red}{\Rightarrow}$)\lstsp If $a\equiv b\pmod n$, then $a,b$ have the same remainder $r_1=r_2$. But then ($\ast$) says that $n\divides(b-a)$.
	\smallbreak
	
	($\textcolor{red}{\Leftarrow}$)\lstsp Assume that $n\mid(b-a)$ so that $b-a=kn$ for some integer $k$. By $(\ast)$, we see that
	\[
		r_2-r_1=(b-a)-(q_2-q_1)n=(k-q_2+q_1)n
	\]
	is divisible by $n$. Since the remainders satisfy $0\le r_1,r_2<n$, we moreover see that
	\[
		-n<r_2-r_1<n
	\]
	The only possibility is $r_2-r_1=0$. Otherwise said, $a,b$ have the same remainder: $a\equiv b\pmod n$.
\end{proof}

If you're having trouble with the last step, think about an example! Suppose $n=26$ and write $x=r_2-r_1$. Hopefully you believe that $x=0$ is the only \emph{integer} satisfying the two conditions,
\[
	x\text{ is divisible by }26\quad\text{and}\quad -26<x<26 %\rule[-10pt]{0pt}{10pt}
\]

Since the result is abstract, it is good to recap the relationship between congruence and divisibility.\vspace{-3pt}
\begin{tcolorbox}[highlight math, left=-4pt]
	\begin{itemize}\itemsep1pt
		\item Each $a\in\Z$ is congruent to \emph{exactly one} of the integers $0,1,2,\ldots,n-1$ modulo $n$: its \emph{remainder}.
		\item $a$ is divisible by $n$ if and only if $a\equiv 0\pmod n$. 
		%\item $a$ is \emph{not} divisible by $n$ if and only if $a\equiv 1,\ 2,\ 3,\ldots,$ or $n-1$ modulo $n$. 
	\end{itemize}
\end{tcolorbox}


\goodbreak


\begin{examples}{}{}
\exstart We describe all integers $x$ which are congruent to 7 on division by 11:
	\[
		x\equiv 7\spmod{11}\iff 11\mid(x-7) \iff x-7=11k\iff x=11k+7
	\]
	for some integer $k$.
	\goodbreak

\begin{enumerate}\setcounter{enumi}{1}
  \item To get more of a feel for the notation, consider the following conjectures:
	\begin{enumerate}
	    \item $a\equiv 8\pmod 6\Longrightarrow a\equiv 2\pmod 3$
	    \item $a\equiv 2\pmod 3\Longrightarrow a\equiv 8\pmod 6$
	\end{enumerate}
	Conjecture (a) is true. If $a\equiv 8\pmod 6$, then $a=6k+8$ for some integer $k$, from which
		\[
			a=6k+8 =3(2k+2)+2 \implies a\equiv 2\pmod 3
		\]
		Conjecture (b) is false. The \emph{counter-example} $a=5$ disproves this (universal) claim:
		\[
			5\equiv 2\spmod 3 \quad\text{and}\quad 5\not\equiv 8\spmod 6
		\]
	\end{enumerate}
\end{examples}



\boldsubsubsection{Modular Arithmetic}

Remainders have a natural arithmetic that is very similar to that of the real numbers. We the same symbols; even the congruence symbol $\equiv$ looks a bit like an equals sign!\footnote{This is no accident. In Chapter \ref{chap:relations} we'll see that congruence is an important example of an \emph{equivalence relation:} a generalized notion of equality. Indeed, two integers are congruent if and only if something about them is equal: their \emph{remainders}!} Apart from being fun, \emph{modular arithmetic} has many applications, including cryptography and data security: cell-phones and computers perform millions of these calculations every day! Here are the basic rules.

\begin{thm}{}{congbasic}
	Suppose $a,b,c,d$ are integers and that $n$ is some modulus. Then,
	\begin{enumerate}
	  \item If $a\equiv c$ and $b\equiv d$ modulo $n$, then
	  \[
	  	a\pm b\equiv c\pm d\quad\text{ and }\quad ab\equiv cd \pmod n
	  \]
	  \item The usual \textcolor{Brown}{associative}, \textcolor{Magenta}{commutative} and \textcolor{blue}{distributive} laws of arithmetic hold for congruences:
		\begin{align*}
			&\textcolor{Brown}{a+(b+c)\equiv (a+b)+c}
			&&\textcolor{Magenta}{a+b\equiv b+a}
			&&\textcolor{blue}{a(b+c)\equiv ab+ac} \hspace{75pt}\\
			&\textcolor{Brown}{a(bc)\equiv (ab)c}
			&&\textcolor{Magenta}{ab\equiv ba}
		\end{align*}
	\end{enumerate}
\end{thm}

The theorem says that the operations `take the remainder' and `add/subtract/multiply' can be performed in any order or combination, the result will be the same.

\begin{example}{}{}
	Find the \textcolor{red}{remainder} when $29+14$ is divided by 6. We do this in two ways:
	\begin{enumeratea}\itemsep0pt
		\item First find the sum 43, then compute its remainder: $43\equiv\textcolor{red}{1}\pmod 6$ since $6\mid(43-1)$.
		\item Alternatively, we could find the remainder of each component and then add:
		\[
			29+14\equiv 5+2\equiv 7\equiv \textcolor{red}{1}\pmod 6
		\]
	\end{enumeratea}
\end{example}

\begin{proof}
	\begin{enumerate}
	  \item We prove the multiplication rule. Suppose that $a\equiv c$ and $b\equiv d$. By Theorem \ref{thm:congdiv}, we have $c=a+kn$ and $d=b+ln$ for some integers $k,l$. Now compute:
		\[
			cd-ab=(a+kn)(b+ln)-ab =(bk+al+kln)n
		\]
		is divisible by $n$, whence $ab\equiv cd$. The addition/subtraction argument is almost identical.
		\item The associative, commutative and distributive laws hold because $x=y\implies x\equiv y\pmod n$, regardless of $n$ (equal numbers have the same remainder!).\qedhere
	\end{enumerate}	  
\end{proof}

The ability to take remainders \emph{before} adding and multiplying is remarkably powerful, and allows us rapidly to perform some surprising calculations.

\begin{examples}{}{}
	\exstart Find the remainder when $37^{423}$ is divided by 10.\vspace{-5pt}
	\begin{enumerate}\setcounter{enumi}{1}
	  \item[] The sheer size of $37^{423}$ makes this appear impossible at first glance.\footnotemark{}
	  Instead we think about the rules of arithmetic modulo 10. Since $37\equiv 7\equiv -3\pmod{10}$, we see that
		\[
			37\cdot 37\equiv (-3)\cdot(-3)\equiv 9\equiv -1\pmod{10}
		\]
		This is more promising, for we can use it to simplify the original expression:
		\begin{align*}
		37^{423}&\equiv \underbrace{(-3)\cdot(-3)\cdots(-3)}_{\text{423 times}} \equiv \bigl((-3)^2\bigr)^{211}(-3)\equiv (-1)^{211}(-3)\equiv 3\pmod{10}
		\end{align*}
	  
% 		\item Find the remainder when $39^{23}$ is divided by 10? At the outset this question appears impossible to answer. Ask your calculator and it will tell you that $39^{23}\approx 3.93\times 10^{36}$, which is of no assistance; we need to discover the \emph{units} digit of $39^{23}$, whereas your calculator reports only a few of the significant digits at the other end of the number.\\
% 		Instead of relying on a calculator, we think about the rules of arithmetic modulo 10. Since $39\equiv 9\equiv -1\pmod{10}$, we quickly notice that
% 		\[39\cdot 39\equiv (-1)\cdot(-1)\equiv 1\pmod{10},\]
% 		whence $39^2\equiv 1\pmod{10}$. Since positive integer exponents signify repeated multiplication, we can repeat the exercise to obtain
% 		\begin{align*}
% 		39^{23}&\equiv \underbrace{(-1)\cdot(-1)\cdots(-1)}_{\text{23 times}}=(-1)^{23}\equiv -1\equiv 9\pmod{10}
% 		\end{align*}
% 		Therefore $39^{23}$ has remainder 9 when divided by 10. Otherwise said, the last digit of $39^{23}$ is a 9. If you ask a computer for all  the digits you can check this yourself.
		
	  \item Here we compute modulo $n=6$:
	  \[
	  	7^9+14^3\equiv 1^9+2^3\equiv 1+8\equiv 9\equiv 3
	  \]
	  It would have been madness to compute $7^9+14^3=40356351$ before finding the remainder!
	  
	  \item Find the \textcolor{blue}{remainder} when $124^{12}\cdot 65^{49}$ is divided by 11.\par
	  This needs several steps and simplifications. Since $124=11^2+3$ and $65=11\cdot 6-1$, we write
	  \begin{align*}
	  	124^{12}\cdot 65^{49}&\equiv 3^{12}\cdot(-1)^{49}\equiv (3^3)^4\cdot (-1)\\
	  	&\equiv -(27^4)\equiv -(5^4)\\
	  	&\equiv -(25^2)\equiv -(3^2)\equiv \textcolor{blue}{2}\pmod{11}
	  \end{align*}
	  %The remainder is therefore 2. There is no way to do this on a pocket calculator, since the original number $124^{12}\cdot 65^{49}\approx 9\times 10^{113}$ is far too large to work with!
	\end{enumerate}
\end{examples}

\footnotetext{Using logarithms, a pocket calculator will tell you that $37^{423}\approx 2.2\times 10^{663}$ has 663 digits! This is no help since what we want is the \emph{units} digit, not its largest few significant figures.}

When performing these calculations:
\begin{itemize}
  \item Replace each integer by something \emph{small} with the same remainder: $37\equiv -3\spmod{10}$ is more helpful than $37\equiv 7\spmod{10}$, since powers of $-3$ are much easier to work with.
  \item The \textcolor{Green}{base} of an exponential expression can be reduced, but \emph{not} the \textcolor{red}{exponent}: $\textcolor{Green}{17}^{23}\equiv \textcolor{Green}{3}^{23}\spmod 7$ is correct, but $3^{\textcolor{red}{23}}\not\equiv 3^{\textcolor{red}{2}}\spmod 7$. Exponentiation is just shorthand for repeated multiplication.
\end{itemize}


\boldsubsubsection{Application: On what day were you born?}

While we all know our \emph{date} of birth, most of us do not know on which \emph{day} of the week we were born. You can answer this question quite easily (perhaps in your head!) using modular arithmetic.
\begin{itemize}
  \item Since $365\equiv 1\pmod 7$, a standard year advances the calendar one weekday.
  \item Each leap year\footnotemark{} advances the calendar an additional day.
%   \item Each month advances the day dependent on its length modulo 7:
%   \begin{itemize}
%     \item April, June, September, November: $30\equiv 2\pmod 7$
%     \item February (non-leap): $28\equiv 0\pmod 7$
%   	\item Everything else: $31\equiv 3\pmod 7$
% 	\end{itemize}
\end{itemize}

Can you figure the weekday today's date in your year of birth? Thinking about the length of each month modulo 7, you should also be able to find your birth\emph{day.}

\begin{example}{}{}
	Paul Revere was born January 1\st, 1735, in Boston. Given that January 1\st, 2024 was a Monday, find the weekday of Revere's birth.\smallbreak
	The dates differ by 289 years, in which time there have been $\frac{288}4-2=70$ leap years (not 1800 and 1900). The calendar has therefore advanced $289+70\equiv 2$ weekdays: Revere was born on a Saturday.
\end{example}

\footnotetext{Leap years occur whenever the year is divisible by 4. Among centuries, years divisible by 400 are \emph{not} leap years: thus 1900 wasn't a leap year but 2000 was. This is only practical back to the invention of the Gregorian calendar in the 1600s.}

\boldsubsubsection{Division and Congruence Equations}

Division in modular arithmetic behaves in unexpected ways. We'll consider it further in the exercises and the next section, but for the present just be careful. Converting congruences to statements about integers is the safest approach.

\begin{examples}{}{}
	\exstart Even when there is a common factor, dividing both sides is perilous. For instance
% 	\[
% 		42\equiv 12\spmod{10} \quad\text{but}\quad 21\not\equiv 6\spmod{10}
% 	\]
% 	However $21\equiv 6\pmod 5$. What went wrong? Think about the meaning of the left hand side:
	\begin{align*}
		42\equiv 12\spmod{\textcolor{red}{10}}&\implies \exists k\in\Z,\ \ 42-12=10k \implies \exists k\in\Z,\ \ 21-6=5k\\
		&\implies 21\equiv 6\spmod{\textcolor{red}{5}}
	\end{align*}
	Division by 2 also divided the \textcolor{red}{modulus}! If we hadn't divided the modulus, the result would be \emph{false}: $21\not\equiv 6\pmod{10}$.
	
% 	 However, both sides are still divisible by 3; dividing out we obtain the true statement $7\equiv 2\pmod 5$, \emph{without} dividing the modulus. What is going on?
		
		\begin{enumerate}\setcounter{enumi}{1}
		  \item Congruence equations are much harder to solve than standard equations. For instance, we cannot solve $2x\equiv 7\pmod 9$ by division: $x\equiv \frac 72$ is meaningless since $\frac 72$ is not an integer!\par 			It won't always work, but in this case sneakily \emph{multiplying} by 5 solves the problem:
			\[
				2x\equiv 7\implies 10x\equiv 35\implies x\equiv 8\spmod 9
			\]
\end{enumerate}
\end{examples}

 %We'll think about these issues more in the next section.% and Exercise 

% \begin{thm}{}{congdivide}
% 	If $ka\equiv kb\pmod{kn}$ then $a\equiv b\pmod n$.
% 	If $ka\equiv kb\pmod n$, then $a\equiv b\pmod{\frac n{\gcd(k,n)}}$
% \end{thm}



		
% 		\begin{minipage}[t]{0.65\linewidth}\vspace{0pt}
% 			\item\label{ex:congex2} Suppose $n$ is an integer. Then
% 			\[
% 				n^2\not\equiv n\spmod 3\iff n\equiv 2\spmod 3
% 			\]
% 			To demonstrate this simply complete the table (proof by cases): there are only three possible remainders modulo 3!
% 		\end{minipage}
% 		\hfill
% 		\begin{minipage}[t]{0.34\linewidth}\vspace{0pt}
% 			\flushright $
% 			\begin{array}{c|c|c}
% 				n&n^2&n^2\equiv n\pmod 3\text{?}\\\hline\hline
% 				0&0&\text{Yes}\\\hline
% 				1&&\\\hline
% 				2&&
% 			\end{array}
% 			$
% 		\end{minipage}

% \paragraph{Self-test Questions}

% \begin{enumerate}
%   \item True or false: $a\equiv b\pmod n\implies a=b$.
%   \item True of false: $a=b\implies a\equiv b\pmod n$.
%   \item An integer $m$ is \emph{divisible by} $n$ if \underline{\phantom{$\exists k\in\Z:m=kn$}\qquad\qquad}
%   \item A \emph{divisor} $b$ of an integer $a$ is \underline{\phantom{an integer $b$ such that $b\divides a$}\qquad\qquad}
%   \item True or false: if $m$ is divisible by $n$ then $n\equiv 0\pmod m$.
% \end{enumerate}

\begin{exercises}{}{}
	A reading quiz and several questions with linked video solutions can be found \href{http://www.math.uci.edu/~ndonalds/math13/selftest/3-1-cong.html}{online}.


\begin{enumerate}
  \item Check explicitly that $3^{23}\not\equiv 3^2\pmod 7$.
  
  \item Find the remainder when $12^9+19^{24}$ is divided by $10$.
  
  \item Compute the remainder when $30^{10}$ is divided by 13.
  
  \item Find all integers $x$ which satisfy the congruence equation $3x\equiv 2\mod 8$.
  
  \item Find the remainder when $17^{251}\cdot 23^{12}-19^{41}$ is divided by 5. \emph{Hint: $17\equiv 2$ and $2^2\equiv -1\pmod 5$.}
  
  \item Find the remainder when $12^{10}+2^{36}\cdot 18^{12}$ is divided by 141. \emph{Hint: what nice number is close to 141? Use a calculator to help with some of the sums.}
  
  \item Is the following statement identical to Example \ref{ex:congex}? Why/why not?
  \[n^2\equiv n\pmod 3\iff n\equiv 0\pmod 3\text{ or }n\equiv 1\pmod 3,\]
  
  \item Prove that if $a\equiv b\pmod n$ and $c\equiv d\pmod n$ then $3a-c^2\equiv 3b-d^2\pmod n$.

  \item Find a natural number $n$ and integers $a,b$ such that $a^2\equiv b^2\pmod n$ but $a\not\equiv b\pmod n$.
  
  \item\begin{enumerate}
    \item Let $n$ be a positive integer. Prove that $n$ is congruent to the sum of its digits modulo 9.\\
    \emph{Hint: first consider an example such as $345=3\cdot 10^2+4\cdot 10+5\ldots$}
    \item Is the integer 123456789 divisible by 9?
  \end{enumerate}
  
  \item Let $p$ be a prime number greater than or equal to 3. Show that if $p\equiv 1\pmod 3$, then $p\equiv 1\pmod 6$. \emph{Hint: $p$ is odd.} 
  
  \item\label{exs:congdivide} Suppose that $7x\equiv 28\pmod{42}$. By Theorem \ref{thm:congdivide}, it follows that $x\equiv 4\pmod{6}$.
  \begin{enumerate}
    \item Check this explicitly using Theorem \ref{thm:congdiv}.
    \item If $7x\equiv 28\pmod{42}$, is it possible that $x\equiv 4\pmod{42}$?
    \item Is it always the case that $7x\equiv 28\pmod{42}\implies x\equiv 4\pmod{42}$? Why/why not?
    \item Prove Theorem \ref{thm:congdivide}.
  \end{enumerate}
  
  \item If $a\divides b$ and $b\divides c$, prove that $a\divides c$.
  
  \item\label{ex:adivb} Let $a,b$ be positive integers. Prove that $a=b\iff a\divides b$ and $b\divides a$.
  
  \item Decide whether each conjecture is true or false and prove/disprove your assertions.
  \begin{description}
    \item[Conjecture 1:] $a\divides b$ and $a\divides c\implies a\divides bc$.
    \item[Conjecture 2:] $a\divides c$ and $b\divides c\implies ab\divides c$.
  \end{description}

	\item Fermat's Little Theorem (to distinguish it from his `Last') states that if $p$ is prime and $a\not\equiv 0\mod p$, then $a^{p-1}\equiv 1\pmod p$.
	\begin{enumerate}
	  \item Use Fermat's Little Theorem to prove that $b^p\equiv b\pmod p$ for \emph{any} integer $b$.
	  \item Prove that if $p$ is prime then $p\divides (2^p-2)$.
	  \item Find a counterexample to the converse: some non-prime $n$ such that $n\divides 2^n-2$.
	\end{enumerate}
	
	\item Abraham Lincoln was born February 12\th, 1809. On what day of the week was this?
	
	\item Prove that $p\mid n\implies p^2\mid n^2$
	
	\item Prove that $\forall n\in\Z, n^2\equiv 0$ or $1\pmod 3$
\end{enumerate}

\end{exercises}

\clearpage


\subsection{Greatest Common Divisors and the Euclidean Algorithm}\label{sec:gcd}

At its most basic, Number Theory involves finding \emph{integer} solutions to equations. Here are two simple-sounding questions:
\begin{enumerate}
  \item The equation $9x-21y=6$ represents a straight line in the plane. Are there any \emph{integer points} on this line? That is, can you find integers $x,y$ satisfying $9x-21y=6$?
  \item What about on the line $4x+6y=1$?
\end{enumerate}
Before you do anything else, try sketching both lines (lined graph paper will help) and try to decide if there are any integer points. If there are integer points, how many are there? Can you find them all?\\

In this section we will see how to answer these questions in general: for which lines $ax+by=c$ with $a,b,c\in\Z$, are there integer solutions, and how can we find them all? The method introduces the appropriately named \emph{Euclidean algorithm,} a famous procedure dating at least as far back as Euclid's \emph{Elements} (c. 300 BCE.).

\begin{defn}{}{}
Let $m$ and $n$ be integers, not both zero. Their \emph{greatest common divisor} $\gcd(m,n)$ is the largest (positive) divisor of both $m$ and $n$. We say that $m$ and $n$ are \emph{relatively prime} if $\gcd(m,n)=1$.
\end{defn}

\begin{example}{}{}
Let $m=60$ and $n=90$. The positive divisors of the two integers are listed in the table:
\[\begin{array}{c|cccccccccccc}
m&1&2&3&4&5&6&10&12&15&20&\underline{30}&60\\\hline
n&1&2&3&5&6&9&10&15&18&\underline{30}&45&90
\end{array}\]
The greatest common divisor is the largest number common to both rows: clearly $\gcd(60,90)=30$.
\end{example}

Finding the greatest common divisor of two integers by listing all the positive divisors of both numbers is extremely inefficient, especially when the integers are large. This is where Euclid rides to the rescue.\\

{\bf Euclidean Algorithm.} To find $\gcd(m,n)$ for two positive integers $m>n$:

\begin{itemize}
\item[(i)] Use the Division Algorithm (Theorem \ref{thm:div}) to write $m=q_1n+r_1$ with $0\le r_1<n$.
\item[(ii)] If $r_1=0$, then $n$ divides $m$ and so $\gcd(m,n)=n$. Otherwise, repeat:\\
	If $r_1>0$, divide $n$ by $r_1$ to obtain $n=q_2r_1+r_2$ with $0\le r_2<r_1$.
\item[(iii)] If $r_2=0$, then $\gcd(m,n)=r_1$. Otherwise, repeat:\\
	If $r_2>0$, divide $r_1$ by $r_2$ to obtain $r_1=q_3r_2+r_3$ with $0\le r_3<r_2$.
\item[(iv)] Repeat the process, obtaining a decreasing sequence of positive integers
\[r_1>r_2>r_3>\ldots>0\]
\end{itemize}


\begin{thm}{}{euclidalg}
The Algorithm eventually produces a remainder of zero: $\exists p$ such that $r_{p+1}=0$. The greatest common divisor of $m$ and $n$ is then the last non-zero remainder: $\gcd(m,n)=r_p$.
\end{thm}

 The proof is in the exercises. If $m$ and $n$ are not both positive, take absolute values first and apply the algorithm. For instance $\gcd(-6,45)=3$.

\begin{example}{}{}
We compute $\gcd(1260,750)$ using the Euclidean Algorithm. Since each line of the algorithm is a single case of the Division Algorithm $m=qn+r$, you might find it easier to create a table and observe each remainder moving diagonally left and down at each successive step.

\begin{minipage}{0.5\textwidth}
\vspace*{10pt}
\begin{itemize}\setlength{\itemsep}{0pt}
  \item[]$\textcolor{orange}{1260}=1\times \textcolor{blue}{750}+\textcolor{brown}{510}$
  \item[]$\textcolor{blue}{750}=1\times \textcolor{brown}{510}+\textcolor{red}{240}$
  \item[]$\textcolor{brown}{510}=2\times \textcolor{red}{240}+\textcolor{Green}{30}$
  \item[]$\textcolor{red}{240}=8\times \textcolor{Green}{30}+0$
\end{itemize}
\end{minipage}
\begin{minipage}{0.5\textwidth}
\renewcommand{\arraystretch}{1.35}
$\begin{array}{c|c|c|c}
m&q&n&r\\\hline
\textcolor{orange}{1260}&1&\textcolor{blue}{750}&\textcolor{brown}{510}\\
\textcolor{blue}{750}&1&\textcolor{brown}{510}&\textcolor{red}{240}\\
\textcolor{brown}{510}&2&\textcolor{red}{240}&\textcolor{Green}{30}\\
\textcolor{red}{240}&8&\textcolor{Green}{30}&0
\end{array}$
\end{minipage}
Theorem \ref{thm:euclidalg} says that $\gcd(\textcolor{orange}{1260},\textcolor{blue}{750})=\textcolor{Green}{30}$, the last non-zero remainder.
\end{example}

 As you can see, the Euclidean Algorithm is very efficient.

\subsubsection*{Reversing the Algorithm: Integer Points on Lines}

To apply the Euclidean Algorithm to the problem of finding integer points on lines, we must reverse it. We start with the penultimate line of the algorithm and substitute the remainders from the previous lines one at a time: the result is an expression of the form $\gcd(m,n)=mx+ny$ for some integers $x,y$. This is easiest to demonstrate by continuing our example.

\begin{example}{continued}{}
We find integers $x,y$ such that $1260x+750y=30$.\\[5pt]
Solve for \textcolor{Green}{30} (the gcd of $1260$ and $750$) using the third step of the algorithm:
\[\textcolor{Green}{30}=\textcolor{brown}{510}-2\times\textcolor{red}{240}.\]
Now use the second line of the algorithm to solve for \textcolor{red}{240} and substitute:
\[\textcolor{Green}{30}=\textcolor{brown}{510}-2\times (\textcolor{blue}{750}-\textcolor{brown}{510})=3\times \textcolor{brown}{510}-2\times \textcolor{blue}{750}.\]
Finally, substitute for \textcolor{brown}{510} using the first line:
\[\textcolor{Green}{30}=3\times (\textcolor{orange}{1260}-\textcolor{blue}{750})-2\times \textcolor{blue}{750}=3\times \textcolor{orange}{1260}-5\times \textcolor{blue}{750}.\]
Rearranging this, we see that the integers $x=3$ and $y=-5$ satisfy the equation $1260x+750y=30$. Otherwise said, the integer point $(3,-5)$ lies on the line with equation $1260x+750y=30$.
\end{example}

 Note how the process for finding an integer point $(x,y)$ is twofold: first we compute $\gcd(m,n)$ using the Euclidean Algorithm, then we perform a series of back-substitutions to recover $x$ and $y$.\\

 This process of reversing the algorithm works in general, and we have the following corollary of Theorem \ref{thm:euclidalg}.

\begin{cor}{Bézout's Identity}{euclid}
Given integers $m,n$, not both zero, there exist integers $x,y$ such that
\[\gcd(m,n)=mx+ny.\]
\end{cor}

We are now in a position to solve our motivating problem: finding all integer points on the line $ax+by=c$ where $a,b,c$ are integers. Again we appeal first to our example.

\begin{example}{take III}{}
We have already found a single integer solution $(x,y)=(3,-5)$ to the equation $1260x+750y=30$. Notice that the equation is equivalent to dividing through by the greatest common divisor $30=\gcd(1260,750)$:
\[42x+25y=1\]
Since 42 and 25 have no common factors, it seems that the only way to alter $x$ and $y$ while keeping the equation in balance is to increase $x$ by a multiple of 25 and decrease $y$ by the same multiple of 42. For example $(x,y)=(3+25,-5-42)=(28,-47)$ is another solution. Indeed, all integer solutions are given by
\[(x,y)=(3,-5)+(25,-42)t,\quad\text{where $t$ is any integer.}\]
\end{example}

In general, we have the following result.

\begin{thm}{}{diophanine}
Let $a,b,c$ be integers where $a,b$ are non-zero, and let $d=\gcd(a,b)$. Then the equation $ax+by=c$ has an integer solution $(x,y)$ if and only if $\,d\divides c$.\\
In such a case, suppose that $(x_0,y_0)$ is some fixed solution. Then all integer solutions are given by
\[x=x_0+\frac bdt,\qquad y=y_0-\frac adt,\tag*{($\ast$)}\]
where $t$ is any integer.
\end{thm}

 The general approach is to use the Euclidean Algorithm to find the initial solution $(x_0,y_0)$, then to apply ($\ast$) to obtain all solutions.\footnote{The astute observer should recognize the similarity between this and the complementary function/particular integral method for linear differential equations: $(x_0,y_0)$ is a `particular solution' to the full equation $ax+by=c$, while $(\frac bdt,-\frac adt)$ comprises all solutions to the `homogeneous equation' $ax+by=0$.} The proof is again in the exercises.\\

 Warning! If $c\neq\gcd(a,b)$, you will need to modify the integers obtained in Bézout's Identity in order to find the initial solution $(x_0,y_0)$. For example, since $1260\times 3+750\times(-5)=30$ we multiply by 3 to see that $(x_0,y_0)=(9,-15)$ is an initial solution to $1260x+750y=90$. All integer points on this line therefore have the form
\[(x,y)=(9+25t,-15-42t), \text{ where }t\in\Z\]

\begin{examples}{}{}
\begin{enumerate}
  \item Consider the line $570x-123y=7$. We calculate the greatest common divisor using the Euclidean algorithm: note that the negative sign is irrelevant.
  \[\renewcommand{\arraystretch}{1.1}\left.\begin{array}{l}
	570=4\times 123+78\\
	123=1\times 78+45\\
	78=1\times 45+33\\
	45=1\times 33+12\\
	33=2\times 12+9\\
	12=1\times 9+3\\
	9=3\times 3+0
  \end{array}\right\}\implies\gcd(570,123)=3.\]
  Since $\,3\nmid 7$, we conclude that the line $570 x-123 y=7$ contains no integer points.
  \item Applied to the line with equation $570x-123y=-6$, we reverse the algorithm to obtain
  \begin{align*}
  3&=12-9=12-(33-2\times 12)\\
  &=3\times 12-33=3(45-33)-33\\
  &=3\times 45-4\times 33=3\times 45-4(78-45)\\
  &=7\times 45-4\times 78=7(123-78)-4\times 78\\
  &=7\times 123-11\times 78=7\times 123-11(570-4\times 123)\\
  &=570\times (-11)-123\times (-51)
  \end{align*}
  Multiplying by $-2$ so that our solution conforms to the desired equation, it follows that $(x_0,y_0)=(22,102)$ is an initial solution. The general solution is then
  \[(x,y)=(22,102)+\left(-\frac{123}3,-\frac{570}3\right)t=(22-41t,102-190t)\]
\end{enumerate}
\end{examples}

% \paragraph{Self-test Questions}
% 
% \begin{enumerate}
%   \item True or false: $\gcd(21,-12)=-3$. What about $\gcd(-21,-12)=-3$?
%   \item Suppose that $a$ is a non-zero integer: which of the numbers $0,\ a$ or $\nm a$ is equal to $\gcd(a,0)$?
%   \item True or false: $1700x-340y=170$ has an integer solution $(x,y)$.
%   \item True or false. If $a$ and $b$ are relatively prime then the equation $ax+by=1$ has an integer solution $(x,y)$.
%   \item True or false: it is possible for a linear equation $ax+by=c$ where $a,b,c$ are integers to have \emph{exactly one} integer solution $(x,y)$.
% \end{enumerate}

\begin{exercises}{}{}
	A reading quiz and several questions with linked video solutions can be found \href{http://www.math.uci.edu/~ndonalds/math13/selftest/3-2-euclidalg.html}{online}.

\begin{enumerate}
  \item\label{qn:gcdea} Use the Euclidean Algorithm to compute the greatest common divisors indicated.
  \begin{enumerate}
    \item $\gcd(20,12)$\qquad (b)\ \ $\gcd(100,36)$\qquad (c)\ \ $\gcd(207,496)$
  \end{enumerate} 
  
  \item For each part of Question \hyperref[qn:gcdea]{\thesubsection.\ref*{qn:gcdea}}, find integers $x,y$ which satisfy Bézout's Identity $\gcd(m,n)=mx+ny$.
  
  
  \item\begin{enumerate}
    \item Answer our motivating problems from the beginning of the section using the above process.
  		\begin{enumerate}
    		\item[(i)] Find all integer points on the line $9x-21y=6$.
    		\item[(ii)] Show that there are no integer points on the line $4x+6y=1$.
  		\end{enumerate}
  	\item Can you give an elementary proof as to why there are no integer points on the line $4x+6y=1$?
  \end{enumerate} 
  
	\item Find all the integer points on the following lines, or show that none exist.
    \begin{enumerate}
      \item $16x-33y=2$.
      \item $122x+36y=3$.
      \item $303x+204y=6$.
      \item $324x-204y=-12$.
    \end{enumerate}
  
  \item Show that there exists no integer $x$ such that $3x\equiv 5\pmod 6$.
  
  \item Find all solutions $x$ to the congruence equation $12x\equiv 1\pmod{17}$
    
  \item Five people each take the same number of candies from a jar. Then a group of seven people does the same: in so doing they empty the jar. If the jar originally contained 239 candies. Can you be sure how much candies each person took?

  \item Here we sketch a proof that the Euclidean Algorithm (Theorem \ref{thm:euclidalg}) terminates with $r_p=\gcd(m,n)$. Note that you \emph{cannot} use Bézout's Identity in to prove any of what follows, since it is a corollary of the algorithm.
  \begin{enumerate}
    \item Suppose you have a decreasing sequence
    \[m>n>r_1>r_2>\cdots>0\tag*{($\ast$)}\]
    of positive integers. Explain why the sequence can only have \emph{finitely many} terms. This shows that the Euclidean Algorithm eventually terminates with some $r_{p+1}=0$.
  	\item Suppose that $m=qn+r$ for some integers $m,n,q,r$. Prove that $\gcd(m,n)\divides r$.
  	\item Explain why $\gcd(m,n)\divides r_p$.
  	\item Explain why $r_p$ divides all of the integers in the sequence ($\ast$), in particular that $r_p\divides m$ and $r_p\divides n$.
  	\item Explain why $r_p\le\gcd(m,n)$. Why does this force us to conclude that $r_p=\gcd(m,n)$?
  \end{enumerate}
  
  \item Suppose that $d\divides m$ and $d\divides n$. Prove that $d\divides\gcd(m,n)$.

  \item\label{ex:gcd1} Prove the following:
  \[\gcd(m,n)=1\iff\exists x,y\in\Z\text{ such that }mx+ny=1.\]
  \emph{One direction can be done by applying Bézout's Identity, but the other direction requires an argument.}
  
  
  \item In this question we prove the Theorem \ref{thm:diophanine} on integer solutions to linear equations. Let $a,b,c\in\Z$. Suppose that $(x_0,y_0)$ and $(x_1,y_1)$ are two integer solutions to the linear Diophantine equation $ax+by=c$.
  \begin{enumerate}
    \item Show that $(x_0-x_1,y_0-y_1)$ satisfies the equation $ax+by=0$.
    \item Suppose that $\gcd(a,b)=d$. Prove that $\gcd(\frac ad,\frac bd)=1$. (\emph{Use Question \hyperref[ex:gcd1]{\thesubsection.\ref*{ex:gcd1}})}
    \item Find all integer solutions $(x,y)$ to $ax+by=0$ (\emph{Don't use the Theorem, it's what you're trying to prove! Think about part (b) and divide through by $d$ first.}).
    \item Use (a) and (b) to conclude that $(x,y)$ is an integer solution to $ax+by=c$ if and only if
    \[x=x_0+\frac bdt\qquad y=y_0-\frac adt,\qquad \text{where }t\in\Z.\]
  \end{enumerate}
  
  \item Show that $\gcd(5n+2,12n+5)=1$ for every integer $n$. \emph{There are two ways to approach this: you can try to use the Euclidean algorithm abstractly, or you can use the result of Exercise \hyperref[ex:gcd1]{\thesubsection.\ref*{ex:gcd1}}.}
  
  \item Let $n$ be a positive integer. Complete the table
  \[\begin{array}{|l||c|c|c|c|c|c|}
  \hline
  n&1&2&3&4&5&6\\\hline
  \gcd(2n,n+1)&&&&&&\\\hline
  \end{array}\]
  Now make a conjecture for the value of $\gcd(2n,n+1)$ and prove it.
  
  \item The set of remainders $\Z_n=\{0,1,2,\ldots,n-1\}$ is called a \emph{ring} when equipped with addition and multiplication modulo $n$. For example $5+6\equiv 3\pmod{8}$. We say that $b\in\Z_n$ is an \emph{inverse} of $a\in\Z_n$ if
	\[ab\equiv 1\pmod n.\]
	\begin{enumerate}
	  \item Show that 2 has no inverse modulo 6.
	  \item Show that if $n=n_1n_2$ is composite ($\exists$ integers $n_1,n_2\ge 2$) then there exist elements of the ring $\Z_n$ which have no inverses.
	  \item Prove that $a$ has an inverse modulo $n$ if and only if $\gcd(a,n)=1$. Conclude that the only sets $\Z_n$ for which all non-zero elements have inverses are those for which $n$ is prime.\\
	  \emph{You will find Exercise \hyperref[ex:gcd1]{\thesubsection.\ref*{ex:gcd1}} helpful.}
	  
	  \item Let $\gcd(c,n)=d$. Prove that $ca\equiv cb\mod n\implies a\equiv b\pmod{\frac nd}$.
	\end{enumerate}
\end{enumerate}


\end{exercises}
