\section{Divisibility and the Euclidean Algorithm}\label{chap:gcd}


In this section we introduce \emph{congruence,} which generalizes the notion of an integer's parity (evenness/oddness). The study of congruence is of fundamental importance to the sub-discipline of number theory, and provides some of the most straightforward examples of groups and rings. We will cover the basics in this section, returning in Chapter \ref{chap:relations} for more formal observations.


\subsection{Divisors, Remainders and Congruence}\label{sec:cong}


\begin{defn}{}{div}
	Let $m,n$ be integers. The proposition $n\mid m$ is read ``\emph{$n$ divides $m$,}'' and means
	\[
		\exists k\in\Z\text{ such that }m=kn
	\]
	We could also say that ``$n$ is a \emph{divisor} of $m$,'' that ``$m$ is \emph{divisible} by $n$,'' or that ``$m$ is a \emph{multiple} of $n$.''
\end{defn}

The negated symbol $\nmid$ is read \emph{does not divide.}

\begin{examples}{}{}
	\exstart We write $4\mid 20$ since $20=4\times 5$. The same equation also says that $5\mid 20$.
	\begin{enumerate}\setcounter{enumi}{1}
		\item The proposition $9\nmid 7$ is read ``\emph{9 does not divide 7.}'' It is shorthand for $\neg(9\mid 7)$.
	\end{enumerate}
\end{examples}

When integers do not divide, there is a \emph{remainder} left over. Your study of remainders likely goes back to elementary school when you first learned division: for instance,\footnote{The common meaning of \emph{divide} is to apportion a quantity equally. Thus to divide 33 apples between 5 people, each person gets 6 apples and 3 are left over. In grade school mathematics, fractions come much later.}
\[
	33\div 5=6\mathbin{\mathsf r}3 \tag{``6 remainder 3''}
\]

An important foundational result says that unique remainders always exist.


\begin{thm}{Division Algorithm}{div}
	Suppose $m,n\in\Z$ with $n$ positive. Then there exist \emph{unique} integers $q,r$ (the \emph{quotient} and \emph{remainder}) for which
	\[
		m=qn+r\quad\text{and}\quad 0\le r<n
	\]
\end{thm}


In elementary school language, $m\div n=q\mathbin{\mathsf r}r$.

\begin{examples}{}{}
	\exstart Given $m=23$ and $n=7$, we have $23=3\cdot 7+2$; that is $q=3$ and $r=2$.
	\begin{enumerate}\setcounter{enumi}{1}
		\item If $m=-11$ and $n=3$, then $-11=(-4)\cdot 3+1$; that is $q=-4$ and $r=1$.
		\item The formula $m=6q+4$, where $q\in\Z$, describes all integers with remainder 4 on division by 6.
	\end{enumerate}
\end{examples}

An \emph{algorithm} is typically a computational process: if $m>0$ one could view this as the repeated subtraction of $n$ from $m$ until the result $r=m-qn$ satisfies $0\le r<n$. A rigorous proof requires foundational ideas related to induction to which we will return in Chapter \ref{chap:induction}. For our current purposes, we just need to know that remainders exist. Indeed our next step is to find a way to compare remainders \emph{without} explicitly invoking the division algorithm.


\begin{defn}{}{}
	Let $a,b$ and $n$ be integers with $n$ positive. The proposition
	\[a\equiv b\spmod n\qquad \text{``\emph{$a$ is congruent to $b$ modulo $n$}''}\]
	means that $a$ and $b$ have the \emph{same remainder} on division by $n$. The integer $n$ is called the \emph{modulus.}
\end{defn}

\begin{examples}{}{}
	Consider remainders modulo \textcolor{blue}{3} (division by \textcolor{blue}{3}).
	\begin{enumerate}\itemsep1pt
	  \item We write $7\equiv 10\pmod{\textcolor{blue}{3}}$, since $7=2\cdot \textcolor{blue}{3}+\textcolor{red}{1}$ and $13=4\cdot \textcolor{blue}{3}+\textcolor{red}{1}$ have the same remainder ($\textcolor{red}{r=1}$).
	  \item We write $6\not\equiv 10\pmod{\textcolor{blue}{3}}$, since $6=2\cdot\textcolor{blue}{3}+\textcolor{red}{0}$ and $17=5\cdot\textcolor{blue}{3}+\textcolor{red}{2}$ have \textcolor{red}{different remainders}.
	\end{enumerate}
\end{examples}

Calculating using the division algorithm is tedious. Our next result is crucial in that it permits the direct comparison of remainders. This can be treated as an equivalent definition of congruence. 

\begin{thm}{}{congdiv}
	$a\equiv b\pmod n\mathrel{\textcolor{red}{\iff}} n\mid (b-a) \mathrel{\textcolor{blue}{\iff}} b=a+kn\text{ for some integer }k$
\end{thm}

\begin{proof}
	The \textcolor{blue}{second} biconditional is nothing more than an application of Definition \ref{defn:div}:
	\begin{align*}
		n\mid(b-a)&\iff \exists k\in\Z\text{ such that }b-a=kn\\
 		&\iff b=a+kn\text{ for some integer }k
	\end{align*}

	Before presenting direct arguments for each direction of the \textcolor{red}{first} biconditional, it is helpful to introduce notation from the division algorithm:
	\begin{align*}
		&a=q_1n+r_1\qquad b=q_2n+r_2\qquad 0\le r_1,r_2<n\\[1pt]
		\implies &b-a=(q_2-q_1)n+(r_2-r_1) \tag{$\ast$}
	\end{align*}
	
	($\textcolor{red}{\Rightarrow}$)\lstsp If $a\equiv b\pmod n$, then $a,b$ have the same remainder $r_1=r_2$. But then ($\ast$) says that $n\mid(b-a)$.
	\smallbreak
	
	($\textcolor{red}{\Leftarrow}$)\lstsp Assume that $n\mid(b-a)$ so that $b-a=kn$ for some integer $k$. By $(\ast)$, we see that
	\[
		r_2-r_1=(b-a)-(q_2-q_1)n=(k-q_2+q_1)n
	\]
	is divisible by $n$. Since the remainders satisfy $0\le r_1,r_2<n$, we moreover see that
	\[
		-n<r_2-r_1<n
	\]
	The only possibility is $r_2-r_1=0$. Otherwise said, $a,b$ have the same remainder: $a\equiv b\pmod n$.
\end{proof}

If you're having trouble with the last step, think about an example! Suppose $n=26$ and write $x=r_2-r_1$. Hopefully you believe that $x=0$ is the only \emph{integer} satisfying the two conditions,
\[
	x\text{ is divisible by }26\quad\text{and}\quad -26<x<26 %\rule[-10pt]{0pt}{10pt}
\]

Since the result is abstract, it is good to recap the relationship between congruence and divisibility.\vspace{-3pt}
\begin{tcolorbox}[highlight math, left=-4pt]
	\begin{itemize}\itemsep1pt
		\item Each $a\in\Z$ is congruent to \emph{exactly one} of the integers $0,1,2,\ldots,n-1$ modulo $n$: its \emph{remainder}.
		\item $a$ is divisible by $n$ if and only if $a\equiv 0\pmod n$. 
		%\item $a$ is \emph{not} divisible by $n$ if and only if $a\equiv 1,\ 2,\ 3,\ldots,$ or $n-1$ modulo $n$. 
	\end{itemize}
\end{tcolorbox}


\goodbreak


\begin{examples}{}{}
\exstart We describe all integers $x$ which are congruent to 7 on division by 11:
	\[
		x\equiv 7\spmod{11}\iff 11\mid(x-7) \iff x-7=11k\iff x=11k+7
	\]
	for some integer $k$.
	\goodbreak

\begin{enumerate}\setcounter{enumi}{1}
  \item To get more of a feel for the notation, consider the following conjectures:
	\begin{enumerate}
	    \item $a\equiv 8\pmod 6\Longrightarrow a\equiv 2\pmod 3$
	    \item $a\equiv 2\pmod 3\Longrightarrow a\equiv 8\pmod 6$
	\end{enumerate}
	Conjecture (a) is true. If $a\equiv 8\pmod 6$, then $a=6k+8$ for some integer $k$, from which
		\[
			a=6k+8 =3(2k+2)+2 \implies a\equiv 2\pmod 3
		\]
		Conjecture (b) is false. The \emph{counter-example} $a=5$ disproves this (universal) claim:
		\[
			5\equiv 2\spmod 3 \quad\text{and}\quad 5\not\equiv 8\spmod 6
		\]
	\end{enumerate}
\end{examples}



\boldsubsubsection{Modular Arithmetic}

Remainders have a natural arithmetic that is very similar to that of the real numbers. We the same symbols; even the congruence symbol $\equiv$ looks a bit like an equals sign!\footnote{This is no accident. In Chapter \ref{chap:relations} we'll see that congruence is an important example of an \emph{equivalence relation:} a generalized notion of equality. Indeed, two integers are congruent if and only if something about them is equal: their \emph{remainders}!} Apart from being fun, \emph{modular arithmetic} has many applications, including cryptography and data security: cell-phones and computers perform millions of these calculations every day! Here are the basic rules, which generalize of most of the results we proved in Section \ref{sec:proof}.

\begin{thm}{}{congbasic}
	Suppose $a,b,c,d$ are integers and that $n$ is some modulus. Then,
	\begin{enumerate}
	  \item If $a\equiv c$ and $b\equiv d$ modulo $n$, then
	  \[
	  	a\pm b\equiv c\pm d\quad\text{ and }\quad ab\equiv cd \pmod n
	  \]
	  \item The usual \textcolor{Brown}{associative}, \textcolor{Magenta}{commutative} and \textcolor{blue}{distributive} laws of arithmetic hold for congruences:
		\begin{align*}
			&\textcolor{Brown}{a+(b+c)\equiv (a+b)+c}
			&&\textcolor{Magenta}{a+b\equiv b+a}
			&&\textcolor{blue}{a(b+c)\equiv ab+ac} \hspace{75pt}\\
			&\textcolor{Brown}{a(bc)\equiv (ab)c}
			&&\textcolor{Magenta}{ab\equiv ba}
		\end{align*}
	\end{enumerate}
\end{thm}

The theorem says that the operations `take the remainder' and `add/subtract/multiply' can be performed in any order or combination, the result will be the same.

\begin{example}{}{}
	Find the \textcolor{red}{remainder} when $29+14$ is divided by 6. We do this in two ways:
	\begin{enumeratea}\itemsep0pt
		\item First find the sum 43, then compute its remainder: $43\equiv\textcolor{red}{1}\pmod 6$ since $6\mid(43-1)$.
		\item Alternatively, we could find the remainder of each component and then add:
		\[
			29+14\equiv 5+2\equiv 7\equiv \textcolor{red}{1}\pmod 6
		\]
	\end{enumeratea}
\end{example}


\begin{proof}
	\begin{enumerate}
	  \item We prove the multiplication rule. Suppose that $a\equiv c$ and $b\equiv d$. By Theorem \ref{thm:congdiv}, we have $c=a+kn$ and $d=b+ln$ for some integers $k,l$. Now compute:
		\[
			cd-ab=(a+kn)(b+ln)-ab =(bk+al+kln)n
		\]
		is divisible by $n$, whence $ab\equiv cd$. The addition/subtraction argument is almost identical.
		\item The associative, commutative and distributive laws hold because $x=y\implies x\equiv y\pmod n$, regardless of $n$ (equal numbers have the same remainder!).\qedhere
	\end{enumerate}	  
\end{proof}

The ability to take remainders \emph{before} adding and multiplying is remarkably powerful, and allows us rapidly to perform some surprising calculations.

\begin{examples}{}{}
	\exstart Find the remainder when $37^{423}$ is divided by 10.\vspace{-5pt}
	\begin{enumerate}\setcounter{enumi}{1}
	  \item[] The sheer size of $37^{423}$ makes this appear impossible at first glance.\footnotemark{}
	  Instead we think about the rules of arithmetic modulo 10. Since $37\equiv 7\equiv -3\pmod{10}$, we see that
		\[
			37\cdot 37\equiv (-3)\cdot(-3)\equiv 9\equiv -1\pmod{10}
		\]
		This is more promising, for we can use it to simplify the original expression:
		\begin{align*}
		37^{423}&\equiv \underbrace{(-3)\cdot(-3)\cdots(-3)}_{\text{423 times}} \equiv \bigl((-3)^2\bigr)^{211}(-3)\equiv (-1)^{211}(-3)\equiv 3\pmod{10}
		\end{align*}
	  
% 		\item Find the remainder when $39^{23}$ is divided by 10? At the outset this question appears impossible to answer. Ask your calculator and it will tell you that $39^{23}\approx 3.93\times 10^{36}$, which is of no assistance; we need to discover the \emph{units} digit of $39^{23}$, whereas your calculator reports only a few of the significant digits at the other end of the number.\\
% 		Instead of relying on a calculator, we think about the rules of arithmetic modulo 10. Since $39\equiv 9\equiv -1\pmod{10}$, we quickly notice that
% 		\[39\cdot 39\equiv (-1)\cdot(-1)\equiv 1\pmod{10},\]
% 		whence $39^2\equiv 1\pmod{10}$. Since positive integer exponents signify repeated multiplication, we can repeat the exercise to obtain
% 		\begin{align*}
% 		39^{23}&\equiv \underbrace{(-1)\cdot(-1)\cdots(-1)}_{\text{23 times}}=(-1)^{23}\equiv -1\equiv 9\pmod{10}
% 		\end{align*}
% 		Therefore $39^{23}$ has remainder 9 when divided by 10. Otherwise said, the last digit of $39^{23}$ is a 9. If you ask a computer for all  the digits you can check this yourself.
		
	  \item Here we compute modulo $n=6$:
	  \[
	  	7^9+14^3\equiv 1^9+2^3\equiv 1+8\equiv 9\equiv 3
	  \]
	  It would have been madness to compute $7^9+14^3=40356351$ before finding the remainder!
	  
	  \item Find the \textcolor{blue}{remainder} when $124^{12}\cdot 65^{49}$ is divided by 11.\par
	  This needs several steps and simplifications. Since $124=11^2+3$ and $65=11\cdot 6-1$, we write
	  \begin{align*}
	  	124^{12}\cdot 65^{49}&\equiv 3^{12}\cdot(-1)^{49}\equiv (3^3)^4\cdot (-1)\\
	  	&\equiv -(27^4)\equiv -(5^4)\\
	  	&\equiv -(25^2)\equiv -(3^2)\equiv \textcolor{blue}{2}\pmod{11}
	  \end{align*}
	  %The remainder is therefore 2. There is no way to do this on a pocket calculator, since the original number $124^{12}\cdot 65^{49}\approx 9\times 10^{113}$ is far too large to work with!
	\end{enumerate}
\end{examples}

\footnotetext{Using logarithms, a pocket calculator will tell you that $37^{423}\approx 2.2\times 10^{663}$ has 663 digits! This is no help since what we want is the \emph{units} digit, not its largest few significant figures.}

When performing these calculations:
\begin{itemize}
  \item Replace each integer by something \emph{small} with the same remainder: $37\equiv -3\spmod{10}$ is more helpful than $37\equiv 7\spmod{10}$, since powers of $-3$ are much easier to work with.
  \item The \textcolor{Green}{base} of an exponential expression can be reduced, but \emph{not} the \textcolor{red}{exponent}: $\textcolor{Green}{17}^{23}\equiv \textcolor{Green}{3}^{23}\spmod 7$ is correct, but $3^{\textcolor{red}{23}}\not\equiv 3^{\textcolor{red}{2}}\spmod 7$. Exponentiation is just shorthand for repeated multiplication.
\end{itemize}


\boldsubsubsection{Application: On what day were you born?}

While we all know our \emph{date} of birth, most of us do not know on which \emph{day} of the week we were born. You can answer this question quite easily (perhaps in your head!) using modular arithmetic.
\begin{itemize}
  \item Since $365\equiv 1\pmod 7$, a standard year advances the calendar one weekday.
  \item Each leap year\footnotemark{} advances the calendar an additional day.
%   \item Each month advances the day dependent on its length modulo 7:
%   \begin{itemize}
%     \item April, June, September, November: $30\equiv 2\pmod 7$
%     \item February (non-leap): $28\equiv 0\pmod 7$
%   	\item Everything else: $31\equiv 3\pmod 7$
% 	\end{itemize}
\end{itemize}

Can you figure the weekday today's date in your year of birth? Thinking about the length of each month modulo 7, you should also be able to find your birth\emph{day.}

\begin{example}{}{}
	Paul Revere was born January 1\st, 1735, in Boston. Given that January 1\st, 2024 was a Monday, find the weekday of Revere's birth.\smallbreak
	The dates differ by 289 years, in which time there have been $\frac{288}4-2=70$ leap years (not 1800 and 1900). The calendar has therefore advanced $289+70\equiv 2$ weekdays: Revere was born on a Saturday.
\end{example}

\footnotetext{Leap years occur whenever the year is divisible by 4. Among centuries, years divisible by 400 are \emph{not} leap years: thus 1900 wasn't a leap year but 2000 was. This is only practical back to the invention of the Gregorian calendar in the 1600s.}

\boldsubsubsection{Division and Congruence Equations}

Division in modular arithmetic behaves in unexpected ways. We'll consider this further in Exercise \ref*{sec:gcd}.\ref{exs:congdivision}, but for the present it is safest to convert congruences to statements about integers.

\begin{examples}{}{congeqn1}
	\exstart Even when there is a common factor, dividing both sides is perilous. For instance
	\begin{align*}
		42\equiv 12\spmod{\textcolor{red}{10}}&\implies \exists k\in\Z,\ \ 42-12=10k \implies \exists k\in\Z,\ \ 21-6=5k\\
		&\implies 21\equiv 6\spmod{\textcolor{red}{5}}
	\end{align*}
	Division by 2 also divided the \textcolor{red}{modulus}! If we hadn't divided the modulus, the result would be \emph{false}: $21\not\equiv 6\pmod{10}$.
			
	\begin{enumerate}\setcounter{enumi}{1}
		\item Congruence equations are much harder to solve than standard equations. For instance, we cannot solve $2x\equiv 7\pmod 9$ by division: $x\equiv \frac 72$ is meaningless since $\frac 72$ is not an integer!\par 		It won't always work, but in this case sneakily \emph{multiplying} by 5 solves the problem:
		\[
			2x\equiv 7\implies 10x\equiv 35\implies x\equiv 8\spmod 9
		\]
	\end{enumerate}
\end{examples}


\begin{exercises}{}{}
	A reading quiz and several questions with linked video solutions can be found \href{http://www.math.uci.edu/~ndonalds/math13/selftest/3-1-cong.html}{online}.


	\begin{enumerate}  
		\item Prove, using the definition of ``divides,'' that $n\mid a$ and $n\mid b\implies n\mid(a+b)$.
		
		
		\item Let $a,b,c$ be integers. Prove or disprove each of the following claims:
		\begin{enumerate}
		  \item\makebox[180pt][l]{$a\mid b$ and $b\mid c\implies a\mid c$ \hfill (b)} \ $a=b\iff a\mid b$ and $b\mid a$
		  \setcounter{enumi}{2}
		  \item\makebox[180pt][l]{$a\mid b$ and $a\mid c\implies a\mid bc$ \hfill (d)} \ $a\mid c$ and $b\mid c\implies ab\mid c$
	  \end{enumerate}
	  
	  
	  \item List all integers $x$ for which $x\equiv 3\pmod 5$ and $-10\le x\le 20$.
	  
	  
	  \item Use the division algorithm to prove that if $p$ is an odd prime, then $p\equiv 1$ or $p\equiv 3\spmod 4$.
	  
	  
	  \item Prove the first part of Theorem \ref{thm:congbasic}: if $a\equiv c$ and $b\equiv d$, then $a+b\equiv c+d\pmod n$.
	 
	 
	  \item Find a positive integer $n$ and integers $a,b$ such that $a^2\equiv b^2\spmod n$ but $a\not\equiv b\spmod n$.
	  
	  
	  \item Check explicitly that $3^{23}\not\equiv 3^2\pmod 7$.\quad (\emph{Hint: $3^3=27\ldots$})
	  
	  
	  \item Compute the following remainders---use a calculator to help!
	  \begin{enumerate}
	  	\item $12^9+19^{24}$ on division by 10.
	  	\item $30^{10}$ on division by 13.
	  	\item $17^{251}\cdot 23^{12}-19^{41}$ on division by 5.\qquad (\emph{Hint: $17\equiv 2$ and $2^2\equiv -1\spmod 5$})
	  	\item (Hard)\lstsp $12^{10}+2^{36}\cdot 18^{12}$ on division by 141.\qquad	(\emph{Hint: what nice number is close to 141?})
	  \end{enumerate}
	  
	  
	  \item Prove that $3\mid(4^n-1)$ for all positive integers $n$.
	  
		
		\item Let $n$ be an integer. Prove that exactly one of $n$, $n+2$ and $n+4$ is divisible by 3.
	  
	  
	  \item\begin{enumerate}
	    \item Let $n$ be a positive integer. Prove that $n$ is congruent to the sum of its digits modulo 9.\par
	    (\emph{Hint: e.g.{} $345=3\cdot 10^2+\cdots$})
	    \item Is the integer 123456789 divisible by 9?
	  \end{enumerate}
	  
	    
	  \item Describe all integers $x$ which satisfy the congruence equations:
	  \begin{enumerate}
	    \item $3x\equiv 2\pmod 8$\qquad\qquad (b) \ $7x\equiv 28\pmod{42}$.
	  \end{enumerate}
		
		
		\item Abraham Lincoln was born February 12\th, 1809. On what day of the week was this?\par
		(\emph{Start by looking up the day for February 12\th{} \emph{this year}})
		
		
		\item\label{exs:nsquaredrem} Let $n$ be an integer.
		\begin{enumerate}
	    \item Prove: $n^2\equiv 0$ or $1\pmod 3$.\quad (\emph{Hint: prove by cases})
	    \item Prove: $n^2\equiv 0$ or $1\pmod 4$.
	    \item Find all possible remainders of $n^2$ on division by $7$.
	    \item Find all possible remainders of $n^3$ on division by $7$.
	  \end{enumerate}
	    
	    
		\item Use some part(s) of Exercise \ref{exs:nsquaredrem} to prove the following.
		\begin{enumerate}
			\item $\sqrt{4m+6}$ is not an integer, for any integer $m\ge -1$.
	    \item Any number which is simultaneously a square and a cube must be of the form $7k$ or $7k+1$ for some integer $k$.
	  \end{enumerate}
	    
	    
	  \item Let $n$ be an integer $\ge 2$ and consider numbers of the form $\underbrace{11\cdots 11}_{n \text{ times}}$
	  \begin{enumerate}
	    \item Prove every such number can be written as $4k+3$ for some $k\in\Z$.
	    \item Prove that no such number is a perfect square.
	  \end{enumerate}
		
	    
	  \item Fermat's Little Theorem states that if $p$ is prime and $a\not\equiv 0\spmod p$, then $a^{p-1}\equiv 1\spmod p$.
		\begin{enumerate}
		  \item Use Fermat's Little Theorem to prove that $b^p\equiv b\pmod p$ for \emph{any} integer $b$.
		  \item Prove that if $p$ is prime then $p\mid(2^p-2)$.
		  \item Verify that $341\mid (2^{341}-2)$. What does this say about the \emph{converse} to part (b)?
		\end{enumerate}
	  
	\end{enumerate}

\end{exercises}

\clearpage


\subsection{Greatest Common Divisors and the Euclidean Algorithm}\label{sec:gcd}

A basic goal for number theorists is to find \emph{integer} solutions to equations. For instance:
\begin{quote}
	Are there any \emph{integer points} on the line with equation $9x-21y=6$? That is, does the equation $9x-21y=6$ have any solutions, where both $x,y$ are \emph{integers}?
\end{quote}

You might start by sketching both lines (lined graph paper will help). What do you observe? If there are integer points, do they seem to follow any pattern?\smallbreak

In this section we will see how to solve all such \emph{linear Diophantine equations.}\footnote{Equations with integer coefficients and solutions honor the number theorist Diophantus of Alexandria (3\rd{}C.\,\CE).} The method introduces a famous procedure dating at least to Euclid's \emph{Elements} (c.\,300\BCE).

\begin{defn}{}{}
	Let $a$ and $b$ be integers, not both zero. Their \emph{greatest common divisor} $\gcd(a,b)$ is the largest (positive) divisor of both $a$ and $b$. We say that $a$ and $b$ are \emph{relatively prime} if $\gcd(a,b)=1$.
\end{defn}

\begin{example}{}{}
	The positive divisors of $a=60$ and $b=90$ are listed in the table. The greatest common divisor $\gcd(60,90)=\textcolor{blue}{30}$ is plainly the largest number common to both rows.
	\[
		\begin{array}{c|cccccccccccc}
			a&1&2&3&4&5&6&10&12&15&20&\textcolor{blue}{30}&60\\\hline
			b&1&2&3&5&6&9&10&15&18&\textcolor{blue}{30}&45&90
		\end{array}
	\]
\end{example}

Listing divisors is very inefficient given large integers. This is where Euclid rides to the rescue.

\begin{thm}{Euclidean Algorithm}{euclidalg}
	Let $a>b$ be positive integers. We construct a decreasing sequence of integers $b=r_0>r_1>r_2>\cdots$
	\begin{enumerate}\itemsep1pt
		\item \makebox[300pt][l]{Apply the division algorithm (Theorem \ref{thm:div}):}$a=q_1b+r_1$ with $0\le r_1<b$
		\item \makebox[300pt][l]{If $r_1>0$, apply the division algorithm again:}$b=q_2r_1+r_2$ with $0\le r_2<r_1$
		\item \makebox[300pt][l]{If $r_2>0$, apply again:}$r_1=q_3r_2+r_3$ with $0\le r_3<r_2$
		\item While $r_i>0$, keep repeating the division algorithm, dividing each $r_{i+1}$ by $r_i$.
	\end{enumerate}
	The algorithm eventually terminates with a remainder of zero: some $r_{t+1}=0$. The greatest common divisor is the last non-zero remainder: $\gcd(a,b)=r_t$.
\end{thm}

Exercise \ref{exs:euclidalgproof} provides a proof. 
If either of $a,b$ are negative just ignore the signs: for instance $\gcd(-4,34)=\gcd(34,4)=2$.

\begin{example}{}{euclidalg}
	We compute $\gcd(1260,750)$ using the Euclidean algorithm. Note how each line is a single instance of the division algorithm $a=qb+r$ and how the remainders move diagonally $\swarrow$ at each step. For this first example, we also summarize the data in a table.\vspace{-10pt}
	\[
		\renewcommand{\arraystretch}{1.1}
		\begin{array}{r@{\,\,}l@{\hspace{150pt}}c|c|c|c}
			& & a & q & b & r \\\cline{3-6}
			\textcolor{orange}{1260} & =1\times\textcolor{blue}{750}+\textcolor{brown}{510} & \textcolor{orange}{1260}& 1 & \textcolor{blue}{750} & \textcolor{brown}{510}\\
			\textcolor{blue}{750} & =1\times\textcolor{brown}{510}+\textcolor{red}{240} & \textcolor{blue}{750}& 1 & \textcolor{brown}{510} & \textcolor{red}{240}\\
			\textcolor{brown}{510} & =2\times\textcolor{red}{240}+\textcolor{Green}{30} & \textcolor{brown}{510} & 2 & \textcolor{red}{240} & \textcolor{Green}{30}\\
			\textcolor{red}{240} & =8\times\textcolor{Green}{30}+0 & \textcolor{red}{240} & 8 & \textcolor{Green}{30} & 0
		\end{array}
	\]
	Theorem \ref{thm:euclidalg} says that $\gcd(\textcolor{orange}{1260},\textcolor{blue}{750})=\textcolor{Green}{30}$, the last non-zero remainder.
\end{example}

\goodbreak


\boldsubsubsection{Reversing the Algorithm}

To apply the Euclidean algorithm to our motivational problem of integer points on lines, we run it backwards. Start with the penultimate line of the algorithm and move upwards, substituting remainders one at a time. The result is an expression $\gcd(a,b)=ax+by$ for some integers $x,y$.

\begin{example*}{\ref{ex:euclidalg}, cont}{}
	We find integers $x,y$ such that $\textcolor{orange}{1260}x+\textcolor{blue}{750}y=\textcolor{Green}{30} \bigm(=\gcd(\textcolor{orange}{1260},\textcolor{blue}{750})\bigr)$.\smallbreak
	Start by expressing $\textcolor{Green}{30}$ using the third step of the algorithm and work upwards:
	\begin{align*}
		\textcolor{Green}{30} &=\textcolor{brown}{510}-2\times\textcolor{red}{240} & &\text{(3\rd/penultimate line)}\\
		&=\textcolor{brown}{510}-2(\textcolor{blue}{750}-\textcolor{brown}{510}) =3\times \textcolor{brown}{510}-2\times \textcolor{blue}{750} &&\text{(2\nd{} line)}\\
		&=3(\textcolor{orange}{1260}-\textcolor{blue}{750})-2\times \textcolor{blue}{750} &&\text{(1\st{} line)}\\
		&=3\times \textcolor{orange}{1260}-5\times \textcolor{blue}{750}
	\end{align*}
	Rearranging, we see that $x=3$ and $y=-5$ satisfy the equation $1260x+750y=30$. To simplify, divide everything by \textcolor{Green}{30} to see that $(3,-5)$ is an integer point on the line $42x+25y=1$.
\end{example*}


This process of reversing the algorithm works in general, yielding a very powerful result.

\begin{cor}{Bézout's Identity}{euclid}
	Given integers $a,b$, not both zero, $\exists x,y\in\Z$ such that\vspace{-8pt}
	\[
		\gcd(a,b)=ax+by
	\]
\end{cor}

If either $a,b$ are negative, apply the algorithm to $\nm{a},\nm{b}$ and correct the signs afterwards.

\begin{cor}{}{moddivision}
	Suppose $a,b,c$ are integers for which $\gcd(a,b)=1$ and $a\mid bc$. Then $a\mid c$.
\end{cor}

\begin{proof}
	By Bézout's identity, $\exists x,y\in\Z$ for which $ax+by=1$, whence $acx+bcy=c$. The left hand side is now a multiple of $a$.
\end{proof}

\begin{examples}{}{2-3div6}
	\exstart The claim ``\textcolor{blue}{$a\mid c$ and $b\mid c\Longrightarrow ab\mid c$}'' is, in general, \emph{false} (e.g.{} $a=b=c=2$). However: Suppose $c=2m=3n$ for some integers $m,n$. Since $\gcd(2,3)=1$ and $2\mid(3n)$, we see that $2\mid n$. Thus $n=2k$ for some $k\in\Z$ and $c=6k$. We conclude:
	\[
		2\mid c\ \text{ and }\ 3\mid c\implies 6\mid c
	\]
	Though we could have done this example without heavy machinery (use Theorem \ref{thm:oddodd2}), Bézout  proves that the \textcolor{blue}{general claim} holds whenever $a,b$ are coprime.
% 	 
	\begin{enumerate}\setcounter{enumi}{1}
	  \item (Example \ref{ex:euclidalg}, mk.\,III)
	We know that $(x_0,y_0)=(3,-5)$ solves the equation $1260x+750y=30$ (equivalently $45x+25y=1$). Suppose $(x,y)$ is any other integer solution and observe that
	\[
		42(x-x_0)+25(y-y_0)=1-1=0 \implies 42(x-x_0)=-25(y-y_0)
	\]
	Since $\gcd(42,25)=1$, the corollary says $42\mid(y-y_0)$. Write $y-y_0=42t$ for some $t\in\Z$, then
	\[
		42(x-x_0)=-25\cdot 42t\implies x-x_0=-25t
	\]
	We've therefore found \emph{all} integer solutions to the original equation:
	\[
		x=3-25t,\ y=-5+42t\ \text{ where }t\in\Z
	\]
	\end{enumerate} 
\end{examples}


\goodbreak

The method in the example works for all solvable linear Diophantine equations.

\begin{thm}{}{diophantine}
	Let $a,b,c$ be integers where $a,b$ are not both zero, and let $d=\gcd(a,b)$.
	\begin{enumerate}\itemsep0pt
	  \item The equation $ax+by=c$ has an integer solution $(x,y)$ if and only if $\,d\mid c$.
	  \item If a solution exists, then there are infinitely many of them:%\footnotemark
		\[
			x=x_0-\frac bd t,\qquad y=y_0+\frac ad t\tag{$\ast$}
		\]
		where $t\in\Z$ and $(x_0,y_0)$ is some fixed solution (say as supplied by the Euclidean algorithm).
	\end{enumerate} 
\end{thm}


\textcolor{red}{Warning!}\lstsp If $c\neq d=\gcd(a,b)$, you will need to scale the integers obtained in Bézout's identity to find an initial solution $(x_0,y_0)$: see below.

\begin{example}{}{}
	We compute $\gcd(123,78) =\textcolor{blue}{3}$: remainders are underlined for clarity.
  \[\def\arraystretch{1.1}
  \def\ul#1{\underline{#1}}
  \left.\begin{array}{l}
	%570=4\times 123+78\\
	\ul{123}=1\times \ul{78}+\ul{45}\\
	\ul{78}=1\times \ul{45}+\ul{33}\\
	\ul{45}=1\times \ul{33}+\ul{12}\\
	\ul{33}=2\times \ul{12}+\ul 9\\
	\ul{12}=1\times \ul 9+\ul{\textcolor{blue}{3}}\\
	\ul 9=3\times \ul{\textcolor{blue}{3}}+\ul 0
  \end{array}\right\}
  \implies
  \left\{\begin{array}{r@{\,\,}l}
	\ul{\textcolor{blue}{3}}&=\ul{12}-\ul 9\\
	&=\ul{12}-(\ul{33}-2\times \ul{12}) =3\times \ul{12}-\ul{33}\\
	&=3(\ul{45}-\ul{33})-\ul{33} =3\times \ul{45}-4\times \ul{33}\\
	&=3\times \ul{45}-4(\ul{78}-\ul{45}) =7\times \ul{45}-4\times \ul{78}\\
	&=7(\ul{123}-\ul{78})-4\times \ul{78}\\
  &=\textcolor{red}{7}\times \ul{123}-\textcolor{red}{11}\times \ul{78}
  \end{array}\right.
  \]
  \begin{enumeratea}
  	\item Since $\textcolor{blue}{3}\nmid 8$, the equation $78x+123y=8$ has no integer solutions.
  	\item By Bézout's identity, the equation $123x-78y=-6$ has a particular solution
  	\[
  		(x_0,y_0)=(-14,22) \tag{\textcolor{red}{Warning}: Multiply $(\textcolor{red}{7},\textcolor{red}{-11})$ by $-2$}
  	\]
  	All integer solutions to the equation are then given by
  	\[
  		(x,y)=(-14,22)+\left(\tfrac{78}3,\tfrac{123}3\right)t=\bigl(-14+26t,22+41t\bigr),\quad\text{where }t\in\Z
  	\]
	\end{enumeratea}
\end{example}



\boldsubsubsection{Linear Congruence Equations}

We began to consider linear congruence equations in Example \ref{ex:congeqn1}. The ``points on lines" method also applies in this context. To see why, observe that
\[
	ax\equiv c\pmod n\iff \exists y\in\Z\text{ such that }ax=ny+c
\]
The Theorem tells us when we can solve the right hand side, and how. To solve the congruence equation, we just need to extract the $x$-part.

\begin{examples}{}{}
	\exstart To solve $123x\equiv -6\pmod{78}$ is to find a solution to the equation $123x=78y-6$. By the above, all solutions have the form $x=-14+26t$. Otherwise said
	\[123x\equiv -6\pmod{78} \implies x\equiv -14\equiv 12\pmod{26}\]
	
	\begin{enumerate}\setcounter{enumi}{1}
	  \item The congruence equation $4x\equiv 6\pmod{20}$ has no solutions. If it did, then the linear equation $4x=20y+6$ would have integer solutions, but it doesn't: $\gcd(4,20)=4$ does not divide 6. 
	  
	  \item Here is a slightly different approach for $7x\equiv 11\pmod{23}$. By applying the algorithm,
		\[
			\gcd(23,7)=1=7\cdot 10-23\cdot 3 \implies 7\cdot 10\equiv 1\pmod{23}
		\]
		The original equation may now be solved via \emph{multiplication}:
		\[
			7x\equiv 11 \implies x\equiv 10\cdot 7x\equiv 110\equiv 18\pmod{23}
		\]
		In this context, \emph{division by 7} really means \emph{multiplication\footnotemark{} by 10}.
	\end{enumerate}
\end{examples}


\footnotetext{In future studies, you'll refer to 10 as the \emph{multiplicative inverse of 7 in the ring $\Z_{23}$,} and write $7^{-1}=10$ (see Exercise \ref{exs:ringintro}).}


\begin{exercises}{}{}
	A reading quiz and several questions with linked video solutions can be found \href{http://www.math.uci.edu/~ndonalds/math13/selftest/3-2-euclidalg.html}{online}. Some of these questions are tricky, particularly the last few, and are included to give you a taste of upper-division number theory and abstract algebra.

	\begin{enumerate}
	  \item\label{qn:gcdea} Use the Euclidean algorithm to compute the greatest common divisors.
	  \begin{enumerate}
	    \item $\gcd(20,12)$\qquad (b)\ \ $\gcd(100,36)$\qquad (c)\ \ $\gcd(207,496)$
	  \end{enumerate} 
	  
	  \item For each part of Exercise \ref*{qn:gcdea}, find integers $x,y$ which satisfy Bézout's identity $\gcd(a,b)=ax+by$.
	  
	  \item Describe all the integer points on the line $9x-21y=6$ using Theorem \ref{thm:diophantine}.
	  
	  
	  \item\begin{enumerate}
	    \item Use Theorem \ref{thm:diophantine} to show that there are no integer points on the line $4x+6y=1$.
	  	\item Give an elementary proof (without using the Theorem) of part (a)?
	  \end{enumerate} 
	  
		\item Find all integer points $(x,y)$ on the following lines, or show that none exist.
	    \begin{enumerate}
	      \item \makebox[180pt][l]{$16x-33y=2$\hfill (b)} \ $122x+36y=3$
	      \setcounter{enumii}{2}
	      \item \makebox[180pt][l]{$303x+204y=6$\hfill (d)} \ $324x-204y=-12$
	    \end{enumerate}
	  
	  \item\begin{enumerate}
	    \item Show that there exists no integer $x$ such that $3x\equiv 5\pmod 6$.
	    \item Find all solutions $x$ to the congruence equation $12x\equiv 1\pmod{17}$.
	  \end{enumerate} 
	    
	  \item Five people each take the same number of candies from a jar. Then a group of seven people does the same: in so doing they empty the jar. If the jar originally contained 93 candies, can you be sure how much candy each person took?
	
	
	  \item\label{exs:euclidalgproof} We complete the proof of the Euclidean algorithm (Theorem \ref{thm:euclidalg}).
	  \begin{enumerate}
	    \item Suppose $a>b>r_1>r_2\cdots$ is a sequence of non-negative integers. Why must there be only \emph{finitely many} terms? This shows that the algorithm terminates with some $r_{t+1}=0$.
	    \item Suppose that $a,b,q,r$ are integers satisfying $a=qb+r$. Prove that $\gcd(a,b)=\gcd(b,r)$.\par
	    (\emph{You cannot use Bézout's identity to do this, since Bézout is a corollary of the algorithm!})
	    \item Argue that $\gcd(a,b)=r_t$.
	  \end{enumerate}
	  
	
		\item\label{exs:gcd0} Suppose $m\neq 0$. What is $\gcd(m,0)$? Why? Why is Bézout's identity trivial in this situation?
			
	  
	  \item Use Bézout's identity to prove that if $k\mid a$ and $k\mid b$, then $k\mid\gcd(a,b)$.
	  
	
	  \item\label{ex:gcd1} Prove: $\gcd(m,n)=1\iff\exists x,y\in\Z$ such that $mx+ny=1$.\par
	  (\emph{Hint: One direction follows from Bézout's identity, but the other\ldots})
	    
	    
		\item Use Example \ref{ex:2-3div6}.1 to prove the following.
		\begin{enumerate}
% 		  \item Let $c\in\Z$. Prove: $2\mid c$ and $3\mid c\implies 6\mid c$.\par
% 		  (\emph{Hint: write $c=2k=\cdots$ and apply Corollary \ref{cor:moddivision}})
		  \item Let $n\ge 3$ be an odd number. Show that $n\equiv 1\spmod 3\implies  n\equiv 1\spmod 6$.
		 	\item $\frac 16n(n+1)(2n+1)$ is an integer.
		\end{enumerate}
	  
	  
		\item Let $a,b,p\in\Z$. Recall Definition \ref{defn:irreducible}: the only positive divisors of a prime are itself and 1.
	  \begin{enumerate}
	    \item Suppose $p$ is prime and that $p\mid ab$. Prove that $p\mid a$ or $p\mid b$.\par
	    (\emph{Hint: if $p\nmid a$, use Corollary \ref{cor:moddivision}})
	    \item Suppose $p\ge 2$ is an integer with the property that $\forall a,b\in\Z$, $p\mid ab\Longrightarrow p\mid a$ or $p\mid b$. Prove that $p$ is prime.\par
	    (\emph{Hint: prove the contrapositive})
	  \end{enumerate}
	    
	    
	    \item (Hard)\lstsp Show that if $a$ is relatively prime to both $b$ and $c$ then it is relatively prime to $bc$.\par
	    (\emph{Hint: suppose $d\mid a$ and $d\mid bc$ and try to prove that $d=\pm 1$})
	    
	    
			\item\label{exs:congdivision} The general rule for congruence division is as follows:
			\[
				ac\equiv bc\spmod n \implies a\equiv b\spmod{\tfrac n{\gcd(c,n)}}
			\]
			\begin{enumerate}
			  \item Use the rule to find all solutions to the congruence equation $22x\equiv 66\pmod{77}$.
			  \item We prove the rule. Let $d=\gcd(c,n)$.
			  \begin{enumerate}
			    \item Explain why $\gcd(\frac cd,\frac nd)=1$
			  	\item If $ac\equiv bc\spmod n$, prove that $\frac nd\mid(a-b)$.
				\end{enumerate}
			\end{enumerate}
			
	    
	  \item\begin{enumerate}
	    \item Prove part 1 of Theorem \ref{thm:diophantine} using Bézout's identity.
	    \item Prove part 2 by mimicking the method in Example \ref{ex:euclidalg}, (mk.\,III).
	  \end{enumerate} 
	    
	    
	  \item (Hard)\lstsp Apply the discussion of the Euclidean algorithm and linear equations to the following.
	  
	  \begin{enumerate}
	    \item Complete the table.\par
	    \begin{minipage}[t]{0.55\linewidth}\vspace{-8pt}
	  	If $n$ is a positive integer, make a conjecture for the value of $\gcd(2n,n+1)$ and prove it.
	    \end{minipage}
	    \hfill
	    \begin{minipage}[t]{0.43\linewidth}\vspace{-20pt}
	    \flushright
	    $\begin{array}{|l||c|c|c|c|c|c|}
	  			\hline
	  			n&1&2&3&4&5&6\\\hline
	  			\gcd(2n,n+1)&&&&&&\\\hline
	  		\end{array}$
	    \end{minipage}
	
	  	\item Show that $\gcd(5n+2,12n+5)=1$ for every integer $n$.
	  \end{enumerate}
	
	
	  \item\label{exs:ringintro} The set of remainders $\Z_n=\{0,1,2,\ldots,n-1\}$ is called a \emph{ring} when equipped with addition and multiplication modulo $n$. We say that $b\in\Z_n$ is an \emph{inverse} of $a\in\Z_n$ if
		\[
			ab\equiv 1\pmod n
		\]
		\begin{enumerate}
		  \item Show that 2 has no inverse modulo 6.
		  \item Prove that $a$ has an inverse modulo $n$ if and only if $\gcd(a,n)=1$. Conclude that the only sets $\Z_n$ for which all non-zero elements have inverses are those for which $n$ is prime.	  
		\end{enumerate}
		
		
	  %\item Use the Euclidean algorithm to show that for any $k\in\N$, we have $\gcd(ka,kb) = k\gcd(a,b)$.
	  
	
	%     \item For nonzero integers $a$ and $b$, the \textbf{least common multiple} $\lcm(a,b)$ is defined to be the least positive integer $m$ which is a multiple of both $a$ and $b$.
	%     \begin{enumerate}
	%         \item If $m = \lcm(a,b)$, $a \mid c$, and $b \mid c$, show $m \mid c$.
	%         \item If $a$ and  $b$ are both positive, show $\gcd(a,b) \lcm(a,b) = ab$.
	%     \end{enumerate}
	  
	\end{enumerate}

\end{exercises}

